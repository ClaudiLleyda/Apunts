\documentclass[../../Main.tex]{subfiles}

\begin{document}
\part{Coses per fer}
\chapter{Aviat}
	Podeu trobar la versió actualitzada d'aquest pdf seguint aquest \href{https://claudilleyda.github.io}{link}.
	%	\subsection{Ignorar}
	%	\begin{enumerate}
	%		\item Definir \(\mathbb{R}^{+}\). (fot mandra)
	%		\item Adjuntar documents: llegir i aplicar punt \textsl{1.8 Big Projects} de lshort.pdf.
	%		\item Afegir més notes explicant la vida quan acabi. %sarcasm anyone?
	%		\item Definir \(\omicron(\vec{h})\) (que tendeix a \(\vec{0}\)).
	%		\item Definir imatge d'un conjunt per una funció (\(\Ima_{f}(U)\), \(f[U]\)...).
	%		\item Definit obert, tancat, compacte...
	%		\item Relacions d'equivalència.
	%	\end{enumerate}
\section{Àlgebra lineal}
	\subsection{Definicions}
	\begin{definition}[Forma bilineal]
		\labelname{forma bilineal}\label{def:forma bilineal}
		Doble la linealitat, doble el poder.
	\end{definition}
	\begin{definition}[Forma bilineal definida estrictament positiva o negativa]
		\labelname{forma bilineal definida estrictament positiva o negativa}\label{def:bilineal definida estrictament positiva/negativa}
		Sigui \(M\) una forma bilineal simètrica.
		Preguntar a en Cedò.
	\end{definition}
	\begin{definition}[Norma d'una aplicació lineal]
		\labelname{norma d'una aplicació lineal}\label{def:Norma d'una aplicació lineal}
	\end{definition}
	\begin{definition}[Producte escalar]
		\labelname{producte escalar}\label{def:producte escalar}
		És lineal i més coses.
	\end{definition}
	\begin{definition}[Norma d'un vector]
		\labelname{norma d'un vector}\label{def:norma d'un vector}
		\(\norm{\vec{v}}=\sqrt{\prodesc{\vec{v}}{\vec{v}}}\).
	\end{definition}
	\begin{definition}[Dependència lineal]
		\labelname{dependència lineal}\label{def:dependència lineal}
		\labelname{vectors linealment dependents}\label{def:vectors linealment dependents}
	\end{definition}
	\subsection{Proposicions}
	\begin{proposition}
		\label{prop:determinant diferent de zero linealment independents}
		Vectors linealment independents \(\sii\) determinant no nul.
	\end{proposition}
	\subsection{Teoremes}
	\begin{theorem}
		Sigui \(A=(a_{i,j})\in M_{n}(\mathbb{R})\) una matriu simètrica.
		Aleshores \(A\) és definida positiva si i només si
		\[\abs{\begin{matrix}
		a_{1,1}&\cdots&a_{1,i}\\
		\vdots&&\vdots\\
		a_{i,1}&\cdots&a_{i,i}\\
		\end{matrix}}>0\]
		per a tot \(i\in\{1,\dots,n\}\).
		\begin{proof}
			Per inducció sobre \(n\).
		\end{proof}
	\end{theorem}
\section{Funcions de variable real}
	\subsection{Definicions}
	\begin{definition}[Imatge]
		\labelname{imatge d'una aplicació}\label{def:imatge d'una aplicació}
		Siguin \(X\) i \(Y\) dos conjunts, \(A\) un subconjunt de \(X\) i \(f\colon X\longrightarrow Y\) una aplicació.
		Aleshores definim
		\[\Ima_{A}(f)=\{f(a)\in Y\mid a\in A\}\]
		com la imatge de \(f\) per \(A\).
	\end{definition}
	\begin{definition}[Antiimatge]
		\labelname{antiimatge d'una aplicació}\label{def:antiimatge d'una aplicació}
		Siguin \(X\) i \(Y\) dos conjunts, \(B\) un subconjunt de \(Y\) i \(f\colon X\longrightarrow Y\) una aplicació.
		Aleshores definim
		\[\Antiima_{B}(f)=\{x\in X\mid f(x)=b\text{ per a cert }b\in B\}\]
		com la antiimatge de \(f\) per \(B\).
	\end{definition}
	\begin{definition}[Límit]
		\labelname{límit}\label{def:límit}\
	\end{definition}
	\begin{definition}[Funció contínua]
		\labelname{funció contínua}\label{def:funcio continua}\label{def:funció contínua}
		Sigui \(f\) una funció.
		Direm que \(f\) és contínua en \(a\) si
		\[\lim_{x\to a}f(x)=f(a).\]
	\end{definition}
	\begin{definition}[Partició, poligonal i longitud d'una poligonal]
		\labelname{partició}\label{def:partició}
		\labelname{poligonal}\label{def:poligonal}
		\labelname{longitud d'una poligonal}\label{def:longitud de poligonal}
		Sigui \((a,b)\) un interval de \(\mathbb{R}\).
		Direm que una partició de \((a,b)\) és un conjunt de escalars \(t_{0},\dots,t_{n}\in[a,b]\) que compleixen \(a=t_{0}<\dots<t_{n}=b\).
		Direm que \(t_{0},\dots,t_{n}\) són els nodes de la partició.
		
		Sigui \(f\colon(a,b)\to\mathbb{R}^{m}\) una funció.
		Definirem la poligonal \(P_{n}\) d'una partició en una funció \(f\) com
		\[P_{n}=f(t_{0}),\dots,f(t_{n}).\]
		Definim la longitud de la poligonal \(P_{n}\) com
		\[\Long(P_{n})=\sum_{i=0}^{n-1}\left\lVert f(t_{i+1})-f(t_{i})\right\rVert.\]
	\end{definition}
	\begin{definition}[Classe de diferenciabilitat d'una funció]
		\labelname{classe de diferenciabilitat d'una funció}\label{def:Classe de diferenciabilitat}
		Sigui \(f\) una funció \(n\)-vegades diferenciable amb \(f^{(n)}\) contínua.
		Direm que \(f\) és de classe \(\mathcal{C}^{n}\) o que \(f\in\mathcal{C}^{n}\).
	\end{definition}
%	\begin{definition}[Funció monòtona]
%		\labelname{funció monòtona}\label{def:funció monòtona}
%		Sigui \(f\colon[a,b]\to\mathbb{R}\) una funció. Direm que és \(f\) és monòtona si, per a qualsevol \(x,y\in[a,b]\), \(x>y\) implica \(f(x)\geq f(y)\).
%	\end{definition}
	\begin{definition}[Derivada]
		\labelname{derivada}\label{def:derivada}
	\end{definition}
	\begin{definition}[Notació de Landau]
		\labelname{notació de Landau}\label{def:Landau}
		\(A(h)=\omicron(B(h))\dots\)
	\end{definition}
	\subsection{Proposicions}
	\begin{proposition}\label{prop:Derivable implica contínua}
		Siguin \(I\subseteq\mathbb{R}\) un interval i \(f\colon I\to\mathbb{R}\) una funció.
		Aleshores, si \(f\) és derivable en un punt \(a\in I\), \(f\) és contínua en \(a\).
	\end{proposition}
	\begin{proposition}
		Sigui \(f\colon[a,b]\to\mathbb{R}\) una funció acotada i monòtona.
		Aleshores \(f\) és integrable Riemann.
	\end{proposition}
	\begin{proposition}
		Sigui~\(f\colon[a,b]\longrightarrow\RR\) una funció monòtona.
		Aleshores el conjunt de punts de discontinuïtat de~\(f\) és com a màxim numerable.
	\end{proposition}
	\subsection{Teoremes}
	\begin{theorem}[Equivalència entre normes]
		\labelname{Teorema de l'equivalència entre normes}\label{thm:Equivalència de normes}
		Si \(q(x)\) és una norma existeixen \(m,M\in\mathbb{R}^{+}\) tals que \(m\left\lVert x\right\rVert\leq q(x)\leq M\left\lVert x\right\rVert\) per a tot \(x\in\mathbb{R}^{m}\).
	\end{theorem}
	\begin{theorem}[Teorema del Valor Mig]
		\label{Teorema del Valor Mig}\label{thm:TVM}
		\emph{hmm trivial}
	\end{theorem}
	\begin{theorem}[Desigualtat de Cauchy-Schwarz]
		\labelname{Teorema de la Desigualtat de Cauchy-Schwarz}\label{thm:Desigualtat de C-S}
	\end{theorem}
	\begin{theorem}[Teorema de Taylor]%Teorema de l'error de Taylor?
		\labelname{Teorema de Taylor}\label{thm:Teorema de Taylor} %caldrà adaptar-lo
		Siguin \((a,b)\) un interval obert de \(\mathbb{R}\) i \(f\colon(a,b)\rightarrow\mathbb{R}\) una funció de classe \(\mathcal{C}^{n}\).
		Aleshores, per a dos punts \(x,c\in(a,b)\), amb \(x<c\), existeix un punt \(x_{1}\in(x,c)\) tal que
		\[f(x)=f(c)+\sum_{k=1}^{n-1}\frac{f^{(k)}(c)}{k!}(x-c)^{k}+\frac{f^{(n)}(x_{1})}{n!}(x-c)^{n}.\]
	\end{theorem}
	\begin{theorem}[Teorema de Weierstrass]
		\labelname{Teorema de Weierstrass}\label{thm:Weierstrass màxims i mínims múltiples variables}
		Siguin \(U\subseteq\mathbb{R}^{d}\) un obert i \(f\colon U\to\mathbb{R}\) una funció.
		Aleshores, donat un compacte \(S\subset U\), si \(f\) és contínua en \(S\), \(f\) té un màxim i un mínim absoluts en \(S\).
	\end{theorem}
	\begin{theorem}[Teorema del sandvitx]
		\labelname{Teorema del sandvitx}\label{thm:sandvitx}
	\end{theorem}
	\begin{theorem}[Teorema de Rolle]
		\labelname{Teorema de Rolle}\label{thm:Teorema de Rolle}
		Sigui \(f\colon[a,b]\subset\mathbb{R}\longrightarrow\mathbb{R}\) una funció derivable en \((a,b)\) tal que \(f(a)=f(b)\).
		Aleshores existeix un cert \(c\in(a,b)\) tal que \(f'(c)=0\).
	\end{theorem}
	\begin{theorem}[Teorema Fonamental del Càlcul]
		\labelname{Teorema Fonamental del Càlcul}\label{thm:Teorema Fonamental del Càlcul}
		Siguin \([a,b]\) un interval de \(\mathbb{R}\), \(f\colon[a,b]\longrightarrow\mathbb{R}\) una funció contínua en \([a,b]\) i
		\[F(x)=\int_{a}^{x}f(t)dt\]
		una funció.
		Aleshores \(F(x)\) és derivable en \([a,b]\) i \(F'(x)=f(x)\) per a tot \(x\) de \([a,b]\).
	\end{theorem}
	\begin{theorem}[Teorema de Bolzano-Weierstrass]
		\labelname{Teorema de Bolzano-Weierstrass}\label{thm:Teorema de Bolzano-Weierstrass}
		Successió acotada té parcial convergent
	\end{theorem}
\section{Trobar lloc per tot això}
	\begin{definition}[Conjunt obert, tancat...]
		\labelname{conjunt obert}\label{def:obert CDVO}
		\labelname{conjunt tancat}\label{def:tancat CDVO}\label{conjunt obert, tancat...}\label{todo:conjunt definit per desigualtats estrictes és compacte}
	\end{definition}
	\begin{definition}[Continuïtat uniforme]
		\labelname{continuïtat uniforme}\label{def:uniformement contínua}
		Siguin \(U\subseteq\mathbb{R}^{d},V\subseteq\mathbb{R}^{m}\) dos oberts i \(f\colon U\to V\) una funció.
		Aleshores direm que \(f\) és uniformement contínua en un conjunt \(S\subseteq U\) si per a tot \(\varepsilon>0\) existeix un \(\delta>0\) tals que
		\[d_{U}(x,y)<\varepsilon\text{ i }d_{V}(f(x)-f(y))<\delta\]
		per a tot punt \(x,y\in U\).
	\end{definition}
	\begin{theorem}[Teorema de Heine]
		\labelname{Teorema de Heine}\label{thm:Teorema de Heine}
		Siguin \(U\subseteq\mathbb{R}^{d}\) dos oberts, \(S\subset U\) un compacte i \(f\colon U\to\mathbb{R}^{m}\) una funció contínua.
		Aleshores \(f\) és uniformement contínua en \(S\).
		%pàg 91 de Mathematical Analysis d'en Apostol (55 del pdf)
	\end{theorem}
	\begin{definition}[Funció indicatriu]
		\labelname{funció indicatriu}\label{def:funció indicatriu}
		Siguin \(X\) un conjunt i \(A\subseteq X\) un subconjunt de \(X\).
		Definim la funció indicatriu de \(A\) com una funció
		\[1_{A}\colon X\longrightarrow\{0,1\}\]
		tal que
		\begin{displaymath}
		1_{A}(x)=\begin{cases}
		1 & x\in A \\
		0 & x\notin A.
		\end{cases}
		\end{displaymath}
	\end{definition}
	\subsection{Complexos}
	\begin{definition}[Nombre complex]
		\labelname{nombre complex}\label{def:nombre complex}
		\labelname{unitat imaginària}\label{def:unitat imaginària}
		Siguin \(a\) i \(b\) dos nombres reals, \(\iu\) una constant que satisfà \(\iu^{2}=-1\) i
		\[z=a+b\iu.\]
		Aleshores direm que \(z\) és un nombre complex i que \(\iu\) és la unitat imaginària.
	\end{definition}
	\begin{definition}[Integral d'una funció complexa]
		\labelname{integral d'una funció complexa}\label{def:integral d'una funció complexa}
		Sigui \(f\) una funció complexa i \(a\) i \(b\) dos reals.
		Aleshores definim
		\[\int_{a}^{b}f(x)\diff x=\int_{a}^{b}\Re(f(x))\diff x+\iu\int_{a}^{b}\Im(f(x))\diff x\]
		com la integral de \(f\).
	\end{definition}
	\begin{definition}[Unió disjunta]
		\labelname{unió disjunta}\label{def:unió disjunta}
		Siguin \(A\) i \(B\) dos conjunts.
		Aleshores definim
		\[A\uniodisjunta B=(A\times\{0\})\cup(B\times\{1\})\]
		com la unió disjunta entre \(A\) i \(B\).
	\end{definition}
	\begin{definition}[Vectors perpendiculars]
		\labelname{vectors perpendiculars}\label{def:vectors perpendiculars}
		\[\prodesc{\vec{u}}{\vec{v}}=0.\]
	\end{definition}
%	\begin{theorem}[Teorema de la divisió Euclidiana]
%		\labelname{Teorema de la divisió Euclidiana}\label{thm:divisió euclidiana}
%	\end{theorem}
%	\begin{history}[Who's it gonna be]
%		In a right triangle, the square of the hypotenuse is equal to the sum of the squares of the catheti.
%		\[a^2+b^2=c^2\]
%	\end{history}
	\section{Equacions diferencials ordinàries \rom{1}}
	\subsection{Teoria qualitativa}
	\begin{observation}
		\label{obs:podem entendre les equacions diferencials autònomes com camps vectorials}
		Sigui
		\[\dot{u}(t)=X(u(t))\]
		una equació diferencial autònoma sobre un obert \(\obert{U}\subseteq\mathbb{R}^{n}\) amb \(X\in\mathcal{C}^{1}\).
		Aleshores la funció \(X\) és un camp vectorial.
	\end{observation}
	\begin{definition}[Òrbites]
		\labelname{òrbita d'una equació diferencial}\label{def:òrbita d'una equació diferencial}
		Sigui \(\varphi\) una solució d'una equació diferencial autònoma
		\[\dot{u}(t)=X(u(t))\]
		on \(X\in\mathcal{C}^{1}\).
		Aleshores direm que \(\varphi\) és una òrbita de l'equació diferencial \(\dot{u}(t)=X(u(t))\).
%		
%		Siguin
%		
%		\[\dot{u}(t)=X(u(t))\]
%		una equació diferencial autònoma sobre un obert \(\obert{U}\subseteq\mathbb{R}^{n}\) i \(\varphi\) una solució de l'equació diferencial. Aleshores direm que \(\varphi\) és una òrbita de l'equació diferencial.
	\end{definition}
	\begin{proposition}
		Siguin
		\[\dot{u}(t)=X(u(t))\]
		una equació diferencial autònoma sobre un obert \(\obert{U}\subseteq\mathbb{R}^{n}\) amb \(X\in\mathcal{C}^{1}\) i \(x\in\obert{U}\) un punt.
		Aleshores existeix una solució maximal \(\varphi_{x}\) tal que \(\varphi_{x}(0)=x\).
		\begin{proof}
			Considerem el problema de Cauchy
			\[\begin{cases*}
				\displaystyle \dot{u}(t)=X(u(t)) \\
				\displaystyle u(0)=x.
			\end{cases*}\]
			
			Aleshores per la proposició \myref{prop:existeixen solucions improrrogables} tenim que existeix una solució maximal \(\varphi_{x}\) d'aquest problema de Cauchy, i per la definició de \myref{def:problema de Cauchy} trobem que \(\varphi_{x}(0)=x\) i hem acabat.
		\end{proof}
	\end{proposition}
	\begin{definition}[Flux]
		\labelname{flux}\label{def:flux}
		Siguin
		\[\dot{u}(t)=X(u(t))\]
		una equació diferencial autònoma sobre un obert \(\obert{U}\subseteq\mathbb{R}^{n}\) amb \(X\in\mathcal{C}^{1}\), \(x\in\obert{U}\) un punt i \(\varphi_{x}\) la solució maximal tal que \(\varphi_{x}(0)=x\).
		Aleshores direm que \(\varphi_{x}\) és el flux de \(x\).
	\end{definition}
	
	
\end{document}
