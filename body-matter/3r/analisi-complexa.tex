\documentclass[../../Main.tex]{subfiles}

\begin{document}
\part{Anàlisi complexa i de Fourier}
\chapter{Els nombres complexos}
\section{Estructura dels nombres complexos}
	\subsection{Cos de nombres complexos}
	\begin{notation}[Conjunt de nombres complexos]
		\label{notation:cos de nombres complexos}
		Denotarem
		\[
            \field{C} = \{a+b\iu\mid a,b\in\field{R}\}.
        \]
%		\[\field{C} = \{\alpha+\beta\iu\mid\alpha,\beta\in\field{R}\}.\]
	\end{notation}
	\begin{definition}[Suma de nombres complexos]
		\labelname{suma de nombres complexos}
        \label{def:suma de nombres complexos}
		Siguin \(a+b\iu\) i \(c+d\iu\) dos nombres complexos.
        Aleshores definim la seva suma com
		\[
            (a+b\iu)+(c+d\iu) = (a+c)+(b+d)\iu.
        \]
	\end{definition}
	\begin{observation}
		\label{obs:els nombres complexos estan tancats per la suma}
		Siguin \(a+b\iu\) i \(c+d\iu\) dos nombres complexos.
        Aleshores
		\[
            (a+b\iu+c+d\iu)\in\field{C}
        \]
	\end{observation}
	\begin{proposition}
		\label{prop:els nombres complexos commuten per la suma}
		Siguin \(a+b\iu\) i \(c+d\iu\) dos nombres complexos.
        Aleshores
		\[
            (a+b\iu)+(c+d\iu) = (c+d\iu)+(a+b\iu).
        \]
		\begin{proof}
			Per la definició de \myref{def:suma de nombres complexos} tenim
			\begin{align*}
				a+b\iu+c+d\iu &= (a+c)+(b+d)\iu \\
				 &= (c+a)+(d+b)\iu = c+d\iu+a+b\iu.\qedhere
			\end{align*}
		\end{proof}
	\end{proposition}
	\begin{proposition}
		\label{prop:els nombres complexos són associatius per la suma}
		Siguin \(a+b\iu\), \(c+d\iu\) i \(u+v\iu\) tres nombres complexos.
        Aleshores
		\[
            (a+b\iu)+\big((c+d\iu)+(u+v\iu)\big)
            = \big((a+b\iu)+(c+d\iu)\big)+(u+v\iu).
        \]
		\begin{proof}
			Per la definició de \myref{def:suma de nombres complexos} tenim
			\begin{align*}
                (a+b\iu)+\big((c+d\iu)+(u+v\iu)\big)
                    &= (a+b\iu)+\big((c+u)+(d+v)\iu\big) \\
                    &= \big(a+(c+u)\big)+\big(b+(d+v)\big)\iu \\
                    &= \big((a+c)+u\big)+\big((b+d)+v\big)\iu \\
                    &= \big((a+c)+(b+d)\iu\big)+(u+v\iu) \\
                    &= \big((a+b\iu)+(c+d\iu)\big)+(u+v\iu).\qedhere
			\end{align*}
		\end{proof}
	\end{proposition}
	\begin{proposition}
		\label{prop:element neutre per la suma dels complexos}
		Sigui \(a+b\iu\) un nombre complex.
        Aleshores
		\[
            (a+b\iu)+0 = a+b\iu.
        \]
		\begin{proof}
			Per la definició de \myref{def:suma de nombres complexos} tenim
			\[
                (a+b\iu)+0 = (a+0)+(b+0)\iu = a+b\iu.\qedhere
            \]
		\end{proof}
	\end{proposition}
	\begin{proposition}
		\label{prop:element invers per la suma dels complexos}
		Sigui \(a+b\iu\) un nombre complex.
        Aleshores
		\[
            (a+b\iu)+(-a-b\iu) = 0.
        \]
		\begin{proof}
			Per la definició de \myref{def:suma de nombres complexos} tenim
			\[
                (a+b\iu)+(-a-b\iu) = (a-a)+(b-b)\iu = 0.\qedhere
            \]
		\end{proof}
	\end{proposition}
	\begin{definition}[Producte de nombres complexos]
		\labelname{producte de nombres complexos}
        \label{def:producte de nombres complexos}
		Siguin \(a+b\iu\) i \(c+d\iu\) dos nombres complexos.
        Aleshores definim el seu producte com
		\[
            (a+b\iu)\cdot(c+d\iu) = (ac-bd)+(ad+bc)\iu.
        \]
	\end{definition}
	\begin{observation}
		\label{obs:els nombres complexos estan tancats pel producte}
		Siguin \(a+b\iu\) i \(c+d\iu\) dos nombres complexos.
        Aleshores
		\[
            (a+b\iu)(c+d\iu)\in\field{C}.
        \]
	\end{observation}
	\begin{proposition}
		\label{prop:els nombres complexos commuten pel producte}
		Siguin \(a+b\iu\) i \(c+d\iu\) dos nombres complexos.
        Aleshores
		\[
            (a+b\iu)(c+d\iu) = (c+d\iu)(a+b\iu).
        \]
		\begin{proof}
			Per la definició de \myref{def:producte de nombres complexos} tenim
			\begin{align*}
				(a+b\iu)(c+d\iu) &= (ac-bd)+(ad+bc)\iu \\
				 &= (ca-bd)+(da+cb)\iu = (c+d\iu)(a+b\iu).\qedhere
			\end{align*}
		\end{proof}
	\end{proposition}
	\begin{proposition}
		\label{prop:els nombres complexos són associatius pel producte}
		Siguin \(a+b\iu\), \(c+d\iu\) i \(u+v\iu\) tres nombres complexos.
        Aleshores
		\[
            (a+b\iu)\big((c+d\iu)(u+v\iu))\big)
            = \big((a+b\iu)(c+d\iu)\big)(u+v\iu).
        \]
		\begin{proof}
			Prenem \(\alpha = a+b\iu\), \(\beta = c+d\iu\)
            i \(\gamma = u+v\iu\).
            Per la definició de \myref{def:producte de nombres complexos} tenim
			\begin{align*}
				\alpha\big(\beta\gamma\big)
                    &= (a+b\iu)\big((c+d\iu)(u+v\iu)\big) \\
                    &= (a+b\iu)\big((cu-dv)+(cv+du)\iu\big) \\
                    &= \big(a(cu-dv)-b(cv+du)\big)
                       +\big(a(cv+du)+b(cu-dv)\big)\iu \\
                    &= (acu-adv-bcv-bdu)+(acv+adu+bcu-bdv)\iu \\
                    &= (acu-bdu-adv-bcv)+(acv-bdv+adu+bcu)\iu \\
                    &= \big((ac-bd)u-(ad+bc)v\big)
                       +\big((ac-bd)v+(ad+bc)u\big)\iu \\
                    &= \big((ac-bd)+(ad+bc)\iu\big)(u+v\iu) \\
                    &= \big((a+b\iu)(c+d\iu)\big)(u+v\iu)
                     = (\alpha\beta)\gamma.\qedhere
			\end{align*}
		\end{proof}
	\end{proposition}
	\begin{proposition}
		\label{prop:element neutre pel producte dels complexos}
		Sigui \(a+b\iu\) un nombre complex.
        Aleshores
		\[
            (a+b\iu)\cdot1 = a+b\iu.
        \]
		\begin{proof}
			Per la definició de \myref{def:producte de nombres complexos} tenim
			\[
                (a+b\iu)\cdot1 = (a\cdot1)+(b\cdot1)\iu = a+b\iu.\qedhere
            \]
		\end{proof}
	\end{proposition}
	\begin{proposition}
		\label{prop:element invers pel producte de nombres complexos}
		Sigui \(a+b\iu\) un nombre complex.
        Aleshores
		\[
            (a+b\iu)\Big(\frac{a}{a^{2}+b^{2}} + \frac{-b}{a^{2}+b^{2}}\iu\Big)
            = 1.
        \]
		\begin{proof}
			Per la definició de \myref{def:producte de nombres complexos} tenim
			\begin{align*}
				(a+b\iu)\Big(\frac{a}{a^{2}+b^{2}}+\frac{-b}{a^{2}+b^{2}}\iu\Big)
                    &= \Big(\frac{a^{2}}{a^{2}+b^{2}}-\frac{-b^{2}}{a^{2}+b^{2}}\Big)
                       +\Big(\frac{ab}{a^{2}+b^{2}}+\frac{-ba}{a^{2}+b^{2}}\Big)\iu \\
                    &= \Big(\frac{a^{2}+b^{2}}{a^{2}+b^{2}}\Big)
                       +\Big(\frac{ab-ab}{a^{2}+b^{2}}\Big)\iu = 1.\qedhere
			\end{align*}
		\end{proof}
	\end{proposition}
	\begin{proposition}
		\label{prop:distribuitva del producte respecte la suma de nombres complexos}
		Siguin \(a+b\iu\), \(c+d\iu\) i \(u+v\iu\) tres nombres complexos.
        Aleshores
		\[
            (a+b\iu)\big((c+d\iu)+(u+v\iu)\big) = (a+b\iu)(c+d\iu)+(a+b\iu)(u+v\iu).
        \]
		\begin{proof}
			Per la definició de \myref{def:suma de nombres complexos}
            i \myref{def:producte de nombres complexos} tenim
			\begin{align*}
				(a+b\iu)\big((c+d\iu)+(u+v\iu)\big)
                    &= (a+b\iu)\big((c+u)+(d+v)\iu\big) \\
                    &= \big(a(c+u)-b(d+v)\big)+\big(a(d+v)+b(c+u)\big)\iu \\
                    &= (ac+au-bd-bv)+(ad+av+bc+bu)\iu \\
                    &= (ac-bd+au-bv)+(ad+bc+av+bu)\iu\\
                    &= (ac-bd)+(ad+bc)\iu+(au-bv)+(av+bu)\iu \\
                    &= (a+b\iu)(c+d\iu)+(a+b\iu)(u+v\iu).\qedhere
			\end{align*}
		\end{proof}
	\end{proposition}
	\begin{corollary}
		\label{cor:els complexos formen un cos}
		El conjunt \(\field{C}\) amb la suma \(+\)
        i el producte \(\cdot\) és un cos.
	\end{corollary}
	\subsection{Propietats de nombres complexos}
	\begin{definition}[Conjugat d'un nombre complex]
		\labelname{conjugat d'un nombre complex}
        \label{def:conjugat d'un nombre complex}
		Sigui \(z = a+b\iu\) un nombre complex.
        Aleshores definim
		\[
            \conjugat{z} = a-b\iu
        \]
		com el conjugat de \(z\).
	\end{definition}
	\begin{proposition}
		\label{prop:el conjugat del conjugat d'un nombre complex és ell mateix}
		Sigui \(z\) un nombre complex.
        Aleshores
		\[
            \conjugat{\conjugat{z}} = z.
        \]
		\begin{proof}
			Per la definició de \myref{def:nombre complex} tenim que
            existeixen \(a\), \(b\in\field{R}\) tals que \(z = a+b\iu\).
            Aleshores per la definició
            de \myref{def:conjugat d'un nombre complex} tenim que
			\begin{align*}
				\conjugat{\conjugat{z}} &= \conjugat{\conjugat{a+b\iu}} \\
                                        &= \conjugat{a-b\iu} \\
                                        &= a+b\iu = z.\qedhere
			\end{align*}
		\end{proof}
	\end{proposition}
	\begin{proposition}
		\label{prop:el conjugat de la suma és la suma de conjugats}
		Siguin \(z\) i \(w\) dos nombres complexos.
        Aleshores
		\[
            \conjugat{z+w} = \conjugat{z}+\conjugat{w}.
        \]
		\begin{proof}
			Per la definició de \myref{def:nombre complex} tenim que
            existeixen \(a\), \(b\), \(c\), \(d\in\field{R}\) tals que
			\[
                z = a+b\iu\qquad\text{i}\qquad w = c+d\iu.
            \]
			Aleshores per la definició de \myref{def:suma de nombres complexos}
            i la definició de \myref{def:conjugat d'un nombre complex} trobem
            que
			\begin{align*}
				\conjugat{z+w} &= \conjugat{(a+b\iu)+(c+d\iu)} \\
                               &= \conjugat{(a+c)+(b+d)\iu} \\
                               &= (a+c)-(b+d)\iu \\
                               &= (a-b\iu)+(c-d\iu) \\
                               &= \conjugat{a+b\iu}+\conjugat{c+d\iu}
                                = \conjugat{z}+\conjugat{w}.\qedhere
			\end{align*}
		\end{proof}
	\end{proposition}
	\begin{proposition}
		\label{prop:el conjugat del producte és el producte de conjugats}
		Siguin \(z\) i \(w\) dos nombres complexos.
        Aleshores
		\[
            \conjugat{zw} = \conjugat{z}\,\conjugat{w}.
        \]
		\begin{proof}
            Per la definició de \myref{def:nombre complex} tenim que existeixen
            \(a\), \(b\), \(c\), \(d\in\field{R}\) tals que
			\[
                z = a+b\iu\qquad\text{i}\qquad w = c+d\iu.
            \]
            Aleshores per la definició
            de \myref{def:producte de nombres complexos} i la definició
            de \myref{def:conjugat d'un nombre complex} trobem que
			\begin{align*}
				\conjugat{zw} &= \conjugat{(a+b\iu)(c+d\iu)} \\
                              &= \conjugat{(ac-bd)+(ad+bc)\iu} \\
                              &= (ac-bd)-(ad+bc)\iu \\
                              &= (ac-bd)+(-ad-bc)\iu \\
                              &= (a-b\iu)(c-d\iu) \\
                              &= \conjugat{a+b\iu}\,\conjugat{c+d\iu}
                               = \conjugat{z}\,\conjugat{w}\qedhere.
			\end{align*}
		\end{proof}
	\end{proposition}
	\begin{proposition}
		\label{prop:el producte d'un nombre complex pel seu conjugat és la suma dels quadrats de la seva part real i imaginaria}
		Sigui \(z = a+b\iu\) un nombre complex.
        Aleshores
		\[
            z\conjugat{z} = a^{2}+b^{2}.
        \]
		\begin{proof}
			Per la definició de \myref{def:producte de nombres complexos} i la
            definició de \myref{def:conjugat d'un nombre complex} trobem que
			\begin{align*}
                z\conjugat{z} &= (a+b\iu)\conjugat{a+b\iu} \\
                              &= (a+b\iu)(a-b\iu) = a^{2}+b^{2}.\qedhere
			\end{align*}
		\end{proof}
	\end{proposition}
	\begin{proposition}
		\label{prop:un nombre complex és igual al seu conjugat si i només si és un real}
		Sigui \(z\) un nombre complex.
        Aleshores \(z = \conjugat{z}\) si i només si \(z\in\field{R}\).
		\begin{proof}
			Per la definició de \myref{def:nombre complex} tenim que
            existeixen \(a\), \(b\in\field{R}\) tals que \(z = a+b\iu\).
            Aleshores si tenim \(z = \conjugat{z}\), per la definició
            de \myref{def:conjugat d'un nombre complex} trobem que
			\[
                a+b\iu = a-b\iu,
            \]
			que és equivalent a \(b = -b\), i per tant ha de ser \(b = 0\) i
            trobem que \(z = a\in\field{R}\).
		\end{proof}
	\end{proposition}
	\begin{proposition}
		\label{prop:inversa d'un nombre complex en funció del seu conjugat}
		Sigui \(z \neq 0\) un nombre complex.
        Aleshores
		\[
            z^{-1} = \frac{\conjugat{z}}{z\conjugat{z}}.
        \]
		\begin{proof}
			Per la definició de \myref{def:nombre complex} tenim que
            existeixen \(a\), \(b\in\field{R}\) tals que \(z = a+b\iu\).
            Aleshores tenim que
			\begin{align*}
				z\frac{\conjugat{z}}{z\conjugat{z}}
                    &= \frac{z\conjugat{z}}{z\conjugat{z}} \\
                    &= \frac{a^{2}+b^{2}}{a^{2}+b^{2}} = 1
                    \tag{\ref{prop:el producte d'un nombre complex pel seu conjugat és la suma dels quadrats de la seva part real i imaginaria}}
			\end{align*}
            i per la definició de \myref{def:l'invers d'un element d'un anell}
            hem acabat.
		\end{proof}
	\end{proposition}
	\begin{definition}[Part real i part imaginària d'un nombre complex]
		\labelname{part real i part imaginària d'un nombre complex}
        \label{def:part real i part imaginària d'un nombre complex}
		\labelname{part real d'un nombre complex}
        \label{def:part real d'un nombre complex}
		\labelname{part imaginària d'un nombre complex}
        \label{def:part imaginària d'un nombre complex}
		Sigui \(z = a+b\iu\) un nombre complex.
        Aleshores definim
		\[
            \Re(z) = a
        \]
		com la part real de \(z\) i
		\[
            \Im(z) = b
        \]
		com la part imaginària de \(z\).
	\end{definition}
	\begin{proposition}
		\label{prop:fórmules per la part real i part imaginària d'un nombre complex}
		\label{prop:fórmula per la part real d'un nombre complex}
		\label{prop:fórmula per la part imaginària d'un nombre complex}
		Sigui \(z\) un nombre complex.
        Aleshores
		\[
            \Re(z) = \frac{z+\conjugat{z}}{2}\qquad\text{i}\qquad\Im(z)
                   = \frac{z-\conjugat{z}}{2\iu}.
        \]
		\begin{proof}
			Per la definició de \myref{def:nombre complex} tenim que
            existeixen \(a\), \(b\in\field{R}\) tals que \(z = a+b\iu\).
            Aleshores tenim que
			\[
                \frac{z+\conjugat{z}}{2} = \frac{a+b\iu+a-b\iu}{2}
                                         = \frac{2a}{2} = a = \Re(z),
            \]
			i
			\[
                \frac{z-\conjugat{z}}{2\iu} = \frac{a+b\iu-a+b\iu}{2\iu}
                                            = \frac{2b\iu}{2\iu} = b = \Im(z),
            \]
			i per la definició
            de \myref{def:part real i part imaginària d'un nombre complex} hem
            acabat.
		\end{proof}
	\end{proposition}
	\subsection{Topologia de nombres complexos}
	\begin{definition}[Mòdul d'un nombre complex]
		\labelname{mòdul d'un nombre complex}
        \label{def:mòdul d'un nombre complex}
		Sigui \(z = a+b\iu\) un nombre complex.
        Aleshores definim el seu mòdul com
		\[
            \modul{z} = \sqrt{a^{2}+b^{2}}.
        \]
	\end{definition}
	\begin{observation}
		\label{obs:les parts real i imaginàries d'un complex són menors que el seu mòdul}
		\label{obs:la part real d'un complex és menor que el seu mòdul}
		\label{obs:la part imaginària d'un complex és menor que el seu mòdul}
		\(\Re(z) \leq \modul{z}\), \(\Im(z) \leq \modul{z}\).
	\end{observation}
	\begin{proposition}
		\label{prop:el mòdul d'un nombre complex és l'arrel del nombre pel seu conjugat}
		Sigui \(z\) un nombre complex.
        Aleshores es satisfà
		\[
            \modul{z} = \sqrt{z\conjugat{z}}.
        \]
		\begin{proof}
			Per la definició de \myref{def:nombre complex} tenim
            que \(z = a+b\iu\), i per la
            proposició \myref{prop:el producte d'un nombre complex pel seu conjugat és la suma dels quadrats de la seva part real i imaginaria}
            trobem que
			\begin{align*}
				z\conjugat{z} &= a^{2}+b^{2} \\
                              &= \big(\sqrt{a^{2}+b^{2}}\big)^{2}
                               = \modul{z}^{2},
                              \tag{\ref{def:mòdul d'un nombre complex}}
			\end{align*}
			i per tant 
			\[
                \modul{z} = \sqrt{z\conjugat{z}}.\qedhere
            \]
		\end{proof}
	\end{proposition}
	\begin{proposition}[Desigualtat triangular]
		\labelname{desigualtat triangular}
        \label{prop:desigualta triangular nombres complexos}
		Siguin \(z\) i \(w\) dos nombres complexos.
        Aleshores
		\[
            \modul{z+w} \leq \modul{z}+\modul{w}.
        \]
		\begin{proof}
			Per la definició de \myref{def:mòdul d'un nombre complex} tenim que
			\begin{align*}
				\modul{z+w}^{2} &= (z+w)\conjugat{(z+w)} \\
                                &= (z+w)(\conjugat{z}+\conjugat{w})
                                \tag{\ref{prop:el conjugat de la suma és la suma de conjugats}} \\
                                &= z\conjugat{z}+z\conjugat{w}
                                   +\conjugat{z}w+w\conjugat{w} \\
                                &= \modul{z}^{2}+\modul{w}^{2}
                                   +\conjugat{z}w+z\conjugat{w}
                                \tag{\ref{prop:el mòdul d'un nombre complex és l'arrel del nombre pel seu conjugat}} \\
                                &= \modul{z}^{2}+\modul{w}^{2}
                                   +\conjugat{\conjugat{\conjugat{z}w}}+z\conjugat{w}
                                   \tag{\ref{prop:el conjugat del conjugat d'un nombre complex és ell mateix}} \\
                                &= \modul{z}^{2}+\modul{w}^{2}
                                   +\conjugat{\conjugat{\conjugat{z}}\,\conjugat{w}}+z\conjugat{w}
                                \tag{\ref{prop:el conjugat del producte és el producte de conjugats}} \\
                                &= \modul{z}^{2}+\modul{w}^{2}
                                   +\conjugat{z\conjugat{w}}+z\conjugat{w}
                                \tag{\ref{prop:el conjugat del conjugat d'un nombre complex és ell mateix}} \\
                                &= \modul{z}^{2}+\modul{w}^{2}+2\Re(zw)
                                \tag{\ref{prop:fórmula per la part real d'un nombre complex}} \\
                                &\leq \modul{z}^{2}+\modul{w}^{2}+2\modul{zw}
                                \tag{\ref{obs:la part real d'un complex és menor que el seu mòdul}} \\
                                &\leq \modul{z}^{2}+\modul{w}^{2}
                                      +2\modul{z}\modul{w} \\
                                &= (\modul{z}+\modul{w})^{2}
                                %TODO
			\end{align*}
		\end{proof}
	\end{proposition}
\end{document} 

