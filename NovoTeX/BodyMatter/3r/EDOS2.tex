\documentclass[../../Main.tex]{subfiles}

\begin{document}
\part{Equacions diferencials ordinàries \rom{2}}
\begin{comment}
%	\emph{\hypersetup{urlcolor=black}\href{https://www.urbandictionary.com/define.php?term=wtf}{wtf} is this assignatura}
\section{Òrbites d'una equació diferencial}
	\subsection{Sistemes autònoms a \ensuremath{\mathbb{R}^{n}}}
\section{Sistemes autònoms a \ensuremath{\mathbb{R}^{n}}}
%	\subsection{Interpretació geomètrica} % Justificació dels retrats de fase? Això a EDOS I
	\subsection{Estructura de les òrbites}
	\subsection{Integrals primeres}
	\subsection{Superfícies invariants}
	\subsection{Retrat de fase i conjugació}
\section{Sistemes integrables}
	\subsection{Sistemes potencials}
	\subsection{Sistemes Hamiltonians}
	\subsection{El model de Lotka-Volterra}
\section{Sistemes no integrables}
	\subsection{Teorema del flux tubular}
	\subsection{Anàlisi qualitativa dels punts d'equilibri}
	\subsection{Comportament límit de les òrbites}
	\subsection{Teorema de Poincaré-Bendixson}
	\subsection{Funcions de Liapunov}
	\subsection{Cicles límit}
	\subsection{Criteri de Bendixson-Dulac}
%	\subsection{Models a l'ecologia}
	\subsection{Sistema de van der Pol}
\chapter{Equacions en derivades parcials}
\section{Equacions en derivades parcials de primer ordre}
	\subsection{Introducció a les equacions en derivades parcials}
	\subsection{Equacions lineals i quasi-lineals de primer odre}
\section{Equacions en derivades parcials de segon ordre}
	\subsection{Equacions de la corda finita}
	\subsection{Principi d'Alembert}
	\subsection{Problemes de contorn}
	\subsection{L'equació de la calor}
	\subsection{Problema de la barra infinita}
	\subsection{Separació de variables i sèries de Fourier}
	\subsection{L'equació de Laplace}
\end{comment}
%	\subsection{Teorema del flux tubular}
%	\begin{theorem}[Teorema del flux tubular]
%		\labelname{Teorema del flux tubular}\label{thm:Teorema del flux tubular}
%		\begin{proof}
%			%TODO
%		\end{proof}
%	\end{theorem}
%	\subsection{Teoremes massa difícils pels nostres cervells (i crèdits)}
%	\begin{theorem}[Teorema de Hartman]
%		\labelname{Teorema de Hartman}\label{thm:Teorema de Hartman}
%	\end{theorem}
%	\begin{theorem}[Teorema de Perron-Hadamard]
%		\labelname{Teorema de Perron-Hadamard}\label{thm:Teorema de Perron-Hadamard}
%	\end{theorem}

\chapter{Sistemes autònoms al pla}
%\section{Òrbites}
%	\subsection{Sistemes autònoms a \ensuremath{\mathbb{R}^{n}}}
%	\subsection{Interpretació geomètrica}
%	\subsection{Estructura de les òrbites}
%	\subsection{Superfícies invariants}
%\section{Sistemes integrables}
%	\subsection{Integral primera}
%%	\subsection{Propietats dels sistemes integrables}
%	\subsection{Sistemes potencials}
%	\subsection{Sistemes Hamiltonians}
%	\subsection{Model de Lotka-Voltera}
\section{Sistemes no integrables}
%	\subsection{Teorema del flux tubular}
%	\subsection{Anàlisi qualitativa dels punts d'equilibri}
	\subsection{Comportament límit de les òrbites} % 4.1 Equações Diferenciais Ordinárias
	\begin{definition}[Conjunts \(\omega\)-límit i \(\alpha\)-límit]
		\labelname{conjunt \ensuremath{\omega}-límit}\label{def:conjunt omega-límit}
		\labelname{conjunt \ensuremath{\alpha}-límit}\label{def:conjunt alpha-límit}
		Sigui \(\varphi_{x}\colon\mathbb{R}\longrightarrow\mathbb{R}^{n}\) l'òrbita d'un punt \(x\in\mathbb{R}^{n}\). Aleshores definim
		\[\omega(x)=\{y\in\mathbb{R}^{n}\mid\text{existeix }(t_{n})_{n\in\mathbb{N}}\text{ amb }t_{n}\overset{n\to\infty}{\longrightarrow}+\infty\text{ tal que }\varphi_{x}(t_{n})\overset{n\to\infty}{\longrightarrow}y\}\]
		com el conjunt \(\omega\)-límit de \(x\) i
		\[\alpha(x)=\{y\in\mathbb{R}^{n}\mid\text{existeix }(t_{n})_{n\in\mathbb{N}}\text{ amb }t_{n}\overset{n\to\infty}{\longrightarrow}-\infty\text{ tal que }\varphi_{x}(t_{n})\overset{n\to\infty}{\longrightarrow}y\}\]
		com el conjunt \(\alpha\)-límit de \(x\).
	\end{definition}
	\begin{observation}
		Siguin \(x\in\mathbb{R}^{n}\) un punt i
		\[\dot{x}=f(x)\quad\text{i}\quad\dot{x}=-f(x)\]
		dues equacions diferencials. Aleshores el conjunt \(\omega\)-límit de \(x\) en la primera equació diferencial és el conjunt \(\alpha\)-límit de \(x\) en la segona equació diferencial i, anàlogament, el conjunt \(\alpha\)-límit de \(x\) en la primera equació diferencial és el conjunt \(\omega\)-límit de \(x\) en la segona equació diferencial.
	\end{observation}
	\begin{definition}[Conjunts invariants]
		\labelname{conjunt positivament invariant}\label{def:conjunt positivament invariant}
		\labelname{conjunt negativament invariant}\label{def:conjunt negativament invariant}
		\labelname{conjunt invariant}\label{def:conjunt invariant}
		Sigui \(\varphi_{x}\) l'òrbita d'un punt \(x\in\mathbb{R}^{n}\). Aleshores definim
		\[\gamma^{+}(x)=\{\varphi_{x}(t)\mid t\geq0\},\quad\gamma^{-}(x)=\{\varphi_{x}(t)\mid t\leq0\}\quad\text{i}\quad\gamma(x)=\gamma^{+}(x)\cup\gamma^{-}(x),\]
		i direm que un conjunt \(C\subseteq\mathbb{R}^{n}\) és positivament invariant si per a tot \(y\in C\) tenim que \(\gamma^{+}(y)\subseteq C\), que és negativament invariant si per a tot \(y\in C\) tenim que \(\gamma^{-}(y)\subseteq C\) i que és invariant si per a tot \(y\in C\) tenim que \(\gamma(y)\subseteq C\).
	\end{definition}
	\begin{theorem}
		\label{thm:propietats topològiques dels omega-límits}
		Siguin
		\(\dot{x}=f(x)\)
		una equació diferencial sobre \(\mathbb{R}^{n}\), \(\tancat{K}\) un compacte de \(\mathbb{R}^{n}\) i \(p\in\mathbb{R}^{n}\) un punt tals que \(\gamma^{+}(x)\subset\tancat{K}\). Aleshores
		\begin{enumerate}
			\item\label{enum1:thm:propietats topològiques dels omega-límits} \(\omega(x)\neq\emptyset\).
			\item\label{enum2:thm:propietats topològiques dels omega-límits} \(\omega(x)\) és un tancat.
			\item\label{enum3:thm:propietats topològiques dels omega-límits} \(\omega(x)=\omega(\varphi_{x}(t))\) per a tot \(t\in\mathbb{R}\).
			\item\label{enum4:thm:propietats topològiques dels omega-límits} \(\omega(x)\) és connex.
		\end{enumerate}
		\begin{proof}
			Comencem veient el punt \eqref{enum1:thm:propietats topològiques dels omega-límits}. Per hipòtesi tenim que \(\gamma^{+}(x)\subset\tancat{K}\). Ara bé, tenim que \(\varphi_{x}(0)=x\in\gamma^{+}(x)\), i per tant \(x\in\tancat{K}\) i tenim que \(\tancat{K}\neq\emptyset\). %TODO (pàg. 132 Equações Diferenciais Ordinárias)
		\end{proof}
	\end{theorem}
	\subsection{Teorema de Poincaré-Bendixson}
	\begin{definition}[Secció transversal]
		\labelname{secció transversal}\label{def:secció transversal}
		Siguin \(\dot{x}=f(x)\) una equació diferencial sobre un obert \(\obert{U}\subseteq\mathbb{R}^{2}\) i \(\Sigma\subseteq\obert{U}\) una corba tal que per a tot \(x,y\in\Sigma\) es satisfà \(f(x)=f(y)\). Aleshores direm que \(\Sigma\) és una secció transversal de l'equació diferencial. % \Sigma no és una corba, és la seva imatge
	\end{definition}
%	\begin{lemma}
%		\label{lemma:Teorema de Poincaré-Bendixson 1}
%		Siguin \(\dot{x}=f(x)\) una equació diferencial sobre un obert \(\obert{U}\subseteq\mathbb{R}^{2}\), \(\Sigma\subseteq\obert{U}\) una secció transversal i \(p\in\obert{U}\) i \(q\in\Sigma\cap\omega(p)\) dos punts. Aleshores existeix una successió de punts \((\varphi_{p}(t_{n}))_{n\in\mathbb{N}}\) tals que
%		\[\lim_{n\to\infty}\varphi_{p}(t_{n})=q.\]
%		\begin{proof}
%			%TODO
%		\end{proof}
%	\end{lemma}
	\begin{definition}[Aplicació de retorn]
		\labelname{aplicació de retorn}\label{def:aplicació de retorn}
		Siguin \(\dot{x}=f(x)\) una equació diferencial sobre un obert \(\obert{U}\subseteq\mathbb{R}^{2}\) i \(A\subseteq\obert{U}\) un conjunt tals que per a tot \(p\in A\) punt regular i tota secció transversal \(\Sigma\subseteq A\) existeix un entorn \(\obert{V}\subseteq\obert{U}\) de \(p\) tal que per a tot \(x\in\Sigma\cap\obert{V}\) existeix un \(t\neq0\) tal que \(\varphi_{x}(t)\in\Sigma\cap\obert{V}\). Aleshores direm que \(A\) admet una aplicació de retorn.
	\end{definition}
	\begin{definition}[Gràfic]
		\labelname{gràfic}\label{def:gràfic}
		Siguin \(\dot{x}=f(x)\) una equació diferencial sobre un obert \(\obert{U}\subseteq\mathbb{R}^{2}\) i \(A\subseteq\obert{U}\) un conjunt invariant que conté punts regulars i punts crítics tal que per a tot punt \(x\in A\) tenim \(\alpha(x)=\{p\}\) i \(\omega(x)=\{q\}\) per a certs punts crítics \(p,q\in A\), i tals que \(A\) admet una aplicació de retorn. Aleshores direm que \(A\) és un gràfic.
	\end{definition}
	\begin{theorem}[Teorema de Poincaré-Bendixson]
		\labelname{Teorema de Poincaré-Bendixson}\label{thm:Teorema de Poincaré-Bendixson}
		Siguin \(\dot{x}=f(x)\) una equació diferencial en \(\Omega\subseteq\mathbb{R}^{2}\) amb un nombre finit de punts crítics i \(\tancat{K}\subseteq\Omega\) un compacte tal que \(\gamma^{+}(x)\subseteq\tancat{K}\). Aleshores
		\begin{enumerate}
			\item\label{enum1:thm:Teorema de Poincaré-Bendixson} si \(\omega(x)\) només conté punts crítics tenim que \(\omega(x)=\{p\}\).
			\item\label{enum2:thm:Teorema de Poincaré-Bendixson} si \(\omega(x)\) conté punts crítics i punts regulars tenim que \(\omega(x)\) és un gràfic.
			\item\label{enum3:thm:Teorema de Poincaré-Bendixson} si \(\omega(x)\) no conté punts crítics tenim que \(\omega(x)\) és una òrbita periòdica.
		\end{enumerate}
		\begin{proof}
%			Aquest enunciat té sentit pel Teorema \myref{thm:propietats topològiques dels omega-límits}. %TODO
		\end{proof}
	\end{theorem}
%	\begin{definition}[Corona circular]
%		\labelname{corona circular}\label{def:corona circular}
%	\end{definition}
	\begin{theorem}
		\label{thm:dins d'una òrbita periòdica hi ha un punt crític}
		Sigui \(\varphi\colon\mathbb{R}\longrightarrow\mathbb{R}^{2}\) una òrbita periòdica d'una equació diferencial \(\dot{x}=f(x)\). Aleshores existeix un \(p\in\interior(\varphi(\mathbb{R}))\) tal que \(p\) és un punt crític.
		\begin{proof}
			%TODO
		\end{proof}
	\end{theorem}
	\subsection{Criteri de Bendixson-Dulac}
	\begin{theorem}[Criteri de Bendixson-Dulac]
		\labelname{criteri de Bendixson-Dulac}\label{thm:Criteri de Bendixson-Dulac}
		\begin{proof}
			%TODO
		\end{proof}
	\end{theorem}
	% 4.2 Equações Diferenciais Ordinárias
%	\subsection{Funcions de Liapunov}
	% 5 Lições de Equações Diferenciais Ordinárias
%	\subsection{Cicles límit}
	
\chapter{Equacions en derivades parcials}
\section{Equacions en derivades parcials de primer ordre}
	\subsection{Equació general de primer ordre}
	\begin{definition}[Equació en derivades parcials de primer ordre]
		\labelname{equació en derivades parcials de primer ordre}\label{def:equació en derivades parcials de primer ordre}
		Siguin \(\Omega\subseteq\mathbb{R}^{2n+1}\) un obert i \(F\colon\Omega\longrightarrow\mathbb{R}\) una funció. Aleshores, denotant
		\[u_{i}(x)=\frac{\partial u(x)}{\partial x_{i}},\]
		direm que l'expressió
		\[F(x_{1},\dots,x_{n},u(x),u_{1}(x),\dots,u_{n}(x))=0\]
		és una equació en derivades parcials de primer ordre sobre \(\Omega\).
	\end{definition}
	\begin{definition}[Solució d'una equació en derivades parcials]
		\labelname{solució d'una equació en derivades parcials}\label{def:solució d'una equació en derivades parcials}
		Siguin
		\[F(x_{1},\dots,x_{n},u(x),u_{1}(x),\dots,u_{n}(x))=0\]
		una equació en derivades parcials de primer ordre sobre un obert \(\Omega\) i \(\Phi(x)\colon\Omega\longrightarrow\mathbb{R}\) una funció de classe \(\mathcal{C}^{2}(\Omega)\) tal que
		\[F(x_{1},\dots,x_{n},\Phi(x),\Phi_{1}(x),\dots,\Phi_{n}(x))=0\]
		per a tot \(x\in\Omega\). Aleshores direm que \(\Phi\) és una solució de l'equació en derivades parcials.
	\end{definition}
	\subsection{Equacions quasi-lineals de primer ordre}
	\begin{definition}[Equació quasi-lineal]
		\labelname{equació diferencial quasi-lineal}\label{def:equació diferencial quasi lineal}
		Siguin \(\Omega\subseteq\mathbb{R}^{n}\times\mathbb{R}\) un obert, \(\{P_{i}\}_{i=1}^{N}\) una família de funcions de \(\Omega\) a \(\mathbb{R}\) i \(R\colon\Omega\longrightarrow\mathbb{R}\) una funció. Aleshores direm que l'expressió
		\[\sum_{i=1}^{N}\left(P_{i}(x,u(x))\frac{\partial u(x)}{\partial x_{i}}\right)=R(x,u(x))\]
		és una equació diferencial quasi-lineal de primer ordre.
	\end{definition}
	\begin{example}
		Exemples d'equacions quasi-lineals són, amb \(K>0\), l'equació de la calor
		\[\frac{\partial^{2} u}{\partial x^{2}}=K\frac{\partial u}{\partial t}\]
		o l'equació de la corda vibrant
		\[\frac{\partial^{2} u}{\partial x^{2}}=K\frac{\partial^{2} u}{\partial t^{2}},\]
		mentre que exemples d'equacions en derivades parcials que no siguin quasi-lineals són l'equació no lineal de la calor
		\[\frac{\partial^{2} u}{\partial x^{2}}+f(u)=K\frac{\partial u}{\partial t}\]
		o la llei de la conservació escalar
		\[\frac{\partial u(t)}{\partial t}+\divergencia(f(u(t)))=0.\]
	\end{example}
	
\end{document} 
