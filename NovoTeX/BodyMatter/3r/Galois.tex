\documentclass[../../Main.tex]{subfiles}

\begin{document}
\part{Teoria de Galois}
%\chapter{Polinomis simètrics}
\chapter{Teoria de Galois}
\section{Extensions de cossos}
\subsection{Elements algebraics i elements transcendents}
	\begin{definition}[Extensió de cossos]
		\labelname{extensió de cossos}\label{def:extensió de cossos}
		Siguin~\(\KK\) i~\(\FF\) dos cossos tals que~\(\KK\subseteq\FF\). Aleshores direm que~\(\FF\) és una extensió de~\(\KK\) i ho denotarem com~\(\FF\extensio\KK\). També direm que~\(\FF\extensio\KK\) és una extensió.
	\end{definition}
	\begin{proposition} %FER Buscar-li un lloc millor. Potser a estructures.
		\label{prop:un cos quocient un irreductible és un cos}
		Siguin~\(\KK\) un cos i~\(p(x)\in\KK[x]\) un polinomi irreductible. Aleshores~\(\KK[x]/(p(x))\) és un cos.
		\begin{proof} % Veure que un cos és un DIP. A estructures lol.
			Per hipòtesi tenim que~\(p(x)\) és irreductible, i per la proposició~\myref{prop:irreductible sii ideal maximal} trobem que l'ideal~\((p(x))\) és maximal. Aleshores per la proposició~\myref{prop:condició equivalent a ideal maximal per R/M cos} tenim que~\(\FF\) és un cos.
		\end{proof}
	\end{proposition}
	\begin{example}
		\label{ex:el cos de polinomis és una extensió}
		Siguin~\(\KK\) un cos i~\(p(x)\in\KK[x]\) un polinomi irreductible. Denotem~\(\FF=\KK[x]/(p(x))\). Aleshores~\(\FF\extensio\KK\) és una extensió.
		\begin{solution}
			Per la proposició~\myref{prop:un cos quocient un irreductible és un cos} tenim que~\(\FF\) és un cos. Veiem que~\(\KK\subseteq\FF\) per l'observació~\myref{obs:un anell està contingut en el seu anell de polinomis}, i per la definició d'\myref{def:extensió de cossos} hem acabat.
		\end{solution}
	\end{example}
	\begin{example}[Morfisme avaluació]
		\labelname{\empty}
		\label{ex:morfisme d'avaluació}
		\label{ex:morfisme avaluació}
		\label{ex:el morfisme avaluació és un morfisme d'anells}
		Siguin~\(\FF\extensio\KK\) una extensió i~\(\alpha\in\FF\) un element. Volem veure que l'aplicació
%		\begin{align*}
%			\ev_{\alpha}\colon\KK[x]&\longrightarrow\FF \\
%			\KK[x]\setminus\KK\ni x&\longmapsto\alpha \\
%			\KK\ni\lambda&\longmapsto\lambda
%		\end{align*}
		\begin{align*}
			\ev_{\alpha}\colon\KK[x]&\longrightarrow\FF \\
			p(x)&\longmapsto p(\alpha)
		\end{align*}
		és un morfisme d'anells.
		\begin{solution}
			%TODO
		\end{solution}
	\end{example}
	\begin{lemma}
		\label{lema:element algebraic}
		\label{lema:element transcendent}
		Siguin~\(\FF\extensio\KK\) una extensió i~\(\alpha\in\FF\) un element. Aleshores
		\[\ker(\ev_{\alpha})=\begin{cases}
			(0) \\
			(p(x)), & p(x)\text{ és un polinomi mònic irreductible a }\KK[x]
		\end{cases}\]
		\begin{proof} % REVISAR. Pot ser molt més fàcil?
			Suposem que~\(\ker(\ev_{\alpha})\neq(0)\). Per la proposició~\myref{prop:el nucli d'un morfisme entre anells és ideal, la imatge d'un morfisme entre anells és subanell} i l'exemple~\myref{ex:el morfisme avaluació és un morfisme d'anells} tenim que~\(\ker(\ev_{\alpha})\) és un ideal, i com que, per hipòtesi,~\(\KK\) és un cos, tenim que~\(\KK[x]\) és un domini d'ideals principals. %REF
			Per tant, per la definició de~\myref{def:domini d'ideals principals} tenim que~\(\ker(\ev_{\alpha})\) és un ideal principal, i per la definició d'\myref{def:ideal principal} trobem que existeix un~\(a\in\KK[x]\) tal que~\(\ker(\ev_{\alpha})=(a)\).
			
			Tenim que~\(a\) ha de ser un polinomi mònic. Suposem que~\(a\in\KK\). Trobem per la definició~\myref{ex:el morfisme avaluació és un morfisme d'anells} que~\(\ev_{\alpha}(a)=a\), i com que hem suposat que~\(\ker(\ev_{\alpha})\neq(0)\), trobem que~\(a\neq0\), i per tant~\(\ev_{\alpha}(a)=a\neq0\), trobant que~\(a\notin\ker(\ev_{\alpha})\), arribant a contradicció. Per tant~\(a\in\KK[x]\setminus\KK\). Suposem que~\(a=\lambda p(x)\), amb~\(\lambda\in\KK\),~\(\lambda\neq0\) i~\(p\in\KK[x]\setminus\KK\). Per l'exemple~\myref{ex:el morfisme avaluació és un morfisme d'anells} tenim que~\(\ev_{\alpha}\) és un morfisme d'anells, i per la definició de~\myref{def:morfisme entre anells} trobem que
			\begin{align*}
				\ev_{\alpha}(\lambda p(x))&=\ev_{\alpha}(\lambda)\ev_{\alpha}(p(x))\\
				&=\lambda\ev_{\alpha}(p(x)). \tag{\ref{ex:el morfisme avaluació és un morfisme d'anells}}
			\end{align*}
			Ara bé, com que hem suposat que~\(\ker(\ev_{\alpha})=(\lambda p(x))\) tenim~\(\ev_{\alpha}(\lambda p(x))=0\) i, com que hem suposat que~\(\lambda\neq0\), ha de ser~\(\ev_{\alpha}(p(x))=0\), i tenim que
			\[\ker(\ev_{\alpha})=(p(x)).\]
			
			Veiem ara que~\(p(x)\) és irreductible. Pel~\myref{thm:Primer Teorema de l'isomorfisme entre anells} trobem que
			\[\KK[x]/(p(x))=\KK[x]/\ker(\ev_{\alpha})\cong\Ima(\ev_{\alpha}).\]
			
			Com que~\(\FF\) és un cos tenim que~\(\Ima(\ev_{\alpha})\) és un domini. %REF
			Com que~\(\Ima(\ev_{\alpha})\cong\KK[x]/(p(x))\) és un domini, tenim per la proposició~\myref{prop:R/I domini d'integritat sii I ideal primer} que~\((p(x))\) és un ideal primer, i per l'observació~\myref{obs:ideals primer iff primer} trobem que~\(p(x)\) és primer, i per la proposició~\myref{prop:en un DI un primer és un irreductible} tenim que~\(p(x)\) és irreductible.
		\end{proof}
	\end{lemma}
	\begin{definition}[Element algebraic i element transcendent]
		\labelname{element algebraic}\label{def:element algebraic}
		\labelname{element transcendent}\label{def:element transcendent}
%		Siguin~\(\FF\extensio\KK\) una extensió i~\(\alpha\in\FF\) un element tal que existeix un polinomi no nul~\(p(x)\in\KK[x]\) satisfent~\(p(\alpha)=0\). Aleshores direm que~\(\alpha\) és {algebraic} sobre~\(\KK\).
		Siguin~\(\FF\extensio\KK\) una extensió i~\(\alpha\in\FF\) un un element tal que no existeix cap polinomi mònic no nul~\(p(x)\in\KK[x]\) tal que~\(p(\alpha)=0\). Aleshores direm que~\(\alpha\) és {algebraic} sobre~\(\KK\). Denotarem~\(\Irr(\alpha,\KK)=p(x)\).
		
		Si~\(\alpha\) no és algebraic sobre~\(\KK\) direm que és {transcendent} sobre~\(\KK\).
	\end{definition}
	\begin{observation}
		\label{obs:polinomi mínim de descomposició}
		\((\Irr(\alpha,\KK))=\ker(\ev_{\alpha})\)
	\end{observation}
	\begin{example}
		\label{ex:elements algebraics}
		Volem veure que els següents són algebraics.
		\begin{enumerate}
			\item~\(\sqrt{2}\) i~\(\sqrt{3}\) en l'extensió~\(\RR\extensio\QQ\).
			\item~\(\sqrt{2}+\sqrt{3}\) en l'extensió~\(\RR\extensio\QQ\)
		\end{enumerate}
		\begin{solution}
			Tenim que %TODO
			\begin{enumerate}
			\item~\(x^{2}-2\),~\(x^{2}-3\).
			\item~\(x^{4}-10x^{2}+1\).\qedhere
			\end{enumerate}
		\end{solution}
	\end{example}
	\begin{observation}
		\label{obs:condició equivalent a element algebraic}
		\label{obs:condició equivalent a element transcendent}
		Siguin~\(\FF\extensio\KK\) una extensió i~\(\alpha\in\FF\) un element. Aleshores~\(\alpha\) és algebraic si i només si existeix un polinomi no nul~\(p(x)\in\KK[x]\) tal que~\(\ker(\ev_{\alpha})=(p(x))\).
		\begin{proof}
			Si~\(\alpha\) és algebraic aleshores, per la definició d'\myref{def:element algebraic}, existeix un polinomi no nul~\(p(x)\in\KK[x]\) tal que~\(p(\alpha)=0\), i per la definició de~\myref{def:nucli d'un morfisme entre grups} tenim que~\(\ker(\ev_{\alpha})\neq(0)\). Aleshores pel lema~\myref{lema:element algebraic} hem acabat.
		\end{proof}
	\end{observation}
\subsection{Grau d'una extensió de cossos}
	\begin{definition}[Grau d'una extensió de cossos]
		\labelname{grau d'una extensió}\label{def:grau d'una extensió}
		\labelname{extensió finita}\label{def:extensió finita}
		Sigui~\(\FF\extensio\KK\) una extensió. Aleshores direm que la dimensió del~\(\KK\)-espai vectorial~\(\FF\) és el {grau de l'extensió} i el denotarem com~\(\grauExtensio{\FF}{\KK}\). Si~\(\grauExtensio{\FF}{\KK}<\infty\) direm que~\(\FF\extensio\KK\) és una extensió finita.
		
		Aquesta definició té sentit per la proposició~\myref{prop:un subcòs és un espai vectorial}.
	\end{definition}
	\begin{example}
		\label{ex:el grau de l'extensió d'un cos i el quocient entre el seu anell de polinomis i un polinomi irreductible}
		Siguin~\(\KK\) un cos i~\(p(x)\in\KK[x]\) un polinomi irreductible. Denotem~\(\FF=\KK[x]/p(x)\). Aleshores
		\[\grauExtensio{\FF}{\KK}=\grau(p(x)).\]
		\begin{solution}
			Aquest enunciat té sentit per l'exemple~\myref{ex:el cos de polinomis és una extensió}. %FER
		\end{solution}
	\end{example}
	\begin{theorem}[Fórmula de les Torres]
		\labelname{fórmula de les torres}\label{thm:fórmula de les torres}
		Siguin~\(\EE\extensio\FF\) i~\(\FF\extensio\KK\) dues extensions de cossos. Aleshores
		\[\grauExtensio{\EE}{\KK}=\grauExtensio{\EE}{\FF}\grauExtensio{\FF}{\KK}.\]
		\begin{proof}
			Si~\(\grauExtensio{\EE}{\FF}=\infty\) ó~\(\grauExtensio{\FF}{\KK}=\infty\) tenim que~\(\grauExtensio{\EE}{\KK}=\infty\).
			
			Suposem doncs que~\(\grauExtensio{\EE}{\FF}=n\) i~\(\grauExtensio{\FF}{\KK}=m\). %TODO %FER
		\end{proof}
	\end{theorem}
	\begin{definition}[Torre]
		\labelname{torre}\label{def:torre de cossos}
		Siguin~\(\KK_{1}\subseteq\KK_{2}\subseteq\dots\subseteq\KK_{1}\) cossos. Aleshores direm que
		\[\KK_{1}\subseteq\KK_{2}\subseteq\dots\subseteq\KK_{1}\]
		és una torre.
	\end{definition}
	\begin{corollary}
		\label{cor:fórmula de les torres}
		Sigui~\(\KK_{1}\subseteq\KK_{2}\subseteq\dots\subseteq\KK_{1}\) una torre. Aleshores
		\[\grauExtensio{\KK_{n}}{\KK_{1}}=\grauExtensio{\KK_{n}}{\KK_{n-1}}\cdots\grauExtensio{\KK_{2}}{\KK_{1}}\]
		\begin{proof}
			Conseqüència de la~\myref{thm:fórmula de les torres}.
		\end{proof}
	\end{corollary}
\subsection{Subanells engendrats per un element}
	\begin{definition}[Mínim subanell engendrat]
		\labelname{mínim subanell engendrat per un element d'una extensió}\label{def:mínim subanell engendrat per un element d'una extensió}
		Siguin~\(\FF\extensio\KK\) una extensió i~\(\alpha\in\FF\) un element. Aleshores direm que
		\[\KK[\alpha]=\{p(\alpha)\in\FF\mid p(x)\in\FF[x]\}\]
		és el mínim subanell engendrat per~\(\alpha\) en~\(\FF\extensio\KK\).
	\end{definition}
	\begin{observation}
	\label{obs:l'anell engendrat per alpha està entre l'extensió}
		\(\KK\subseteq\KK[\alpha]\subseteq\FF\).
	\end{observation}
	\begin{proposition}
		\label{prop:el mínim subanell engendrat per un element és l'anell de polinomis quocient l'irreductible de l'element}
		Siguin~\(\FF\extensio\KK\) una extensió i~\(\alpha\in\FF\) un element. Aleshores
		\[\KK[\alpha]\cong\KK[x]/(\Irr(\alpha,\KK)).\]
		\begin{proof}
			Tenim que
			\begin{align*}
				\KK[x]/(\Irr(\alpha,\KK))&=\KK[x]/\ker(\ev_{\alpha}) \tag{\ref{obs:polinomi mínim de descomposició}}\\
				&\cong\Ima(\ev_{\alpha}) \tag{\ref{thm:Primer Teorema de l'isomorfisme entre anells}}\\
				&=\{p(\alpha)\in\FF\mid p(x)\in\KK[x]\} \tag{\ref{def:imatge d'un morfisme entre anells} i \ref{ex:morfisme avaluació}}\\
				&= \KK[\alpha], \tag{\ref{def:mínim subanell engendrat per un element d'una extensió}}
			\end{align*}
			i per tant
			\[\KK[\alpha]\cong\KK[x]/(\Irr(\alpha,\KK)).\qedhere\]
		\end{proof}
	\end{proposition}
	\begin{corollary}
		\label{cor:grau d'un mínim subanell engendrat per un element d'una extensió}
		\(\grauExtensio{\KK[\alpha]}{\KK}=\grau(\Irr(\alpha,\KK))\).
		\begin{proof}
			Per l'exemple~\myref{ex:el grau de l'extensió d'un cos i el quocient entre el seu anell de polinomis i un polinomi irreductible}.
		\end{proof}
	\end{corollary}
	\begin{corollary}
		\label{cor:el mínim subanell engendrat per un transcendent és isomorfa a l'anell de polinomis}
		Siguin~\(\FF\extensio\KK\) una extensió i~\(\gamma\in\FF\) un element transcendent sobre~\(\KK\). Aleshores
		\[\KK[\gamma]\cong\KK[x].\]
	\end{corollary}
	\begin{definition}[Mínim subcòs engendrat]
		\labelname{mínim subcòs engendrat per un element d'una extensió}\label{def:mínim subcòs engendrat per un element d'una extensió}
		Siguin~\(\FF\extensio\KK\) una extensió i~\(\alpha\in\FF\) un element. Aleshores direm que
		\[\KK(\alpha)=\cosdefraccions(\KK[\alpha])\]
		és el mínim subcòs engendrat per~\(\alpha\).
		
		Aquesta definició té sentit pel Teorema~\myref{thm:cos de fraccions}.
	\end{definition}
	\begin{observation}
		\label{obs:inclusió del mínim subanell i subcòs engendrats per un element}
		\(\KK\subseteq\KK[\alpha]\subseteq\KK(\alpha)\subseteq\FF\).
	\end{observation}
	\begin{lemma}
		\label{lema:un element és algebraic si i només si el mínim subanell engendrat per l'element és un cos}
		Siguin~\(\FF\extensio\KK\) una extensió i~\(\alpha\in\FF\) un element. Aleshores~\(\alpha\) és algebraic sobre~\(\KK\) si i només si
		\[\KK(\alpha)=\KK[\alpha].\]
		\begin{proof} %FER %Revisar
			Veiem que la condició és suficient~\((\implica)\). Suposem que~\(\alpha\in\FF\) és algebraic sobre~\(\KK\). Per la definició d'\myref{def:element algebraic} trobem que~\(\Irr(\alpha,\KK)\) és un polinomi irreductible no nul, i per la proposició~\myref{prop:un cos quocient un irreductible és un cos} tenim que~\(\KK[\alpha]\) és un cos.
			
			Ara bé, pel Teorema~\myref{thm:cos de fraccions} tenim que~\(\KK(\alpha)\subseteq\KK[\alpha]\), però per l'observació~\myref{def:mínim subcòs engendrat per un element d'una extensió} tenim que~\(\KK[\alpha]\subseteq\KK(\alpha)\), i pel \myref{thm:doble inclusió} tenim que~\(\KK(\alpha)=\KK[\alpha]\).
			
			Acabem veient que la condició és necessària~\((\implicatper)\). Suposem doncs que
			\[\KK(\alpha)=\KK[\alpha].\]
			Per tant~\(\KK[\alpha]\) és un cos, i com que, per l'observació~\myref{obs:l'anell engendrat per alpha està entre l'extensió}, tenim que~\(\alpha\in\KK[\alpha]\), i per la definició de~\myref{def:cos per anells} tenim que~\(-1/\alpha\in\KK[\alpha]\). Aleshores, per la definició d'\myref{def:mínim subanell engendrat per un element d'una extensió} trobem que existeix un polinomi~\(a_{0}+a_{1}x+\dots+a_{k}x^{k}\in\KK[x]\), amb~\(a_{k}\neq0\), tal que
			\[\frac{-1}{\alpha}=a_{0}+a_{1}\alpha+a_{2}\alpha^{2}+\dots+a_{k}\alpha^{k},\]
			i per tant
			\[0=1+a_{0}\alpha+a_{1}\alpha^{2}+\dots+a_{r}\alpha^{r+1},\]
			i trobem que~\(\alpha\) és una arrel del polinomi~\(p(x)=1+a_{0}x+a_{1}x^{2}+\dots+a_{r}x^{r+1}\). Aleshores per la definició de~\myref{ex:morfisme d'avaluació} tenim que
			\[p(x)\in\ker(\ev_{\alpha})\]
			i per tant~\(\ker(\ev_{\alpha})\neq(0)\) i pel lema~\myref{lema:element algebraic} trobem que~\(\alpha\) és algebraic.
		\end{proof}
	\end{lemma}
\subsection{Extensions algebraiques}
	\begin{definition}[Extensió algebraica]
		\labelname{extensió algebraica}\label{def:extensió algebraica}
		Sigui~\(\FF\extensio\KK\) una extensió tal que tot~\(\alpha\in\FF\) és algebraic sobre~\(\KK\). Aleshores direm que~\(\FF\extensio\KK\) és una extensió algebraica.
	\end{definition}
	\begin{proposition}
		\label{prop:les extensions finites són algebraiques}
		Sigui~\(\FF\extensio\KK\) una extensió finita. Aleshores~\(\FF\extensio\KK\) és una extensió algebraica.
		\begin{proof}
			Prenem un element~\(\alpha\in\FF\) i suposem que~\(\grauExtensio{\FF}{\KK}=n\). Per la definició~d'\myref{def:extensió finita} tenim que~\(\dim_{\KK}(\FF)=n\), i per tant els~\(n+1\) elements~\(1,\alpha,\dots,\alpha^{n}\) són~\(\KK\)-linealment independents, %REF
			i per la definició de~\myref{def:vectors linealment dependents} trobem que existeixen~\(\lambda_{0},\dots,\lambda_{n}\in\KK\) no nuls tals que
			\[\lambda_{0}+\lambda_{1}\alpha+\dots+\lambda_{n}\alpha^{n}=0,\]
			i per tant trobem que~\(\alpha\) és arrel del polinomi~\(p(x)=\lambda_{0}+\lambda x+\dots+\lambda_{n}x^{n}\in\KK[x]\). Aleshores per la definició~d'\myref{def:element algebraic} trobem que~\(\alpha\in\FF\) és algebraic sobre~\(\KK\) i per la definició~d'\myref{def:extensió algebraica} trobem que l'extensió~\(\FF\extensio\KK\) és algebraica.
			% Fer per 1,\alpha,\dots,\alpha^{\grau} són \KK-linealment dependents i per tant existeixen \lambda_{i}\in\KK tals que se'n va a la merda.
		\end{proof}
	\end{proposition}
	\begin{lemma}
		\label{lema:condicions equivalents a polinomi irreductible}
		Siguin~\(\FF\extensio\KK\) una extensió,~\(\alpha\in\FF\) un element algebraic sobre~\(\KK\) i~\(p(x)\in\KK[x]\) un polinomi mònic tal que~\(p(\alpha)=0\). Aleshores són equivalents
		\begin{enumerate}
		\item\label{lema:condicions equivalents a polinomi irreductible:eq1} \(p(x)=\Irr(\alpha,\KK)\).
		\item\label{lema:condicions equivalents a polinomi irreductible:eq2} \(p(x)\) és irreductible a~\(\KK[x]\).
		\item\label{lema:condicions equivalents a polinomi irreductible:eq3} Si~\(q(x)\in\KK[x]\) tal que~\(q(\alpha)=0\), aleshores~\(p(x)\divides q(x)\).
		\item\label{lema:condicions equivalents a polinomi irreductible:eq4} Si~\(q(x)\in\KK[x]\) tal que~\(q(\alpha)=0\), aleshores~\(\grau(p(x))\leq\grau(q(x))\).
		\end{enumerate}
		\begin{proof}
			La implicació~\(\eqref{lema:condicions equivalents a polinomi irreductible:eq1}\implica\eqref{lema:condicions equivalents a polinomi irreductible:eq2}\) és conseqüència de la definició~d'\myref{def:element algebraic}.
			
			Veiem ara la implicació~\(\eqref{lema:condicions equivalents a polinomi irreductible:eq2}\implica\eqref{lema:condicions equivalents a polinomi irreductible:eq3}\). Suposem que~\(p(\alpha)=0\) i prenem~\(q(x)\in\KK[x]\) tal que~\(q(\alpha)=0\). Calculem~\(\mcd(p(x),q(x))\). Calculem~\(\mcd(p(x),q(x))\). Si~\(\mcd(p(x),q(x))=p(x)\) aleshores~\(p(x)\divides q(x)\) i tenim el que volíem.
			
			Si~\(\mcd(p(x),q(x))\neq p(x)\), com que per la proposició~\myref{prop:en DIP irreductible implica primer} tenim que~\(p(x)\) és primer tenim que~\(\mcd(p(x),q(x))=1\), i per la~\myref{thm:identitat de Bézout} trobem que existeixen~\(r(x)\) i~\(s(x)\) polinomis en~\(\KK[x]\) tals que
			\[p(x)r(x)+q(x)s(x)=1.\]
			Ara bé, tindríem que
			\[
				1=p(\alpha)r(\alpha)+q(\alpha)s(\alpha)=0r(\alpha)+0s(\alpha)=0,
			\]
			i per tant arribem a contradicció i hem acabat.
			
			Veiem la implicació~\(\eqref{lema:condicions equivalents a polinomi irreductible:eq3}\implica\eqref{lema:condicions equivalents a polinomi irreductible:eq4}\). Suposem doncs que~\(q(x)\in\KK[x]\) és un polinomi tal que~\(q(\alpha)=0\) i tal que~\(p(x)\divides q(x)\). Tenim que existeix un polinomi~\(d(x)\in\KK[x]\) tal que
			\[q(x)=p(x)d(x),\]
			i per tant trobem que~\(\grau(p(x))\leq\grau(q(x))\).
			
			Veiem que~\(\eqref{lema:condicions equivalents a polinomi irreductible:eq4}\implica\eqref{lema:condicions equivalents a polinomi irreductible:eq2}\). Suposem doncs que existeix un polinomi~\(q(x)\in\KK[x]\) satisfent~\(q(\alpha)=0\) i~\(\grau(p(x))\leq\grau(q(x))\). Aleshores tenim que existeixen dos polinomis~\(p_{1}(x)\),~\(p_{2}(x)\in\KK[x]\) tals que~\(p(x)=p_{1}(x)p_{2}(x)\), i com que~\(p(\alpha)=0\) trobem que~\(p_{1}(\alpha)=0\) ó~\(p_{2}(\alpha)=0\). Suposem que~\(p_{1}(\alpha)=0\). Per tant tenim que
			\[\grau(p_{1}(x))\leq\grau(p(x)),\]
			i per hipòtesi
			\[\grau(p(x))\leq\grau(p_{1}(x)),\]
			i per tant~\(p_{2}(x)\in\KK\).
			
			Veiem que~\(\eqref{lema:condicions equivalents a polinomi irreductible:eq3}\implica\eqref{lema:condicions equivalents a polinomi irreductible:eq1}\). Suposem doncs que si existeix un polinomi~\(q(x)\in\KK[x]\) tal que~\(q(\alpha)=0\), aleshores~\(p(x)\divides q(x)\). Considerem~\(\ker(\ev_{\alpha})\). Com que, per hipòtesi, tenim que~\(p(\alpha)=0\), trobem que~\(\ker(\ev_{\alpha})\subseteq(p(x))\). Per altra banda, si~\(\ker(\ev_{\alpha})=(q(x))\) aleshores tenim que~\(q(\alpha)=0\), i per hipòtesi~\(p(x)\divides q(x)\), i per tant~\((p(x))\subseteq(q(x))\) i tenim que~\((p(x))=\ker(\ev_{\alpha})\), i per la definició~d'\myref{def:element algebraic} trobem que~\(p(x)=\Irr(\alpha,\KK)\).
		\end{proof}
	\end{lemma}
	\begin{example}
		Volem calcular~\(\Irr(\sqrt{5}+\sqrt{3},\QQ)\).
		\begin{solution}
			%TODO
			\(\Irr(\sqrt{5}+\sqrt{3},\QQ)=x^{4}+4x^{2}+64\).
		\end{solution}
	\end{example}
\subsection{El grup de Galois}
	\begin{definition}[El grup de Galois]
		\labelname{el grup de Galois}\label{def:el grup de Galois}
		\label{def:grup de Galois}
		Sigui~\(\FF\extensio\KK\) una extensió de cossos. Aleshores definim
		\[\Gal(\FF\extensio\KK)=\{f\colon\FF\longrightarrow\FF\mid f\text{ és un isomorfisme d'anells i }\left.f\right|_{\KK}=\Id_{\KK}\}\]
		com el grup de Galois de l'extensió~\(\FF\extensio\KK\).
	\end{definition}
	\begin{proposition}
		\label{prop:el grup de Galois és un grup}
		Sigui~\(\FF\extensio\KK\) una extensió. Aleshores~\(\Gal(\FF\extensio\KK)\) és un grup.
		\begin{proof} % Revisar. %REFS
			Prenem un element~\(f\in\Gal(\FF\extensio\KK)\). Per la definició~d'\myref{def:el grup de Galois} tenim que~\(f\) és un isomorfisme d'anells, i per la definició~d'\myref{def:isomorfisme entre anells} trobem que~\(f\) és invertible, i per tant existeix un isomorfisme~\(g\colon\FF\longrightarrow\FF\) tal que~\(f\circ g=g\circ f=\Id_{\FF}\). Hem de veure que~\(\left.g\right|_{\KK}=\Id_{\KK}\). Tenim que
			\begin{align*}
				\Id_{\KK}&=\left.\Id_{\FF}\right|_{\KK} \\
				&=\left.g\circ f\right|_{\KK} \\
				&=g\circ\Id_{\KK}=\left.g\right|_{\KK}
			\end{align*}
			i per tant~\(g\in\Gal(\FF\extensio\KK)\).
			
			Prenem dos elements~\(f\),~\(g\in\Gal(\FF\extensio\KK)\) i considerem~\(g\circ f\). Tenim que~\(g\colon f\colon\FF\longrightarrow\FF\) és un isomorfisme d'anells, i ens queda veure que~\(\left.g\circ f\right|_{\KK}=\Id_{\KK}\). Tenim que
			\[\left.g\circ f\right|_{\KK}
			=\left.g\circ\Id_{\KK}\right|_{\KK}
			=\left.g\right|_{\KK}
			=\Id_{\KK},\]
			i hem acabat, i per la definició~\myref{def:grup} tenim que~\(\Gal(\FF\extensio\KK)\) és un grup.
		\end{proof}
	\end{proposition}
	\begin{proposition}
		\label{prop:els morfismes del grup de Galois conserven les arrels dels polinomis sobre K}
		Siguin~\(\FF\extensio\KK\) una extensió,~\(p(x)\in\KK[x]\) un polinomi i~\(\alpha\in\FF\) una arrel de~\(p(x)\). Aleshores per a tot~\(f\in\Gal(\FF\extensio\KK)\) tenim que~\(f(\alpha)\) és una arrel de~\(p(x)\).
		\begin{proof}
			Tenim que existeixen~\(a_{0},\dots,a_{n}\in\KK\) tals que
			\[p(x)=a_{0}+a_{1}x+\dots+a_{n}x^{n},\]
			i tenim que
			\[0=p(\alpha)=a_{0}+a_{1}\alpha+\dots+a_{n}\alpha^{n}.\]
			Aleshores tenim, per la definició de \myref{def:morfisme entre anells}, que
			\begin{align*}
				f\Big(\sum_{i=0}^{n}a_{i}\alpha^{i}\Big)&=\sum_{i=0}^{n}f(a_{i}\alpha^{i}) \\
				&=\sum_{i=0}^{n}f(a_{i})f(\alpha^{i}) \\
				&=\sum_{i=0}^{n}f(a_{i})f(\alpha)^{i} \\
				&=\sum_{i=0}^{n}a_{i}f(\alpha)^{i}=f(0)=0
			\end{align*}
			i per tant~\(f(\alpha)\) és una arrel de~\(p(x)\).
		\end{proof}
	\end{proposition}
	\begin{corollary}
		\label{cor:els morfismes del grup de Galois conserven les arrels dels polinomis sobre K}
		Siguin~\(\FF\extensio\KK\) una extensió i~\(\alpha\in\FF\) un element algebraic sobre~\(\KK\). Aleshores per a tot~\(f\in\Gal(\FF\extensio\KK)\) tenim que~\(f(\alpha)\) és una arrel de~\(\Irr(\alpha,\KK)\).
	\end{corollary}
	\begin{example}
		\label{ex:càlcul d'un grup de Galois}
		Calcular un grup de Galois.
		\begin{solution}
			%TODO
		\end{solution}
	\end{example}
\subsection{Extensions de morfismes}
	\begin{lemma}
		\label{lema:primer lema de l'extensió de morfismes}
		Siguin~\(\FF_{1}\extensio\KK\) i~\(\FF_{2}\extensio\KK\) dues extensions,~\(\alpha\in\FF_{1}\) un element algebraic sobre~\(\KK\) i~\(\beta\in\FF_{2}\) un element. Aleshores existeix un morfisme
		\[f\colon\KK(\alpha)\longrightarrow\FF_{2}\]
		tal que~\(\left.f\right|_{\KK}=\Id_{\KK}\) i~\(f(\alpha)=\beta\)
		si i només si~\(\beta\) és una arrel de~\(\Irr(\alpha,\KK)\).
		\begin{proof}
			Comencem veient que la condició és suficient~(\(\implica\)). Suposem que existeix un morfisme~\(f\colon\KK(\alpha)\longrightarrow\FF_{2}\) tal que~\(\left.f\right|_{\KK}=\Id_{\KK}\) i~\(f(\alpha)=\beta\). Volem veure que~\(\beta\) és una arrel de~\(\Irr(\alpha,\KK)\). Posem
			\[\Irr(\alpha,\KK)=a_{0}+a_{1}x+a_{2}x^{2}+\dots+a_{n}x^{n}\]
			amb~\(a_{0},\dots,a_{n}\in\KK\) i per la definició de~\myref{def:morfisme entre anells} calculem
			\begin{align*}
				a_{0}+a_{1}\beta+\dots+a_{n}\beta^{n}&=a_{0}+a_{1}f(\alpha)+\dots+a_{n}{f(\alpha)}^{n} \\
				&=a_{0}+a_{1}f(\alpha)+\dots+a_{n}f(\alpha^{n}) \\
				&=f(a_{0})+f(a_{1})f(\alpha)+\dots+f(a_{n})f(\alpha^{n}) \\
				&=f(a_{0})+f(a_{1}\alpha)+\dots+f(a_{n}\alpha^{n}) \\
				&=f(a_{0}+a_{1}\alpha+\dots+a_{n}\alpha^{n})\\
				&=f(0)=0,
			\end{align*}
			i per tant tenim que~\(\beta\) és una arrel de~\(\Irr(\alpha,\KK)\).
			
			Veiem ara que la condició és necessària~(\(\implicatper\)). Suposem que~\(\beta\) és una arrel de~\(\Irr(\alpha,\KK)\). Considerem el morfisme d'avaluació~\(\ev_{\beta}\colon\KK[x]\longrightarrow\FF_{2}\). Com que, per hipòtesi, tenim que~\(\beta\) és una arrel de~\(\Irr(\alpha,\KK)\) trobem que~\(\Irr(\alpha,\KK)\in\ker(\ev_{\beta})\). Ara bé, pe la definició d'\myref{def:element algebraic} trobem que~\(\Irr(\alpha,\KK)\) és irreductible, i per tant~\((\Irr(\alpha,\beta))=\ker(\ev_{\beta})\). %  Veure a estructures. Ideal principal.
			Aleshores tenim que
			\begin{align*}
				\KK(\alpha)&\cong\KK[x]/(\Irr(\alpha,\KK)) \tag{\ref{prop:el mínim subanell engendrat per un element és l'anell de polinomis quocient l'irreductible de l'element}}\\
				&=\KK[x]/\ker(\ev_{\beta}) \\
				&\cong\Ima(\ev_{\beta})\subseteq\FF_{2}. \tag{\ref{thm:Primer Teorema de l'isomorfisme entre anells}}
			\end{align*}
			Considerem doncs els morfismes entre anells
			\begin{align*}
				f_{1}\colon\KK(\alpha)&\longrightarrow\KK[x]/(\Irr(\alpha,\KK)) \\
				\KK\ni\lambda&\longmapsto\overline{\lambda} \\
				\alpha&\longmapsto\overline{x}
			\end{align*}
			i
			\begin{align*}
				f_{2}\colon\KK[x]/\ker(\ev_{\beta})&\longrightarrow\Ima(\ev_{\beta})\subseteq\FF_{2} \\
				\overline{\lambda}&\longmapsto\lambda \\
				\overline{x}&\longmapsto\beta
			\end{align*}
			i~\(f=f_{2}\circ f_{1}\) és el morfisme que buscàvem.
		\end{proof}
	\end{lemma}
%	\begin{example}
%		Exemple d'un grup de Galois on no cal aquest lema, a diferència de l'exemple que ve
%	\end{example}
	\begin{notation}
		Siguin~\(\FF\) i~\(\KK\) dos cossos i~\(f\colon\KK\longrightarrow\FF\) un morfisme entre anells. Aleshores denotem
		\begin{align*}
			\hat{f}\colon\KK[x]&\longrightarrow\FF[x] \\
			a_{0}+a_{1}x+\dots+a_{n}x^{n}&\longmapsto f(a_{0})x+f(a_{1})x+\dots+f(a_{n})x^{n}.
		\end{align*}
	\end{notation}
	\begin{lemma}[Extensió de morfismes]
		\labelname{\empty}\label{lema:extensió de morfismes}
		Siguin~\(\FF_{1}\extensio\KK\) i~\(\FF_{2}\extensio\KK\) dues extensions,~\(\alpha\in\FF_{1}\) un element algebraic sobre~\(\KK\),~\(\beta\in\FF_{2}\) un element i~\(f\colon\KK\longrightarrow\FF_{2}\) un morfisme. Aleshores existeix un morfisme
		\[g\colon\KK(\alpha)\longrightarrow\FF_{2}\]
		tal que~\(\left.g\right|_{\KK}=f\) i~\(g(\alpha)=\beta\)
		si i només si~\(\beta\) és una arrel de~\(\hat{f}(\Irr(\alpha,\KK))\).
		\begin{proof}
			%TODO
		\end{proof}
	\end{lemma}
	\begin{example}
		\label{ex:càlcul d'un altre grup de Galois}
		Volem calcular alguna cosa com~\(\Gal(\QQ(\sqrt{2},\sqrt{3}),\QQ)\).
		\begin{solution}
			\(\Gal(\QQ(\sqrt{2},\sqrt{3}),\QQ)\cong\ZZ/(2)\times\ZZ/(2)\).
		\end{solution}
	\end{example}
\subsection{El cos de descomposició}
	\begin{lemma}
		\label{lema:condicions equivalents a extensió finita}
		Siguin~\(\FF\extensio\KK\) una extensió. Aleshores són equivalents
		\begin{enumerate}
		\item L'extensió~\(\FF\extensio\KK\) és finita.
		\item Existeixen~\(\alpha_{1},\dots,\alpha_{n}\in\KK\) algebraics sobre~\(\KK\) tals que
		\[\FF=\KK(\alpha_{1},\dots,\alpha_{n}).\]
		\item Existeixen~\(\alpha_{1},\dots,\alpha_{n}\in\KK\) tals que~\(\alpha_{i}\) és algebraic sobre~\(\KK(\alpha_{1},\dots,\alpha_{i-1})\) i
		\[\FF=\KK(\alpha_{1},\dots,\alpha_{n}).\]
		\end{enumerate}
		\begin{proof}
			%TODO
		\end{proof}
	\end{lemma}
	\begin{theorem}
		\label{thm:teorema de les extensions algebraiques}
		Siguin~\(\LL\extensio\FF\) i~\(\FF\extensio\KK\) dues extensions. Aleshores~\(\LL\extensio\KK\) és algebraica
		si i només si~\(\LL\extensio\FF\) i~\(\FF\extensio\KK\) són algebraiques.
		\begin{proof}
			%TODO
		\end{proof}
	\end{theorem}
	\begin{theorem}
		\label{thm:els nombres algebraics són un cos}
		Siguin~\(\FF\extensio\KK\) una extensió i~\(\alpha\),~\(\beta\in\FF\) dos elements algebraics sobre~\(\KK\). Aleshores
		\begin{itemize}
		\item \(\alpha+\beta\) és algebraic sobre~\(\KK\).
		\item \(\alpha\beta\) és algebraic sobre~\(\KK\).
		\item Si~\(\alpha\neq0\),~\(\alpha^{1}\) és algebraic sobre~\(\KK\).
		\end{itemize}
	\end{theorem}
	\begin{corollary}
		\label{cor:els nombres algebraics són un cos}
		Sigui~\(\FF\extensio\KK\) una extensió. Aleshores
		\[\EE=\{\alpha\in\FF\mid\alpha\text{ és algebraic sobre }\KK\}\subseteq\FF\]
		és un cos.
	\end{corollary}
\subsection{El cos de descomposició d'una família de polinomis}
	\begin{theorem}
		\label{thm:existència del cos de descomposició d'una família de polinomis}
		Sigui~\(\Lambda=\{p_{i}(x)\}_{i=1}^{n}\) una família de polinomis sobre l'anell de polinomis d'un cos~\(\KK\). Aleshores existeix un cos~\(\LL\) tal que~\(\LL\extensio\KK\) és una extensió finita i~\(p_{i}(x)\) descompon en~\(\LL\) per a tot~\(i\in\{1,\dots,n\}\).
		\begin{proof}
			\myref{thm:Teorema de Kronecker}
			%TODO % Veure lo de l'extensió finita
		\end{proof}
	\end{theorem}
	\begin{theorem}
		\label{thm:unicitat del cos de descomposició d'una família de polinomis}
		Siguin~\(\LL_{1}\) i~\(\LL_{2}\) dos cossos de descomposició d'una família de polinomis~\(\Lambda=\{p_{i}(x)\}_{i=1}^{n}\) sobre un cos~\(\KK\). Aleshores existeix un isomorfisme d'anells~\(\varphi\colon\LL_{1}\longrightarrow\LL_{2}\) tal que~\(\left.\varphi\right|_{\KK}=\Id\).
		\begin{proof}
			%TODO
		\end{proof}
	\end{theorem}
	\begin{theorem}[Extensió de morfismes a un cos de descomposició]
		\labelname{extensió de morfismes a un cos de descomposició}\label{thm:extensió de morfismes a un cos de descomposició}
		Siguin~\(\EE\extensio\FF\extensio\KK\) una extensió de cossos~\(p(x)\in\KK[x]\) un polinomi i~\(\LL\) el seu cos de descomposició tals que~\(p(x)\) descompon en~\(\EE[x]\) i tals que existeix un morfisme~\(f\colon\FF\longrightarrow\EE\) tal que~\(\left.f\right|_{\KK}=\Id\). Aleshores existeix una funció~\(\tilde{f}\colon\LL\longrightarrow\EE\) tal que~\(\left.\tilde{f}\right|_{\FF}=f\).
		\begin{proof}
			%TODO
		\end{proof}
	\end{theorem}
	\begin{definition}[Subgrup transitiu]
		\labelname{subgrup transitiu}\label{def:subgrup transitiu}
		Siguin~\(\GrupSimetric_{n}\) el grup simètric~d'\(n\) elements i~\(G\subgrup\GrupSimetric_{n}\) un subgrup tal que per a tots~\(i\),~\(j\in\{1,\dots,n\}\) existeix una permutació~\(\sigma\in\GrupSimetric_{n}\) tal que~\(\sigma(i)=j\). Aleshores direm que~\(G\) és un subgrup transitiu de~\(\GrupSimetric_{n}\).
	\end{definition}
	\begin{proposition}
		\label{prop:l'ordre d'un subgrup transitiu divideix n}
		Sigui~\(G\) un subgrup transitiu de~\(\GrupSimetric_{n}\). Aleshores
		\[n\divides\card{G}.\]
		\begin{proof}
			%TODO
		\end{proof}
	\end{proposition}
	\begin{definition}[Grup de Galois d'un polinomi]
		\labelname{grup de Galois d'un polinomi}\label{def:grup de Galois d'un polinomi}
		Sigui~\(p(x)\in\KK[x]\) un polinomi amb cos de descomposició~\(\LL\). Aleshores definirem
		\[\Gal(p(x))=\Gal(\LL\extensio\KK)\]
		com el grup de Galois de~\(p(x)\).
	\end{definition}
	\begin{corollary}
		\label{cor:grup de Galois d'una extensió amb un cos de descomposició}
		Siguin~\(p(x)\in\KK[x]\) un polinomi sobre un cos~\(\KK\) i~\(A=\{\alpha_{1},\dots,\alpha_{n}\}\) el conjunt de les arrels de~\(p(x)\). Aleshores~\(\Gal(p(x))\) és isomorf un subgrup transitiu de~\(\GrupSimetric_{A}\).
		\begin{proof}
			%TODO
		\end{proof}
	\end{corollary}
%\subsection{Discriminant}
	\begin{definition}[Discriminant]
		\labelname{discriminant}\label{def:discriminant}
		Sigui~\(p(x)\in\KK[x]\) un polinomi sobre un cos~\(\KK\) amb cos de descomposició~\(\LL\) i arrels~\(\alpha_{1},\dots,\alpha_{n}\in\LL\). Aleshores definim
		\[\discriminant(p(x))=\prod_{i<j}^{n}(\alpha_{i}-\alpha_{j})^{2}\]
		com el discriminant de~\(p(x)\). També definim
		\[\delta(p(x))=\prod_{i<j}^{n}(\alpha_{i}-\alpha_{j}).\]
	\end{definition}
	\begin{observation}
		\label{obs:discriminant}
		\(\discriminant(p(x))=\delta(p(x))^{2}\).
	\end{observation}
	\begin{example}
		\label{ex:discriminant d'un polinomi quadràtic}
		Volem calcular el discriminant d'un polinomi de grau~\(2\).	
		\begin{solution}
			%TODO
		\end{solution}
	\end{example}
	\begin{proposition}
		\label{prop:el discriminant pertany al cos del polinomi}
		Sigui~\(p(x)\in\KK[x]\) un polinomi sobre un cos~\(\KK\). Aleshores~\(\discriminant(p(x))\in\KK\).
	\end{proposition}
	\begin{lemma}
		\label{lema:discriminant}
		Siguin~\(p(x)\in\KK[x]\) un polinomi sobre un cos~\(\KK\) amb~\(n\) arrels i~\(\sigma\in\GrupSimetric_{n}\) una permutació. Aleshores
		\[\prod_{i<j}^{n}(\alpha_{\sigma(i)}-\alpha_{\sigma(j)})=\sig(\sigma)\delta(p(x))\]
		\begin{proof}
			%TODO
		\end{proof}
	\end{lemma}
	\begin{corollary}
		\label{cor:l'arrel del discriminant d'un polinomi pertany al cos si i només si el grup de Galois del polinomi és un subgrup dels alternats de les arrels}
		Siguin~\(p(x)\in\KK[x]\) un polinomi sense arrels múltiples sobre un cos~\(\KK\) amb~\(\ch(\KK)\neq2\), amb conjunt d'arrels~\(A=\{\alpha_{1},\dots,\alpha_{n}\}\) i sigui~\(\varphi\colon\Gal(p(x))\longrightarrow\GrupSimetric_{A}\) un epimorfisme de grups.
		Aleshores
		\[\varphi(\Gal(p(x))\subseteq\GrupAlternat(A)\Sii\delta(p(x))\in\KK.\]
		\begin{proof}
			Aquest enunciat té sentit pel \corollari~\myref{cor:grup de Galois d'una extensió amb un cos de descomposició}.
			%TODO
		\end{proof}
	\end{corollary}
	% Exemple dels polinomis de grau 2 i 3.
	
	\section{Extensions normals i extensions separables}
\subsection{Extensions normals}
	\begin{definition}[Extensió normal]
		\labelname{extensió normal}\label{def:extensió normal}
		Sigui~\(\FF\extensio\KK\) una extensió de cossos tal que
		\begin{enumerate}
			\item L'extensió~\(\FF\extensio\KK\) és algebraica.
			\item Per a tot polinomi irreductible~\(p(x)\in\KK[x]\) tal que~\(p(x)\) té una arrel en~\(\FF\) tenim que~\(p(x)\) té totes les arrels a~\(\FF\).
		\end{enumerate}
		Aleshores direm que l'extensió~\(\FF\extensio\KK\) és una extensió normal.
	\end{definition}
	\begin{theorem}[Teorema de normalitat]
		\labelname{Teorema de normalitat}\label{thm:Teorema de normalitat}
		Sigui~\(\FF\extensio\KK\) una extensió finita. Aleshores~\(\FF\extensio\KK\) és una extensió normal si i només si existeix un polinomi~\(p(x)\in\KK[x]\setminus\KK\) tal que~\(\FF\) és el cos de descomposició de~\(p(x)\).
		\begin{proof}
			%TODO
		\end{proof}
	\end{theorem}
\subsection{Elements separables}
	\begin{definition}[Polinomi separable]
		\labelname{polinomi separable}\label{def:polinomi separable}
		Sigui~\(p(x)\) un polinomi irreductible sense arrels múltiples en el seu cos de descomposició. Aleshores direm que~\(p(x)\) és separable.
	\end{definition}
	\begin{definition}[Element separable]
		\labelname{element separable}\label{def:element separable}
		Siguin~\(\FF\extensio\KK\) una extensió de cossos i~\(\alpha\in\FF\) un element algebraic sobre~\(\KK\) tal que~\(\Irr(\alpha,\KK)\) és separable. Aleshores direm que~\(\alpha\) és separable sobre~\(\KK\).
	\end{definition}
	\begin{definition}[Extensió separable]
		\labelname{extensió separable}\label{def:extensió separable}
		Sigui~\(\FF\extensio\KK\) una extensió de cossos tal que per a tot~\(\alpha\in\FF\) tenim que~\(\alpha\) és separable sobre~\(\KK\). Aleshores direm que~\(\FF\extensio\KK\) és una extensió separable.
	\end{definition}
	\begin{observation}
		\label{obs:les extensions separables són algebraiques}
		Sigui~\(\FF\extensio\KK\) una extensió separable. Aleshores~\(\FF\extensio\KK\) és una extensió algebraica.
%		Les extensions separables són extensions algebraiques.
	\end{observation}
	\begin{proposition}
		Sigui~\(\FF\extensio\KK\) una extensió amb~\(\ch(\KK)\neq0\). Aleshores l'extensió~\(\FF\extensio\KK\) és separable si i només si és algebraica.
		\begin{proof}
			%TODO
			%Faltaran refs per lo de la característica.
		\end{proof}
	\end{proposition}
	\begin{theorem}
		\label{thm:teorema de les extensions separables}
		Siguin~\(\LL\extensio\FF\) i~\(\FF\extensio\KK\) dues extensions de cossos tals que l'extensió~\(\LL\extensio\KK\) sigui separable. Aleshores les extensions~\(\LL\extensio\FF\) i~\(\FF\extensio\KK\) són separables.
		\begin{proof}
			%TODO
		\end{proof}
	\end{theorem}
	\begin{lemma}
		\label{lema:un polinomi és separable si i només si ho és adaptant els coeficients}
		Siguin~\(p(x)\in\KK[x]\) un polinomi,~\(\FF\) un cos i~\(f\colon\KK\longrightarrow\FF\) un morfisme d'anells. Aleshores~\(p(x)\) és separable si i només si~\(\hat{f}(p(x))\) és separable.	
	\end{lemma}
	\begin{lemma}
		\label{lema:condicions pel nombre d'extensions de morfismes}
		Siguin~\(\KK(\alpha)\extensio\KK\) una extensió,~\(\FF\) un cos i~\(f\colon\KK\longleftarrow\FF\) un morfisme d'anells tals que~\(\hat{f}(\Irr(\alpha,\KK))\) descompon en factors lineals. Aleshores són equivalents
		\begin{enumerate}
			\item\label{lema:condicions pel nombre d'extensions de morfismes:eq1} L'extensió~\(\KK(\alpha)\extensio\KK\) és separable.
			\item\label{lema:condicions pel nombre d'extensions de morfismes:eq2} L'element~\(\alpha\) és separable.
			\item\label{lema:condicions pel nombre d'extensions de morfismes:eq3} \(\grauExtensio{\KK(\alpha)}{\KK}=\card{\{g\colon\KK(\alpha)\longrightarrow\FF\mid\left.g\right|_{\KK}=f\text{ i }g\text{ és un morfisme d'anells}\}}\).
		\end{enumerate}
		\begin{proof}
%			Tenim que~(\(\ref{lema:condicions pel nombre d'extensions de morfismes:eq1}\implica\ref{lema:condicions pel nombre d'extensions de morfismes:eq2}\)) per la definició de~\myref{def:extensió separable}.
			%TODO
		\end{proof}
	\end{lemma}
	\begin{theorem}[Teorema de separabilitat]
		\labelname{Teorema de separabilitat}\label{thm:Teorema de separabilitat}
		Sigui~\(\KK(\alpha_{1},\dots,\alpha_{n})\extensio\KK\) una extensió algebraica,~\(\FF\) un cos i~\(f\colon\FF\longrightarrow\FF\) un morfisme d'anells tal que per a tot~\(i\in\{1,\dots,n\}\) tenim que~\(\hat{f}(\Irr(\alpha_{i},\KK))\) descompon en factors lineals. Aleshores són equivalents
		\begin{enumerate}
			\item L'extensió~\(\KK(\alpha_{1},\dots,\alpha_{n})\) és separable.
			\item Els elements~\(\alpha_{1},\dots,\alpha_{n}\) són separables sobre~\(\KK\).
			\item L'element~\(\alpha_{1}\) és separable sobre~\(\KK\), i per a tot~\(i\in\{2,\dots,n\}\) l'element~\(\alpha_{i}\) és separable sobre~\(\KK(\alpha_{1},\dots,\alpha_{i-1})\).
				\item Denotant~\(\LL=\KK(\alpha_{1},\dots,\alpha_{n})\) tenim que
				\[\grauExtensio{\LL}{\KK}=\card{\{g\colon\LL\longrightarrow\KK\mid\left.g\right|_{\KK}=f\text{ i }g\text{ és un morfisme d'anells}\}}.\]
		\end{enumerate}
	\end{theorem}
\section{El Teorema Fonamental de la Teoria de Galois}
\subsection{Extensions de Galois}
	\begin{definition}[Extensió de Galois]
		\labelname{extensió de Galois}\label{def:extensió de Galois}
		Sigui~\(\FF\extensio\KK\) una extensió normal i separable. Aleshores direm que l'extensió~\(\FF\extensio\KK\) és de Galois.
	\end{definition}
	\begin{theorem}[Teorema de les extensions de Galois]
		\labelname{Teorema de les extensions de Galois}\label{thm:Teorema de les extensions de Galois}
		Sigui~\(\FF\extensio\KK\) una extensió finita. Aleshores~\(\FF\extensio\KK\) és de Galois si i només si
		\[\card{\Gal(\FF\extensio\KK)}=\grauExtensio{\FF}{\KK}.\]
		\begin{proof}
			%TODO
		\end{proof}
	\end{theorem}
\subsection{Reticles}
	\begin{definition}[Reticle de subgrups]
		\labelname{reticle de subgrups}\label{def:reticle de subgrups}
		Sigui~\(G\) un grup. Aleshores definim
		\[\ReticleGrup(G)=\{H\subseteq G\mid H\text{ és un subgrup de }G\}\]
		com el reticle de subgrups de~\(G\).
	\end{definition}
	\begin{definition}[Reticle de subcossos]
		\labelname{reticle de subcossos}\label{def:reticle de subcossos}
		Sigui~\(\FF\extensio\KK\) una extensió de cossos. Aleshores definim
		\[\ReticleCos(\FF\extensio\KK)=\{\EE\subseteq\FF\mid\EE\text{ és un cos i }\KK\subseteq\EE\}\]
		com el reticle de subcossos de l'extensió~\(\FF\extensio\KK\).
	\end{definition}
	\begin{lemma}
		\label{lema:cos fix}
		Siguin~\(\FF\extensio\KK\) una extensió,~\(\EE\in\ReticleCos(\FF\extensio\KK)\) un subcòs i~\(H\in\ReticleGrup(\Gal(\FF\extensio\KK))\) un grup. Aleshores
		\[\EE^{H}=\{a\in\EE\mid f(a)=a\text{ per a tot }f\in H\}\]
		és un cos i satisfà~\(\KK\subseteq\EE^{H}\subseteq\FF\).
		\begin{proof}
			%TODO
		\end{proof}
	\end{lemma}
	\begin{definition}[Cos fix]
		\labelname{cos fix}\label{def:cos fix}
		Siguin~\(\FF\extensio\KK\) una extensió,~\(\EE\in\ReticleCos(\FF\extensio\KK)\) un subcòs i~\(H\in\ReticleGrup(\Gal(\FF\extensio\KK))\) un grup. Aleshores definim
		\[\EE^{H}=\{a\in\EE\mid f(a)=a\text{ per a tot }f\in H\}\]
		com el cos fix de~\(H\). % Té sentit pel lemma anterior
	\end{definition}
	\begin{proposition}
		\label{prop:els grups de Galois inverteixen les inclusions}
		Siguin~\(\FF\extensio\EE\) i~\(\EE\extensio\KK\) dues extensions de cossos. Aleshores~\(\Gal(\FF\extensio\EE)\) és un subgrup de~\(\Gal(\FF\extensio\KK)\).
		\begin{proof}
			%TODO
		\end{proof}
	\end{proposition}
	\begin{lemma}
		% Escriure el lemma aquest tant llarg
	\end{lemma}
\end{document}
