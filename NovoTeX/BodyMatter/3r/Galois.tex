\documentclass[../Apunts.tex]{subfiles}

\begin{document}
\part{Teoria de Galois}
\chapter[Capítol primer]{Primer}
\section{Extensions de cossos}
\subsection{Elements algebraics i elements transcendents}
	\begin{definition}[Extensió d'un cos]
		\labelname{extensió d'un cos}\label{def:extensió d'un cos}
		Siguin~\(\field{K}\) i~\(\field{F}\) dos cossos tals que~\(\field{F}\subseteq\field{K}\). Aleshores direm que~\(\field{K}\) és una extensió de~\(\field{F}\) i ho denotarem com~\(\field{K}\extensio\field{F}\). També direm que~\(\field{K}\extensio\field{F}\) és una extensió.
	\end{definition}
	\begin{definition}[Grau d'una extensió]
		\labelname{grau d'una extensió}\label{def:grau d'una extensió}
		Sigui~\(\field{K}\extensio\field{F}\) una extensió. Aleshores direm que la dimensió del \(\field{K}\)-espai vectorial~\(\field{F}\) és el grau de l'extensió i el denotarem com~\(\grauExtensio{\field{K}}{\field{F}}\).
		
		Aquesta definició té sentit per la proposició~\myref{prop:un subcòs és un espai vectorial}.
	\end{definition}
\end{document}