\documentclass[../Apunts.tex]{subfiles}

\begin{document}
\part{Geometria diferencial}
\chapter{Corbes}
	\section{Difeomorfismes de classe \ensuremath{\mathcal{C}^{\infty}}}
	\subsection{Funcions analítiques}
	\begin{definition}[Classe de diferenciabilitat infinita]
		\labelname{classe de diferenciabilitat infinita}\label{def:classe de diferenciabilitat infinita}
		Sigui \(\obert{U}\subseteq\mathbb{R}^{d}\) un obert i
		\begin{align*}
			f\colon\obert{U}&\longrightarrow\mathbb{R}^{m} \\
			x&\longmapsto(f_{1}(x),\dots,f_{m}(x))
		\end{align*}
		una funció tal que per a tot \(i\in\{1,\dots,m\}\) la funció \(f_{i}\) és de classe \(\mathcal{C}^{k}\) per a tot natural \(k\in\mathbb{N}\). Aleshores direm que \(f\) és de classe de diferenciabilitat infinita. També direm que \(f\) és de classe \(\mathcal{C}^{\infty}\).
	\end{definition}
	\begin{definition}[Funció analítica]
		\labelname{funció analítica}\label{def:funció analítica}
		Sigui \(\obert{U}\subseteq\mathbb{R}^{d}\) un obert,
		\begin{align*}
			f\colon\obert{U}&\longrightarrow\mathbb{R}^{m} \\
			x&\longmapsto(f_{1}(x),\dots,f_{m}(x))
		\end{align*}
		una funció de classe \(\mathcal{C}^{\infty}\) i \(x_{0}\in\obert{U}\) un punt tal que existeix una sèrie de potències \(\sum_{n=0}^{\infty}a_{n}(x-x_{0})^{n}\) tal que existeix un entorn \(N_{x_{0}}\subseteq\obert{U}\) satisfent que per a tot \(x\in N_{x_{0}}\) la sèrie de potències \(\sum_{n=0}^{\infty}a_{n}(x-x_{0})^{n}\) convergeix puntualment a \(f(x)\). Aleshores direm que \(f\) és una funció analítica. També direm que \(f\) és de classe \(\mathcal{C}^{\omega}\).
	\end{definition}
	\begin{example}
		\label{ex:els polinomis són funcions analítiques}
		Volem veure que tot polinomi és una funció analítica.
		\begin{solution}
			Prenem un polinomi
			\[p(x)=a_{0}+a_{1}x+a_{2}x^{2}+\dots+a_{n}x^{n}\]
			i un natural \(k\in\mathbb{N}\). Si \(k\leq n\) tenim que % REFS
			\[p^{(k)}(x)=k!a_{k}+(k+1)!a_{k+1}x+\dots+\frac{n!}{(n-k)!}a_{n}x^{n-k},\]
			i si \(k>n\) tenim que
			\[p^{(k)}(x)=0.\]
			Aleshores per la definició de \myref{def:classe de diferenciabilitat infinita} tenim que \(p(x)\) és de classe \(\mathcal{C}^{\infty}\). Observem que
			\[p(x)=\sum_{i=0}^{n}a_{i}x^{i}\]
			i per la definició de \myref{def:sèrie de potències} tenim que \(p(x)\) és una sèrie de potències, i per la definició de \myref{def:convergència puntual} trobem que \(p(x)\) convergeix puntualment a \(\sum_{i=0}^{n}a_{i}x^{i}\) per a tot \(x\in\mathbb{R}\), i per la definició de \myref{def:funció analítica} tenim que \(p(x)\) és una funció analítica per a tot \(x\in\mathbb{R}\).
		\end{solution}
	\end{example}
	\begin{definition}[Difeomorfisme de classe \ensuremath{\mathcal{C}^{\infty}}]
		\labelname{difeomorfisme de classe \ensuremath{\mathcal{C}^{\infty}}}\label{def:difeomorfisme de classe C infinit}\label{def:difeomorfisme de classe de diferenciabilitat infinita}
		Siguin \(\obert{U}\) i \(\obert{V}\) dos oberts de \(\mathbb{R}^{d}\) i \(\Phi\colon\obert{U}\longleftrightarrow\obert{V}\) un difeomorfisme tal que les funcions \(\Phi\) i \(\Phi^{-1}\) siguin de classe \(\mathcal{C}^{\infty}\). Aleshores direm que \(\Phi\) és un difeomorfisme de classe de diferenciabilitat infinita o que \(\Phi\) és un difeomorfisme de classe \(\mathcal{C}^{\infty}\).
		
		Aquesta definició té sentit per la definició de \myref{def:difeomorfisme}.
	\end{definition}
	\subsection{Reparametrització d'una corba}
	\begin{definition}[Corba]
		\labelname{corba}\label{def:corba}
		Siguin \(I\subseteq\mathbb{R}\) un interval i \(\alpha\colon I\longrightarrow\mathbb{R}^{n}\) una funció. Aleshores direm que \(\alpha\) és una corba sobre \(I\).
	\end{definition}
	\begin{example}
		\label{ex:corba que uniex dos punts}
		Volem veure que donats dos punts \(a\) i \(b\) de \(\mathbb{R}^{n}\) existeix una corba contínua que els uneix.
		\begin{solution}
			Observem que la funció contínua % Veure que és contínua per ser un polinomi
			\begin{align*}
				\alpha\colon[0,1]&\longrightarrow\mathbb{R}^{n} \\
				t&\longmapsto a+t(b-a)
			\end{align*}
			satisfà que \(\alpha(0)=a\) i \(\alpha(1)=b\), i per la definició de \myref{def:corba} trobem que és una corba contínua sobre \([0,1]\).
		\end{solution}
	\end{example}
	\begin{definition}[Corba regular]
		\labelname{corba regular}\label{def:corba regular}
		Sigui \(\alpha\) una corba sobre \(I\) tal que per a tot \(t\in I\) es satisfà \(\alpha'(t)\neq\vec{0}\). Aleshores direm que \(\alpha\) és regular.
	\end{definition}
	\begin{definition}[Reparametrització]
		\labelname{reparametrització d'una corba}\label{def:reparametrització d'una corba}
		\labelname{canvi de paràmetre}\label{def:canvi de paràmetre}
		Siguin \(\alpha\) una corba sobre \(I\) i \(h\colon J\longrightarrow I\) un difeomorfisme. Aleshores direm que la funció
		\[\beta(t)=(\alpha\circ h)(t)\]
		és una reparametrització en \(J\) de \(\alpha\) i que \(h\) és el canvi de paràmetre.
	\end{definition}
	\begin{proposition}
		Sigui \(\beta\) una reparametrització en \(J\) d'una corba regular \(\alpha\) sobre \(I\). Aleshores \(\beta\) és una corba regular.
		\begin{proof}
			Per la definició de \myref{def:reparametrització d'una corba} trobem que existeix un canvi de paràmetre \(h:J\longrightarrow I\) tal que \(\beta=(\alpha\circ h)(t)\). Aleshores tenim que
			\begin{align*}
				\beta\colon J&\longrightarrow\mathbb{R}^{n} \\
				t&\longmapsto\alpha(h(t)),
			\end{align*}
			i per la definició de \myref{def:corba} trobem que \(\beta\) és una corba.
			
			Per la definició de \myref{def:canvi de paràmetre} trobem que \(h\) és un difeomorfisme, i per la definició de \myref{def:difeomorfisme} trobem que \(h\) és derivable, i per la \myref{thm:regla de la cadena} trobem que
			\[\beta'(t)=h'(t)\alpha'(h(t)).\]
			
			Com que per hipòtesi \(\alpha\) és regular, per la definició de \myref{def:corba regular} trobem que \(\alpha'(h(t))\neq\vec{0}\) per a tot \(t\in J\). Ara bé, pel \corollari \myref{cor:difeomorfisme és equivalent a ser injectiva i tenir diferencial amb determinant no nul} trobem que \(h'(t)\neq\vec{0}\) per a tot \(t\in J\), i per tant tenim que \(\beta'(t)\neq\vec{0}\) per a tot \(t\in J\), i per la definició de \myref{def:corba regular} trobem que \(\beta\) és una corba regular.
		\end{proof}
	\end{proposition}
	\subsection{La longitud d'una corba i el paràmetre arc}
	\begin{definition}[Longitud d'una corba]
		\labelname{longitud d'una corba}\label{def:longitud d'una corba}
		Siguin \(\alpha\) una corba sobre \(I\) i \(a\), \(b\) dos punts de \(I\). Aleshores definim
		\[\Long_{a}^{b}(\alpha)=\int_{a}^{b}\norm{\alpha'(t)}\diff t\]
		com la longitud de \(\alpha\) entre \(a\) i \(b\).
	\end{definition}
	\begin{example}
		\label{ex:longitud d'una corba}
		Volem calcular la longitud de la corba
		\[\alpha(t)=(\cos(t),\sin(t),t)\]
		entre \(0\) i \(x\) per a tot \(x\) positiu.
		\begin{solution}
			Per la definició de \myref{def:longitud d'una corba} trobem que això és
			\begin{align*}
				\Long_{0}^{x}(\alpha)&=\int_{0}^{x}\norm{\alpha'(t)}\diff t \\
				&=\int_{0}^{x}\norm{(-\sin(t),\cos(t),1)}\diff t \\
				&=\int_{0}^{x}\sqrt{\sin^{2}(t)+\cos^{2}(t)+1}\diff t \\
				&=\int_{0}^{x}\sqrt{2}\diff t \\
				&=[\sqrt{2}t]_{0}^{x}=\sqrt{2}x.\qedhere
			\end{align*}
		\end{solution}
	\end{example}
	\begin{proposition}
		Sigui \(\beta\) una reparametrització d'una corba \(\alpha\) en \(I\) amb canvi de paràmetre \(h\). Aleshores per a tot \(a\) i \(b\) de \(I\), amb \(c=h^{-1}(a)\) i \(d=h^{-1}(d)\) tenim
		\[\Long_{a}^{b}(\alpha)=\Long_{c}^{d}(\beta).\]
		\begin{proof}
			Això té sentit per la definició de \myref{def:canvi de paràmetre} i la definició de \myref{def:difeomorfisme}.
			
			Per la definició de \myref{def:longitud d'una corba} trobem que
			\begin{align*}
				\Long_{c}^{d}(\beta)&=\int_{c}^{d}\norm{\beta'(s)}\diff s \\
				&=\int_{c}^{d}\norm{(\alpha\circ h)'(s)}\diff s, \tag{\ref{def:canvi de paràmetre}}
			\end{align*}
			i com que, per la definició de \myref{def:canvi de paràmetre} tenim que \(h\) és un difeomorfisme, per la definició de \myref{def:difeomorfisme} trobem que \(h\) és diferenciable i per la \myref{thm:regla de la cadena} trobem que
			\[\int_{c}^{d}\norm{(\alpha\circ h)'(s)}\diff s=\int_{c}^{d}\norm{h'(s)\alpha'(h(s))}\diff s,\]
			i per la definició de norma tenim que% REF
			\[\int_{c}^{d}\norm{h'(s)\alpha'(h(s))}\diff s=\int_{c}^{d}\abs{h'(s)}\norm{\alpha'(h(s))}\diff s.\]
			
			Ara bé, com que \(h\) és un difeomorfisme tenim per la definició de \myref{def:difeomorfisme} que \(h'\) és contínua, i pel \corollari \myref{cor:difeomorfisme és equivalent a ser injectiva i tenir diferencial amb determinant no nul} trobem que \(h'(t)\neq0\) per a tot \(t\in[c,d]\). Aleshores tenim que ha de ser o bé \(h'(t)>0\) per a tot \(t\in[c,d]\) o bé \(h'(t)<0\) per a tot \(t\in[c,d]\).
			
			Comencem suposant que \(h'(t)>0\) per a tot \(t\in[c,d]\). Aleshores tenim que
			\[\int_{c}^{d}\abs{h'(s)}\norm{\alpha'(h(s))}\diff s=\int_{c}^{d}h'(s)\norm{\alpha'(h(s))}\diff s,\]
			i pel Teorema del canvi de variable tenim que %REF
			\begin{align*}
				\int_{c}^{d}h'(s)\norm{\alpha'(h(s))}\diff s&=\int_{h(c)}^{h(d)}\norm{\alpha(t)}\diff t \\
				&=\int_{a}^{b}\norm{\alpha(t)}\diff t=\Long_{a}^{b}(\alpha),
			\end{align*}
			i per tant
			\[\Long_{c}^{d}(\beta)=\Long_{a}^{b}(\alpha).\]
			
			Suposem ara que \(h'(t)>0\) per a tot \(t\in[c,d]\). Aleshores tenim que
			\begin{align*}
				\int_{c}^{d}\abs{h'(s)}\norm{\alpha'(h(s))}\diff s&=\int_{c}^{d}-h'(s)\norm{\alpha'(h(s))}\diff s \\
				&=\int_{d}^{c}h'(s)\norm{\alpha'(h(s))}\diff s
			\end{align*}
			i pel Teorema del canvi de variable tenim que %REF
			\begin{align*}
				\int_{d}^{c}h'(s)\norm{\alpha'(h(s))}\diff s&=\int_{h(c)}^{h(d)}\norm{\alpha(t)}\diff t \\
				&=\int_{a}^{b}\norm{\alpha(t)}\diff t=\Long_{a}^{b}(\alpha),
			\end{align*}
			i per tant
			\[\Long_{c}^{d}(\beta)=\Long_{a}^{b}(\alpha).\qedhere\]
		\end{proof}
	\end{proposition}
	\begin{definition}[Funció longitud d'arc]
		\labelname{funció longitud d'arc}\label{def:funció longitud d'arc}
		Siguin \(\alpha\) una corba sobre \(I\) i \(a\in I\) un punt. Aleshores direm que la funció
		\[\funciolongituddarc_{\alpha}(a)(t)=\int_{a}^{t}\norm{\alpha'(s)}\diff s\]
		és la funció longitud d'arc de \(\alpha\) amb origen en \(a\).
%		\ref{def:diferencial}
	\end{definition}
	\begin{observation}
		\label{obs:la funció longitud d'arc és creixent}
		Sigui \(\alpha\) una corba sobre \(I\) i \(a\in I\) un punt. Aleshores per a tot \(t\in I\)
		\[\frac{\diff\funciolongituddarc_{\alpha}(a)}{\diff t}(t)\geq0.\]
		\begin{proof}
			Per la definició de \myref{def:funció longitud d'arc} trobem que
			\[\funciolongituddarc_{\alpha}(a)(t)=\int_{a}^{t}\norm{\alpha'(s)}\diff s,\]
			i pel \myref{thm:Teorema Fonamental del Càlcul} tenim que
			\[\frac{\diff\funciolongituddarc_{\alpha}(a)}{\diff t}(t)=\norm{\alpha'(t)}\geq0.\qedhere\] % REFS
		\end{proof}
	\end{observation}
	\begin{proposition}
		\label{prop:la funció longitud arc d'una corba regular és un difeomorfisme}
		Siguin \(\alpha\) una corba regular sobre \(I\) i \(a\in I\) un punt. Aleshores la funció \(\funciolongituddarc_{\alpha}(a)\) és un difeomorfisme. % SOBRE J
		\begin{proof}
			Per la definició de \myref{def:corba regular} trobem que \(\alpha'(t)\neq0\) per a tot \(t\in I\), i pel \myref{thm:Teorema Fonamental del Càlcul} tenim que
			\[\frac{\diff\funciolongituddarc_{\alpha}(a)}{\diff t}(t)=\norm{\alpha'(t)}.\]
			Per tant trobem que % REFS
			\[\frac{\diff\funciolongituddarc_{\alpha}(a)}{\diff t}(t)\neq0,\]
			i per l'observació \myref{obs:diferencial en d=1 és com derivar} i el \corollari 		\myref{cor:difeomorfisme és equivalent a ser injectiva i tenir diferencial amb determinant no nul} tenim que \(\funciolongituddarc_{\alpha}(a)\) és un difeomorfisme, com volíem veure. % SOBRE J
		\end{proof}
	\end{proposition}
	\begin{proposition} % Veure després de definir corba parametritzada per l'arc
		\label{prop:podem trobar una reparametrització amb velocitat unitaria de qualsevol corba regular}
		Sigui \(\alpha\) una corba regular sobre \(I\). Aleshores existeix una reparametrització \(\beta\) de \(\alpha\) tal que per a tot \(t\in I\)
		\[\norm{\beta'(t)}=1.\]
		\begin{proof}
			Per la proposició \ref{prop:la funció longitud arc d'una corba regular és un difeomorfisme} tenim que per a tot \(a\in I\) la funció \(\funciolongituddarc_{\alpha}(a)\) és un difeomorfisme, i per la definició de \myref{def:difeomorfisme} trobem que la funció \(\funciolongituddarc_{\alpha}(a)\) és bijectiva, i pel Teorema \myref{thm:bijectiva iff invertible} trobem que \(\funciolongituddarc_{\alpha}(a)\) és invertible i per la definició de \myref{def:aplicació invertible} tenim que existeix una funció \(t_{a}\) tal que \(t_{a}\) sigui la inversa de \(\funciolongituddarc_{\alpha}(a)\). Considerem
			\[\beta(s)=\alpha(t(s)).\]
			Aleshores tenim que
			\begin{align*} % És positiva pel corol·lari del teorema de la funció inversa?
				\norm{\beta'(s)}&=\norm{\frac{\diff t}{\diff s}\alpha'(s)} \\
				&=\abs{\frac{\diff t}{\diff s}}\norm{\alpha'(s)} \\
				&=\frac{\diff t}{\diff s}(s(t))\frac{\diff s}{\diff t}(t(s))\\
				&=\frac{1}{\frac{\diff s}{\diff t}(t(s))}\frac{\diff s}{\diff t}(t(s))=1. \qedhere
			\end{align*}
		\end{proof}
	\end{proposition}
	\begin{definition}[Corba parametritzada per l'arc]
		\labelname{corba parametritzada per l'arc}\label{def:corba parametritzada per l'arc}
		Sigui \(\alpha\) una corba regular sobre \(I\) tal que per a tot \(t\in I\) es satisfà
		\[\norm{\alpha'(t)}=1.\]
		Aleshores direm que \(\alpha\) està parametritzada per l'arc.
	\end{definition}
	\begin{observation}
		Sigui \(\alpha\) una corba en \(I\) parametritzada per l'arc i \(a\in I\) un punt. Aleshores per a tot \(t\in I\) tenim
		\[\funciolongituddarc_{\alpha}(a)(t)=t-a.\]
		\begin{proof}
			Per la definició de \myref{def:funció longitud d'arc} trobem que
			\[\funciolongituddarc_{\alpha}(a)(t)=\int_{a}^{t}\norm{\alpha'(s)}\diff s,\]
			i per la definició de \myref{def:corba parametritzada per l'arc} tenim que  per a tot \(t\in I\) es satisfà \(\norm{\alpha'(t)}=1\), i per tant
			\begin{align*}
				\funciolongituddarc_{\alpha}(a)(t)&=\int_{a}^{t}\norm{\alpha'(s)}\diff s \\
				&=\int_{a}^{t}\diff s=t-a.\qedhere
			\end{align*}
		\end{proof}
	\end{observation}
	\begin{definition}[Contacte]
		\labelname{contacte}\label{def:contacte entre dues corbes parametritzades per l'arc}
		Siguin \(\alpha\) i \(\beta\) dues corbes en \(I\) parametritzades per l'arc i \(t_{0}\in I\) un punt tals que existeix un \(r\) satisfent
		\[\lim_{t\to t_{0}}\frac{\alpha(t)-\beta(t)}{(t-t_{0})^{p}}=0\quad\text{i}\quad\lim_{t\to t_{0}}\frac{\alpha(t)-\beta(t)}{(t-t_{0})^{r+1}}\neq0\]
		per a tot \(p\leq r\). Aleshores direm que \(\alpha\) i \(\beta\) tenen contacte d'ordre \(r\) en \(t_{0}\).
	\end{definition}
	\begin{proposition}
		\label{prop:contacte r és equivalent a tenir les r primeres derivades iguals}
		Siguin \(\alpha\) i \(\beta\) dues corbes en \(I\) parametritzades per l'arc i \(t_{0}\in I\) un punt. Aleshores \(\alpha\) i \(\beta\) tenen contacte d'ordre \(r\) en \(t_{0}\) si i només si
		\[\alpha^{(p)}(t_{0})=\beta^{(p)}(t_{0})\quad\text{i}\quad\alpha^{(r+1)}(t_{0})\neq\beta^{(r+1)}(t_{0})\]
		per a tot \(p\leq r\).
		\begin{proof}
			Suposem que
			\[\alpha^{(p)}(t_{0})=\beta^{(p)}(t_{0})\quad\text{i}\quad\alpha^{(r+1)}(t_{0})\neq\beta^{(r+1)}(t_{0})\]
			per a tot \(p\leq r\). Tenim que, per a tot \(p\leq r\) es satisfà
			\[\alpha^{(p)}(t_{0})=\beta^{(p)}(t_{0}),\]
			o equivalentment
			\begin{equation}
				\label{prop:contacte r és equivalent a tenir les r primeres derivades iguals:eq1}
				\alpha^{(p)}(t_{0})-\beta^{(p)}(t_{0})=0.
			\end{equation}
			Prenem \(n\leq r\), i per la definició de \myref{def:derivada} trobem que
			\begin{align*}
			\alpha^{(n)}(t_{0})-\beta^{(n)}(t_{0})&=\lim_{t\to t_{0}}\left(\frac{\alpha^{(n-1)}(t_{0})-\alpha^{(n-1)}(t)}{t-t_{0}}-\frac{\beta^{(n-1)}(t_{0})-\beta^{(n-1)}(t)}{t-t_{0}}\right) \\
			&=\lim_{t\to t_{0}}\left(\frac{\alpha^{(n-1)}(t_{0})-\alpha^{(n-1)}(t)-\beta^{(n-1)}(t_{0})+\beta^{(n-1)}(t)}{t-t_{0}}\right) \\
			&=\lim_{t\to t_{0}}\left(\frac{\alpha^{(n-1)}(t_{0})-\beta^{(n-1)}(t_{0})}{t-t_{0}}-\frac{\alpha^{(n-1)}(t)-\beta^{(n-1)}(t)}{t-t_{0}}\right)
			\intertext{Ara bé, tenim que \(\alpha^{(n-1)}(t_{0})-\beta^{(n-1)}(t_{0})=0\), i per tant}
			&=\lim_{t\to t_{0}}\frac{\beta^{(n-1)}(t)-\alpha^{(n-1)}(t)}{t-t_{0}}=0,
			\end{align*}
			 i tenim que
			 \[\lim_{t\to t_{0}}\frac{\alpha(t)-\beta(t)}{(t-t_{0})^{p}}=0.\]
			 
			 Considerem ara
			 \[\alpha^{(r+1)}(t_{0})\neq\beta^{(r+1)}(t_{0}).\]
			 Tenim que
			 \[\alpha^{(r+1)}(t_{0})-\beta^{(r+1)}(t_{0})\neq0.\]
			 Ara bé, per la definició de \myref{def:derivada} trobem que
			 \begin{align*}
				 \alpha^{(r+1)}(t_{0})-\beta^{(r+1)}(t_{0})&=\lim_{t\to t_{0}}\left(\frac{\alpha^{(r)}(t_{0})-\alpha^{(r)}(t)}{t-t_{0}}-\frac{\beta^{(r)}(t_{0})-\beta^{(r)}(t)}{t-t_{0}}\right) \\
				 &=\lim_{t\to t_{0}}\frac{\alpha^{(r)}(t_{0})-\alpha^{(r)}(t)-\beta^{(r)}(t_{0})+\beta^{(r)}(t)}{t-t_{0}} \\
				 &=\lim_{t\to t_{0}}\frac{\beta^{(r)}(t)-\alpha^{(r)}(t)}{t-t_{0}}\neq0 \tag{\ref{prop:contacte r és equivalent a tenir les r primeres derivades iguals:eq1}}
			 \end{align*}
			 i per tant
			 \[\lim_{t\to t_{0}}\frac{\alpha(t)-\beta(t)}{(t-t_{0})^{r+1}}\neq0\]
			 i per la definició de \myref{def:contacte entre dues corbes parametritzades per l'arc} tenim que \(\alpha\) i \(\beta\) tenen contacte \(r\).
		\end{proof}
	\end{proposition}
	\subsection{Fórmules de Frenet} % de corbes parametritzades per l'arc
	\begin{proposition}
		\label{prop:unicitat del producte vectorial entre dos vectors}
		Siguin \(\vec{u}\) i \(\vec{v}\) dos vectors de \(\mathbb{R}^{3}\). Aleshores existeix un únic vector \(\vec{w}\) de \(\mathbb{R}^{3}\) tal que per a tot vector \(\vec{x}\) de \(\mathbb{R}^{3}\)
		\[\prodesc{\vec{w}}{\vec{x}}=\det(\vec{u},\vec{v},\vec{x}).\]
		\begin{proof}
			%TODO
		\end{proof}
	\end{proposition}
	\begin{definition}[Producte vectorial]
		\labelname{producte vectorial}\label{def:producte vectorial}
		Siguin \(\vec{u}\) i \(\vec{v}\) dos vectors de \(\mathbb{R}^{3}\) i \(\vec{w}\) el vector de \(\mathbb{R}^{3}\) tal que
		\[\prodesc{\vec{w}}{\vec{x}}=\det(\vec{u},\vec{v},\vec{x}).\]
		Aleshores definim el producte vectorial de \(\vec{u}\) i \(\vec{v}\) com
		\[\vec{u}\prodvec\vec{v}=\vec{w}.\]
		Aquesta definició té sentit per la proposició \myref{prop:unicitat del producte vectorial entre dos vectors}.
	\end{definition}
	\begin{observation}
		\label{obs:fórmula del determinant segons el producte vectorial i el producte escalar}
		\(\prodesc{\vec{u}\prodvec\vec{v}}{\vec{x}}=\det(\vec{u},\vec{v},\vec{x})\).
	\end{observation}
	\begin{proposition}
	\label{prop:el producte vectorial canvia de signe en permutar els vectors}
		Siguin \(\vec{u}\) i \(\vec{v}\) dos vectors de \(\mathbb{R}^{3}\). Aleshores
		\[\vec{u}\prodvec\vec{v}=-\vec{v}\prodvec\vec{u}.\]
		\begin{proof}
			%TODO
		\end{proof}
	\end{proposition}
	\begin{proposition}
		\label{prop:el producte vectorial és zero si i només si els vectors no són linealment independents}
		Siguin \(\vec{u}\) i \(\vec{v}\) dos vectors. Aleshores \(\vec{u}\prodvec\vec{v}\neq0\) si i només si \(\vec{u}\) i \(\vec{v}\) són linealment independents.
		\begin{proof}
			%TODO
		\end{proof}
	\end{proposition}
	\begin{proposition}
		\label{prop:dos vectors linealment independents són perpendiculars al seu producte vectorial}
		\label{prop:el producte vectorial és perpendicular als vectors}
		Siguin \(\vec{u}\) i \(\vec{v}\) dos vectors linealment independents de \(\mathbb{R}^{3}\). Aleshores \(\vec{u}\prodvec\vec{v}\) és perpendicular a \(\vec{u}\) i \(\vec{v}\).
		\begin{proof}
			%TODO
		\end{proof}
	\end{proposition}
	\begin{proposition}
		\label{prop:el determinant de dos vectors linealment independents i el seu producte vectorial és diferent de zero}
		Siguin \(\vec{u}\) i \(\vec{v}\) dos vectors linealment independents de \(\mathbb{R}^{3}\). Aleshores \(\det(\vec{u},\vec{v},\vec{u}\prodvec\vec{u})\neq0\).
		\begin{proof}
			%TODO
		\end{proof}
	\end{proposition}
	\begin{definition}[Orientació d'una base]
		\labelname{orientació d'una base}\label{orientació d'una base}
		\labelname{base positiva}\label{def:base positiva}
		Sigui \((\vec{u},\vec{v},\vec{w})\) una base tal que \(\det(\vec{u},\vec{v},\vec{w})>0\). Aleshores direm que \((\vec{u},\vec{v},\vec{w})\) és una base positiva.
	\end{definition}
	\begin{proposition}
		\label{prop:dos vectors linealment independents i el seu producte vectorial formen una base positiva}
		Siguin \(\vec{u}\) i \(\vec{v}\) dos vectors linealment independents. Aleshores la base \((\vec{u},\vec{v},\vec{u}\prodvec\vec{v})\) és una base positiva.
		\begin{proof}
			%TODO % Sentit per la proposició \ref{prop:el producte vectorial és perpendicular als vectors}.
		\end{proof}
	\end{proposition}
	\begin{proposition}
		\label{prop:fórmula per la derivada del producte vectorial de dues corbes}
		Siguin \(\alpha\) i \(\beta\) dues corbes sobre \(I\) en \(\mathbb{R}^{3}\) diferenciables en \(I\). Aleshores
		\[\frac{\diff(\alpha(t)\prodvec\beta(t))}{\diff t}=\frac{\diff\alpha(t)}{\diff t}\prodvec\beta(t)+\alpha(t)\prodvec\frac{\diff\beta(t)}{\diff t}.\]
		\begin{proof}
			%TODO
		\end{proof}
	\end{proposition}
%	\begin{proposition}
%		Siguin \(\vec{u}\) i \(\vec{v}\) dos vectors de \(\mathbb{R}^{3}\). Aleshores per a tots \(\vec{x}\) i \(\vec{y}\) de \(\mathbb{R}^{3}\) es satisfà
%		\[\prodesc{\vec{u}\prodvec\vec{v}}{\vec{x}\prodvec\vec{y}}=\det\left|\begin{matrix}
%			\prodesc{\vec{u}}{\vec{x}} & \prodesc{\vec{v}}{\vec{x}} \\
%			\prodesc{\vec{u}}{\vec{y}} & \prodesc{\vec{v}}{\vec{y}}
%		\end{matrix}\right|.\]
%		\begin{proof}
%			%TODO
%		\end{proof}
%	\end{proposition}
	
\end{document}
