\documentclass[../Apunts.tex]{subfiles}

\begin{document}
\part{Geometria diferencial}
\chapter{Corbes}
\section{Difeomorfismes de classe \ensuremath{\mathcal{C}^{\infty}}}
	\subsection{Funcions analítiques}
	\begin{definition}[Classe de diferenciabilitat infinita]
		\labelname{classe de diferenciabilitat infinita}\label{def:classe de diferenciabilitat infinita}
		Sigui \(\obert{U}\subseteq\mathbb{R}^{d}\) un obert i
		\begin{align*}
			f\colon\obert{U}&\longrightarrow\mathbb{R}^{m} \\
			x&\longmapsto(f_{1}(x),\dots,f_{m}(x))
		\end{align*}
		una funció tal que per a tot \(i\in\{1,\dots,m\}\) la funció \(f_{i}\) és de classe \(\mathcal{C}^{k}\) per a tot natural \(k\in\mathbb{N}\). Aleshores direm que \(f\) és de classe de diferenciabilitat infinita. També direm que \(f\) és de classe \(\mathcal{C}^{\infty}\).
	\end{definition}
	\begin{definition}[Funció analítica]
		\labelname{funció analítica}\label{def:funció analítica}
		Sigui \(\obert{U}\subseteq\mathbb{R}^{d}\) un obert,
		\begin{align*}
			f\colon\obert{U}&\longrightarrow\mathbb{R}^{m} \\
			x&\longmapsto(f_{1}(x),\dots,f_{m}(x))
		\end{align*}
		una funció de classe \(\mathcal{C}^{\infty}\) i \(x_{0}\in\obert{U}\) un punt tal que existeix una sèrie de potències \(\sum_{n=0}^{\infty}a_{n}(x-x_{0})^{n}\) tal que existeix un entorn \(N_{x_{0}}\subseteq\obert{U}\) satisfent que per a tot \(x\in N_{x_{0}}\) la sèrie de potències \(\sum_{n=0}^{\infty}a_{n}(x-x_{0})^{n}\) convergeix puntualment a \(f(x)\). Aleshores direm que \(f\) és una funció analítica. També direm que \(f\) és de classe \(\mathcal{C}^{\omega}\).
	\end{definition}
	\begin{example}
		\label{ex:els polinomis són funcions analítiques}
		Volem veure que tot polinomi és una funció analítica.
		\begin{solution}
			Prenem un polinomi
			\[p(x)=a_{0}+a_{1}x+a_{2}x^{2}+\dots+a_{n}x^{n}\]
			i un natural \(k\in\mathbb{N}\). Si \(k\leq n\) tenim que % REFS
			\[p^{(k)}(x)=k!a_{k}+(k+1)!a_{k+1}x+\dots+\frac{n!}{(n-k)!}a_{n}x^{n-k},\]
			i si \(k>n\) tenim que
			\[p^{(k)}(x)=0.\]
			Aleshores per la definició de \myref{def:classe de diferenciabilitat infinita} tenim que \(p(x)\) és de classe \(\mathcal{C}^{\infty}\). Observem que
			\[p(x)=\sum_{i=0}^{n}a_{i}x^{i}\]
			i per la definició de \myref{def:sèrie de potències} tenim que \(p(x)\) és una sèrie de potències, i per la definició de \myref{def:convergència puntual} trobem que \(p(x)\) convergeix puntualment a \(\sum_{i=0}^{n}a_{i}x^{i}\) per a tot \(x\in\mathbb{R}\), i per la definició de \myref{def:funció analítica} tenim que \(p(x)\) és una funció analítica per a tot \(x\in\mathbb{R}\).
		\end{solution}
	\end{example}
	\begin{definition}[Difeomorfisme de classe \ensuremath{\mathcal{C}^{\infty}}]
		\labelname{difeomorfisme de classe \ensuremath{\mathcal{C}^{\infty}}}\label{def:difeomorfisme de classe C infinit}\label{def:difeomorfisme de classe de diferenciabilitat infinita}
		Siguin \(\obert{U}\) i \(\obert{V}\) dos oberts de \(\mathbb{R}^{d}\) i \(\Phi\colon\obert{U}\longleftrightarrow\obert{V}\) un difeomorfisme tal que les funcions \(\Phi\) i \(\Phi^{-1}\) siguin de classe \(\mathcal{C}^{\infty}\). Aleshores direm que \(\Phi\) és un difeomorfisme de classe de diferenciabilitat infinita o que \(\Phi\) és un difeomorfisme de classe \(\mathcal{C}^{\infty}\).
		
		Aquesta definició té sentit per la definició de \myref{def:difeomorfisme}.
	\end{definition}
	\subsection{Reparametrització d'una corba}
	\begin{definition}[Corba]
		\labelname{corba}\label{def:corba}
		Siguin \(I\subseteq\mathbb{R}\) un interval i \(\alpha\colon I\longrightarrow\mathbb{R}^{n}\) una funció. Aleshores direm que \(\alpha\) és una corba sobre \(I\).
	\end{definition}
	\begin{example}
		\label{ex:corba que uniex dos punts}
		Volem veure que donats dos punts \(a\) i \(b\) de \(\mathbb{R}^{n}\) existeix una corba contínua que els uneix.
		\begin{solution}
			Observem que la funció contínua % Veure que és contínua per ser un polinomi
			\begin{align*}
				\alpha\colon[0,1]&\longrightarrow\mathbb{R}^{n} \\
				t&\longmapsto a+t(b-a)
			\end{align*}
			satisfà que \(\alpha(0)=a\) i \(\alpha(1)=b\), i per la definició de \myref{def:corba} trobem que és una corba contínua sobre \([0,1]\).
		\end{solution}
	\end{example}
	\begin{definition}[Corba regular]
		\labelname{corba regular}\label{def:corba regular}
		Sigui \(\alpha\) una corba sobre \(I\) tal que per a tot \(t\in I\) es satisfà \(\alpha'(t)\neq\vec{0}\). Aleshores direm que \(\alpha\) és regular.
	\end{definition}
	\begin{definition}[Reparametrització]
		\labelname{reparametrització d'una corba}\label{def:reparametrització d'una corba}
		\labelname{canvi de paràmetre}\label{def:canvi de paràmetre}
		Siguin \(\alpha\) una corba sobre \(I\) i \(h\colon J\longrightarrow I\) un difeomorfisme. Aleshores direm que la funció
		\[\beta(t)=(\alpha\circ h)(t)\]
		és una reparametrització en \(J\) de \(\alpha\) i que \(h\) és el canvi de paràmetre.
	\end{definition}
	\begin{proposition}
		Sigui \(\beta\) una reparametrització en \(J\) d'una corba regular \(\alpha\) sobre \(I\). Aleshores \(\beta\) és una corba regular.
		\begin{proof}
			Per la definició de \myref{def:reparametrització d'una corba} trobem que existeix un canvi de paràmetre \(h:J\longrightarrow I\) tal que \(\beta=(\alpha\circ h)(t)\). Aleshores tenim que
			\begin{align*}
				\beta\colon J&\longrightarrow\mathbb{R}^{n} \\
				t&\longmapsto\alpha(h(t)),
			\end{align*}
			i per la definició de \myref{def:corba} trobem que \(\beta\) és una corba.
			
			Per la definició de \myref{def:canvi de paràmetre} trobem que \(h\) és un difeomorfisme, i per la definició de \myref{def:difeomorfisme} trobem que \(h\) és derivable, i per la \myref{thm:regla de la cadena} trobem que
			\[\beta'(t)=h'(t)\alpha'(h(t)).\]
			
			Com que per hipòtesi \(\alpha\) és regular, per la definició de \myref{def:corba regular} trobem que \(\alpha'(h(t))\neq\vec{0}\) per a tot \(t\in J\). Ara bé, pel \corollari{} \myref{cor:difeomorfisme és equivalent a ser injectiva i tenir diferencial amb determinant no nul} trobem que \(h'(t)\neq\vec{0}\) per a tot \(t\in J\), i per tant tenim que \(\beta'(t)\neq\vec{0}\) per a tot \(t\in J\), i per la definició de \myref{def:corba regular} trobem que \(\beta\) és una corba regular.
		\end{proof}
	\end{proposition}
	\subsection{La longitud d'una corba i el paràmetre arc}
	\begin{definition}[Longitud d'una corba]
		\labelname{longitud d'una corba}\label{def:longitud d'una corba}
		Siguin \(\alpha\) una corba sobre \(I\) i \(a\), \(b\) dos punts de \(I\). Aleshores definim
		\[\Long_{a}^{b}(\alpha)=\int_{a}^{b}\norm{\alpha'(t)}\diff t\]
		com la longitud de \(\alpha\) entre \(a\) i \(b\).
	\end{definition}
	\begin{example}
		\label{ex:longitud d'una corba}
		Volem calcular la longitud de la corba
		\[\alpha(t)=(\cos(t),\sin(t),t)\]
		entre \(0\) i \(x\) per a tot \(x\) positiu.
		\begin{solution}
			Per la definició de \myref{def:longitud d'una corba} trobem que això és
			\begin{align*}
				\Long_{0}^{x}(\alpha)&=\int_{0}^{x}\norm{\alpha'(t)}\diff t \\
				&=\int_{0}^{x}\norm{(-\sin(t),\cos(t),1)}\diff t \\
				&=\int_{0}^{x}\sqrt{\sin^{2}(t)+\cos^{2}(t)+1}\diff t \\
				&=\int_{0}^{x}\sqrt{2}\diff t \\
				&=[\sqrt{2}t]_{0}^{x}=\sqrt{2}x.\qedhere
			\end{align*}
		\end{solution}
	\end{example}
	\begin{proposition}
		Sigui \(\beta\) una reparametrització d'una corba \(\alpha\) en \(I\) amb canvi de paràmetre \(h\). Aleshores per a tot \(a\) i \(b\) de \(I\), amb \(c=h^{-1}(a)\) i \(d=h^{-1}(d)\) tenim
		\[\Long_{a}^{b}(\alpha)=\Long_{c}^{d}(\beta).\]
		\begin{proof}
			Això té sentit per la definició de \myref{def:canvi de paràmetre} i la definició de \myref{def:difeomorfisme}.
			
			Per la definició de \myref{def:longitud d'una corba} trobem que
			\begin{align*}
				\Long_{c}^{d}(\beta)&=\int_{c}^{d}\norm{\beta'(s)}\diff s \\
				&=\int_{c}^{d}\norm{(\alpha\circ h)'(s)}\diff s, \tag{\ref{def:canvi de paràmetre}}
			\end{align*}
			i com que, per la definició de \myref{def:canvi de paràmetre} tenim que \(h\) és un difeomorfisme, per la definició de \myref{def:difeomorfisme} trobem que \(h\) és diferenciable i per la \myref{thm:regla de la cadena} trobem que
			\[\int_{c}^{d}\norm{(\alpha\circ h)'(s)}\diff s=\int_{c}^{d}\norm{h'(s)\alpha'(h(s))}\diff s,\]
			i per la definició de norma tenim que% REF
			\[\int_{c}^{d}\norm{h'(s)\alpha'(h(s))}\diff s=\int_{c}^{d}\abs{h'(s)}\norm{\alpha'(h(s))}\diff s.\]
			
			Ara bé, com que \(h\) és un difeomorfisme tenim per la definició de \myref{def:difeomorfisme} que \(h'\) és contínua, i pel \corollari{} \myref{cor:difeomorfisme és equivalent a ser injectiva i tenir diferencial amb determinant no nul} trobem que \(h'(t)\neq0\) per a tot \(t\in[c,d]\). Aleshores tenim que ha de ser o bé \(h'(t)>0\) per a tot \(t\in[c,d]\) o bé \(h'(t)<0\) per a tot \(t\in[c,d]\).
			
			Comencem suposant que \(h'(t)>0\) per a tot \(t\in[c,d]\). Aleshores tenim que
			\[\int_{c}^{d}\abs{h'(s)}\norm{\alpha'(h(s))}\diff s=\int_{c}^{d}h'(s)\norm{\alpha'(h(s))}\diff s,\]
			i pel Teorema del canvi de variable tenim que %REF
			\begin{align*}
				\int_{c}^{d}h'(s)\norm{\alpha'(h(s))}\diff s&=\int_{h(c)}^{h(d)}\norm{\alpha(t)}\diff t \\
				&=\int_{a}^{b}\norm{\alpha(t)}\diff t=\Long_{a}^{b}(\alpha),
			\end{align*}
			i per tant
			\[\Long_{c}^{d}(\beta)=\Long_{a}^{b}(\alpha).\]
			
			Suposem ara que \(h'(t)<0\) per a tot \(t\in[c,d]\). Aleshores tenim que			\begin{align*}
				\int_{c}^{d}\abs{h'(s)}\norm{\alpha'(h(s))}\diff s&=\int_{c}^{d}-h'(s)\norm{\alpha'(h(s))}\diff s \\
				&=\int_{d}^{c}h'(s)\norm{\alpha'(h(s))}\diff s
			\end{align*}
			i pel Teorema del canvi de variable tenim que %REF
			\begin{align*}
				\int_{d}^{c}h'(s)\norm{\alpha'(h(s))}\diff s&=\int_{h(c)}^{h(d)}\norm{\alpha(t)}\diff t \\
				&=\int_{a}^{b}\norm{\alpha(t)}\diff t=\Long_{a}^{b}(\alpha),
			\end{align*}
			i per tant
			\[\Long_{c}^{d}(\beta)=\Long_{a}^{b}(\alpha).\qedhere\]
		\end{proof}
	\end{proposition}
	\begin{definition}[Funció longitud d'arc]
		\labelname{funció longitud d'arc}\label{def:funció longitud d'arc}
		Siguin \(\alpha\) una corba sobre \(I\) i \(a\in I\) un punt. Aleshores direm que la funció
		\begin{align*}
			\funciolongituddarc_{\alpha}(a)(t)\colon I&\longrightarrow\mathbb{R}\\
			t&\longmapsto\int_{a}^{t}\norm{\alpha'(s)}\diff s
		\end{align*}
		és la funció longitud d'arc de \(\alpha\) amb origen en \(a\).
%		\ref{def:diferencial}
	\end{definition}
	\begin{observation}
		\label{obs:la funció longitud d'arc és creixent}
		Sigui \(\alpha\) una corba sobre \(I\) i \(a\in I\) un punt. Aleshores per a tot \(t\in I\)
		\[\frac{\diff\funciolongituddarc_{\alpha}(a)}{\diff t}(t)\geq0.\]
		\begin{proof}
			Per la definició de \myref{def:funció longitud d'arc} trobem que
			\[\funciolongituddarc_{\alpha}(a)(t)=\int_{a}^{t}\norm{\alpha'(s)}\diff s,\]
			i pel \myref{thm:Teorema Fonamental del Càlcul} tenim que
			\[\frac{\diff\funciolongituddarc_{\alpha}(a)}{\diff t}(t)=\norm{\alpha'(t)}\geq0.\qedhere\] % REFS
		\end{proof}
	\end{observation}
	\begin{proposition}
		\label{prop:la funció longitud arc d'una corba regular és un difeomorfisme}
		Siguin \(\alpha\) una corba regular sobre \(I\) i \(a\in I\) un punt. Aleshores la funció \(\funciolongituddarc_{\alpha}(a)\) és un difeomorfisme. % SOBRE J
		\begin{proof}
			Per la definició de \myref{def:corba regular} trobem que \(\alpha'(t)\neq0\) per a tot \(t\in I\), i pel \myref{thm:Teorema Fonamental del Càlcul} tenim que
			\[\frac{\diff\funciolongituddarc_{\alpha}(a)}{\diff t}(t)=\norm{\alpha'(t)}.\]
			Per tant trobem que % REFS
			\[\frac{\diff\funciolongituddarc_{\alpha}(a)}{\diff t}(t)\neq0,\]
			i per l'observació \myref{obs:diferencial en d=1 és com derivar} i el \corollari{} \myref{cor:difeomorfisme és equivalent a ser injectiva i tenir diferencial amb determinant no nul} tenim que \(\funciolongituddarc_{\alpha}(a)\) és un difeomorfisme, com volíem veure. % SOBRE J
		\end{proof}
	\end{proposition}
	\begin{proposition} % Veure després de definir corba parametritzada per l'arc
		\label{prop:podem trobar una reparametrització amb velocitat unitaria de qualsevol corba regular}
		Sigui \(\alpha\) una corba regular sobre \(I\). Aleshores existeix una reparametrització \(\beta\) de \(\alpha\) tal que per a tot \(t\in I\)
		\[\norm{\beta'(t)}=1.\]
		\begin{proof}
			Per la proposició \ref{prop:la funció longitud arc d'una corba regular és un difeomorfisme} tenim que per a tot \(a\in I\) la funció \(\funciolongituddarc_{\alpha}(a)\) és un difeomorfisme, i per la definició de \myref{def:difeomorfisme} trobem que la funció \(\funciolongituddarc_{\alpha}(a)\) és bijectiva, i pel Teorema \myref{thm:bijectiva iff invertible} trobem que \(\funciolongituddarc_{\alpha}(a)\) és invertible i per la definició de \myref{def:aplicació invertible} tenim que existeix una funció \(t_{a}\) tal que \(t_{a}\) sigui la inversa de \(\funciolongituddarc_{\alpha}(a)\). Considerem
			\[\beta(s)=\alpha(t(s)).\]
			Aleshores tenim que
			\begin{align*} % És positiva pel corol·lari del teorema de la funció inversa?
				\norm{\beta'(s)}&=\norm{\frac{\diff t}{\diff s}\alpha'(s)} \\
				&=\abs{\frac{\diff t}{\diff s}}\norm{\alpha'(s)} \\
				&=\frac{\diff t}{\diff s}(s(t))\frac{\diff s}{\diff t}(t(s))\\
				&=\frac{1}{\frac{\diff s}{\diff t}(t(s))}\frac{\diff s}{\diff t}(t(s))=1. \qedhere
			\end{align*}
		\end{proof}
	\end{proposition}
	\begin{definition}[Corba parametritzada per l'arc]
		\labelname{corba parametritzada per l'arc}\label{def:corba parametritzada per l'arc}
		Sigui \(\alpha\) una corba regular sobre \(I\) tal que per a tot \(t\in I\) es satisfà
		\[\norm{\alpha'(t)}=1.\]
		Aleshores direm que \(\alpha\) està parametritzada per l'arc.
	\end{definition}
	\begin{example} % Veure cas general amb Teorema de la funció implícita
		\label{ex:reparametrització per l'arc del cercle de radi R}
		\label{ex:circumferència de radi R parametritzat per l'arc}
		Volem donar una reparametrització de la corba
		\[\alpha(t)=(R\cos(t),R\sin(t))\]
		tal que estigui parametritzada per l'arc.
		\begin{solution}
			Tenim que
			\[\alpha'(t)=(-R\sin(t),R\cos(t)),\]
			i per tant
			\begin{align*}
				\norm{\alpha'(t)}&=\norm{(-R\sin(t),R\cos(t))} \\
				&=\sqrt{{(-R\sin(t))}^{2}+{(R\cos(t))}^{2}} \\
				&=\sqrt{R^{2}(\sin^{2}(t)+\cos^{2}(t))}=R.
			\end{align*}
			Per la definició de \myref{def:funció longitud d'arc} trobem que
			\begin{align*}
				s(t)&=\int_{0}^{t}\norm{\alpha'(x)}\diff x \\
				&=\int_{0}^{t}R\diff x=Rt.
			\end{align*}
			Prenem ara el canvi de paràmetre
			\[h(s)=\frac{s}{R}.\]
			Tenim que
			\[\beta(s)=(\alpha\circ h)(s)=\left(-R\sin\left(\frac{s}{R}\right),R\cos\left(\frac{s}{R}\right)\right).\]
			
			Tenim que \(\beta\) és una reparametrització de \(\alpha\). Veiem ara que \(\beta\) està parametritzada per l'arc. Tenim que
			\begin{align*}
				\norm{\beta'(s)}&=\norm{\alpha(h(s))'} \\
				&=\norm{h'(s)\alpha'(h(s))} \tag{\ref{thm:regla de la cadena}}\\
				&=\norm{\frac{1}{R}\left(-R\sin\left(\frac{s}{R}\right),R\cos\left(\frac{s}{R}\right)\right)} \\
				&=\norm{\left(-\sin\left(\frac{s}{R}\right),\cos\left(\frac{s}{R}\right)\right)} \\
				&=\sqrt{\sin^{2}\left(\frac{s}{R}\right)+\cos^{2}\left(\frac{s}{R}\right)}=1,
			\end{align*}
			i per la definició de \myref{def:corba parametritzada per l'arc} hem acabat.
		\end{solution}
	\end{example}
	\begin{observation}
		Sigui \(\alpha\) una corba en \(I\) parametritzada per l'arc i \(a\in I\) un punt. Aleshores per a tot \(t\in I\) tenim
		\[\funciolongituddarc_{\alpha}(a)(t)=t-a.\]
		\begin{proof}
			Per la definició de \myref{def:funció longitud d'arc} trobem que
			\[\funciolongituddarc_{\alpha}(a)(t)=\int_{a}^{t}\norm{\alpha'(s)}\diff s,\]
			i per la definició de \myref{def:corba parametritzada per l'arc} tenim que  per a tot \(t\in I\) es satisfà \(\norm{\alpha'(t)}=1\), i per tant
			\begin{align*}
				\funciolongituddarc_{\alpha}(a)(t)&=\int_{a}^{t}\norm{\alpha'(s)}\diff s \\
				&=\int_{a}^{t}\diff s=t-a.\qedhere
			\end{align*}
		\end{proof}
	\end{observation}
	\begin{definition}[Contacte]
		\labelname{contacte}\label{def:contacte entre dues corbes parametritzades per l'arc}
		Siguin \(\alpha\) i \(\beta\) dues corbes en \(I\) parametritzades per l'arc i \(t_{0}\in I\) un punt tals que existeix un \(r\) satisfent
		\[\lim_{t\to t_{0}}\frac{\alpha(t)-\beta(t)}{(t-t_{0})^{p}}=0\quad\text{i}\quad\lim_{t\to t_{0}}\frac{\alpha(t)-\beta(t)}{(t-t_{0})^{r+1}}\neq0\]
		per a tot \(p\leq r\). Aleshores direm que \(\alpha\) i \(\beta\) tenen contacte d'ordre \(r\) en \(t_{0}\).
	\end{definition}
	\begin{proposition}
		\label{prop:contacte r és equivalent a tenir les r primeres derivades iguals}
		Siguin \(\alpha\) i \(\beta\) dues corbes en \(I\) parametritzades per l'arc i \(t_{0}\in I\) un punt. Aleshores \(\alpha\) i \(\beta\) tenen contacte d'ordre \(r\) en \(t_{0}\) si i només si
		\[\alpha^{(p)}(t_{0})=\beta^{(p)}(t_{0})\quad\text{i}\quad\alpha^{(r+1)}(t_{0})\neq\beta^{(r+1)}(t_{0})\]
		per a tot \(p\leq r\).
		\begin{proof}
			Suposem que
			\[\alpha^{(p)}(t_{0})=\beta^{(p)}(t_{0})\quad\text{i}\quad\alpha^{(r+1)}(t_{0})\neq\beta^{(r+1)}(t_{0})\]
			per a tot \(p\leq r\). Tenim que, per a tot \(p\leq r\) es satisfà
			\[\alpha^{(p)}(t_{0})=\beta^{(p)}(t_{0}),\]
			o equivalentment
			\begin{equation}
				\label{prop:contacte r és equivalent a tenir les r primeres derivades iguals:eq1}
				\alpha^{(p)}(t_{0})-\beta^{(p)}(t_{0})=0.
			\end{equation}
			Prenem \(n\leq r\), i per la definició de \myref{def:derivada} trobem que
			\begin{align*}
			\alpha^{(n)}(t_{0})-\beta^{(n)}(t_{0})&=\lim_{t\to t_{0}}\left(\frac{\alpha^{(n-1)}(t_{0})-\alpha^{(n-1)}(t)}{t-t_{0}}-\frac{\beta^{(n-1)}(t_{0})-\beta^{(n-1)}(t)}{t-t_{0}}\right) \\
			&=\lim_{t\to t_{0}}\left(\frac{\alpha^{(n-1)}(t_{0})-\alpha^{(n-1)}(t)-\beta^{(n-1)}(t_{0})+\beta^{(n-1)}(t)}{t-t_{0}}\right) \\
			&=\lim_{t\to t_{0}}\left(\frac{\alpha^{(n-1)}(t_{0})-\beta^{(n-1)}(t_{0})}{t-t_{0}}-\frac{\alpha^{(n-1)}(t)-\beta^{(n-1)}(t)}{t-t_{0}}\right)
			\intertext{Ara bé, tenim que \(\alpha^{(n-1)}(t_{0})-\beta^{(n-1)}(t_{0})=0\), i per tant}
			&=\lim_{t\to t_{0}}\frac{\beta^{(n-1)}(t)-\alpha^{(n-1)}(t)}{t-t_{0}}=0,
			\end{align*}
			 i tenim que
			 \[\lim_{t\to t_{0}}\frac{\alpha(t)-\beta(t)}{(t-t_{0})^{p}}=0.\]
			 
			 Considerem ara
			 \[\alpha^{(r+1)}(t_{0})\neq\beta^{(r+1)}(t_{0}).\]
			 Tenim que
			 \[\alpha^{(r+1)}(t_{0})-\beta^{(r+1)}(t_{0})\neq0.\]
			 Ara bé, per la definició de \myref{def:derivada} trobem que
			 \begin{align*}
				 \alpha^{(r+1)}(t_{0})-\beta^{(r+1)}(t_{0})&=\lim_{t\to t_{0}}\left(\frac{\alpha^{(r)}(t_{0})-\alpha^{(r)}(t)}{t-t_{0}}-\frac{\beta^{(r)}(t_{0})-\beta^{(r)}(t)}{t-t_{0}}\right) \\
				 &=\lim_{t\to t_{0}}\frac{\alpha^{(r)}(t_{0})-\alpha^{(r)}(t)-\beta^{(r)}(t_{0})+\beta^{(r)}(t)}{t-t_{0}} \\
				 &=\lim_{t\to t_{0}}\frac{\beta^{(r)}(t)-\alpha^{(r)}(t)}{t-t_{0}}\neq0 \tag{\ref{prop:contacte r és equivalent a tenir les r primeres derivades iguals:eq1}}
			 \end{align*}
			 i per tant
			 \[\lim_{t\to t_{0}}\frac{\alpha(t)-\beta(t)}{(t-t_{0})^{r+1}}\neq0\]
			 i per la definició de \myref{def:contacte entre dues corbes parametritzades per l'arc} tenim que \(\alpha\) i \(\beta\) tenen contacte \(r\).
		\end{proof}
	\end{proposition}
	\subsection{Producte escalar i producte vectorial}
	\begin{proposition}
		\label{prop:derivada del producte escalar de dues corbes}
		Siguin \(\alpha\) i \(\beta\) dues corbes diferenciables sobre \(I\). Aleshores
		\[\frac{\diff\prodesc{\alpha(t)}{\beta(t)}}{\diff t}=\prodesc{\alpha'(t)}{\beta(t)}+\prodesc{\alpha(t)}{\beta'(t)}\]
		\begin{proof}
			Tenim per la definició de \myref{def:derivada} que
			\[\frac{\diff\prodesc{\alpha(t)}{\beta(t)}}{\diff t}=\lim_{h\to0}\frac{\prodesc{\alpha(t+h)}{\beta(t+h)}-\prodesc{\alpha(t)}{\beta(t)}}{h},\]
			i que
			\[\alpha'(t)=\lim_{h\to0}\frac{\alpha(t+h)-\alpha(t)}{h}\quad\text{i}\quad\beta'(t)=\lim_{h\to0}\frac{\beta(t+h)-\beta(t)}{h}.\]
			Per tant trobem que quan \(h\to0\) tenim que
			\[\alpha(t+h)=\alpha(t)+h\alpha'(t)\quad\text{i}\quad\beta(t+h)=\beta(t)+h\beta'(t).\]
			Per tant tenim que
			\[\frac{\diff\prodesc{\alpha(t)}{\beta(t)}}{\diff t}=\lim_{h\to0}\frac{\prodesc{\alpha(t)+h\alpha'(t)}{\beta(t)+h\beta'(t)}-\prodesc{\alpha(t)}{\beta(t)}}{h},\]
			i per la definició de \myref{def:producte escalar} trobem que
			\begin{multline*}
				\prodesc{\alpha(t)+h\alpha'(t)}{\beta(t)+h\beta'(t)}-\prodesc{\alpha(t)}{\beta(t)}=\\
				=\prodesc{\alpha(t)}{\beta(t)+h\beta'(t)}+\prodesc{h\alpha'(t)}{\beta(t)+h\beta'(t)}-\prodesc{\alpha(t)}{\beta(t)}
			\end{multline*}
			i
			\begin{multline*}
				\prodesc{\alpha(t)+h\alpha'(t)}{\beta(t)+h\beta'(t)}=\\
				=\prodesc{\alpha(t)}{\beta(t)}+\prodesc{\alpha(t)}{h\beta'(t)}+\prodesc{h\alpha'(t)}{\beta(t)+h\beta'(t)}=\\
				=\prodesc{\alpha(t)}{\beta(t)}+\prodesc{\alpha(t)}{h\beta'(t)}+\prodesc{h\alpha'(t)}{\beta(t)}+\prodesc{h\alpha'(t)}{h\beta'(t)}
			\end{multline*}
			i per tant
			\begin{multline*}
				\prodesc{\alpha(t)+h\alpha'(t)}{\beta(t)+h\beta'(t)}-\prodesc{\alpha(t)}{\beta(t)}=\\
				=\prodesc{\alpha(t)}{h\beta'(t)}+\prodesc{h\alpha'(t)}{\beta(t)}+\prodesc{h\alpha'(t)}{h\beta'(t)}=\\
				=h\prodesc{\alpha(t)}{\beta'(t)}+h\prodesc{\alpha'(t)}{\beta(t)}+h^{2}\prodesc{\alpha'(t)}{\beta'(t)}.
			\end{multline*}
			Per tant ens queda que
			\begin{multline*}
				\lim_{h\to0}\frac{\prodesc{\alpha(t)+h\alpha'(t)}{\beta(t)+h\beta'(t)}-\prodesc{\alpha(t)}{\beta(t)}}{h}=\\
				=\lim_{h\to0}\frac{h\prodesc{\alpha(t)}{\beta'(t)}+h\prodesc{\alpha'(t)}{\beta(t)}+h^{2}\prodesc{\alpha'(t)}{\beta'(t)}}{h}=\\
				=\prodesc{\alpha(t)}{\beta'(t)}+\prodesc{\alpha'(t)}{\beta(t)}+\lim_{h\to0}h\prodesc{\alpha'(t)}{\beta'(t)}=\\
				=\prodesc{\alpha(t)}{\beta'(t)}+\prodesc{\alpha'(t)}{\beta(t)},
			\end{multline*}
			i trobem
			\[\frac{\diff\prodesc{\alpha(t)}{\beta(t)}}{\diff t}=\prodesc{\alpha'(t)}{\beta(t)}+\prodesc{\alpha(t)}{\beta'(t)}.\qedhere\]
		\end{proof}
	\end{proposition}
	\begin{proposition}
		\label{prop:unicitat del producte vectorial entre dos vectors}
		Siguin \(\vec{u}\) i \(\vec{v}\) dos vectors de \(\mathbb{R}^{3}\). Aleshores existeix un únic vector \(\vec{w}\) de \(\mathbb{R}^{3}\) tal que per a tot vector \(\vec{x}\) de \(\mathbb{R}^{3}\)
		\[\prodesc{\vec{w}}{\vec{x}}=\det(\vec{u},\vec{v},\vec{x}).\]
		\begin{proof}
			%TODO
		\end{proof}
	\end{proposition}
	\begin{definition}[Producte vectorial]
		\labelname{producte vectorial}\label{def:producte vectorial}
		Siguin \(\vec{u}\) i \(\vec{v}\) dos vectors de \(\mathbb{R}^{3}\) i \(\vec{w}\) el vector de \(\mathbb{R}^{3}\) tal que
		\[\prodesc{\vec{w}}{\vec{x}}=\det(\vec{u},\vec{v},\vec{x}).\]
		Aleshores definim el producte vectorial de \(\vec{u}\) i \(\vec{v}\) com
		\[\vec{u}\prodvec\vec{v}=\vec{w}.\]
		Aquesta definició té sentit per la proposició \myref{prop:unicitat del producte vectorial entre dos vectors}.
	\end{definition}
	\begin{observation}
		\label{obs:fórmula del determinant segons el producte vectorial i el producte escalar}
		\(\prodesc{\vec{u}\prodvec\vec{v}}{\vec{x}}=\det(\vec{u},\vec{v},\vec{x})\).
	\end{observation}
	\begin{proposition}
	\label{prop:el producte vectorial canvia de signe en permutar els vectors}
		Siguin \(\vec{u}\) i \(\vec{v}\) dos vectors de \(\mathbb{R}^{3}\). Aleshores
		\[\vec{u}\prodvec\vec{v}=-\vec{v}\prodvec\vec{u}.\]
		\begin{proof}
			Sigui \(\vec{x}\) un vector de \(\mathbb{R}^{3}\). Considerem
			\[\det(\vec{u},\vec{v},\vec{x}).\]
			Per la definició de \myref{def:determinant d'una matriu} trobem que % REF, no és la def del det
			\begin{equation}
				\label{prop:el producte vectorial canvia de signe en permutar els vectors:eq1}
				\det(\vec{u},\vec{v},\vec{x})=-\det(\vec{v},\vec{u},\vec{x}),
			\end{equation}
			i per l'observació \myref{obs:fórmula del determinant segons el producte vectorial i el producte escalar} tenim que
			\begin{align*}
				\prodesc{\vec{u}\prodvec\vec{v}}{\vec{x}}&=\det(\vec{u},\vec{v},\vec{x}) \\
				&=-\det(\vec{v},\vec{u},\vec{x}) \tag{\ref{prop:el producte vectorial canvia de signe en permutar els vectors:eq1}}\\
				&=-\prodesc{\vec{v}\prodvec\vec{u}}{\vec{x}},
			\end{align*}
			i per tant \(\vec{u}\prodvec\vec{v}=-\vec{v}\prodvec\vec{u}\), com volíem veure. % REF
		\end{proof}
	\end{proposition}
	\begin{proposition}
		\label{prop:el producte vectorial és zero si i només si els vectors no són linealment independents}
		Siguin \(\vec{u}\) i \(\vec{v}\) dos vectors. Aleshores \(\vec{u}\prodvec\vec{v}\neq\vec{0}\) si i només si \(\vec{u}\) i \(\vec{v}\) són linealment independents.
		\begin{proof}
			Suposem que \(\vec{u}\prodvec\vec{v}=\vec{0}\). Aleshores tenim, per a tot vector \(\vec{x}\) de \(\mathbb{R}^{3}\), que
			\[\prodesc{\vec{u}\prodvec\vec{v}}{\vec{x}}=\prodesc{\vec{0}}{\vec{x}},\]
			i per la definició de \myref{def:producte escalar} trobem que
			\[\prodesc{\vec{u}\prodvec\vec{v}}{\vec{x}}=0.\]
			Ara bé, per l'observació \myref{obs:fórmula del determinant segons el producte vectorial i el producte escalar} trobem que per a tot vector \(\vec{x}\) de \(\mathbb{R}^{3}\) tenim
			\[\det(\vec{u},\vec{v},\vec{x})=0,\]
			i per tant tenim que \(\vec{u}\) i \(\vec{v}\) no són linealment independents. %REF
		\end{proof}
	\end{proposition}
	\begin{proposition}
		\label{prop:dos vectors linealment independents són perpendiculars al seu producte vectorial}
		\label{prop:el producte vectorial és perpendicular als vectors}
		Siguin \(\vec{u}\) i \(\vec{v}\) dos vectors linealment independents de \(\mathbb{R}^{3}\). Aleshores \(\vec{u}\prodvec\vec{v}\) és perpendicular a \(\vec{u}\) i \(\vec{v}\).
		\begin{proof}
			%TODO
		\end{proof}
	\end{proposition}
	\begin{proposition}
		\label{prop:el determinant de dos vectors linealment independents i el seu producte vectorial és diferent de zero}
		Siguin \(\vec{u}\) i \(\vec{v}\) dos vectors linealment independents de \(\mathbb{R}^{3}\). Aleshores \(\det(\vec{u},\vec{v},\vec{u}\prodvec\vec{v})\neq0\).
		\begin{proof}
			Per l'observació \myref{obs:fórmula del determinant segons el producte vectorial i el producte escalar} tenim que
			\[\prodesc{\vec{u}\prodvec\vec{v}}{\vec{u}\prodvec\vec{v}}=\det(\vec{u},\vec{v},\vec{u}\prodvec\vec{v}).\]
			Ara bé, com que per hipòtesi els vectors \(\vec{v}\) i \(\vec{u}\) són linealment independents, per la proposició \myref{prop:el producte vectorial és zero si i només si els vectors no són linealment independents} trobem que
			\[\vec{u}\prodvec\vec{v}\neq\vec{0},\]
			i per la definició de \myref{def:producte escalar} trobem que
			\[\prodesc{\vec{u}\prodvec\vec{v}}{\vec{u}\prodvec\vec{v}}\neq0.\]
			Per tant tenim que
			\[\det(\vec{u},\vec{v},\vec{u}\prodvec\vec{v})\neq0.\qedhere\]
		\end{proof}
	\end{proposition}
	\begin{definition}[Orientació d'una base]
		\labelname{orientació d'una base}\label{orientació d'una base}
		\labelname{base positiva}\label{def:base positiva}
		Sigui \((\vec{u},\vec{v},\vec{w})\) una base tal que \(\det(\vec{u},\vec{v},\vec{w})>0\). Aleshores direm que \((\vec{u},\vec{v},\vec{w})\) és una base positiva.
	\end{definition}
	\begin{proposition}
		\label{prop:dos vectors linealment independents i el seu producte vectorial formen una base positiva}
		Siguin \(\vec{u}\) i \(\vec{v}\) dos vectors linealment independents. Aleshores la base \((\vec{u},\vec{v},\vec{u}\prodvec\vec{v})\) és una base positiva.
		\begin{proof}
			Aquest enunciat té sentit per la proposició \ref{prop:el determinant de dos vectors linealment independents i el seu producte vectorial és diferent de zero}.
			
			Per l'observació \myref{obs:fórmula del determinant segons el producte vectorial i el producte escalar} trobem que
			\[\prodesc{\vec{u}\prodvec\vec{v}}{\vec{u}\prodvec\vec{v}}=\det(\vec{u},\vec{v},\vec{u}\prodvec\vec{v}),\]
			i per la definició de \myref{def:producte escalar} trobem que
			\[\prodesc{\vec{u}\prodvec\vec{v}}{\vec{u}\prodvec\vec{v}}>0.\]
			Per tant tenim que
			\[\det(\vec{u},\vec{v},\vec{u}\prodvec\vec{v})>0,\]
			i per la definició de \myref{def:base positiva} hem acabat.
		\end{proof}
	\end{proposition}
	\begin{proposition}[Fórmula de Lagrange]
		\labelname{fórmula de Lagrange}\label{prop:fórmula de Lagrange}
		Siguin \(\vec{u}\) i \(\vec{v}\) dos vectors de \(\mathbb{R}^{3}\). Aleshores per a tots \(\vec{x}\) i \(\vec{y}\) de \(\mathbb{R}^{3}\) es satisfà
		\[\prodesc{\vec{u}\prodvec\vec{v}}{\vec{x}\prodvec\vec{y}}=\det\left[\begin{matrix}
			\prodesc{\vec{u}}{\vec{x}} & \prodesc{\vec{v}}{\vec{x}} \\
			\prodesc{\vec{u}}{\vec{y}} & \prodesc{\vec{v}}{\vec{y}}
		\end{matrix}\right].\]
		\begin{proof}
			%TODO
		\end{proof}
	\end{proposition}
	\begin{proposition}[Fórmula de Leibniz]
		\labelname{}
		\label{prop:fórmula de Leibniz}
		\label{prop:fórmula per la derivada del producte vectorial de dues corbes}
		Siguin \(\alpha\) i \(\beta\) dues corbes sobre \(I\) en \(\mathbb{R}^{3}\) diferenciables en \(I\). Aleshores
		\[\frac{\diff(\alpha(t)\prodvec\beta(t))}{\diff t}=\frac{\diff\alpha(t)}{\diff t}\prodvec\beta(t)+\alpha(t)\prodvec\frac{\diff\beta(t)}{\diff t}.\]
		\begin{proof}
			Per la definició de \myref{def:derivada} tenim que
			\begin{equation}
				\label{prop:fórmula de Leibniz:eq2}
				\frac{\diff(\alpha(t)\prodvec\beta(t))}{\diff t}=\lim_{h\to0}\frac{\alpha(t+h)\prodvec\beta(t+h)-\alpha(t)\prodvec\beta(t)}{h}.
			\end{equation}
			Per la definició de \myref{def:producte vectorial} tenim que per a tot vector \(\vec{x}\) de \(\mathbb{R}^{3}\) es satisfà
			\[\prodesc{\alpha(t+h)\prodvec\beta(t+h)}{\vec{x}}=\det(\alpha(t+h),\beta(t+h),\vec{x}).\]
			De nou per la definició de \myref{def:derivada} trobem que quan \(h\to0\) es satisfà
			\begin{equation}
				\label{prop:fórmula de Leibniz:eq1}
				\alpha(t+h)=\alpha(t)+h\alpha'(t)\quad\text{i}\quad\beta(t+h)=\beta(t)+h\beta'(t),
			\end{equation}
			i per tant tenim que per a tot vector \(\vec{x}\) de \(\mathbb{R}^{3}\) es satisfà
			\begin{align*}
				\prodesc{\alpha(t+h)\prodvec\beta(t+h)}{\vec{x}}&=\det(\alpha(t+h),\beta(t+h),\vec{x}) \\
				&=\det(\alpha(t)+h\alpha'(t),\beta(t)+h\beta'(t),\vec{x}) \tag{\ref{prop:fórmula de Leibniz:eq1}}
			\end{align*}
			i per la definició de \myref{def:determinant d'una matriu} tenim que
			\begin{multline*}
				\det(\alpha(t)+h\alpha'(t),\beta(t)+h\beta'(t),\vec{x})=\\
				=\det(h\alpha'(t),\beta(t)+h\beta'(t),\vec{x})+\det(\alpha(t),\beta(t)+h\beta'(t),\vec{x})
			\end{multline*}
			i de nou per la definició de \myref{def:determinant d'una matriu} trobem que
			\begin{multline*}
				\det(\alpha(t),\beta(t)+h\beta'(t),\vec{x})+\det(h\alpha'(t),\beta(t)+h\beta'(t),\vec{x})=\\
				=\det(\alpha(t),\beta(t),\vec{x})+\det(\alpha(t),h\beta'(t),\vec{x})+\\
				+\det(h\alpha'(t),\beta(t),\vec{x})+\det(h\alpha'(t),h\beta'(t),\vec{x}),
			\end{multline*}
			i per la definició de \myref{def:producte vectorial} això és
			\begin{multline*}
				\prodesc{\alpha(t+h)\prodvec\beta(t+h)}{\vec{x}}=\\
				=\prodesc{\alpha(t)\prodvec\beta(t)}{\vec{x}}+\prodesc{\alpha(t)\prodvec h\beta'(t)}{\vec{x}}+\prodesc{h\alpha'(t)\prodvec\beta(t)}{\vec{x}}+\prodesc{h\alpha'(t)\prodvec h\beta'(t)}{\vec{x}},
			\end{multline*}
			i de nou per la definició de \myref{def:producte vectorial} trobem que
			\begin{multline*}
				\prodesc{\alpha(t+h)\prodvec\beta(t+h)}{\vec{x}}=\\
				=\prodesc{\alpha(t)\prodvec\beta(t)+h\alpha(t)\prodvec\beta'(t)+h\alpha'(t)\prodvec\beta(t)+h^{2}\alpha'(t)\prodvec \beta'(t)}{\vec{x}}.
			\end{multline*}
			
			Ara bé, ens queda que per a tot vector \(\vec{x}\) de \(\mathbb{R}^{3}\) es satisfà
			\[\prodesc{\alpha(t+h)\prodvec\beta(t+h)}{\vec{x}}=\det(\alpha(t+h),\beta(t+h),\vec{x})\]
			i
			\begin{multline*}
				\prodesc{\alpha(t)\prodvec\beta(t)+h\alpha(t)\prodvec\beta'(t)+h\alpha'(t)\prodvec\beta(t)+h^{2}\alpha'(t)\prodvec \beta'(t)}{\vec{x}}=\\=\det(\alpha(t+h),\beta(t+h),\vec{x}),
			\end{multline*}
			i per la proposició \ref{prop:unicitat del producte vectorial entre dos vectors} trobem que
			\[\alpha(t+h)\prodvec\beta(t+h)=\alpha(t)\prodvec\beta(t)+h\alpha(t)\prodvec\beta'(t)+h\alpha'(t)\prodvec\beta(t)+h^{2}\alpha'(t)\prodvec \beta'(t).\]
			Per tant per \eqref{prop:fórmula de Leibniz:eq2} tenim que
			\begin{multline*}
				\frac{\diff(\alpha(t)\prodvec\beta(t))}{\diff t}=\lim_{h\to0}\frac{\alpha(t+h)\prodvec\beta(t+h)-\alpha(t)\prodvec\beta(t)}{h}\\
				=\lim_{h\to0}\frac{h\alpha(t)\prodvec\beta'(t)+h\alpha'(t)\prodvec\beta(t)+h^{2}\alpha'(t)\prodvec \beta'(t)}{h}\\
				=\alpha(t)\prodvec\beta'(t)+\alpha'(t)\prodvec\beta(t)+\lim_{h\to0}\frac{h^{2}\alpha'(t)\prodvec \beta'(t)}{h}\\
				=\alpha(t)\prodvec\beta'(t)+\alpha'(t)\prodvec\beta(t),
			\end{multline*}
			i trobem
			\[\frac{\diff(\alpha(t)\prodvec\beta(t))}{\diff t}=\frac{\diff\alpha(t)}{\diff t}\prodvec\beta(t)+\alpha(t)\prodvec\frac{\diff\beta(t)}{\diff t}.\qedhere\]
		\end{proof}
	\end{proposition}
	\subsection{Fórmules de Frenet}
	\begin{proposition}
		\label{prop:la primera derivada i la segona derivada d'una corba són perpendiculars}
		Sigui \(\alpha\) una corba sobre \(I\) tal que \(\alpha\) és dues vegades diferenciable en un punt \(t_{0}\). Aleshores tenim que \(\alpha'(t_{0})\) i \(\alpha''(t_{0})\) són perpendiculars.
		\begin{proof}
			Per la definició de \myref{def:derivada} trobem que
			\[\alpha''(t_{0})=\lim_{h\to0}\frac{\alpha'(t_{0}+h)+\alpha'(t_{0})}{h},\]
			i per tant tenim que
			\[\prodesc{\alpha'(t_{0})}{\alpha''(t_{0})}=\prodesc[\Big]{\alpha'(t_{0})}{\lim_{h\to0}\frac{\alpha'(t_{0}+h)+\alpha'(t_{0})}{h}},\]
			i per la definició de \myref{def:producte escalar} trobem que
			\begin{align*}
				\prodesc{\alpha'(t_{0})}{\alpha''(t_{0})}	&=\prodesc[\big]{\alpha'(t_{0})}{\lim_{h\to0}\frac{\alpha'(t_{0})}{h}}+\prodesc[\big]{\alpha'(t_{0})}{\lim_{h\to0}\frac{\alpha'(t_{0}+h)}{h}} \\
				&=\prodesc{\alpha'(t_{0})}{0}+\prodesc{\alpha'(t_{0})}{0}=0,
			\end{align*}
			i per la definició de \myref{def:vectors perpendiculars} hem acabat.
		\end{proof}
	\end{proposition}
	\begin{definition}[Curvatura]
		\labelname{curvatura}\label{def:curvatura}
		Sigui \(\alpha\) una corba parametritzada per l'arc sobre \(I\) i dues vegades diferenciable. Aleshores direm que l'aplicació
		\begin{align*}
			\curvatura_{\alpha}\colon I&\longrightarrow\mathbb{R} \\
			t&\longmapsto\norm{\alpha''(t)}
		\end{align*}
		és la curvatura de \(\alpha\).
		
		Si \(\curvatura_{\alpha}(t)\neq0\) direm que \(\alpha\) té curvatura no \nulla{} en \(t\).
	\end{definition}
	\begin{example}
		\label{ex:curvatura de la circumferència}
		Volem calcular la curvatura d'una circumferència.% de radi \(R\).
		\begin{solution}
			Per l'exercici \myref{ex:circumferència de radi R parametritzat per l'arc} tenim que podem parametritzar una circumferència de radi \(R\) per l'arc com
			\[\alpha(t)=\left(-R\sin\left(\frac{t}{R}\right),R\cos\left(\frac{t}{R}\right)\right).\]
			
			Aleshores tenim que
			\[\alpha''(t)=\frac{\diff}{\diff t}\left(-\cos\left(\frac{t}{R}\right),-\sin\left(\frac{t}{R}\right)\right)=\left(\frac{1}{R}\sin\left(\frac{t}{R}\right),\frac{-1}{R}\cos\left(\frac{t}{R}\right)\right),\]
			i trobem que
			\[\norm{\alpha''(t)}=\sqrt{\frac{1}{R^{2}}\sin^{2}\left(\frac{t}{R}\right)+\frac{1}{R^{2}}\cos^{2}\left(\frac{t}{R}\right)}=\frac{1}{R},\]
			i per la definició de \myref{def:curvatura} trobem que \(\curvatura_{\alpha}(s)=\frac{1}{R}\).
		\end{solution}
	\end{example}
	\begin{proposition}
		\label{prop:una corba té curvatura zero si i només si és una recta}
		Sigui \(\alpha\) una corba sobre \(I\) parametritzada per l'arc i dues vegades diferenciable. Aleshores \(\alpha\) és una recta si i només si \(\kappa(t)=0\) per a tot \(s\in I\).
		\begin{proof}
			Comencem veient que la condició és necessària (\(\implica\)). Suposem doncs que \(\alpha\) és una recta sobre \(\mathbb{R}^{n}\). Tenim que existeixen \(K_{1}\) i \(K_{2}\) de \(\mathbb{R}^{n}\) tals que
			\[\alpha(t)=K_{1}s+K_{2}.\]
			Aleshores tenim que \(\alpha''(t)=\vec{0}\), i per la definició de \myref{def:curvatura} trobem que \(\kappa(t)=0\).
			
			Veiem ara que la condició és suficient (\(\implicatper\)). Suposem doncs que \(\kappa(t)=0\) per a tot \(s\in I\). Tenim doncs que \(\alpha''(t)=\vec{0}\), i per tant existeixen \(K_{1}\) i \(K_{2}\) de \(\mathbb{R}^{n}\) tals que \(\alpha'(t)=K_{1}\) i \(\alpha''(t)=K_{1}s+K_{2}\) i trobem que \(\alpha\) és una recta.
		\end{proof}
	\end{proposition}
	\begin{definition}[Normal]
		\labelname{normal}\label{def:normal}
		Sigui \(\alpha\) una corba sobre \(I\) parametritzada per l'arc i dues vegades diferenciable i tal que \(\curvatura_{\alpha}(t)\neq0\) per a tot \(s\in I\). Aleshores direm que l'aplicació
		\begin{align*}
			\normal_{\alpha}\colon I&\longrightarrow\mathbb{R} \\
			t&\longmapsto\frac{\alpha''(t)}{\norm{\alpha''(t)}}
		\end{align*}
		és la normal de \(\alpha\).
	\end{definition}
	\begin{observation}
		\label{obs:la normal a una corba és unitària}
		Tenim que \(\norm{\normal_{\alpha}(t)}=1\).
	\end{observation}
	\begin{definition}[Tangent]
		\labelname{tangent}\label{def:tangent}
		Sigui \(\alpha\) una corba sobre \(I\) parametritzada per l'arc. Aleshores direm que l'aplicació
		\begin{align*}
			\tangent_{\alpha}\colon I&\longrightarrow\mathbb{R} \\
			t&\longmapsto\alpha'(t)
		\end{align*}
		és la tangent de \(\alpha\).
	\end{definition}
	\begin{observation}
		\label{obs:la derivada de la tangent és la curvatura per la normal}
		Tenim que \(\tangent_{\alpha}'(t)=\curvatura_{\alpha}(t)\normal_{\alpha}(t)\).
	\end{observation}
	\begin{proposition}
		\label{prop:la tangent i la normal d'una corba amb curvatura no nula són perpendiculars}
		Sigui \(\alpha\) una corba sobre \(I\) amb curvatura no \nulla{} i \(t\in I\) un real. Aleshores els vectors \(\tangent_{\alpha}(t)\) i \(\normal_{\alpha}(t)\) són perpendiculars.
		\begin{proof}
			Considerem
			\[\prodesc{\tangent_{\alpha}(t)}{\normal_{\alpha}(t)}.\]
			Per la definició de \myref{def:tangent} i \myref{def:normal} trobem que
			\begin{align*}
				\prodesc{\tangent_{\alpha}(t)}{\normal_{\alpha}(t)}&=\prodesc[\big]{\alpha'(t)}{\frac{\alpha''(t)}{\norm{\alpha''(t)}}} \\
				&=\frac{1}{\norm{\alpha''(t)}}\prodesc{\alpha'(t)}{\alpha''(t)} \tag{\ref{def:producte escalar}} \\
				&=0. \tag{\ref{prop:la primera derivada i la segona derivada d'una corba són perpendiculars}}
			\end{align*}
			Aleshores per la definició de \myref{def:vectors perpendiculars} trobem que \(\tangent_{\alpha}(t)\) i \(\normal_{\alpha}(t)\) són perpendiculars.
		\end{proof}
	\end{proposition}
	\begin{proposition}
		\label{prop:circumferència osculadora}
		Siguin \(\alpha\) una corba parametritzada per l'arc sobre \(I\) i \(t_{0}\in I\) un punt tal que \(\alpha''(t_{0})\neq\vec{0}\). Aleshores existeix una única circumferència \(\beta\) de \(\mathbb{R}^{3}\) tal que \(\alpha\) i \(\beta\) tenen contacte d'ordre \(2\) en \(\alpha(t_{0})\), que té radi \(\radicircosculadora_{\alpha}(s_{0})=\frac{1}{\curvatura_{\alpha}(t_{0})}\) i és de la forma
		\[\beta(t)=Q+\radicircosculadora_{\alpha}(t_{0})\left(-N(t_{0})\cos\left(\frac{t}{\radicircosculadora_{\alpha}(t_{0})}\right)+T(t_{0})\sin\left(\frac{t}{\radicircosculadora_{\alpha}(t_{0})}\right)\right).\]
		\begin{proof}
			%TODO
%			Sigui \((\vec{u}_{1},\vec{u}_{2})\) una base ortonormal de \(\mathbb{R}^{2}\). Aleshores les circumferències de \(\mathbb{R}^{3}\) de radi \(R\) i centre \(Q\) parametritzades per l'arc són de la forma
%			\[\beta(t)=Q+R\left(\cos\left(\frac{t}{R}\right),\sin\left(\frac{t}{R}\right)\right).\]
%			
%			Per la proposició \myref{prop:contacte r és equivalent a tenir les r primeres derivades iguals} tenim que
%			\[\alpha(t_{0})=\beta(t_{0}),\quad\alpha'(t_{0})=\beta'(t_{0})\quad\text{i}\quad\alpha''(t_{0})=\beta''(t_{0}).\]
%			Tenim que
%			\[\beta'(t)=\]
		\end{proof}
	\end{proposition}
	\begin{definition}[Circumferència osculadora]
		\labelname{circumferència osculadora}\label{def:circumferència osculadora}
		Sigui \(\alpha\) una corba parametritzada per l'arc sobre \(I\) i \(t_{0}\in I\) un punt tal que \(\alpha''(t_{0})\neq\vec{0}\) i \(\radicircosculadora_{\alpha}(t_{0})=\frac{1}{\curvatura_{\alpha}(t_{0})}\). Aleshores direm que la circumferència
		\[\beta(t)=Q+\radicircosculadora_{\alpha}(t_{0})\left(-N(t_{0})\cos\left(\frac{t}{\radicircosculadora_{\alpha}(t_{0})}\right)+T(t_{0})\sin\left(\frac{t}{\radicircosculadora_{\alpha}(t_{0})}\right)\right).\]
		és la circumferència osculadora de \(\alpha\) en \(t_{0}\). Denotarem per \(\radicircosculadora_{\alpha}(t_{0})=\frac{1}{\curvatura_{\alpha}(t_{0})}\) el radi de la circumferència osculadora.
		
		Aquesta definició té sentit per la proposició \myref{prop:circumferència osculadora}.
	\end{definition}
	\begin{definition}[Binormal]
		\labelname{binormal}\label{def:binormal}
		Sigui \(\alpha\) una corba amb curvatura no \nulla{}. Aleshores definim
		\[\binormal_{\alpha}(t)=\tangent_{\alpha}(t)\prodvec\normal_{\alpha}(t)\]
		com la binormal de \(\alpha\).
	\end{definition}
	\begin{proposition}
		\label{prop:triedre de Frenet}
		Sigui \(\alpha\) una corba sobre \(I\) amb curvatura no \nulla{}. Aleshores \((\tangent_{\alpha}(t),\normal_{\alpha}(t),\binormal_{\alpha}(t))\) és una base ortonormal i positiva de \(\mathbb{R}^{3}\).
		\begin{proof}
			Per la proposició \myref{prop:la tangent i la normal d'una corba amb curvatura no nula són perpendiculars} trobem que \(\tangent_{\alpha}(t)\) i \(\normal_{\alpha}(t)\) són perpendiculars. Aleshores per la definició de \myref{def:binormal} trobem que \(\binormal_{\alpha}(t)=\tangent_{\alpha}(t)\prodvec\normal_{\alpha}(t)\) i per la proposició \myref{prop:dos vectors linealment independents i el seu producte vectorial formen una base positiva} tenim que la base \((\tangent_{\alpha}(t),\normal_{\alpha}(t),\binormal_{\alpha}(t))\) és positiva.
			
			Per la definició de \myref{def:producte vectorial} tenim que per a tot \(\vec{x}\) de \(\mathbb{R}^{3}\) es satisfà
			\[\prodesc{\binormal_{\alpha}(t)}{\vec{x}}=\det(\tangent_{\alpha}(t),\normal_{\alpha}(t),\vec{x}).\]
			Si prenem \(\vec{x}\) tal que \((\tangent_{\alpha}(t),\normal_{\alpha}(t),\vec{x})\) sigui una base ortonormal de \(\mathbb{R}^{3}\) trobem que
			\[\det(\tangent_{\alpha}(t),\normal_{\alpha}(t),\vec{x})=1,\]
			i per tant
			\[\prodesc{\binormal_{\alpha}(t)}{\vec{x}}=1,\]
			i tenim que \(\norm{\binormal_{\alpha}(t)}=1\). Aleshores per l'observació \myref{obs:la normal a una corba és unitària} i la definició de \myref{def:corba parametritzada per l'arc} tenim que \(\norm{\tangent_{\alpha}(t)}=1\) i que \(\norm{\normal_{\alpha}(t)}=1\), i per tant \((\tangent_{\alpha}(t),\normal_{\alpha}(t),\binormal_{\alpha}(t))\) és una base ortonormal i positiva de \(\mathbb{R}^{3}\). % REF
		\end{proof}
	\end{proposition}
	\begin{definition}[Triedre de Frenet]
		\labelname{triedre de Frenet}\label{def:triedre de Frenet}
		Sigui\(\alpha\) una corba amb curvatura no \nulla{}. Aleshores direm que la base \((\tangent_{\alpha}(t),\normal_{\alpha}(t),\binormal_{\alpha}(t))\) és el triedre de Frenet.
		
		Aquesta definició té sentit per la proposició \myref{prop:triedre de Frenet}.
	\end{definition}
	\begin{proposition}
		\label{prop:torsió}
		Sigui \(\alpha\) una corba parametritzada per l'arc amb curvatura no \nulla{}. Aleshores existeix un \(\torsio_{\alpha}(t)\in\mathbb{R}\) tal que
		\[\binormal'_{\alpha}(t)=\torsio_{\alpha}(t)\normal(t).\]
		\begin{proof}
			Tenim per la proposició \myref{prop:triedre de Frenet} que \((\tangent_{\alpha}(t),\normal_{\alpha}(t),\binormal_{\alpha}(t))\) és una base ortonormal de \(\mathbb{R}^{3}\), i per tant tenim que %REF
			\[\binormal'_{\alpha}(t)=\prodesc{\binormal'_{\alpha}(t)}{\tangent_{\alpha}(t)}\tangent_{\alpha}(t)+\prodesc{\binormal'_{\alpha}(t)}{\normal_{\alpha}(t)}\normal_{\alpha}(t)+\prodesc{\binormal'_{\alpha}(t)}{\binormal_{\alpha}(t)}\binormal_{\alpha}(t).\]
			
			Tenim que \(\prodesc{\binormal_{\alpha}(t)}{\binormal_{\alpha}(t)}=1\), per tant trobem que
			\[\frac{\diff\prodesc{\binormal_{\alpha}(t)}{\binormal_{\alpha}(t)}}{\diff t}=0,\]
			i per la proposició \myref{prop:fórmula de Leibniz} trobem que
			\[\prodesc{\binormal'_{\alpha}}{\binormal_{\alpha}}+\prodesc{\binormal_{\alpha}}{\binormal'_{\alpha}}=0,\]
			i per la definició de \myref{def:producte escalar} tenim que
			\[2\prodesc{\binormal'_{\alpha}(t)}{\binormal_{\alpha}(t)}=0,\]
			i per tant
			\[\prodesc{\binormal'_{\alpha}(t)}{\binormal_{\alpha}(t)}=0.\]
			
			Per la proposició \ref{prop:triedre de Frenet} trobem que
			\[\prodesc{\binormal_{\alpha}(t)}{\tangent_{\alpha}(t)}=0,\]
			i per tant
			\[\frac{\diff\prodesc{\binormal_{\alpha}(t)}{\tangent_{\alpha}(t)}}{\diff t}=0,\]
			i de nou per la proposició \myref{prop:fórmula de Leibniz} tenim que
			\[\frac{\diff\prodesc{\binormal_{\alpha}(t)}{\tangent_{\alpha}(t)}}{\diff t}=\prodesc{\binormal'_{\alpha}(t)}{\tangent_{\alpha}(t)}+\prodesc{\binormal_{\alpha}(t)}{\tangent'_{\alpha}(t)}.\]
			Ara bé, tenim que
			\begin{align*}
				\prodesc{\binormal_{\alpha}(t)}{\tangent'_{\alpha}(t)}&=\prodesc{\binormal_{\alpha}(t)}{\curvatura_{\alpha}(t)\normal_{\alpha}(t)} \tag{\ref{obs:la derivada de la tangent és la curvatura per la normal}} \\
				&=0, \tag{\ref{prop:triedre de Frenet}}
			\end{align*}
			i per tant
			\[\prodesc{\binormal'_{\alpha}(t)}{\tangent_{\alpha}(t)}=0,\]
			i per tant, denotant \(\torsio_{\alpha}(t)=\prodesc{\binormal'_{\alpha}(t)}{\normal_{\alpha}(t)}\) trobem que
			\[\binormal'_{\alpha}(t)=\torsio_{\alpha}(t)\normal_{\alpha}(t).\qedhere\]
		\end{proof}
	\end{proposition}
	\begin{definition}[Torsió]
		\labelname{torsió}\label{def:torsió}
		Sigui \(\alpha\) una corba parametritzada per l'arc amb curvatura no \nulla{} i \(\torsio_{\alpha}(t)\) un real tal que
		\[\binormal'_{\alpha}(t)=\torsio_{\alpha}(t)\normal(t).\]
		Aleshores direm que \(\torsio_{\alpha}(t)\) és la torsió de \(\alpha\).
		
		Aquesta definició té sentit per la proposició \myref{prop:torsió}.
	\end{definition}
	% FER Exemple calcular torsió?
	\begin{proposition}[Fórmules de Frenet]
		\labelname{fórmules de Frenet}\label{prop:fórmules de Frenet per corbes parametritzades per l'arc}
		Sigui \(\alpha\) una corba parametritzada per l'arc amb curvatura no \nulla{}. Aleshores
		\[\begin{bmatrix}
			\tangent'_{\alpha}(t) \\
			\normal'_{\alpha}(t) \\
			\binormal'_{\alpha}(t)
		\end{bmatrix}=
		\begin{bmatrix}
			0 & \curvatura_{\alpha}(t) & 0 \\
			-\curvatura_{\alpha}(t) & 0 & -\torsio_{\alpha}(t) \\
			0 & \torsio_{\alpha}(t) & 0
		\end{bmatrix}
		\begin{bmatrix}
			\tangent_{\alpha}(t) \\
			\normal_{\alpha}(t) \\
			\binormal_{\alpha}(t)
		\end{bmatrix}.\]
		\begin{proof}
			
			Per l'observació \myref{obs:la derivada de la tangent és la curvatura per la normal} trobem que
			\begin{equation}
				\label{prop:fórmules de Frenet:eq1}
				\tangent_{\alpha}'(t)=\curvatura_{\alpha}(t)\normal_{\alpha}(t),
			\end{equation}
			i per la definició de \myref{def:torsió} trobem que
			\begin{equation}
				\label{prop:fórmules de Frenet:eq2}
				\binormal'_{\alpha}(t)=\torsio_{\alpha}(t)\normal(t).
			\end{equation}
			
			Per la definició de \myref{def:binormal} trobem que
			\[\binormal_{\alpha}(t)=\tangent_{\alpha}(t)\prodvec\normal_{\alpha}(t),\]
			i per tant % REF
			\[\normal_{\alpha}(t)=\binormal_{\alpha}(t)\prodvec\tangent_{\alpha}(t).\]
			Aleshores per la proposició \myref{prop:fórmula de Leibniz} tenim que
			\[\normal'_{\alpha}(t)=\binormal'_{\alpha}(t)\prodvec\tangent_{\alpha}(t)+\binormal_{\alpha}(t)\prodvec\tangent'_{\alpha}(t).\]
			Ara bé, per \eqref{prop:fórmules de Frenet:eq1} i \eqref{prop:fórmules de Frenet:eq2} trobem que
			\begin{align*}
				\normal'_{\alpha}(t)&=\torsio_{\alpha}(t)\normal_{\alpha}(t)\prodvec\tangent_{\alpha}(t)+\binormal_{\alpha}(t)\prodvec\curvatura_{\alpha}(t)\normal_{\alpha}(t) \\
				&=\torsio_{\alpha}(t)\big(\normal_{\alpha}(t)\prodvec\tangent_{\alpha}(t)\big)+\curvatura_{\alpha}(t)\big(\binormal_{\alpha}(t)\prodvec\normal_{\alpha}(t)\big) \\ % REF
				&=\torsio_{\alpha}(t)\binormal_{\alpha}(t)+\curvatura_{\alpha}(t)\big(\binormal_{\alpha}(t)\prodvec\normal_{\alpha}(t)\big) \tag{\ref{def:binormal}} \\
				&=-\torsio_{\alpha}(t)\binormal_{\alpha}(t)-\curvatura_{\alpha}(t)\tangent_{\alpha}(t). \tag{\ref{def:binormal}}
			\end{align*}
			
			Per tant ens queda
			\begin{align*}
				\tangent_{\alpha}'(t)&=\curvatura_{\alpha}(t)\normal_{\alpha}(t) \\
				\normal'_{\alpha}(t)&=-\torsio_{\alpha}(t)\binormal_{\alpha}(t)-\curvatura_{\alpha}(t)\tangent_{\alpha}(t) \\
				\binormal'_{\alpha}(t)&=\torsio_{\alpha}(t)\normal(t)
			\end{align*}
			i per la definició de \myref{def:producte de matrius} hem acabat.
		\end{proof}
	\end{proposition}
	\begin{proposition}
		\label{prop:condició equivalent per que la reparametrització per l'arc d'una corba tingui curvatura no nula}
		Sigui \(\alpha\) una corba amb curvatura no \nulla{} i \(h\) un canvi de paràmetre de \(\alpha\) tal que \(\tilde{\alpha}=\alpha\circ h\) estigui parametritzada per l'arc. Aleshores \(\tilde{\alpha}\) té curvatura no \nulla{} si i només si \(\alpha'\prodvec\alpha''\neq\vec{0}\).
		\begin{proof}
			Tenim que \(\tilde{\alpha}=\alpha\circ h\), i per la proposició \myref{prop:fórmula de Leibniz} i la \myref{thm:regla de la cadena} tenim que
			\[\alpha'=h'(\tilde{\alpha}'\circ h)\]
			i
			\[\alpha''=h''(\tilde{\alpha}'\circ h)+(h')^{2}(\tilde{\alpha}''\circ h).\]
			Per tant tenim que
			\begin{align*}
				\alpha'\prodvec\alpha''&=(h'(\tilde{\alpha}'\circ
				 h))\prodvec(h''(\tilde{\alpha}'\circ h)+(h')^{2}(\tilde{\alpha}''\circ h)) \\
				 &=(h'(\tilde{\alpha}'\circ h))\prodvec(h''(\tilde{\alpha}'\circ h))+(h'(\tilde{\alpha}'\circ h))\prodvec((h')^{2}(\tilde{\alpha}''\circ h)) \\
				 &=h'h''(\tilde{\alpha}'\circ h)\prodvec(\tilde{\alpha}'\circ h)+(h')^{2}(h'(\tilde{\alpha}'\circ h))\prodvec(\tilde{\alpha}''\circ h) \\
				&=(h')^{2}(\tilde{\alpha}'\circ h)\prodvec(\tilde{\alpha}''\circ h).
			\end{align*}
			
			Ara bé, per la definició de \myref{def:canvi de paràmetre} tenim que \(h\) és un difeomorfisme, i per la definició de \myref{def:difeomorfisme} trobem que \(h\neq0\). Tenim també que \(h''\neq0\), i per tant trobem que
			\[\alpha'\prodvec\alpha''=\vec{0}\Sii(\tilde{\alpha}'\circ h)\prodvec(\tilde{\alpha}''\circ h)=\vec{0},\]
			i com que, per hipòtesi, \(\tilde{\alpha}\) està parametritzada per l'arc, per la definició de \myref{def:corba parametritzada per l'arc} tenim que \(\tilde{\alpha}'\circ h\neq\vec{0}\), i per la proposició \myref{prop:la primera derivada i la segona derivada d'una corba són perpendiculars} trobem que \((\tilde{\alpha}'\circ h)\prodvec(\tilde{\alpha}''\circ h)=\vec{0}\) si i només si \(\tilde{\alpha}''=\vec{0}\), i per tant
			\[\alpha'\prodvec\alpha''=\vec{0}\Sii\norm{\tilde{\alpha}''}=\vec{0}.\qedhere\]
		\end{proof}
	\end{proposition}
	\begin{definition}[Curvatura per una reparametrització]
		\labelname{curvatura}\label{def:curvatura  per una reparametrització}
		Sigui \(\alpha\) una corba amb curvatura no \nulla{} i \(\alpha'\prodvec\alpha''=\vec{0}\) i \(h\) un canvi de paràmetre de \(\alpha\) tal que \(\tilde{\alpha}=\alpha\circ h\) estigui parametritzada per l'arc. Aleshores direm que
		\[\curvatura_{\alpha}=\curvatura_{\tilde{\alpha}}\circ h^{-1}\]
		és la curvatura de \(\alpha\).
		
		Això té sentit per la definició de \myref{def:curvatura} i la proposició \myref{prop:condició equivalent per que la reparametrització per l'arc d'una corba tingui curvatura no nula}.
	\end{definition}
	\begin{definition}[Torsió per una reparametrització]
		\labelname{torsió}\label{def:torsió  per una reparametrització}
		Sigui \(\alpha\) una corba amb curvatura no \nulla{} i \(h\) un canvi de paràmetre de \(\alpha\) tal que \(\tilde{\alpha}=\alpha\circ h\) estigui parametritzada per l'arc. Aleshores direm que
		\[\torsio_{\alpha}=\torsio_{\tilde{\alpha}}\circ h^{-1}\]
		és la curvatura de \(\alpha\).
		
		Aquesta definició té sentit per la definició de \myref{def:torsió}.
	\end{definition}
	\begin{definition}[Rapidesa]
		\labelname{rapidesa}\label{def:rapidesa}
		Sigui \(\alpha\) una corba regular. Aleshores direm que
		\[\rapidesa_{\alpha}(t)=\norm{\alpha'(t)}\]
		és la rapidesa de \(\alpha\).
	\end{definition}
	\begin{proposition}[Fórmules de Frenet]
		\labelname{fórmules de Frenet}\label{prop:fórmules de Frenet}
		Sigui \(\alpha\) una corba amb curvatura no \nulla{} i tal que \(\alpha'\prodvec\alpha''=\vec{0}\). Aleshores
		\[\begin{bmatrix}
			\tangent_{\alpha}'(t) \\
			\normal_{\alpha}'(t) \\
			\binormal_{\alpha}'(t)
		\end{bmatrix}
		=\rapidesa_{\alpha}(t)\begin{bmatrix}
			0 & \curvatura_{\alpha}(t) & 0 \\
			-\curvatura_{\alpha}(t) & 0 & -\curvatura_{\alpha}(t) \\
			0 & \curvatura_{\alpha}(t) & 0
		\end{bmatrix}
		\begin{bmatrix}
			\tangent_{\alpha}(t) \\
			\normal_{\alpha}(t) \\
			\binormal_{\alpha}(t)
		\end{bmatrix}.\]
		\begin{proof}
			Sigui \(h\) una reparametrització de \(\alpha\) tal que \(\tilde{\alpha}=\alpha\circ h\) estigui parametritzada per l'arc.
			
			Ai haig de tornar a definir les coses aquestes per corbes no parametritzades per l'arc. Quina mandra. Imagineu una demostració patrocinada per \myref{thm:regla de la cadena} i les \myref{prop:fórmules de Frenet per corbes parametritzades per l'arc}. Us prometo que tot quadra, com volíem veure.
		\end{proof}
	\end{proposition}
	\subsection{Teorema Fonamental de la teoria local de corbes}
	\begin{note} %TODO
		Es venen moltes coses que moure a geometria lineal.
	\end{note}
	\begin{example}[Grup ortogonal]
		\labelname{}\label{ex:grup ortogonal}
		Volem veure que el conjunt
		\[\GrupOrtogonal(n)=\{A\in\matrius_{n\times n}(\mathbb{R})\mid\text{per a tot }\vec{u},\vec{v}\in\matrius_{n\times1}(\mathbb{R})\text{ tenim }\prodesc{A\vec{u}}{A\vec{v}}=\prodesc{\vec{u}}{\vec{v}}\}\]
		amb el producte de matrius és un grup.
		\begin{solution}
			%TODO
			% Moure a estructures
		\end{solution}
	\end{example}
	\begin{example}[Grup especial ortogonal]
		\labelname{}\label{ex:grup especial ortogonal}
		Volem veure que
		\[\GrupEspecialOrtogonal(n)=\{A\in\GrupOrtogonal\mid\det(A)=1\}\]
		és un subgrup de \(\GrupOrtogonal(n)\).
		\begin{solution}
			%TODO
			% Moure a estructures
		\end{solution}
	\end{example}
	\begin{proposition}
		\label{prop:els valors propis d'una matriu ortogonal són -1 o 1}
		Sigui \(A\in\GrupOrtogonal(n)\) una matriu. Aleshores els valors propis reals de \(A\) són \(-1\) ó \(1\).
		\begin{proof}
			%TODO
		\end{proof}
	\end{proposition}
	\begin{proposition}
		\label{prop:caracterització de les matrius ortogonals 2x2}
		Es satisfà
		\[\GrupEspecialOrtogonal(2)=\left\{\begin{bmatrix}
			\cos(t) & -\sin(t) \\
			\sin(t) & \cos(t)
		\end{bmatrix}\mid t\in\mathbb{R}\right\}\]
		i
		\[\GrupOrtogonal(2)\setminus\GrupEspecialOrtogonal(2)=\left\{\begin{bmatrix}
			\cos(t) & \sin(t) \\
			\sin(t) & -\cos(t)
		\end{bmatrix}\mid t\in\mathbb{R}\right\}.\]
		\begin{proof}
			%TODO
		\end{proof}
	\end{proposition}
	\begin{proposition}
		Sigui \(A\in\GrupEspecialOrtogonal(3)\) una matriu. Aleshores existeixen una base ortonormal \(\base{B}\) de \(\mathbb{R}^{3}\) i un \(t\in\mathbb{R}\) tals que
		\[M(\base{B},\basecanonica)AM(\basecanonica,\base{B})=\begin{bmatrix}
			\pm1 & 0 & 0 \\
			0 & \cos(t) & \sin(t) \\
			0 & \sin(t) & -\cos(t)
		\end{bmatrix}.\]
		\begin{proof}
			%TODO
		\end{proof}
	\end{proposition}
	\begin{proposition}
		\label{prop:forma matricial de les aplicacions que conserven distàncies}
		Sigui \(f\colon\mathbb{R}^{n}\longrightarrow\mathbb{R}^{n}\) una aplicació conserva distàncies. Aleshores existeixen una matriu \(A\in\GrupOrtogonal(n)\) i un \(\vec{C}\in\mathbb{R}^{n}\) tals que
		\[f(\vec{v})=A\vec{v}+\vec{C}.\]
		\begin{proof}
			%TODO
		\end{proof}
	\end{proposition}
	\begin{proposition}
		\label{prop:derivada del producte d'una matriu per una corba}
		Siguin \(\alpha\) una corba regular i \(A\in\matrius_{n}\) una matriu. Aleshores
		\[\frac{\diff(A\alpha(t))}{\diff t}=A\alpha'(t).\]
		\begin{proof}
			%TODO
		\end{proof}
	\end{proposition}
	\begin{proposition}
		\label{prop:les matrius especials ortogonals conserven el producte vectorial}
		Siguin \(A\in\GrupEspecialOrtogonal(n)\) una matriu i \(\vec{u}\), \(\vec{v}\) dos vectors de \(\mathbb{R}^{n}\). Aleshores
		\[A(\vec{u}\prodvec\vec{v})=(A\vec{u})\prodvec(A\vec{v}).\]
		\begin{proof}
			%TODO
		\end{proof}
	\end{proposition}
	\begin{corollary}
		\label{cor:una corba parametritzada per l'arc i la seva imatge per una aplicació que conserva les distàncies són equivalents}
		Siguin \(\alpha\) una corba parametritzada per l'arc, \(A\in\GrupEspecialOrtogonal(3)\) una matriu, \(\vec{C}\) un vector de \(\mathbb{R}^{3}\) i \(\beta\) una corba tal que
		\[\beta(t)=A\alpha(t)+\vec{C}.\]
		Aleshores \(\beta(t)\) està parametritzada per l'arc i es satisfà
		\[\tangent_{\beta}=A\tangent_{\alpha},\quad\normal_{\beta}=A\normal_{\alpha},\quad\binormal_{\beta}=A\binormal_{\alpha},\quad\curvatura_{\beta}=A\curvatura_{\alpha}\quad\text{i}\quad\torsio_{\beta}=A\torsio_{\alpha}.\]
		\begin{proof}
			%TODO
		\end{proof}
	\end{corollary}
	\begin{lemma}
		\label{lemma:Teorema Fonamental de la teoria local de corbes}
		Siguin \(\alpha\) i \(\beta\) dues corbes. Aleshores la relació
		\[\alpha\sim\beta \Sii \text{Existeixen }A\in\GrupEspecialOrtogonal(3)\text{ i }\vec{C}\in\mathbb{R}^{3}\text{ tals que }\alpha=A\beta+\vec{C}\]
		és una relació d'equivalència.
		\begin{proof}
			%TODO
		\end{proof}
	\end{lemma}
	\begin{theorem}[Teorema Fonamental de la teoria local de corbes]
		\labelname{Teorema Fonamental de la teoria local de corbes}\label{thm:Teorema Fonamental de la teoria local del corbes}
		Siguin \(\curvatura(t)\) i \(\torsio(t)\) dues funcions diferenciables sobre \(I\), amb \(\curvatura(t)>0\) per a tot \(t\in I\). Aleshores existeix una única corba \(\alpha\) sobre \(I\) parametritzada per l'arc satisfent \(\curvatura_{\alpha}(t)=\curvatura(t)\) i \(\torsio_{\alpha}(t)=\torsio(t)\), llevat d'equivalència. % Reescriure la part de llevat d'equivalència
		\begin{proof}
			%TODO
		\end{proof}
	\end{theorem}
\section{Superfícies}
	\subsection{Immersions i submersions}
	\begin{definition}[Immersió]
		\labelname{immersió}\label{def:immersió}
		Sigui \(\obert{U}\subseteq\mathbb{R}^{n}\) un obert, \(x_{0}\in\obert{U}\) un punt i \(f\colon\obert{U}\longrightarrow\mathbb{R}^{n}\) una funció diferenciable tal que \(df(a)\) sigui injectiva. Aleshores direm que \(f\) és una immersió en \(x_{0}\).
	\end{definition}
	\begin{observation} % REF álgebra lineal
		\label{obs:immersió si i només si té rang maximal}
		\(f\) és una immersió en \(x_{0}\) si i només si \(\rang(df(a))=n\).
	\end{observation}
	\begin{example}
		\label{ex:la inclusió canònica és una immersió}
		Volem veure que la inclusió canònica és una immersió.
		\begin{solution}
			La inclusió canònica ve definida com
			\begin{align*}
				i\colon\mathbb{R}^{p}&\longrightarrow\mathbb{R}^{p+q} \\
				x&\longmapsto(x;0), \tag{\ref{notation:punts per grups}}
			\end{align*}
			i per la definició de \myref{def:Jacobiana} tenim que la seva diferencial és
			\[\left[\begin{array}{c}
				I_{p} \\\hline
				0_{q\times(p-q)}
			\end{array}\right].\]
			Aleshores per la definició de \myref{def:rang d'una matriu} trobem que aquesta té rang \(p\), i per l'observació \myref{obs:immersió si i només si té rang maximal} tenim que \(i\) és una immersió.
		\end{solution}
	\end{example}
	\begin{definition}[Submersió]
		\labelname{submersió}\label{def:submersió}
		Sigui \(\obert{U}\subseteq\mathbb{R}^{n}\) un obert, \(x_{0}\in\obert{U}\) un punt i \(f\colon\obert{U}\longrightarrow\mathbb{R}^{n}\) una funció diferenciable tal que \(df(a)\) sigui exhaustiva. Aleshores direm que \(f\) és una submersió en \(x_{0}\).
	\end{definition}
	\begin{observation}
		\label{obs:submersió si i només si té rang més petit o igual}
		\(f\) és una submersió en \(x_{0}\) si i només si \(\rang(df(a))\leq n\).
	\end{observation}
	\begin{example}
		Volem veure que la projecció canònica és una submersió.
		\begin{solution}
			La projecció canònica ve definida com
			\begin{align*}
				\pi\colon\mathbb{R}^{p+q}&\longrightarrow\mathbb{R}^{p} \\
				(x;y)&\longmapsto x, \tag{\ref{notation:punts per grups}}
			\end{align*}
			i per la definició de \myref{def:Jacobiana} trobem que la seva diferencial és
			\[\left[\begin{array}{c|c}
			I_{p} & 0_{p\times(p+q)}
			\end{array}\right].\]
			Aleshores per la definició de \myref{def:rang d'una matriu} trobem que aquesta té rang \(p\) i per l'observació \myref{obs:submersió si i només si té rang més petit o igual} tenim que \(\pi\) és una submersió.
		\end{solution}
	\end{example}
	\begin{theorem}[Teorema d'estructura local de les immersions]
		\labelname{Teorema d'estructura local de les immersions}\label{thm:Teorema d'estructura local de les immersions}
		Siguin \(\obert{U}\subseteq\mathbb{R}^{n}\) un obert i \(f\colon\obert{U}\longrightarrow\mathbb{R}^{m}\) una immersió en un punt \(x_{0}\in\obert{U}\). Aleshores existeixen dos oberts \(\obert{U}'\subseteq\mathbb{R}^{n}\) i \(\obert{V}\subseteq\mathbb{R}^{m}\) satisfent \(x_{0}\subseteq\obert{U}'\subseteq\obert{U}\) i existeix un difeomorfisme \(g\colon\obert{V}\longleftrightarrow\Ima_{\obert{U}'}(g)\subseteq\mathbb{R}^{n}\) i satisfent \(i(x_{0})\in\obert{V}\) tals que el diagrama
		\[\begin{tikzcd}
			\obert{U}' \arrow{r}{i} \arrow[swap]{dr}{f} & \obert{V} \arrow{d}{g} \\
			& \mathbb{R}^{m}
		\end{tikzcd}\]
		és un diagrama commutatiu.
		\begin{proof}
			%TODO
		\end{proof}
	\end{theorem}
	\begin{theorem}[Teorema d'estructura local de les submersions]
		\labelname{Teorema d'estructura local de les submersions}\label{thm:Teorema d'estructura local de les submersions}
		Siguin \(\obert{U}\subseteq\mathbb{R}^{n}\) un obert i \(f\colon\obert{U}\longrightarrow\mathbb{R}^{m}\) una submersions en un punt \(x_{0}\in\obert{U}\). Aleshores existeix un obert \(\obert{U}'\) satisfent \(x_{0}\subseteq\obert{U}'\subseteq\obert{U}\) i existeix un difeomorfisme \(g\colon\obert{U}'\longleftrightarrow\Ima_{\obert{U}'}(g)\subseteq\mathbb{R}^{n}\) tals que el diagrama
		\[\begin{tikzcd}
			\obert{U}' \arrow{r}{g} \arrow[swap]{dr}{f} & \Ima_{\obert{U}'}(g) \arrow{d}{\pi} \\
			& \mathbb{R}^{m}
		\end{tikzcd}\]
		és un diagrama commutatiu.
		\begin{proof}
			%TODO
		\end{proof}
	\end{theorem}
\end{document}
