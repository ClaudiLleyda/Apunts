\documentclass[../Apunts.tex]{subfiles}

\begin{document}
\chapter{Sèries}
	\section{Sèries numèriques}
	\subsection{Convergència d'una sèrie numèrica}
	\begin{definition}[Sèrie numèrica]
		\labelname{sèrie numèrica}\label{def:sèrie numèrica}
		Sigui \((a_{n})\) una successió de nombres reals. Aleshores direm que
		\[\sum_{n=1}^{\infty}a_{n}=a_{1}+a_{2}+a_{3}+\cdots\]
		és una sèrie numèrica o la sèrie de \((a_{n})\).
	\end{definition}
	\begin{definition}[Sèrie convergent]
		\labelname{sèrie convergent}\label{def:sèrie convergent}
		\labelname{sèrie divergent}\label{def:sèrie divergent}
		Siguin \(\sum_{n=1}^{\infty}a_{n}\) una sèrie numèrica i
		\[S_{N}=\sum_{n=1}^{N}a_{n}\]
		una successió tals que
		\[\lim_{n\to\infty}S_{n}=L\]
		on \(L\) és un nombre real. Aleshores direm que la sèrie \(\sum_{n=1}^{\infty}a_{n}\) és una sèrie convergent amb valor \(L\).
		
		Si \(L\) no és un nombre real direm que la sèrie \(\sum_{n=1}^{\infty}a_{n}\) és una sèrie divergent.
	\end{definition}
	\begin{example}[Sèries geomètriques]
		\labelname{sèries geomètriques}\label{ex:sèries geomètriques}
		Sigui \(r\neq0\) un nombre real. Volem estudiar la convergència de la sèrie
		\begin{equation}
			\label{ex:sèries geomètriques:eq1}
			\sum_{n=0}^{\infty}r^{n}.
		\end{equation}
		\begin{solution}
			Tenim que \(\sum_{n=0}^{N}r^{n}=r^{0}+r^{1}+r^{2}+\cdots\), i per tant
			\begin{align*}
				(1-r)\sum_{n=0}^{N}r^{n}&=(1-r)(r^{0}+r^{1}+r^{2}+\cdots)\\
				&=r^{0}+r^{1}+r^{2}+\dots+r^{n}-(r^{1}+r^{2}+r^{3}+\dots+r^{n+1})\\
				&=1-r^{N+1}\\
			\end{align*}
			i trobem
			\begin{equation*}
				\sum_{n=0}^{N}r^{n}=\frac{1-r^{N+1}}{1-r}
			\end{equation*}
			Ara bé, tenim
			\[\lim_{N\to\infty}\frac{1-r^{N+1}}{1-r}=
			\begin{cases}
				\displaystyle \frac{1}{1-r} & \text{si }-1<r<1 \\
				+\infty & \text{si }r\geq1 \\
				\text{no existeix} & \text{si }r\leq-1 
			\end{cases}\]
			i per la definició de \myref{def:sèrie convergent} trobem que \eqref{ex:sèries geomètriques:eq1} és convergent si i només si \(r\) pertany a l'interval \((-1,1)\), i si \(r\) pertany a l'interval \((-1,1)\) aleshores es compleix
			\begin{equation*}
				\sum_{n=0}^{N}r^{n}=\frac{1-r^{N+1}}{1-r}.\qedhere
			\end{equation*}
		\end{solution}
	\end{example}
	\begin{theorem}[Condició de Cauchy]
		\labelname{condició de Cauchy}\label{thm:Condició de Cauchy per sèries numèriques}
		Sigui \(\sum_{n=0}^{\infty}a_{n}\) una sèrie numèrica. Aleshores \(\sum_{n=0}^{\infty}a_{n}\) és convergent si i només si per a tot \(\varepsilon>0\) existeix un \(n_{0}\) natural tal que per a tot \(N\) i \(M\) naturals amb \(N\),\(M\geq n_{0}\) tenim
		\[\abs{\sum_{n=N}^{M}a_{n}}<\varepsilon.\]
		\begin{proof}
			Suposem que \(\sum_{n=0}^{\infty}a_{n}\) és convergent. Per la definició de \myref{def:sèrie convergent} això és que
			\[\lim_{N\to\infty}\sum_{n=1}^{N}{a_{n}}=L\]
			on \(L\) és un nombre real.
			
			Per la definició de \myref{def:límit} tenim que per a tot \(\delta_{1}>0\) real existeix un \(N\) tal que
			\[\abs{\sum_{n=1}^{N-1}{a_{n}}-L}<\delta_{1}\]
			i de nou per la definició de \myref{def:límit} tenim que per a tot \(\varepsilon>0\) real existeix un \(\delta_{2}>0\) real tal que \(\delta_{1}-\delta_{2}<\varepsilon\) i que existeix un \(M\) natural satisfent
			\[\abs{\sum_{n=1}^{M}a_{n}-L}<\delta_{2}.\]
			Per tant trobem
			\[\abs{\sum_{n=1}^{N-1}a_{n}-L}-\abs{\sum_{n=1}^{M}a_{n}-L}<\delta_{1}-\delta_{2}\]
			i per la desigualtat triangular %TODO %REF
			tenim que
			\begin{align*}
				\abs{\sum_{n=N}^{M}a_{n}}&=\abs{\sum_{n=1}^{N-1}a_{n}-\sum_{n=1}^{M}a_{n}}\\
				&=\abs{\sum_{n=1}^{N-1}a_{n}-L-\sum_{n=1}^{M}a_{n}+L}\\
				&\leq\abs{\sum_{n=1}^{N-1}a_{n}-L}-\abs{\sum_{n=1}^{M}a_{n}-L}\\
				&<\delta_{1}-\delta_{2}<\varepsilon.\qedhere
			\end{align*}
		\end{proof}
	\end{theorem}
	\begin{corollary}
		\label{cor:condició de Cauchy}\label{cor:terme general tendeix a zero en una sèrie convergent}
		Sigui \(\sum_{n=1}^{\infty}a_{n}\) una sèrie convergent. Aleshores
		\[\lim_{n\to\infty}a_{n}=0.\]
		\begin{proof}
			Per la \myref{thm:Condició de Cauchy per sèries numèriques} tenim que per a tot \(\varepsilon>0\) existeix un \(n_{0}\) natural tal que per a tot \(N\) i \(M\) naturals amb \(N\),\(M\geq n_{0}\) tenim
			\[\abs{\sum_{n=N}^{M}a_{n}}<\varepsilon.\]
			En particular, si triem \(M=N+1\), tenim \(\abs{a_{n}}<\varepsilon\) per a tot \(n>n_{0}\), i per la definició de \myref{def:límit} veiem que \(\lim_{n\to\infty}a_{n}=0\), com volíem veure.
		\end{proof}
	\end{corollary}
	\begin{example}[Sèrie harmònica]
		\labelname{sèrie harmònica}\label{ex:sèrie harmònica}
		Sigui \(\alpha\) un nombre real. Volem estudiar la convergència de la sèrie
		\begin{equation}
			\label{ex:sèrie harmònica:eq1}
			\sum_{n=1}^{\infty}\frac{1}{n^{\alpha}}.
		\end{equation}
		\begin{solution}
			Observem que si \(\alpha\leq0\) tenim que
			\[\lim_{n\to\infty}\frac{1}{n^{\alpha}}=\infty\]
			i pel coro{\lgem}ari \myref{cor:condició de Cauchy} tenim que la sèrie és divergent.
			
			Suposem que \(0<\alpha\leq1\). Definim la successió
			\[S_{N}=\sum_{n=1}^{N}\frac{1}{n^{\alpha}}\]
			i tenim que
			\begin{align*}
				S_{2^{N}}&=1+\frac{1}{2^{\alpha}}+\frac{1}{3^{\alpha}}+\dots+\frac{1}{2^{N\alpha}}\\
				&\geq1+\frac{1}{2^{\alpha}}+\frac{2}{4^{\alpha}}+\dots+\frac{2^{N-1}}{2^{N\alpha}}\\
				&=1+\sum_{i=0}^{N-1}\frac{2^{i}}{2^{(i+1)\alpha}}=1+\frac{1}{2^{\alpha}}\sum_{i=0}^{N-1}2^{i(1-\alpha)}.
			\end{align*}
			Ara bé, tenim que \(1-\alpha\geq0\), i per tant trobem
			\[\lim_{N\to\infty}1+\frac{1}{2^{\alpha}}\sum_{i=0}^{N-1}2^{i(1-\alpha)}=\infty,\]
			i tenim que \(\lim_{N\to\infty}S_{2^{N}}=\infty\) i per la definició de \myref{def:sèrie divergent} trobem que \(\sum_{n=1}^{\infty}\frac{1}{n^{\alpha}}\) és divergent.
			
			Veiem ara el cas \(\alpha>1\). Definim de nou la successió
			\[S_{N}=\sum_{n=1}^{N}\frac{1}{n^{\alpha}}.\]
			Tenim que
			\begin{align*}
				S_{2^{N}}&=1+\frac{1}{2^{\alpha}}+\frac{1}{3^{\alpha}}+\dots+\frac{1}{2^{N\alpha}}\\
				&\leq1+\frac{1}{2^{\alpha}}+\frac{2}{4^{\alpha}}+\dots+\frac{2^{N-1}}{2^{N\alpha}}\\
				&=\frac{1}{2^{N\alpha}}+\sum_{i=0}^{N-1}\frac{2^{i}}{2^{i\alpha}}\\
				&=\frac{1}{2^{N\alpha}}+\sum_{i=0}^{N-1}2^{(1-\alpha)i}\\
				&=\frac{1}{2^{N\alpha}}+\frac{1-2^{(1-\alpha)N}}{1-2^{1-\alpha}}\tag{\myref{ex:sèries geomètriques}}
			\end{align*}
			i tenim que
			\[\lim_{N\to\infty}\frac{1}{2^{N\alpha}}+\frac{1-2^{(1-\alpha)N}}{1-2^{1-\alpha}}=\frac{1}{1-2^{1-\alpha}}.\]
			Per tant tenim que la successió \((S_{N})\) està fitada, i degut a que \(S_{N+1}=S_{N}+\frac{1}{(N+1)^{\alpha}}\) tenim que és creixen i pel \myref{thm:Weierstrass màxims i mínims múltiples variables} tenim que és convergent. Per tant per la definició de \myref{def:sèrie convergent} tenim que la sèrie \eqref{ex:sèrie harmònica:eq1} és convergent.
			
			Per tant tenim que la sèrie
			\[\sum_{n=1}^{\infty}\frac{1}{n^{\alpha}}\]
			és convergent si i només si \(\alpha>1\).
		\end{solution}
	\end{example}
	\subsection{Sèries de termes positius}
	\begin{definition}[Sèrie de termes positius]
		\labelname{sèrie de termes positius}\label{def:sèrie de termes positius}
		Sigui \(\sum_{n=0}^{\infty}a_{n}\) una sèrie numèrica tal que \(a_{n}>0\) per a tot \(n\) natural. Aleshores direm que \(\sum_{n=0}^{\infty}a_{n}\) és una sèrie de termes positius.
	\end{definition}
	\begin{notation}
		Sigui \(\sum_{n=0}^{\infty}a_{n}\) una sèrie de termes positius. Aleshores escriurem \(\sum_{n=0}^{\infty}a_{n}<\infty\) si \(\sum_{n=0}^{\infty}a_{n}\) és convergent i \(\sum_{n=0}^{\infty}a_{n}=\infty\) si \(\sum_{n=0}^{\infty}a_{n}\) és divergent
	\end{notation}
	\begin{lemma}
		\label{lemma:criteri de comparació de sèries de termes positius}
		Siguin \(\sum_{n=0}^{\infty}a_{n}\) una sèrie de termes positius, \(\sum_{n=0}^{\infty}b_{n}\) una sèrie de termes positius convergent, \(K\) un nombre real i \(n_{0}\) un natural tal que per a tot \(n>n_{0}\) es satisfà \(a_{n}\leq Kb_{n}\). Aleshores tenim que \(\sum_{n=0}^{\infty}a_{n}\) és convergent.
		\begin{proof}
			Tenim que
			\begin{align*}
				\sum_{n=0}^{N}a_{n}&=\sum_{n=0}^{n_{0}-1}a_{n}+\sum_{n=n_{0}}^{N}a_{n}\\
				&\leq\sum_{n=0}^{n_{0}-1}a_{n}+K\sum_{n=n_{0}}^{N}b_{n}
			\end{align*}
			i per la definició de \myref{def:sèrie convergent} tenim que la sèrie \(K\sum_{n=n_{0}}^{N}b_{n}\) és convergent i, de nou per la definició de \myref{def:sèrie convergent}, tenim que la sèrie \(\sum_{n=0}^{\infty}a_{n}\) és convergent, com volíem veure.
		\end{proof}
	\end{lemma}
	\begin{theorem}[Criteri de comparació]
		\labelname{criteri de comparació}\label{thm:criteri de comparació de sèries de termes positius}
		Siguin \(\sum_{n=0}^{\infty}a_{n}\) i \(\sum_{n=0}^{\infty}b_{n}\) dues sèries de termes positius i
		\[\lim_{n\to\infty}\frac{a_{n}}{b_{n}}=L.\]
		tals que
		\begin{enumerate}
			\item\label{thm:criteri de comparació de sèries de termes positius:enum1} \(L\neq0\) i \(L\neq\infty\). Aleshores \(\sum_{n=0}^{\infty}a_{n}\) és convergent si i només si \(\sum_{n=0}^{\infty}b_{n}\)
			\item\label{thm:criteri de comparació de sèries de termes positius:enum2} \(L=0\) i \(\sum_{n=0}^{\infty}b_{n}\) és convergent. Aleshores \(\sum_{n=0}^{\infty}a_{n}\) és convergent.
			\item\label{thm:criteri de comparació de sèries de termes positius:enum3} \(L=\infty\) i \(\sum_{n=0}^{\infty}a_{n}\) és convergent. Aleshores \(\sum_{n=0}^{\infty}b_{n}\) és convergent.
		\end{enumerate}
		\begin{proof}
			Comencem veient el punt \eqref{thm:criteri de comparació de sèries de termes positius:enum1}. Suposem que \(L\neq0\) ó \(L\neq\infty\). Observem que \(L>0\). Per tant prenem un \(\varepsilon>0\) tal que \(L-\varepsilon>0\). Per la definició de \myref{def:límit} tenim que existeix un natural \(n_{0}\) tal que per a tot \(n>n_{0}\) es satisfà
			\[\abs{\frac{a_{n}}{b_{n}}-L}<\varepsilon.\]
			Per tant tenim que per a tot \(n>n_{0}\)
			\[l-\varepsilon<\frac{a_{n}}{b_{n}}<l+\varepsilon.\]
			
			Suposem que \(\sum_{n=0}^{\infty}b_{n}\) és convergent. Aleshores tenim
			\[b_{n}(l-\varepsilon)<a_{n}<b_{n}(l+\varepsilon),\]
			i pel lemma \myref{lemma:criteri de comparació de sèries de termes positius} trobem que la sèrie \(\sum_{n=0}^{\infty}a_{n}\) és convergent.
			
			Suposem ara que \(\sum_{n=0}^{\infty}a_{n}\) és convergent. Aleshores tenim
			\[\frac{a_{n}}{L-\varepsilon}<b_{n}<\frac{a_{n}}{L+\varepsilon},\]
			i pe lemma \myref{lemma:criteri de comparació de sèries de termes positius} trobem que la sèrie \(\sum_{n=0}^{\infty}b_{n}\) és convergent.
			
			Veiem ara el punt \eqref{thm:criteri de comparació de sèries de termes positius:enum2}. Suposem doncs que \(L=0\) i que \(\sum_{n=0}^{\infty}b_{n}\) és convergent. Per la definició de \myref{def:límit} tenim que per a tot \(\varepsilon>0\) existeix un natural \(n_{0}\) tal que
			\[\abs{\frac{a_{n}}{b_{n}}}<\varepsilon,\]
			i això és equivalent a
			\[a_{n}<\varepsilon b_{n}\]
			i pel lemma \myref{lemma:criteri de comparació de sèries de termes positius} trobem que la sèries \(\sum_{n=0}^{\infty}a_{n}\) és convergent.
			
			Veiem ara el punt \eqref{thm:criteri de comparació de sèries de termes positius:enum3}. Suposem doncs que \(L=\infty\) i que \(\sum_{n=0}^{\infty}a_{n}\) és convergent. Tenim que
			\[\lim_{n\to\infty}\frac{b_{n}}{a_{n}}=0,\] %REF
			i per tant pel punt \eqref{thm:criteri de comparació de sèries de termes positius:enum2} tenim que \(\sum_{n=0}^{\infty}b_{n}\) és convergent.
		\end{proof}
	\end{theorem}
	\begin{example}
		Considerem la sèrie
		\[\sum_{n=1}^{\infty}\left(1+\frac{1}{n}\right)^{3}3^{-n}.\]
		Volem estudiar si aquesta sèrie és convergent o divergent.
		\begin{solution}
			Definim
			\[a_{n}=\left(1+\frac{1}{n}\right)^{3}3^{-n}\quad\text{i}\quad b_{n}=\frac{1}{3^{n}}\]
			Observem que per a tot \(n\) natural tenim \(a_{n}>0\) i \(b_{n}>0\), i per la definició de \myref{def:sèrie de termes positius} tenim que les sèries \(\sum_{n=0}^{\infty}a_{n}\) i \(\sum_{n=0}^{\infty}\) són sèries de termes positius.
			
			Considerem el límit
			\begin{align*}
				\lim_{n\to\infty}\frac{a_{n}}{b_{n}}&=\lim_{n\to\infty}\frac{\left(1+\frac{1}{n}\right)^{3}3^{-n}}{3^{-n}}\\
				&=\lim_{n\to\infty}\left(1+\frac{1}{n}\right)^{3}=1.
			\end{align*}
			
			Ara bé, per l'exercici \myref{ex:sèries geomètriques} tenim que la sèrie \(\sum_{n=0}^{\infty}\frac{1}{3^{n}}\) és convergent, ja que \(-1<\frac{1}{3}<1\), i pel \myref{thm:criteri de comparació de sèries de termes positius} trobem que la sèrie \(\sum_{n=0}^{\infty}a_{n}\) és convergent.
		\end{solution}
	\end{example}
	\begin{proposition}[Criteri de l'arrel]
		\labelname{criteri de l'arrel}\label{prop:criteri de l'arrel}
		Siguin \(\sum_{n=0}^{\infty}a_{n}\) una sèrie de termes positius i
		\[\lim_{n\to\infty}\sqrt[n]{a_{n}}=L\]
		tal que
		\begin{enumerate}
			\item\label{prop:criteri de l'arrel:enum1} \(L<1\). Aleshores la sèrie \(\sum_{n=0}^{\infty}a_{n}\) és convergent.
			\item\label{prop:criteri de l'arrel:enum2} \(L>1\). Aleshores la sèrie \(\sum_{n=0}^{\infty}a_{n}\) és divergent.
		\end{enumerate}
		\begin{proof}
			Comencem veient el punt \eqref{prop:criteri de l'arrel:enum1}.  Suposem doncs que \(L<1\). Prenem un \(\varepsilon>0\) tal que \(L+\varepsilon<1\). Per la definició de \myref{def:límit} tenim que existeix un natural \(n_{0}\) tal que per a tot \(n>n_{0}\) es satisfà
			\[\abs{\sqrt[n]{a_{n}}-L}<\varepsilon,\]
			i per tant per a tot \(n>n_{0}\) tenim
			\[\sqrt[n]{a_{n}}<L+\varepsilon,\]
			i per tant
			\[a_{n}<\left(L+\varepsilon\right)^{n}.\]
			Ara bé, tenim que \(L+\varepsilon<1\), per tant per l'exercici \myref{ex:sèries geomètriques} tenim que \(\sum_{n=0}^{\infty}(L+\varepsilon)^{n}\) és convergent, i per tant pel lemma \myref{lemma:criteri de comparació de sèries de termes positius} tenim que la sèrie \(\sum_{n=0}^{\infty}a_{n}\) és convergent.
			
			Veiem ara el punt \eqref{prop:criteri de l'arrel:enum2}. Suposem doncs que \(L>1\). Prenem \(\varepsilon>0\) tal que \(L-\varepsilon>1\). Per la definició de \myref{def:límit} tenim que existeix un natural \(n_{0}\) tal que per a tot \(n>n_{0}\) es satisfà
			\[\abs{\sqrt[n]{a_{n}}-L}<\varepsilon,\]
			i per tant
			\[L-\varepsilon<\sqrt[n]{a_{n}}.\]
			Ara bé, tenim que \(L-\varepsilon>1\), per tant per l'exercici \myref{ex:sèries geomètriques} tenim que \(\sum_{n=0}^{\infty}(L-\varepsilon)^{n}\) és divergent, i per tant pel lemma \myref{lemma:criteri de comparació de sèries de termes positius} tenim que la sèrie \(\sum_{n=0}^{\infty}a_{n}\) és divergent.
		\end{proof}
	\end{proposition}
	\begin{example}
		Siguin \(\alpha\geq0\) i \(\beta\geq0\) dos reals. Considerem la sèrie numèrica
		\[\sum_{n=1}^{\infty}n^{\beta}\alpha^{n}.\]
		Volem estudiar la convergència d'aquesta sèrie en funció dels valors de \(\alpha\) i \(\beta\).
		\begin{solution}
			Definim
			\[a_{n}=n^{\beta}\alpha^{n}.\]
			Observem primer que \(a_{n}>0\) per a tot \(n\) natural. Per tant per la definició de \myref{def:sèrie de termes positius} tenim que \(\sum_{n=1}^{\infty}a_{n}\) és una sèrie de termes positius. Veiem també que si \(\alpha=1\) tenim
			\[\sum_{n=1}^{\infty}n^{\beta},\]
			i com que \(\beta\geq0\) la sèrie és divergent.
			
			Considerem el límit
			\begin{align*}
			\lim_{n\to\infty}\sqrt[n]{a_{n}}&=\lim_{n\to\infty}\sqrt[n]{n^{\beta}\alpha^{n}}\\
			&=\lim_{n\to\infty}\sqrt[n]{n^{\beta}}\sqrt[n]{\alpha^{n}}\\
			&=\lim_{n\to\infty}\sqrt[n]{n}^{\beta}\alpha=\alpha.
			\end{align*}
			Per tant, pel \myref{prop:criteri de l'arrel} tenim que la sèrie és convergent quan \(\alpha>1\) i divergent quan \(\alpha\leq1\). 
		\end{solution}
	\end{example}
	\begin{proposition}[Criteri del quocient]
		\labelname{criteri del quocient}\label{prop:criteri del quocient}
		Siguin \(\sum_{n=0}^{\infty}a_{n}\) una sèrie de termes positius i
		\[\lim_{n\to\infty}\frac{a_{n+1}}{a_{n}}=L\]
		tal que
		\begin{enumerate}
			\item\label{prop:criteri del quocient:enum1} \(L<1\). Aleshores la sèrie \(\sum_{n=0}^{\infty}a_{n}\) és convergent.
			\item\label{prop:criteri del quocient:enum2} \(L>1\). Aleshores la sèrie \(\sum_{n=0}^{\infty}a_{n}\) és divergent.
		\end{enumerate}
		\begin{proof}
			Comencem veient el punt \eqref{prop:criteri del quocient:enum1}. Suposem doncs que \(L<1\). Prenem un \(\varepsilon>0\) tal que \(L+\varepsilon<1\). Aleshores per la definició de \myref{def:límit} tenim que existeix un \(n_{0}\) tal que per a tot \(n>n_{0}\) es satisfà
			\[\abs{\frac{a_{n+1}}{a_{n}}-L}<\varepsilon,\]
			i per tant per a tot \(n>n_{0}\) tenim
			\[\frac{a_{n+1}}{a_{n}}<L+\varepsilon,\]
			i per tant
			\[a_{n+1}<(L+\varepsilon)a_{n}\]
			i trobem
			\begin{align*}
				a_{n+1}<&(L+\varepsilon)a_{n}\\
				<&(L+\varepsilon)^{2}a_{n-1}\\
				&\vdots\\
				<&(L+\varepsilon)^{n-n_{0}+1}a_{n_{0}}.
			\end{align*}
			Ara bé, com que \(L+\varepsilon<1\) tenim per l'exemple \myref{ex:sèries geomètriques} que la sèrie \(\sum_{n=0}^{\infty}(L+\varepsilon)^{n}\) és convergent, i pel lemma \myref{lemma:criteri de comparació de sèries de termes positius} trobem que la sèrie \(\sum_{n=0}^{\infty}a_{n}\) és convergent.
			
			Veiem ara el punt \eqref{prop:criteri del quocient:enum2}. Suposem doncs que \(L>1\). Prenem un \(\varepsilon>0\) tal que \(L-\varepsilon>1\). Aleshores per la definició de \myref{def:límit} tenim que existeix un \(n_{0}\) natural tal que per a tot \(n>n_{0}\) es satisfà
			\[\abs{\frac{a_{n+1}}{a_{n}}-L}<\varepsilon,\]
			i per tant per a tot \(n>n_{0}\) tenim
			\[L-\varepsilon<\frac{a_{n+1}}{a_{n}}\]
			i per tant
			\[(L-\varepsilon)a_{n}<a_{n+1}\]
			i trobem
			\begin{align*}
				a_{n+1}>&(L-\varepsilon)a_{n}\\
				>&(L-\varepsilon)^{2}a_{n-1}\\
				&\vdots\\
				>&(L-\varepsilon)^{n-n_{0}+1}a_{n_{0}}.
			\end{align*}
			Ara bé, com que \(L-\varepsilon>1\) tenim que
			\[\lim_{n\to\infty}(L-\varepsilon)^{n}=\infty,\]
			i per tant trobem
			\[\lim_{n\to\infty}a_{n}=\infty\]
			i tenim que la sèrie \(\sum_{n=0}^{\infty}a_{n}\) és divergent, com volíem veure.
		\end{proof}
	\end{proposition}
	\begin{lemma}
		\label{lemma:criteri de Raabe}
		Siguin \(x\) i \(\alpha\) dos nombres reals no negatius. Aleshores
		\[1-\frac{\alpha}{x}\leq\left(1-\frac{1}{x+1}\right)^{\alpha}.\]
		\begin{proof}
		:(
%			Tenim que
%			\[\left(1-\frac{1}{x+1}\right)^{\alpha}=\left(\frac{x}{x+1}\right)^{\alpha}\]
%			i trobem %REF
%			\[\left(\frac{x}{x+1}\right)^{\alpha}=\alpha\log\left(\frac{x}{x+1}\right),\]
%			d'on veiem que %REF
%			\[\alpha\log\left(\frac{x}{x+1}\right)=\log\left(x^{\alpha}\right)-\alpha\log(x+1)\]
		\end{proof}
	\end{lemma}
	\begin{proposition}[Criteri de Raabe]
		\labelname{criteri de Raabe}\label{prop:criteri de Raabe}
		Siguin \(\sum_{n=0}^{\infty}a_{n}\) una sèrie de termes positius i
		\[\lim_{n\to\infty}n\left(1-\frac{a_{n+1}}{a_{n}}\right)=L\]
		tal que
		\begin{enumerate}
		\item\label{prop:criteri de Raabe:enum1} \(L>1\). Aleshores la sèrie \(\sum_{n=0}^{\infty}a_{n}\) és convergent.
		\item\label{prop:criteri de Raabe:enum2} \(L<1\). Aleshores la sèrie \(\sum_{n=0}^{\infty}a_{n}\) és divergent.
		\end{enumerate}
		\begin{proof}
			Comencem veient el cas \eqref{prop:criteri de Raabe:enum1}. Suposem doncs que \(L>1\). Prenem un \(\varepsilon>0\) tal que \(L-\varepsilon>1\). Aleshores per la definició de \myref{def:límit} tenim que existeix un \(n_{0}\) natural tal que per a tot \(n>n_{0}\) es satisfà
			\[\abs{n\left(1-\frac{a_{n+1}}{a_{n}}\right)-L}<\varepsilon,\]
			i per tant per a tot \(n>n_{0}\) tenim
			\[L-\varepsilon<n\left(1-\frac{a_{n+1}}{a_{n}}\right),\]
			i per tant
			\[a_{n+1}<a_{n}\left(1-\frac{L-\varepsilon}{n}\right).\]
			
			Aleshores, pel lemma \myref{lemma:criteri de Raabe} trobem que
			\[a_{n+1}<a_{n}\left(1-\frac{1}{n+1}\right)^{L-\varepsilon}\]
			i tenim
			\begin{equation}
				\label{prop:criteri de Raabe:eq1}
				a_{n+1}<a_{n}\left(\frac{n}{n+1}\right)^{L-\varepsilon}.
			\end{equation}
			Podem aplicar la desigualtat \eqref{prop:criteri de Raabe:eq1} recursivament per obtenir
			\begin{align*}
				a_{n+1}<&a_{n}\left(\frac{n}{n+1}\right)^{L-\varepsilon}\\
				<&a_{n-1}\left(\frac{n-1}{n}\frac{n}{n+1}\right)^{L-\varepsilon}\\
				&\vdots\\
				<&a_{n_{0}}\left(\frac{n_{0}}{n_{0}+1}\frac{n_{0}+1}{n_{0}+2}\dots\frac{n-1}{n}\frac{n}{n+1}\right)^{L-\varepsilon}
			\end{align*}
			i per tant trobem
			\[a_{n+1}<a_{n_{0}}\left(\frac{n_{0}}{n+1}\right)^{L-\varepsilon},\]
			i com que \(L-\varepsilon>1\) trobem per l'exemple \myref{ex:sèrie harmònica} que la sèrie
			\[\sum_{n=0}^{\infty}\left(\frac{1}{n+1}\right)^{L-\varepsilon}\]
			és convergent, i pel lemma \myref{lemma:criteri de comparació de sèries de termes positius} tenim que la sèrie \(\sum_{n=0}^{\infty}a_{n}\) és convergent.
			
			Veiem ara el punt \eqref{prop:criteri de Raabe:enum2}. Suposem doncs que \(L<1\).
		\end{proof}
	\end{proposition}
	\begin{example}
		\label{ex:quocient + Raabe}
		Sigui \(\alpha>0\) un nombre real. Considerem la sèrie numèrica
		\begin{equation}
			\label{ex:quocient + Raabe:eq1}
			\sum_{n=1}^{\infty}\frac{\alpha^{n}n!}{n^{n}}.
		\end{equation}
		Volem estudiar la convergència d'aquesta sèrie en funció del valor de \(\alpha\).
		\begin{solution}
			Considerem el límit
			\begin{align*}
				\lim_{n\to\infty}\frac{\frac{\alpha^{n+1}(n+1)!}{(n+1)^{n+1}}}{\frac{\alpha^{n}n!}{n^{n}}}&=\lim_{n\to\infty}\frac{\alpha^{n+1}(n+1)!n^{n}}{\alpha^{n}n!(n+1)^{n+1}} \\
				&=\lim_{n\to\infty}\frac{\alpha(n+1)!n^{n}}{(n+1)!(n+1)^{n}} \\
				&=\alpha\lim_{n\to\infty}\left(\frac{n}{n+1}\right)^{n}=\frac{\alpha}{\e} %REF
			\end{align*}
			Per tant, si \(\alpha>\e\) tenim que \(\frac{\alpha}{\e}>1\), i pel \myref{prop:criteri del quocient} trobem que la sèrie és divergent. Si \(\alpha<\e\) aleshores \(\frac{\alpha}{\e}<1\) i pel \myref{prop:criteri del quocient} trobem que la sèrie és convergent.
			
			Estudiem el cas \(\alpha=\e\). Considerem el límit
			\begin{align*}
				\lim_{n\to\infty}n\left(1-\frac{\frac{\e^{n+1}(n+1)!}{(n+1)^{n+1}}}{\frac{\e^{n}n!}{n^{n}}}\right)&=\lim_{n\to\infty}n\left(1-\frac{\e^{n+1}(n+1)!n^{n}}{e^{n}n!(n+1)^{n+1}}\right) \\
				&=\lim_{n\to\infty}n\left(1-\e\frac{(n+1)!n^{n}}{(n+1)!(n+1)^{n}}\right) \\
				&=\lim_{n\to\infty}n\left(1-\e\left(\frac{n}{n+1}\right)^{n}\right) \\
				&=\lim_{n\to\infty}n(1-e^{2})=-\infty %REF
			\end{align*}
			i pel \myref{prop:criteri de Raabe} trobem que la sèrie \eqref{ex:quocient + Raabe:eq1} és divergent quan \(\alpha=e\).
			
			Per tant la sèrie
			\[\sum_{n=1}^{\infty}\frac{\alpha^{n}n!}{n^{n}}\]
			és convergent si i només si \(\alpha<\e\).
		\end{solution}
	\end{example}
%	\subsection{Sèries alternades}
%	\subsection{Convergència absoluta}
%	\subsection{Reordenació de sèries}
%	
%	\section{Successions i sèries de funcions}
%	\subsection{Successions de funcions}
%	\subsection{Sèries de funcions}
%	\subsection{Sèries de potències}
%	\subsection{Teorema d'aproximació polinòmica de Weierstrass}
%	\printbibliography
\end{document}


% Sèries de funcions amb Perelló (Càlcul infinitesimal)
% Encara no se d'on treure integrals impròpies
% Fourier amb Tolstov (Fourier Series)