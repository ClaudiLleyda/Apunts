\documentclass[../Apunts.tex]{subfiles}

\begin{document}
\chapter{Sèries numèriques}
	\section{Convergència d'una sèrie numèrica}
	\subsection{TBD}
	\begin{definition}[Sèrie numèrica]
		\labelname{sèrie numèrica}\label{def:sèrie numèrica}
		Sigui \((a_{n})\) una successió de nombres reals. Aleshores direm que
		\[\sum_{n=1}^{\infty}a_{n}=a_{1}+a_{2}+a_{3}+\cdots\]
		és una sèrie numèrica o la sèrie de \((a_{n})\).
	\end{definition}
	\begin{definition}[Sèrie convergent]
		\labelname{sèrie convergent}\label{def:sèrie convergent}
		\labelname{sèrie divergent}\label{def:sèrie divergent}
		Siguin \(\sum_{n=1}^{\infty}a_{n}\) una sèrie numèrica i
		\[S_{N}=\sum_{n=1}^{N}a_{n}\]
		una successió tals que
		\[\lim_{n\to\infty}S_{n}=L\]
		on \(L\) és un nombre real. Aleshores direm que la sèrie \(\sum_{n=1}^{\infty}a_{n}\) és una sèrie convergent amb valor \(L\). Si \(L\) no és un nombre real direm que la sèrie \(\sum_{n=1}^{\infty}a_{n}\) és una sèrie divergent.
	\end{definition}
	\begin{example}[Sèries geomètriques]
		\labelname{sèries geomètriques}\label{ex:sèries geomètriques}
		Sigui \(r\neq0\) un nombre real. Volem estudiar la convergència de la sèrie
		\[\sum_{n=0}^{\infty}r^{n}.\]
		\begin{solution}
			Tenim que per a tot \(N\) natural
			\begin{equation}
				\label{ex:sèries geomètriques:eq1}
				\sum_{n=0}^{N}r^{n}=\frac{1-r^{N+1}}{1-r}
			\end{equation}
			ja que \(\sum_{n=0}^{N}r^{n}=r^{0}+r^{1}+r^{2}+\cdots\), i per tant
			\begin{align*}
				(1-r)\sum_{n=0}^{N}r^{n}&=(1-r)(r^{0}+r^{1}+r^{2}+\cdots)\\
				&=r^{0}+r^{1}+r^{2}+\dots+r^{n}-(r^{1}+r^{2}+r^{3}+\dots+r^{n+1})\\
				&=1-r^{N+1}\\
			\end{align*}
			i trobem
			\[\sum_{n=0}^{N}r^{n}=\frac{1-r^{N+1}}{1-r}.\]
			Ara bé, tenim
			\[\lim_{N\to\infty}\frac{1-r^{N+1}}{1-r}=
			\begin{cases}
				\displaystyle \frac{1}{1-r} & \text{si }-1<r<1 \\
				+\infty & \text{si }r\leq1 \\
				\text{no existeix} & \text{si }r\leq-1 
			\end{cases}\]
			i per la definició de \myref{def:sèrie convergent} trobem que \eqref{ex:sèries geomètriques:eq1} és convergent si i només si \(r\) pertany a l'interval \((-1,1)\).
		\end{solution}
	\end{example}
	\begin{theorem}[Condició de Cauchy]
		\labelname{condició de Cauchy}\label{thm:condició de Cauchy}
		Sigui \(\sum_{n=0}^{\infty}a_{n}\) una sèrie numèrica. Aleshores \(\sum_{n=0}^{\infty}a_{n}\) és convergent si i només si per a tot \(\varepsilon>0\) existeix un \(n_{0}\) natural tal que per a tot \(N\) i \(M\) naturals amb \(N,M\geq n_{0}\) tenim
		\[\abs{\sum_{n=N}^{M}a_{n}}<\varepsilon.\]
		\begin{proof}
			Suposem que \(\sum_{n=0}^{\infty}a_{n}\) és convergent. Per la definició de \myref{def:sèrie convergent} això és que
			\[\lim_{N\to\infty}\sum_{n=1}^{N}{a_{n}}=L\]
			on \(L\) és un nombre real. Per la definició de \myref{def:límit} tenim que per a tot \(\delta_{1}>0\) existeix un \(N\) tal que
			\[\abs{\sum_{n=1}^{N-1}{a_{n}}-L}<\delta_{1}\]
			i de nou per la definició de \myref{def:límit} tenim que per a tot \(\varepsilon>0\) existeix un \(\delta_{2}>0\) amb \(\delta_{1}-\delta_{2}<\varepsilon\) tal que existeix un \(M\) satisfent
			\[\abs{\sum_{n=1}^{M}-L}<\delta_{2}\]
			Per tant trobem
			\[\abs{\sum_{n=1}^{N-1}-L}-\abs{\sum_{n=1}^{M}-L}<\delta_{1}-\delta_{2}\]
			i per la desigualtat triangular %TODO %REF
			tenim que
			\begin{align*}
				\abs{\sum_{n=N}^{M}a_{n}}&=\abs{\sum_{n=1}^{N-1}-\sum_{n=1}^{M}}\\
				&=\abs{\sum_{n=1}^{N-1}-L-\sum_{n=1}^{M}+L}\\
				&\leq\abs{\sum_{n=1}^{N-1}-L}-\abs{\sum_{n=1}^{M}-L}\\
				&<\delta_{1}-\delta_{2}<\varepsilon.\qedhere
			\end{align*}
		\end{proof}
	\end{theorem}
	\begin{corollary}
		\label{cor:condició de Cauchy}\label{cor:terme general tendeix a zero en una sèrie convergent}
		Sigui \(\sum_{n=1}^{\infty}a_{n}\) una sèrie convergent. Aleshores
		\[\lim_{n\to\infty}a_{n}=0.\]
		\begin{proof}
			Per la \myref{thm:condició de Cauchy} tenim que per a tot \(\varepsilon>0\) existeix un \(n_{0}\) natural tal que per a tot \(N\) i \(M\) naturals amb \(N,M\leq n_{0}\) tenim
			\[\sum_{n=N}^{M}a_{n}<\varepsilon.\]
			En particular, si triem \(M=N+1\), tenim \(\abs{a_{n}}<\varepsilon\) per a tot \(n\leq n_{0}\), i per la definició de \myref{def:límit} veiem que \(\lim_{n\to\infty}a_{n}=0\), com volíem veure.
		\end{proof}
	\end{corollary}
	\begin{example}[Sèrie harmònica]
		\labelname{sèrie harmònica}\label{ex:sèrie harmònica}
		Sigui \(\alpha\) un nombre real. Volem estudiar la convergència de la sèrie
		\[\sum_{n=1}^{\infty}\frac{1}{n^{\alpha}}.\]
		\begin{solution}
			Observem que si \(\alpha\leq0\) tenim que
			\[\lim_{n\to\infty}\frac{1}{n^{\alpha}}=\infty\]
			i pel coro{\lgem}ari \myref{cor:condició de Cauchy} tenim que la sèrie és divergent.
			
			Suposem que \(0<\alpha\leq1\). Definim la successió
			\[S_{N}=\sum_{n=1}^{N}\frac{1}{n^{\alpha}}\]
			i tenim que
		\end{solution}
	\end{example}
\end{document}