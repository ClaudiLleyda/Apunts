\documentclass[../Apunts.tex]{subfiles}

\begin{document}
\chapter{Matrius}
\section{Matrius i operacions}
\subsection{Cossos}
	\begin{definition}[Cos]
		Aclarir on va aquest apartat.
	\end{definition}
\subsection{Les matrius}
	\begin{definition}[Matriu]
		\labelname{matriu}\labelname{matriu quadrada}
	\end{definition}
	\begin{definition}[Conjunt de matrius]
	\end{definition}
	\begin{definition}[Suma de matrius]
	\end{definition}
	\begin{definition}[Producte de matrius]
	\end{definition}
\subsection{Operacions amb matrius}
\section{Rang d'una matriu}
\subsection{Transformacions elementals}
\subsection{Matrius esglaonades i mètode de Gauss}
\subsection{Teorema de la PAQ}
\subsection{Càlcul de la inversa d'una matriu invertible}
\section{Sistemes d'equacions lineals}
\subsection{Teorema de Rouché-Frobenius}
\subsection{Determinant d'una matriu}

\chapter{Espais vectorials i aplicacions lineals}
\section{Espais vectorials}

\section{Aplicacions lineals}

\chapter{Classificació d'endomorfismes}
\section{Endomorfismes similars}
\section{Diagonalització}
\section{Forma canònica de Jordan}
\section{Forma canònica racional}

\chapter{Espais vectorials euclidians}

\end{document}
