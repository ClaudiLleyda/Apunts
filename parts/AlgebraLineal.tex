\documentclass[../Apunts.tex]{subfiles}

\begin{document}
\chapter{Matrius}
	\section{Matrius i operacions}
	\subsection{Cossos}
	\begin{definition}[Cos]
		\labelname{cos}\label{def:cos}
		Aclarir on va aquest apartat.
	\end{definition}
	% elements 1 i 0 i tal i qual
	\subsection{Les matrius}
	\begin{definition}[Matriu]
		\labelname{matriu}\label{def:matriu}
		\labelname{matriu quadrada}\label{def:matriu quadrada}
		Siguin \(m\) i \(n\) dos enters i \(\{a_{i,j}\}^{1\leq i\leq m}_{1\leq j\leq n}\) elements d'un cos \(\mathbb{K}\). Aleshores direm que
		\[[a_{i,j}]=
		\left[\begin{matrix}
		a_{1,1} & a_{1,2} & \cdots & a_{1,n} \\
		a_{2,1} & a_{2,2} & \cdots & a_{2,n} \\
		\vdots & \vdots &  & \vdots \\
		a_{m,1} & a_{m,2} & \cdots & a_{m,n}
		\end{matrix}\right]\]
		és una matriu de mida \(m\times n\) sobre \(\mathbb{K}\).
		
		Direm que \(a_{i,j}\) és el component \((i,j)\) de la matriu \([a_{i,j}]\). Si \(A=[a_{i,j}]\) també denotarem \([A]_{i,j}=a_{i,j}\).
		
		Si \(n=m\) direm que \((a_{i,j})\) és una matriu quadrada d'ordre \(n\).
	\end{definition}
	\begin{notation}[Conjunt de matrius]
		\label{notation:conjunt de matrius}
		Siguin \(m\) i \(n\) dos enters, \(\mathbb{K}\) un cos i
		\[X=\{A\mid A\text{ és una matriu de mida }m\times n\text{ sobre }\mathbb{K}\}\]
		un conjunt. Aleshores denotarem \(X=M_{m\times n}(\mathbb{K})\).
		
		Si \(n=m\) denotarem \(X=M_{n}(\mathbb{K})\).
	\end{notation}
	%Veure que estàn ben definides %TODO %FER
	%Quan veiem que el producte té sentit veure que pertany a \(M_{m\times n}(\mathbb{K})\).
	\begin{definition}[Suma de matrius]
		\labelname{suma de matrius}\label{def:suma de matrius}
		Siguin \(A\) i \(B\) dues matrius de mida \(m\times n\) sobre un cos \(\mathbb{K}\). Aleshores definim la seva suma com una operació \(+\) que satisfà
		\[[A+B]_{i,j}=[A]_{i,j}+[B]_{i,j}.\]
	\end{definition}
	\begin{definition}[Producte de matrius]
		\labelname{producte de matrius}\label{def:producte de matrius}
		Siguin \(A\) una matriu de mida \(m\times l\) sobre un cos \(\mathbb{K}\) i \(B\) una matriu de mida \(l\times n\) sobre un cos \(\mathbb{K}\). Aleshores definim el seu producte com una operació \(\cdot\) que satisfà
		\[[A\cdot B]_{i,j}=\sum_{k=1}^{l}[A]_{i,k}[B]_{k,j}.\]
		
		Escriurem \(A\cdot B=AB\).
	\end{definition}
	\subsection{Propietats de les operacions amb matrius}
	\begin{proposition}
		\label{prop:associativitat suma matrius}
		Siguin \(A\), \(B\) i \(C\) tres matrius de mida \(m\times n\) sobre un cos \(\mathbb{K}\). Aleshores
		\[(A+B)+C=A+(B+C).\]
		\begin{proof}
			Per la definició de \myref{def:suma de matrius} tenim que
			\begin{align*}
			[(A+B)+C]_{i,j}&=[A+B]_{i,j}+[C]_{i,j}\\
			&=[A]_{i,j}+[B]_{i,j}+[C]_{i,j}\\
			&=[A]_{i,j}+([B]_{i,j}+[C]_{i,j})\tag{\myref{def:cos}}\\
			&=[A]_{i,j}+[B+C]_{i,j}\\
			&=[A+(B+C)]_{i,j}.\qedhere
			\end{align*}
		\end{proof}
	\end{proposition}
	\begin{proposition}
		\label{prop:commutativitat suma matrius}
		Siguin \(A\) i \(B\) dues matrius de mida \(m\times n\) sobre un cos \(\mathbb{K}\). Aleshores
		\[A+B=B+A.\]
		\begin{proof}
			Per la definició de \myref{def:suma de matrius} tenim que
			\begin{align*}
			[A+B]_{i,j}&=[A]_{i,j}+[B]_{i,j}\\
			&=[B]_{i,j}+[A]_{i,j}\tag{\myref{def:cos}}\\
			&=[B+A]_{i,j}.\qedhere
			\end{align*}
		\end{proof}
	\end{proposition}
	\begin{notation}[Matriu nul·la]
		\labelname{matriu nul·la}\label{notation:matriu nula}\label{notation:matriu zero}
		Denotarem la matriu de mida \(m\times n\)
		\[\left[\begin{matrix}
		0 & \cdots & 0 \\
		\vdots & & \vdots \\
		0 & \cdots & 0
		\end{matrix}\right]
		\in M_{m\times n}(\mathbb{K})\]
		com \(0_{m\times n}\), ó \(0\) si és clar pel context.
	\end{notation}
	\begin{proposition}
		\label{prop:element neutre per la suma de matrius}
		Sigui \(A\) una matriu de mida \(m\times n\) sobre un cos \(\mathbb{K}\). Aleshores
		\[A+0=A.\]
		\begin{proof}
			Per la definició de \myref{def:suma de matrius} tenim que
			\begin{align*}
			[A+0]_{i,j}&=[A]_{i,j}+[0]_{i,j}\\
			&=[A]_{i,j}+0\\
			&=[A]_{i,j},\tag{\myref{def:cos}}
			\end{align*}
			com volíem veure.
		\end{proof}
	\end{proposition}
	\begin{definition}[Producte d'una matriu per escalars]
		\labelname{producte d'una matriu per escalars}\label{def:producte d'una matriu per escalars}
		Sigui \(A\) una matriu de mida \(m\times n\) sobre un cos \(\mathbb{K}\) i \(\alpha\) un escalar de \(\mathbb{K}\). Aleshores definim
		\[[\alpha A]_{i,j}=\alpha[A]_{i,j}.\]
		
		Denotarem \((-1)A\) com \(-A\).
	\end{definition}
	\begin{proposition}
		\label{prop:inverses per la suma de matrius}
		Sigui \(A\) una matriu de mida \(m\times n\) sobre un cos \(\mathbb{K}\). Aleshores
		\[A-A=0.\]
		\begin{proof}
			Per la definició de \myref{def:suma de matrius} tenim que
			\begin{align*}
			[A-A]_{i,j}&=[A]_{i,j}+[-A]_{i,j}\\
			&=[A]_{i,j}-[A]_{i,j}\tag{\myref{def:producte d'una matriu per escalars}}\\
			&=0\tag{\myref{def:cos}} %REF?
			\end{align*}
			i trobem \(A-A=0\).
		\end{proof}
	\end{proposition}
	\begin{proposition}
		\label{prop:associativitat mixta producte escalars per matrius}
		Siguin \(A\) una matriu de mida \(m\times n\) i \(\alpha\) i \(\beta\) dos escalars de \(\mathbb{K}\). Aleshores
		\[(\alpha\beta)A=\alpha(\beta A).\]
		\begin{proof}
			Per la definició de \myref{def:producte d'una matriu per escalars} tenim que
			\begin{align*}
			[(\alpha\beta)A]_{i,j}&=\alpha\beta[A]_{i,j}\\
			&=\alpha(\beta[A_{i,j}])\tag{\myref{def:cos}}\\
			&=\alpha[\beta A]_{i,j}.\qedhere
			\end{align*}
		\end{proof}
	\end{proposition}
	\begin{proposition}
		\label{prop:distributiva respecta la suma d'escalars del producte de matrius}
		Siguin \(A\) una matriu de mida \(m\times n\) sobre un cos \(\mathbb{K}\) i \(\alpha\) i \(\beta\) dos escalars de \(\mathbb{K}\). Aleshores
		\[(\alpha+\beta)A=\alpha A+\beta A.\]
		\begin{proof}
			Per la definició de \myref{def:producte d'una matriu per escalars} tenim que
			\begin{align*}
			[(\alpha+\beta)A]_{i,j}&=(\alpha+\beta)[A]_{i,j}\\
			&=\alpha[A]_{i,j}+\beta[A]_{i,j},\tag{\myref{def:cos}}\\
			&=[\alpha A]_{i,j}+[\beta A]_{i,j}
			\end{align*}
			i hem acabat.
		\end{proof}
	\end{proposition}
	\begin{proposition}
		Sigui \(A\) una matriu de mida \(m\times n\) sobre un cos \(\mathbb{K}\). Aleshores
		\[1A=A.\]
		\begin{proof}
			Per la definició de \myref{def:producte d'una matriu per escalars} tenim que
			\[[1A]_{i,j}=1[A]_{i,j}\]
			i per la definició de \myref{def:cos} trobem que
			\[1[A]_{i,j}=[A]_{i,j},\]
			com volíem veure.
		\end{proof}
	\end{proposition}
	\begin{proposition}
		\label{prop:associativitat producte de matrius}
		Siguin \(\mathbb{K}\) un cos i \(A\) una matriu de mida \(m\times l\) sobre \(\mathbb{K}\), \(B\) una matriu de mida \(l\times s\) sobre \(\mathbb{K}\) i \(C\) una matriu de mida \(s\times m\) sobre \(\mathbb{K}\). Aleshores
		\[(AB)C=A(BC).\]
		\begin{proof}
			Per la definició de \myref{def:producte de matrius} tenim que
			\begin{align*}
			[(AB)C]_{i,j}&=\sum_{r=1}^{s}[AB]_{i,r}[C]_{r,j}\\
			&=\sum_{r=1}^{s}\left(\sum_{k=1}^{l}[A]_{i,k}[B]_{k,r}\right)[C]_{r,j}\\
			&=\sum_{r=1}^{s}\sum_{k=1}^{l}[A]_{i,k}[B]_{k,r}[C]_{r,j}\tag{\myref{def:cos}}\\
			&=\sum_{k=1}^{l}\sum_{r=1}^{s}[A]_{i,k}[B]_{k,r}[C]_{r,j}\tag{\myref{def:cos}}\\
			&=\sum_{k=1}^{l}[A]_{i,k}\sum_{r=1}^{s}\left([B]_{k,r}[C]_{r,j}\right)\tag{\myref{def:cos}}\\
			&=\sum_{k=1}^{l}[A]_{i,k}[BC]_{k,j}\\
			&=[A(BC)]_{i,j},
			\end{align*}
			com volíem veure.
		\end{proof}
	\end{proposition}
	\begin{proposition}
		\label{prop:distributiva del producte de matrius respecte la suma}
		Siguin \(A\) i \(A'\) dues matrius de mida \(m\times l\) sobre un cos \(\mathbb{K}\) i \(B\), \(B'\) dues matrius de mida \(l\times n\) sobre un cos \(\mathbb{K}\). Aleshores
		\[(A+A')B=AB+A'B\quad\text{i}\quad A(B+B')=AB+AB'.\]
		\begin{proof}
			Per la definició de \myref{def:producte de matrius} tenim que
			\begin{align*}
			[(A+A')B]_{i,j}&=\sum_{k=1}^{l}[A+A']_{i,k}[B]_{k,j}\\
			&=\sum_{k=1}^{l}\left([A]_{i,k}+[A']_{i,k}\right)[B]_{k,j}\tag{\myref{def:suma de matrius}}\\
			&=\sum_{k=1}^{l}\left([A]_{i,k}[B]_{k,j}+[A']_{i,k}[B]_{k,j}\right)\tag{\myref{def:cos}}\\
			&=\sum_{k=1}^{l}[A]_{i,k}[B]_{k,j}+\sum_{k=1}^{l}[A']_{i,k}[B]_{k,j}\tag{\myref{def:cos}}\\
			&=[AB]_{i,j}+[A'B]_{i,j}\\
			&=[AB+AB']_{i,j}.\tag{\myref{def:suma de matrius}}
			\end{align*}
			
			L'altre demostració és anàloga.
%			i també tenim que
%			\begin{align*}
%			[A(B+B')]_{i,j}&=\sum_{k=1}^{l}[A]_{i,k}[B+B']_{k,j}\\
%			&=\sum_{k=1}^{l}[A]_{i,k}\left([B]_{k,j}+[B']_{k,j}\right)\tag{\myref{def:suma de matrius}}\\
%			&=\sum_{k=1}^{l}\left([A]_{i,k}[B]_{k,j}+[A]_{i,k}[B']_{k,j}\right)\tag{\myref{def:cos}}\\
%			&=\sum_{k=1}^{l}[A]_{i,k}[B]_{k,j}+\sum_{k=1}^{l}[A]_{i,k}[B]_{k,j}\tag{\myref{def:cos}}\\
%			&=[AB]_{i,j}+[AB']_{i,j}\\
%			&=[AB+AB']_{i,j},\tag{\myref{def:suma de matrius}}
%			\end{align*}
%			com volíem veure.
		\end{proof}
	\end{proposition}
	\subsection{Matrius inverses i matrius transposades}
	\begin{notation}[Matriu identitat]
		\labelname{matriu nul·la}\label{notation:matriu identitat}
		Denotarem la matriu d'ordre \(n\)
		\[\left[\begin{matrix}
		1 & 0 & \cdots & 0 \\
		0 & 1 & \ddots & \vdots \\
		\vdots & \ddots & \ddots & 0\\
		0 & \cdots & 0 & 1
		\end{matrix}\right]
		\in M_{n}(\mathbb{K})\]
		com \(I_{n}\) i direm que \(I_{n}\) és la matriu identitat d'ordre \(n\).
	\end{notation}
	\begin{proposition}
		\label{prop:producte per la matriu identitat}
		Sigui \(A\) una matriu de mida \(m\times n\) sobre un cos \(\mathbb{K}\). Aleshores
		\[I_{m}A=A=AI_{n}.\]
		\begin{proof}
			Per la definició de \myref{def:producte de matrius} tenim que
			\[[I_{m}A]_{i,j}=\sum_{k=1}^{m}[I_{m}]_{i,k}[A]_{k,j}.\]
			Ara bé, tenim que
			\[[I_{m}]_{i,j}=
			\begin{cases}
			1 & i=j\\
			0 & i\neq j
			\end{cases}\]
			i per la definició de \myref{def:cos} tenim que
			\[\sum_{k=1}^{m}[I_{m}]_{i,k}[A]_{k,j}=[A]_{i,j}\]
			i per tant \(I_{m}A=A\).
			
			La demostració de l'altre igualtat és anàloga.
		\end{proof}
	\end{proposition}
	\begin{proposition}
		\label{prop:unicitat inverses de matrius pel producte}
		Sigui \(A\) una matriu d'ordre \(n\) sobre un cos \(\mathbb{K}\) tal que existeixin dues matrius \(A_{1}\) i \(A_{2}\) d'ordre \(n\) sobre \(\mathbb{K}\) tal que
		\[AA_{1}=A_{1}A=I_{n}\quad\text{i}\quad AA_{2}=A_{2}A=I_{n}.\]
		Aleshores \(A_{1}=A_{2}\).
		\begin{proof}
			Per la proposició \myref{prop:associativitat producte de matrius} tenim que
			\[A_{1}(AA_{2})=(A_{1}A)A_{2}\]
			i per hipòtesi trobem que \(A_{1}I_{n}=I_{n}A_{2}\), i per la proposició \myref{prop:producte per la matriu identitat} trobem que \(A_{1}=A_{2}\), com volíem veure.
		\end{proof}
	\end{proposition}
	\begin{definition}[Matriu invertible]
		\labelname{matriu invertible}\label{def:matriu invertible}
		\labelname{inversa d'una matriu}\label{def:inversa d'una matriu}
		Sigui \(A\) una matriu d'ordre \(n\) sobre un cos \(\mathbb{K}\) tal que existeixi una matriu \(A'\) d'ordre \(n\) sobre \(\mathbb{K}\) tal que
		\[AA'=A'A=I_{n}.\]
		Aleshores direm que \(A\) és una matriu invertible i que \(A'\) és la inversa de \(A\). També denotarem \(A'=A^{-1}\).
		
		Observem que aquesta definició té sentit per la proposició \myref{prop:unicitat inverses de matrius pel producte}.
	\end{definition}
	\begin{observation}
		\label{obs:matrius amb una fila nula no són invertibles}
		\label{obs:matrius amb una columna nula no són invertibles}
		Sigui \(A=(a_{i,j}\) una matriu de mida \(m\times n\) sobre un cos \(\mathbb{K}\) tal que existeix un \(i\in\{1,\dots,n\}\) satisfent
		\[a_{i,j}=0\text{ ó }a_{j,i}=0\quad\text{per a tot }j\in\{1,\dots,n\}.\]
		Aleshores \(A\) no és invertible.
	\end{observation}
	\begin{proposition}
		\label{prop:el producte de matrius invertibles és invertible}
		Siguin \(A\) i \(B\) dues matrius invertibles d'ordre \(n\) sobre un cos \(\mathbb{K}\). Aleshores la matriu \(AB\) és invertible.
		\begin{proof}
			Per la proposició \ref{prop:associativitat producte de matrius} i la definició de \myref{def:matriu invertible} tenim que
			\[(AB)(B^{-1}A^{-1})=\Id\quad\text{i}\quad(B^{-1}A^{-1})(AB)=\Id,\]
			i de nou per la definició tenim que \(AB\) és invertible.
		\end{proof}
	\end{proposition}
	\begin{proposition}
		\label{prop:inversa del producte de matrius}
		Siguin \(A\) i \(B\) dues matrius invertibles d'ordre \(n\) sobre un cos \(\mathbb{K}\). Aleshores
		\[(AB)^{-1}=B^{-1}A^{-1}.\]
		
		Aquest enunciat té sentit per la proposició \myref{prop:el producte de matrius invertibles és invertible}.
		\begin{proof}
			Per la definició de \myref{def:matriu invertible} tenim que
			\[(AB)^{-1}AB=I_{n},\]
			i com que, per hipòtesi, les matrius \(A\) i \(B\) són invertibles trobem
			\begin{align*}
			B^{-1}A^{-1}&=(AB)^{-1}ABB^{-1}A^{-1}\\
			&=(AB)^{-1}AA^{-1}\\
			&=(AB)^{-1}.\qedhere
			\end{align*}
		\end{proof}
	\end{proposition}
	\begin{proposition}
		\label{prop:inversa de l'inversa d'una matriu és la matriu}
		Sigui \(A\) una matriu invertible d'ordre \(n\) sobre un cos \(\mathbb{K}\). Aleshores
		\[(A^{-1})^{-1}=A.\]
		\begin{proof}
			Per la proposició \myref{prop:inversa del producte de matrius} tenim que
			\[A^{-1}(A^{-1})^{-1}=A^{-1}A\]
			i com que, per hipòtesi, la matriu \(A\) és invertible trobem
			\[AA^{-1}(A^{-1})^{-1}=AA^{-1}A\]
			i tenim
			\[(A^{-1})^{-1}=A.\qedhere\]
		\end{proof}
	\end{proposition}
	\begin{definition}[Matriu transposada]
		\labelname{matriu transposada}\label{def:matriu transposada}
		\labelname{transposició d'una matriu}\label{def:transposició d'una matriu}
		Sigui \(A\) una matriu de mida \(m\times n\) sobre un cos \(\mathbb{K}\). Definim la transposada de \(A\) com una matriu \(A^{t}\) de mida \(n\times m\) sobre \(\mathbb{K}\) que satisfà
		\[[A^{t}]_{i,j}=[A]_{j,i}.\]
	\end{definition}
	\begin{proposition}
		Siguin \(A\) i \(B\) dues matrius de mida \(m\times n\) sobre un cos \(\mathbb{K}\). Aleshores
		\[(A+B)^{t}=A^{t}+B^{t}.\]
		\begin{proof}
			Per la definició de \myref{def:matriu transposada} tenim que
			\begin{align*}
			[(A+B)^{t}]_{i,j}&=[A+B]_{j,i}\\
			&=[A]_{j,i}+[B]_{j,i}\tag{\myref{def:suma de matrius}}\\
			&=[A^{t}]_{i,j}+[B^{t}]_{i,j}\\
			&=[A^{t}+B^{t}]_{i,j},\tag{\myref{def:suma de matrius}}
			\end{align*}
			com volíem veure.
		\end{proof}
	\end{proposition}
	\begin{proposition}
		\label{prop:producte de matrius transposades}
		Siguin \(A\) una matriu de mida \(m\times l\) sobre un cos \(\mathbb{K}\) i \(B\) una matriu de mida \(l\times n\) sobre \(\mathbb{K}\). Aleshores
		\[(AB)^{t}=B^{t}A^{t}.\]
		\begin{proof}
			Per la definició de \myref{def:matriu transposada} tenim que
			\begin{align*}
			[(AB)^{t}]_{i,j}&=[AB]_{j,i}\\
			&=\sum_{k=1}^{l}[A]_{j,k}[B]_{k,i}\tag{\myref{def:producte de matrius}}\\
			&=\sum_{k=1}^{l}[A^{t}]_{k,j}[B]_{i,k}\\
			&=\sum_{k=1}^{l}[B^{t}]_{i,k}[A^{t}]_{k,j}\tag{\myref{def:cos}}\\
			&=[B^{t}A^{t}]_{i,j},\tag{\myref{def:cos}}
			\end{align*}
			com volíem veure.
		\end{proof}
	\end{proposition}
	\begin{proposition}
		\label{prop:transposada d'una invertible és invertible}
		Sigui \(A\) una matriu invertible d'ordre \(n\) sobre un cos \(\mathbb{K}\). Aleshores la matriu \(A^{t}\) és invertible i \(\left(A^{t}\right)^{-1}=\left(A^{-1}\right)^{t}\).
		\begin{proof}
			Per la proposició \myref{prop:producte de matrius transposades} tenim que
			\[A^{t}\left(A^{-1}\right)^{t}=\left(A^{-1}A\right)^{t}\quad\text{i}\quad [\left(A^{-1}\right)^{t}A^{t}=\left(AA^{-1}\right)^{t},\]
			i per la definició de \myref{def:matriu invertible} tenim que \(A^{t}\) és invertible i que \(\left(A^{t}\right)^{-1}=\left(A^{-1}\right)^{t}\), com volíem veure.
		\end{proof}
	\end{proposition}
	\begin{definition}[Matriu simètrica]
		\labelname{matriu simètrica}\label{def:matriu simètrica}
		Sigui \(A\) una matriu d'ordre \(n\) sobre un cos \(\mathbb{K}\) tal que \(A^{t}=A\). Aleshores direm que \(A\) és una matriu simètrica.
	\end{definition}
	\subsection{Producte de matrius en blocs}%FER producte de matrius en blocs
	\begin{definition}[Files i columnes]
		\labelname{fila d'una matriu}\label{def:fila d'una matriu}
		\labelname{columna d'una matriu}\label{def:columna d'una matriu}
		Sigui \(A=(a_{i,j})\) una matriu de mida \(m\times n\) sobre un cos \(\mathbb{K}\). Aleshores definim
		\[F_{i}=\left[\begin{matrix}
		a_{i,1} & \cdots & a_{i,n}
		\end{matrix}\right]\]
		com la \(i\)-èsima fila de \(A\) i
		\[C_{i}=\left[\begin{matrix}
		a_{1,i} \\
		\vdots \\
		a_{m,i}
		\end{matrix}\right]\]
		com la \(i\)-èsima columna de \(A\).
	\end{definition}
	\begin{observation}
		\label{obs:on pertanyen les files i les columnes d'una matriu}
		Si \(A\in M_{m\times n}(\mathbb{K})\) tenim \(F_{i}\in M_{1\times n}(\mathbb{K})\) i \(C_{i}\in M_{m\times 1}(\mathbb{K})\).
	\end{observation}
	\begin{notation}
		Si \(A\) és una matriu de mida \(m\times n\) sobre un cos \(\mathbb{K}\), \(F_{1},\dots,F_{m}\) les files de \(A\) i \(C_{1},\dots,C_{n}\) les columnes de \(A\). Aleshores denotem
		\[A=\left[\begin{matrix}
		F_{1} \\
		\vdots \\
		F_{m}
		\end{matrix}\right]=\left[\begin{matrix}
		C_{1} & \cdots & C_{n}
		\end{matrix}\right].\]
	\end{notation}
	\begin{notation}
		Siguin \(A=(a_{i,j})\) una matriu de mida \(m\times n\) sobre un cos \(\mathbb{K}\), \(m_{1},\dots,m_{p}\) i \(n_{1},\dots,n_{q}\) nombres naturals tals que \(m_{1}+\dots+m_{p}=m\) i \(n_{1}+\dots+n_{q}=n\), \(m_{0}=n_{0}=1\) i
		\[A_{i,j}=\left[\begin{matrix}
		a_{m_{i-1},n_{i-1}} & \cdots & a_{m_{i-1},n_{i}}\\
		\vdots & & \vdots \\
		a_{m_{i},n_{i-1}} & \cdots & a_{m_{i},n_{i}}
		\end{matrix}\right]\]
		matrius sobre un cos \(\mathbb{K}\). Aleshores denotem
		\[A=\left[\begin{matrix}
		A_{1,1} & \cdots & A_{1,q} \\
		\vdots & & \vdots \\
		A_{p,1} & \cdots & A_{p,q}
		\end{matrix}\right].\]
	\end{notation}
	\begin{proposition}
		\label{prop:producte de matrius en blocs}
		Siguin \(A\) una matriu de mida \(m\times l\) sobre un cos \(\mathbb{K}\), \(B\) una matriu de mida \(l\times n\) sobre \(\mathbb{K}\), \(m_{1},\dots,m_{p}\), \(l_{1},\dots,l_{q}\) i \(n_{1},\dots,n_{r}\) nombres naturals satisfent \(m_{1}+\dots+m_{p}=m\), \(l_{1}+\dots+l_{q}=l\) i \(n_{1}+\dots+n_{r}=n\), i per a tot \(i\in\{1,\dots,p\}\), \(k\in\{1,\dots,q\}\) i \(j\in\{1,\dots,r\}\) tals que \(A_{i,k}\) és una matriu de mida \(m_{i}\times l_{k}\) sobre \(\mathbb{K}\) i \(B_{k,j}\) és una matriu de mida \(l_{k}\times n_{j}\) sobre \(\mathbb{K}\) tals que
		\[A=\left[\begin{matrix}
		A_{1,1} & \cdots & A_{1,q} \\
		\vdots & & \vdots \\
		A_{p,1} & \cdots & A_{p,q}
		\end{matrix}\right]\quad\text{i}\quad B=\left[\begin{matrix}
		B_{1,1} & \cdots & B_{1,r} \\
		\vdots & & \vdots \\
		B_{q,1} & \cdots & B_{q,r}
		\end{matrix}\right].\]
		Aleshores
		\[AB=\left[\begin{matrix}
		\sum_{k=1}^{q}A_{1,k}B_{k,1} & \sum_{k=1}^{q}A_{1,k}B_{k,2} & \cdots & \sum_{k=1}^{q}A_{1,k}B_{k,r} \\
		\sum_{k=1}^{q}A_{2,k}B_{k,1} & \sum_{k=1}^{q}A_{2,k}B_{k,2} & \cdots & \sum_{k=1}^{q}A_{2,k}B_{k,r} \\
		\vdots & \vdots & & \vdots \\
		\sum_{k=1}^{q}A_{p,k}B_{k,2} & \sum_{k=1}^{q}A_{p,k}B_{k,2} & \cdots & \sum_{k=1}^{q}A_{p,k}B_{k,r}
		\end{matrix}\right].\]
		\begin{proof}
			Per la definició de \myref{def:producte de matrius} tenim que
			\[[AB]_{i,j}=\sum_{k=1}^{l}[A]_{i,k}[B]_{k,j}.\]
			
			També tenim que \(i=m_{1}+\dots+m_{k_{i}}+i'\) i \(j=n_{1}+\dots+n_{k_{j}}+j'\) per a certs \(k_{i}\in\{1,\dots,p-1\}\), \(k_{j}\in\{1,\dots,r-1\}\), \(i'\leq m_{k_{i}+1}-m_{k_{i}}\) i \(j'\leq n_{k_{j}+1}-n_{k_{j}}\). Per tant hem de veure que
			\[\sum_{k=1}^{l}[A]_{i,k}[B]_{k,j}=\left[\sum_{k=1}^{q}A_{m_{k_{i}},k}B_{k,n_{k_{j}}}\right]_{i',j'}.\]
			
			Per la definició de \myref{def:suma de matrius} tenim que
			\[\left[\sum_{k=1}^{q}A_{m_{k_{i}},k}B_{k,n_{k_{j}}}\right]_{i',j'}=\sum_{k=1}^{q}[A_{m_{k_{i}},k}B_{k,n_{k_{j}}}]_{i',j'},\]
			i per la definició de \myref{def:producte de matrius} trobem
			\[\sum_{k=1}^{q}[A_{m_{k_{i}},k}B_{k,n_{k_{j}}}]_{i',j'}=\sum_{k=1}^{q}\left(\sum_{h=1}^{l_{k}}[A_{m_{k_{i}},k}]_{i',h}[B_{k,n_{k_{j}}}]_{h,j'}\right).\]
			
			Ara bé, com que per hipòtesi tenim que
			\[A=\left[\begin{matrix}
			A_{1,1} & \cdots & A_{1,q} \\
			\vdots & & \vdots \\
			A_{p,1} & \cdots & A_{p,q}
			\end{matrix}\right]\quad\text{i}\quad B=\left[\begin{matrix}
			B_{1,1} & \cdots & B_{1,r} \\
			\vdots & & \vdots \\
			B_{q,1} & \cdots & B_{q,r}
			\end{matrix}\right].\]
			i que \(i=m_{1}+\dots+m_{k_{i}}+i'\) i \(j=n_{1}+\dots+n_{k_{j}}+j'\), trobem
			\[\sum_{k=1}^{q}\left(\sum_{h=1}^{l_{k}}[A_{m_{k_{i}},k}]_{i',h}[B_{k,n_{k_{j}}}]_{h,j'}\right)=\sum_{k=1}^{l}[A]_{i,k}[B]_{k,j}.\qedhere\]
		\end{proof}
	\end{proposition}
	%https://math.hecker.org/2011/04/07/multiplying-block-matrices/
	\section{Rang d'una matriu}
	\subsection{Transformacions elementals}
	\begin{definition}[Transformacions elementals] %REFER per columnes i files
		\labelname{transformacions elementals}\label{def:transformacions elementals}
		Siguin \(A=(a_{i,j})\) una matriu de mida \(m\times n\) sobre un cos \(\mathbb{K}\) i \(i\) i \(j\) dos naturals amb \(i,j\leq m\). Aleshores definim les funcions
		\[P_{i,j}(A)=\left[\begin{matrix}
		a_{1,1} & \cdots & a_{1,n} \\
		\vdots & & \vdots \\
		a_{i-1,1} & \cdots & a_{i-1,n} \\
		a_{j,1} & \cdots & a_{j,n} \\
		a_{i+1,1} & \cdots & a_{i+1,n} \\
		\vdots & & \vdots \\
		a_{j-1,1} & \cdots & a_{j-1,n} \\
		a_{i,1} & \cdots & a_{i,n} \\
		a_{j+1,1} & \cdots & a_{j+1,n} \\
		\vdots & & \vdots \\
		a_{m,1} & \cdots & a_{m,n}
		\end{matrix}\right],\quad D_{i,\lambda}(A)=\left[\begin{matrix}
		a_{1,1} & \cdots & a_{1,n} \\
		\vdots & & \vdots \\
		a_{i-1,1} & \cdots & a_{i-1,n} \\
		\lambda a_{i,1} & \cdots & \lambda a_{i,n} \\
		a_{i+1,1} & \cdots & a_{i+1,n} \\
		\vdots & & \vdots \\
		a_{m,1} & \cdots & a_{m,n}
		\end{matrix}\right]\]
		\[\text{i}\quad E_{i,j,\lambda}(A)=\left[\begin{matrix}
		a_{1,1} & \cdots & a_{1,n} \\
		\vdots & & \vdots \\
		a_{i-1,1} & \cdots & a_{i-1,n} \\
		a_{i,1}-\lambda a_{j,1} & \cdots & a_{i,n}-\lambda a_{j,n}\\
		a_{i+1,1} & \cdots & a_{i+1,n} \\
		\vdots & & \vdots \\
		a_{m,1} & & a_{m,n}
		\end{matrix}\right]\]
		com les transformacions elementals per files de \(M_{n}(\mathbb{K})\).
	\end{definition}
%	\begin{observation}
%		\label{obs:transformacions elementals}
%		Tenim que
%		\[P_{i,j}(I_{n})=\left[\begin{matrix}
%		1 & & & & & & & & & & \\
%		& \ddots & & & & & & & & & \\
%		& & 1 & & & & & & & & \\
%		& & & 0 & \cdots & \cdots & \cdots & 1 & & & \\
%		& & & \vdots & 1 & & & \vdots & & & \\
%		& & & \vdots & & \ddots & & \vdots & & & \\
%		& & & \vdots & & & 1 & \vdots & & & \\
%		& & & 1 & \cdots & \cdots & \cdots & 0 & & & \\
%		& & & & & & & & 1 & & \\
%		& & & & & & & & & \ddots & \\
%		& & & & & & & & & & 1
%		\end{matrix}\right],\]
%		\[D_{i,\lambda}(I_{n})=\left[\begin{matrix}
%		1 & & & & & & & \\
%		& \ddots & & & & & \\
%		& & 1 & & & & \\
%		& & & \lambda & & & \\
%		& & & & 1 & & \\
%		& & & & & \ddots & \\
%		& & & & & & 1
%		\end{matrix}\right],\]
%		i
%		\[E_{i,j,\lambda}(I_{n})=\left[\begin{matrix}
%		1 & & & & & & \\
%		& \ddots & & & & & \\
%		& & 1 & \cdots & \lambda & & \\
%		& & & \ddots & \vdots & \\
%		& & & & 1 & & \\
%		& & & & & \ddots & \\
%		& & & & & & 1
%		\end{matrix}\right].\]
%	\end{observation}
	\begin{proposition}
		\label{prop:transformacions elementals i matrius elementals}
		Sigui \(A\) una matriu de mida \(m\times n\) sobre un cos \(\mathbb{K}\). Aleshores per a tota successió de transformacions elementals per files \(f\) existeix una matriu \(P\) d'ordre \(m\) sobre \(\mathbb{K}\) tal que \(f(A)=PA\) i per a tota successió de transformacions elementals per columnes \(g\) existeix una matriu \(Q\) d'ordre \(n\) sobre \(\mathbb{K}\) tal que \(g(A)=AQ\).
%		Siguin \(A\) una matriu de mida \(m\times n\) sobre un cos \(\mathbb{K}\), \(f\) una transformació elemental per files i \(g\) una transformació elemental per columnes.
%		
%		Aleshores existeix una matriu \(P\) d'ordre \(m\) sobre \(\mathbb{K}\) tal que \(f(A)=PA\) i existeix una matriu \(Q\) d'ordre \(n\) sobre \(\mathbb{K}\) tal que \(g(A)=AQ\).
		\begin{proof}
			En tenim prou amb veure que \(P_{i,j}(A)=P_{i,j}(I_{m})A\), \(D_{i,\lambda}(A)=D_{i,\lambda}(I_{m})A\) i \(E_{i,j,\lambda}(A)=E_{i,j,\lambda}(I_{m})A\) i que \(P_{i,j}(A^{t})^{t}=AP_{i,j}(I_{n})\), \(D_{i,\lambda}(A^{t})^{t}=AD_{i,\lambda}(I_{n})\) i \(E_{i,j,\lambda}(A^{t})^{t}=AE_{i,j,\lambda}(I_{n})\), i això és conseqüència de la definició de \myref{def:producte de matrius}.
		\end{proof}
	\end{proposition}
	\begin{definition}[Matriu elemental]
		\labelname{matriu elemental}\label{def:matriu elemental}
		Sigui \(P\) una matriu d'ordre \(n\) sobre un cos \(\mathbb{K}\) tal que existeixi una sèrie de transformacions elementals tals que per a tota matriu \(A\) de mida \(m\times n\) sobre \(\mathbb{K}\) tenim \(f(A)=PA\). Aleshores direm que \(P\) és una matriu elemental.
		
		Aquesta definició té sentit per la proposició \myref{prop:transformacions elementals i matrius elementals}.
	\end{definition}
	\begin{observation}
		\label{obs:producte de matrius elementals és matriu elemental}
		Siguin \(P\) i \(Q\) dues matrius elementals d'ordre \(n\) sobre un cos \(\mathbb{K}\). Aleshores \(PQ\) és una matriu elemental.
	\end{observation}
	\begin{notation}
		Denotarem les transformacions elementals per files \(P_{i,j}\) com \(F_{i}\leftrightarrow F_{j}\), \(D_{i,\lambda}\) com \(F_{i}\rightarrow\lambda F_{i}\) i \(E_{i,j,\lambda}\) com \(F_{i}\rightarrow F_{i}+\lambda F_{j}\).
		
		Aprofitant la proposició \myref{prop:transformacions elementals i matrius elementals} també denotarem una successió de transformacions elementals com una matriu quadrada.
		
		Si apliquem una transformació elemental per files a una matriu transposada i transposem el resultat denotarem \(P_{i,j}\) com \(C_{i}\leftrightarrow C_{j}\), \(D_{i,\lambda}\) com \(C_{i}\rightarrow\lambda C_{i}\) i \(E_{i,j,\lambda}\) com \(C_{i}\rightarrow C_{i}+\lambda C_{j}\) i direm que són transformacions elementals per columnes.
	\end{notation}
	\begin{proposition}
		\label{prop:les matrius elementals són invertibles}
		Sigui \(P\) una matriu elemental d'ordre \(n\) sobre un cos \(\mathbb{K}\). Aleshores \(P\) és una matriu invertible.
		\begin{proof}
			En tenim prou amb veure que \(P_{i,j}^{-1}=P_{i,j}\), \(D_{i,\lambda}^{-1}=D_{i,\lambda^{-1}}\) i \(E_{i,j,\lambda}^{-1}=E_{i,j,-\lambda}\). Per tant, per la definició de \myref{def:matriu elemental} i la definició de \myref{def:matriu invertible} tenim que \(P\) és una matriu invertible.
		\end{proof}
	\end{proposition}
	\subsection{Matrius esglaonades i mètode de Gauss}
	\begin{definition}[Matriu esglaonada]
		\labelname{matriu esglaonada}\label{def:matriu esglaonada}
		Sigui \(A=(a_{i,j})\) una matriu de mida \(m\times n\) sobre un cos \(\mathbb{K}\) tal que existeixen \(j_{1}<\dots<j_{r}\) nombres naturals tals que per a tot \(i\in\{1,\dots,r\}\) i per a tot \(j<j_{i}\) tenim \(a_{i,j}=0\) i \(a_{i,j_{i}}=1\), i \(a_{i,j}=0\) per a tot \(i>j_{r}\) i \(j\in\{1,\dots,n\}\).
		
		Aleshores direm que \(A\) està esglaonada per files i que \(A^{t}\) està esglaonada per columnes.
		
		Direm que una matriu \(A\) es pot esglaonar per files si existeix una successió \(P\) de transformacions elementals per files tals que \(PA\) està esglaonada per files.
	\end{definition}
	\begin{proposition}
		\label{prop:es pot esglaonar qualsevol matriu per files}
		Sigui \(A\) una matriu de mida \(m\times n\) sobre un cos \(\mathbb{K}\). Aleshores \(A\) es pot esglaonar per files.
		\begin{proof} %REFER a partir de l'hipotesi d'inducció
			Denotem \(A=(a_{i,j})\). Ho farem per inducció sobre \(n\).
			
			Suposem que \(n=1\). Si \(a_{i,1}=0\) per a tot \(i\in\{1,\dots,m\}\) per la definició de \myref{def:matriu esglaonada} hem acabat. Suposem doncs que existeix un \(i\in\{1,\dots,m\}\) tal que \(a_{i,1}\neq0\). Aleshores podem fer les transformacions elementals
			\[A
			\overset{F_{i}\leftrightarrow F_{1}}{\longrightarrow}
			\left[\begin{matrix}
			a_{i,1}\\
			\vdots\\
			a_{i-1,1}\\
			a_{1,1}\\
			a_{i+1,1}\\
			\vdots\\
			a_{m,1}
			\end{matrix}\right]
			\overset{F_{1}\rightarrow a_{1,1}^{-1}F_{1}}{\longrightarrow}
			\left[\begin{matrix}
			1\\
			a_{2,1}\\
			\vdots\\
			a_{m,1}
			\end{matrix}\right]
			\overset{\overset{F_{i}\rightarrow F_{i}-a_{i,1}F_{1}}{\text{per a tot }i\in\{2,\dots,m\}}}{\longrightarrow}
			\left[\begin{matrix}
			1\\
			0\\
			\vdots\\
			0
			\end{matrix}\right]\]
			i per la definició de \myref{def:matriu esglaonada} hem acabat.
			
			Suposem doncs que la hipòtesi és certa per a \(n\). Veiem que també ho és per \(n+1\). Sigui \(A=(a_{i,j})\) una matriu de mida \(m\times(n+1)\) sobre \(\mathbb{K}\). Considerem la matriu
			\[A'=\left[\begin{matrix}
			a_{1,2} & \dots & a_{1,n+1} \\
			\vdots & & \vdots \\
			a_{m,2} & \dots & a_{m,n+1}
			\end{matrix}\right].\]
			Per la definició de \myref{def:matriu} tenim que \(A'\) és una matriu de mida \(m\times n\) sobre \(\mathbb{K}\), i per l'hipòtesi d'inducció tenim que podem esglaonar-la. Sigui doncs \(B\) la matriu \(A'\) esglaonada. Considerem la matriu
			\[\left[\begin{matrix}
			a_{1,1}\\
			\vdots\\
			a_{1,m}\\
			\end{matrix}\right],\]
			que correspon a la primera columna de la matriu \(A\). Hem vist que podem esglaonar-la, i per tant per la definició de \myref{def:matriu esglaonada} tenim que podem esglaonar la matriu \(A\), ja que
			\[A=\left[\begin{array}{c|c}
			a_{1,1} & \\
			\vdots & A' \\
			a_{1,m} & 
			\end{array}\right],\]
			i pel \myref{thm:principi d'inducció} hem acabat.
		\end{proof}
	\end{proposition}
	\begin{corollary}
		\label{corollary:es pot esglaonar qualsevol matriu per columnes}
		Sigui \(A\) una matriu de mida \(m\times n\) sobre un cos \(\mathbb{K}\). Aleshores \(A\) es pot esglaonar per columnes.
	\end{corollary}
	\subsection{Teorema de la \texorpdfstring{\(PAQ\)}{}-reducció}
	\begin{theorem}[Teorema de la \(PAQ\)-reducció]
		\labelname{Teorema de la $PAQ$-reducció}\label{thm:Teorema de la PAQ-reducció}
		Sigui \(A\) una matriu de mida \(m\times n\) sobre un cos \(\mathbb{K}\). Aleshores existeixen un únic nombre natural \(r\) satisfent \(r\leq\min(m,n)\), una matriu elemental \(P\) d'ordre \(m\) sobre \(\mathbb{K}\) i una matriu elemental \(Q\) d'ordre \(n\) sobre \(\mathbb{K}\) tals que
		\[PAQ=\left[\begin{array}{c|c}
		I_{r} & 0\\\hline
		0 & 0
		\end{array}\right].\]
		\begin{proof}
			Denotem \(A=(a_{i,j})\). Si \(a_{i,j}=0\) per a tot \(i\in\{1,\dots,m\}\) i \(j\in\{1,\dots,n\}\) aleshores tenim \(r=0\) i hem acabat. Suposem doncs que existeixen \(i\in\{1,\dots,m\}\) i \(j\in\{1,\dots,n\}\) tals que \(a_{i,j}\neq0\).
			
			Per la proposició \myref{prop:es pot esglaonar qualsevol matriu per files} tenim que existeix una matriu elemental \(P\) d'ordre \(m\) sobre \(\mathbb{K}\) tal que \(PA\) està esglaonada per files. Per la definició de \myref{def:matriu esglaonada} tenim que existeixen \(j_{i}\) per a tot \(i\in\{1,\dots,r\}\) amb \(j_{1}<\dots<j_{r}\) tals que per a tot \(j<j_{i}\) tenim \(a_{i,j}=0\) i \(a_{i,j_{i}}=1\). Fent les transformacions elementals
			\[C_{1}\leftrightarrow C_{j_{1}},\quad C_{2}\leftrightarrow C_{j_{2}},\dots,C_{r}\leftrightarrow C_{j_{r}}\]
			a la matriu \(PA\) obtenim la matriu
			\[A_{1}=\left[\begin{array}{cccc|ccc}
			1 & b_{1,2} & \cdots & b_{1,r} & b_{1,r+1} & \cdots & b_{1,n} \\
			0 & 1 & \ddots & \vdots & \vdots & & \vdots\\
			\vdots & \ddots & \ddots & b_{r-1,r} & \vdots & & \vdots \\
			0 & \cdots & 0 & 1 & b_{r,r+1} & \cdots & b_{r,n} \\\hline
			0 & \cdots & 0 & 0 & 0 & \cdots & 0 \\
			\vdots & & \vdots & \vdots & \vdots & & \vdots \\
			0 & \cdots & 0 & 0 & 0 & \cdots & 0
			\end{array}\right],\]
			i fent a la matriu \(A_{1}\) les transformacions elementals
			\begin{align*}
			C_{k_{1}}\rightarrow C_{k_{1}}-b_{1,k_{1}}C_{1},\quad&\text{per a }k_{1}\in\{2,\dots,n\} \\
			C_{k_{2}}\rightarrow C_{k_{2}}-b_{2,k_{2}}C_{2},\quad&\text{per a }k_{2}\in\{3,\dots,n\} \\
			\vdots & \\
			C_{k_{r}}\rightarrow C_{k_{r}}-b_{r,k_{r}}C_{r},\quad&\text{per a }k_{r}\in\{r+1,\dots,n\},
			\end{align*}
			i per la proposició \myref{prop:transformacions elementals i matrius elementals} tenim que existeix una matriu \(P\) d'ordre \(n\) sobre \(\mathbb{K}\) tal que
			\[PAQ=\left[\begin{array}{c|c}
			I_{r} & 0\\\hline
			0 & 0
			\end{array}\right],\]
			i per la definició de \myref{prop:transformacions elementals i matrius elementals} tenim que les matrius \(P\) i \(Q\) són matrius elementals. %FER %TODO ?
			
			Veiem ara que \(r\) és únic. Suposem que existeixen un nombre natural \(s\leq\min(m,n)\), una matriu elemental \(P_{1}\) d'ordre \(m\) sobre \(\mathbb{K}\) i una matriu elemental \(Q_{1}\) d'ordre \(n\) sobre \(\mathbb{K}\) tals que
			\[P_{1}AQ_{1}=\left[\begin{array}{c|c}
			I_{s} & 0\\\hline
			0 & 0
			\end{array}\right].\]
			Hem de veure que \(r=s\). Suposem que \(r\leq s\).
			
			Tenim que
			\[PAQ=\left[\begin{array}{c|c}
			I_{r} & 0\\\hline
			0 & 0
			\end{array}\right]\]
			i
			\[P_{1}AQ_{1}=\left[\begin{array}{c|c}
			I_{s} & 0\\\hline
			0 & 0
			\end{array}\right].\]
			Com que, per la proposició \myref{prop:les matrius elementals són invertibles}, les matrius \(P\), \(Q\), \(P_{1}\) i \(Q_{1}\) són invertibles tenim que
			\[A=P^{-1}\left[\begin{array}{c|c}
			I_{r} & 0\\\hline
			0 & 0
			\end{array}\right]Q^{-1}=P_{1}^{-1}\left[\begin{array}{c|c}
			I_{s} & 0\\\hline
			0 & 0
			\end{array}\right]Q_{1}^{-1},\]
			i per tant
			\[P_{1}P^{-1}\left[\begin{array}{c|c}
			I_{r} & 0\\\hline
			0 & 0
			\end{array}\right]=
			\left[\begin{array}{c|c}
			I_{s} & 0\\\hline
			0 & 0
			\end{array}\right]Q_{1}^{-1}Q.\]
			
			Denotem \(P_{2}=P_{1}P^{-1}\), \(Q_{2}=Q_{1}^{-1}Q\), \(P_{2}=(p_{i,j})\) i \(Q_{2}=(q_{i,j})\) i tenim
			\begin{multline*}
			\left[\begin{array}{ccc|ccc}
			p_{1,1} & \cdots & p_{1,r} & p_{1,r+1} & \cdots & p_{1,m} \\
			\vdots & & \vdots & \vdots & & \vdots \\
			p_{s,1} & \cdots & p_{s,r} & p_{s,r+1} & \cdots & p_{s,m} \\\hline
			p_{s+1,1} & \cdots & p_{s+1,r} & p_{s+1,r+1} & \cdots & p_{s+1,m} \\
			\vdots & & \vdots & \vdots & & \vdots \\
			p_{m,1} & \cdots & p_{m,r} & p_{m,r+1} & \cdots & p_{m,m}
			\end{array}\right]
			\left[\begin{array}{c|c}
			I_{r} & 0\\\hline
			0 & 0
			\end{array}\right]=\\=
			\left[\begin{array}{c|c}
			I_{s} & 0\\\hline
			0 & 0
			\end{array}\right]
			\left[\begin{array}{ccc|ccc}
			q_{1,1} & \cdots & q_{1,r} & q_{1,r+1} & \cdots & q_{1,n} \\
			\vdots & & \vdots & \vdots & & \vdots \\
			q_{s,1} & \cdots & q_{s,r} & q_{s,r+1} & \cdots & q_{s,n} \\\hline
			q_{s+1,1} & \cdots & q_{s+1,r} & q_{s+1,r+1} & \cdots & q_{s+1,n} \\
			\vdots & & \vdots & \vdots & & \vdots \\
			q_{n,1} & \cdots & q_{n,r} & q_{n,r+1} & \cdots & q_{n,n}
			\end{array}\right],
			\end{multline*}
			i per la proposició \myref{prop:producte de matrius en blocs} tenim que
			\[\left[\begin{array}{ccc|ccc}
			p_{1,1} & \cdots & p_{1,r} & 0 & \cdots & 0 \\
			\vdots & & \vdots & \vdots & & \vdots \\
			p_{s,1} & \cdots & p_{s,r} & 0 & \cdots & 0 \\\hline
			p_{s+1,1} & \cdots & p_{s+1,r} & 0 & \cdots & 0 \\
			\vdots & & \vdots & \vdots & & \vdots \\
			p_{m,1} & \cdots & p_{m,r} & 0 & \cdots & 0
			\end{array}\right]=
			\left[\begin{array}{ccc|ccc}
			q_{1,1,} & \cdots & q_{1,r} & q_{1,r+1} & \cdots & q_{1,n} \\
			\vdots & & \vdots & \vdots & & \vdots \\
			q_{s,1} & \cdots & q_{s,r} & q_{s,r+1} & \cdots & q_{s,n} \\\hline
			0 & \cdots & 0 & 0 & \cdots & 0 \\
			\vdots & & \vdots & \vdots & & \vdots \\
			0 & \cdots & 0 & 0 & \cdots & 0
			\end{array}\right].\]
			Per tant
			\[\left[\begin{matrix}
			0 & \cdots & 0 \\
			\vdots & & \vdots \\
			0 & \cdots & 0
			\end{matrix}\right]=
			\left[\begin{matrix}
			q_{1,r+1} & \cdots & q_{1,n} \\
			\vdots & & \vdots \\
			q_{s,r+1} & \cdots & q_{s,n}
			\end{matrix}\right]\]
			i trobem que
			\[Q_{2}=\left[\begin{array}{ccc|ccc}
			q_{1,1} & \cdots & q_{1,r} & 0 & \cdots & 0 \\
			\vdots & & \vdots & \vdots & & \vdots \\
			q_{s,1} & \cdots & q_{s,r} & 0 & \cdots & 0 \\\hline
			q_{s+1,1} & \cdots & q_{s+1,r} & q_{s+1,r+1} & \cdots & q_{s+1,n} \\
			\vdots & & \vdots & \vdots & & \vdots \\
			q_{n,1} & \cdots & q_{n,r} & q_{n,r+1} & \cdots & q_{n,n}
			\end{array}\right].\]
			
			Denotem per tant
			\[Q_{2}=\left[\begin{array}{c|c}
			Q'_{2} & 0_{s\times(n-r)} \\\hline
			Q''_{2} & Q'''_{2}
			\end{array}\right].\]
			
			Ara bé, pel que hem vist abans, existeixen un nombre natural \(t\leq\min(r,s)\), una matriu invertible \(P_{3}\) d'ordre \(s\) sobre \(\mathbb{K}\) i una matriu invertible \(Q_{3}\) d'ordre \(r\) sobre \(\mathbb{K}\) tals que
			\[P_{3}Q''_{2}Q_{3}=\left[\begin{array}{c|c}
			I_{t} & 0 \\\hline
			0 & 0
			\end{array}\right],\]
			i com que, per hipòtesi, \(r\leq s\) tenim \(t\leq r\leq s\).
			
			Per tant per la proposició \myref{prop:producte de matrius en blocs} tenim que
			\begin{multline*}
			\left[\begin{array}{c|c}
			P_{3} & 0_{s\times(n-s)} \\\hline
			0_{(n-s)\times s} & I_{n-s}
			\end{array}\right]
			\left[\begin{array}{c|c}
			Q'_{2} & 0_{s\times(n-r)} \\\hline
			Q''_{2} & Q'''_{2}
			\end{array}\right]
			\left[\begin{array}{c|c}
			Q_{3} & 0_{r\times(n-r)} \\\hline
			0_{(n-r)\times r} & I_{n-r}
			\end{array}\right]=\\=
			\left[\begin{array}{c|c}
				\begin{array}{c|c}
				I_{t} & 0_{t\times(r-t)} \\\hline
				0_{(r-t)\times t} & 0_{(s-t)\times(r-t)}
				\end{array}
			& 0_{s\times(n-r)} \\\hline
			Q''_{2}Q_{3} & Q'''_{2}
			\end{array}\right]=B.
			\end{multline*}
			Per la proposició \myref{prop:el producte de matrius invertibles és invertible} trobem que la matriu \(B\) és invertible. Ara bé, per l'observació \myref{obs:matrius amb una fila nula no són invertibles} trobem que ha de ser \(r-t=s-t=0\), i per tant \(r=s\), com volíem veure. La demostració del cas \(s\leq r\) és anàloga.
		\end{proof}
	\end{theorem}
	\begin{definition}[Rang]
		\labelname{rang d'una matriu}\label{def:rang d'una matriu}
		Siguin \(A\) una matriu de mida \(m\times n\) sobre un cos \(\mathbb{K}\), \(P\) una matriu invertible d'ordre \(n\) sobre \(\mathbb{K}\) i \(Q\) una matriu invertible d'ordre \(m\) sobre \(\mathbb{K}\) tals que
		\[PAQ=\left[\begin{array}{c|c}
		I_{r} & 0 \\\hline
		0 & 0
		\end{array}\right].\]
		Aleshores direm que \(r\) és el rang de \(A\) i denotarem \(\rang(A)=r\).
		
		Observem que aquesta definició té sentit pel \myref{thm:Teorema de la PAQ-reducció}.
	\end{definition}
	\begin{proposition}
		\label{prop:rang de la matriu transposada és el rang de la matriu}
		Sigui \(A\) una matriu de mida \(m\times n\) sobre un cos \(\mathbb{K}\). Aleshores \(\rang(A)=\rang(A^{t})\).
		\begin{proof}
			Per la definició de \myref{def:rang d'una matriu} tenim que existeixen una matriu \(P\) invertible d'ordre \(n\) sobre \(\mathbb{K}\) i una matriu \(Q\) invertible d'ordre \(m\) sobre \(\mathbb{K}\) tals que
			\[PAQ=\left[\begin{array}{c|c}
			I_{r} & 0 \\\hline
			0 & 0
			\end{array}\right].\]
			
			Ara bé, per la proposició \myref{prop:producte de matrius transposades} tenim que
			\[Q^{t}A^{t}P^{t}=\left[\begin{array}{c|c}
			I_{r} & 0 \\\hline
			0 & 0
			\end{array}\right].\]
			Per la proposició \myref{prop:transposada d'una invertible és invertible} tenim que les matrius \(Q^{t}\) i \(P^{t}\) són invertibles, i per tant, per la definició de \myref{def:rang d'una matriu} tenim que \(\rang{A^{t}}=r\).
		\end{proof}
	\end{proposition}
	\begin{proposition}
		\label{prop:una matriu invertible és producte de matrius elementals}
		Sigui \(A\) una matriu invertible d'ordre \(n\) sobre un cos \(\mathbb{K}\). Aleshores \(A\) és una matriu elemental.
		\begin{proof}
			Pel \myref{thm:Teorema de la PAQ-reducció} tenim que existeixen dues matrius invertibles \(P\) i \(Q\) d'ordre \(n\) sobre \(\mathbb{K}\) tals que
			\[PAQ=\left[\begin{array}{c|c}
			I_{r} & 0\\\hline
			0 & 0
			\end{array}\right].\]
			
			Ara bé, per la proposició \myref{prop:el producte de matrius invertibles és invertible} tenim que la matriu
			\[\left[\begin{array}{c|c}
			I_{r} & 0\\\hline
			0 & 0
			\end{array}\right]\]
			ha de ser invertible, i per l'observació \myref{obs:matrius amb una fila nula no són invertibles} trobem que ha de ser \(r=n\). Per tant tenim \(PAQ=I_{n}\) i tenim que \(A=P^{-1}Q^{-1}\), i per l'observació \myref{obs:producte de matrius elementals és matriu elemental} tenim que \(A\) és una matriu elemental.
		\end{proof}
	\end{proposition}
	\begin{corollary}
		Sigui \(A\) una matriu d'ordre \(n\) sobre un cos \(\mathbb{K}\). Aleshores \(A\) és invertible si i només si \(\rang(A)=n\).
	\end{corollary}
	\begin{observation}
		\label{obs:rang d'una matriu és invariant pel producte amb invertibles}
		Siguin \(A\) una matriu de mida \(m\times n\) sobre un cos \(\mathbb{K}\), \(P\) una matriu invertible d'ordre \(m\) sobre \(\mathbb{K}\) i \(Q\) una matriu invertible d'ordre \(n\) sobre \(\mathbb{K}\). Aleshores
		\[\rang(A)=\rang(PAQ).\]
		%FER \begin{proof}\end{proof}
	\end{observation}
%	\subsection{Càlcul de la inversa d'una matriu invertible}
	\section{Sistemes d'equacions lineals}
	\subsection{Sistemes d'equacions lineals}
	\begin{definition}[Sistema d'equacions lineals]
		\labelname{sistema d'equacions lineals}\label{def:sistema d'equacions lineals}
		Siguin \(\mathbb{K}\) un cos i \(\{a_{i,j}\}_{1\leq i\leq m}^{1\leq j\leq n}\), \(\{b_{i}\}_{1\leq i\leq m}\) i \(\{x_{i}\}_{1\leq i\leq n}\) elements de \(\mathbb{K}\). Aleshores direm que l'expressió
		\begin{equation}
		\label{eq:def:sistema d'equacions lineals: 1}
		\begin{cases}
		a_{1,1}x_{1}+\dots+a_{1,n}x_{n}=b_{1}\\
		\hfill\vdots\hfill\hfill\\
		a_{m,1}x_{1}+\dots+a_{m,n}x_{n}=b_{m}
		\end{cases}
		\end{equation}
		és un sistema d'equacions lineals sobre \(\mathbb{K}\). Direm que \(x_{1},\dots,x_{n}\) són les incògnites.
		
		Si existeixen \(a_{1},\dots,a_{n}\) elements de \(\mathbb{K}\) tals que
		\[\begin{cases}
		a_{1,1}a_{1}+\dots+a_{1,n}a_{n}=b_{1} \\
		\hfill\vdots\hfill\hfill\\
		a_{m,1}a_{1}+\dots+a_{m,n}a_{n}=b_{m}
		\end{cases}\]
		aleshores direm que \((a_{1},\dots,a_{n})\in\mathbb{K}^{n}\) és una solució del sistema \eqref{eq:def:sistema d'equacions lineals: 1}.
		
		Direm que el sistema \eqref{eq:def:sistema d'equacions lineals: 1} és compatible si existeix alguna solució, i direm que o bé és compatible determinat si aquesta solució és única o bé compatible indeterminat si la solució no és única.
	\end{definition}
	\begin{notation}
		Sigui
		\begin{equation}
		\begin{cases}
		a_{1,1}x_{1}+\dots+a_{1,n}x_{n}=b_{1}\\
		\hfill\vdots\hfill\hfill\\
		a_{m,1}x_{1}+\dots+a_{m,n}x_{n}=b_{m}
		\end{cases}
		\end{equation}
		un sistema d'equacions lineals sobre un cos \(\mathbb{K}\). 
	\end{notation}
	\subsection{Teorema de Rouché-Frobenius}	% Mètode de Gauss?
%	\subsection{Determinant d'una matriu}
%
%\chapter{Espais vectorials i aplicacions lineals}
%	\section{Espais vectorials}
%	
%	\section{Aplicacions lineals}
%	
%\chapter{Classificació d'endomorfismes}
%	\section{Endomorfismes similars}
%	\section{Diagonalització}
%	\section{Forma canònica de Jordan}
%	\section{Forma canònica racional}
%
%\chapter{Espais vectorials euclidians}

\end{document}
