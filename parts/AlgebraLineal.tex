\documentclass[../Apunts.tex]{subfiles}

\begin{document}
\chapter{Matrius}
	\section{Matrius i operacions}
	\subsection{Cossos}
	\begin{definition}[Cos]
		\labelname{cos}\label{def:cos}
		Aclarir on va aquest apartat.
	\end{definition}
	% elements 1 i 0 i tal i qual
	\subsection{Les matrius}
	\begin{definition}[Matriu]
		\labelname{matriu}\label{def:matriu}
		\labelname{matriu quadrada}\label{def:matriu quadrada}
		Siguin \(m\) i \(n\) dos enters i \(\{a_{i,j}\}^{1\leq i\leq m}_{1\leq j\leq n}\) elements d'un cos \(\mathbb{K}\). Aleshores direm que
		\[[a_{i,j}]=
		\left[\begin{matrix}
		a_{1,1} & a_{1,2} & \cdots & a_{1,n} \\
		a_{2,1} & a_{2,2} & \cdots & a_{2,n} \\
		\vdots & \vdots &  & \vdots \\
		a_{m,1} & a_{m,2} & \cdots & a_{m,n}
		\end{matrix}\right]\]
		és una matriu de mida \(m\times n\) sobre \(\mathbb{K}\).
		
		Direm que \(a_{i,j}\) és el component \((i,j)\) de la matriu \([a_{i,j}]\). Si \(A=[a_{i,j}]\) també denotarem \([A]_{i,j}=a_{i,j}\).
		
		Si \(n=m\) direm que \((a_{i,j})\) és una matriu quadrada d'ordre \(n\).
	\end{definition}
	\begin{notation}[Conjunt de matrius]
		\label{notation:conjunt de matrius}
		Siguin \(m\) i \(n\) dos enters, \(\mathbb{K}\) un cos i
		\[X=\{A\mid A\text{ és una matriu de mida }m\times n\text{ sobre }\mathbb{K}\}\]
		un conjunt. Aleshores denotarem \(X=M_{m\times n}(\mathbb{K})\).
		
		Si \(n=m\) denotarem \(X=M_{n}(\mathbb{K})\).
	\end{notation}
	%Veure que estàn ben definides %TODO %FER
	%Quan veiem que el producte té sentit veure que pertany a \(M_{m\times n}(\mathbb{K})\).
	\begin{definition}[Suma de matrius]
		\labelname{suma de matrius}\label{def:suma de matrius}
		Siguin \(A\) i \(B\) dues matrius de mida \(m\times n\) sobre un cos \(\mathbb{K}\). Aleshores definim la seva suma com una operació \(+\) que satisfà
		\[[A+B]_{i,j}=[A]_{i,j}+[B]_{i,j}.\]
	\end{definition}
	\begin{definition}[Producte de matrius]
		\labelname{producte de matrius}\label{def:producte de matrius}
		Siguin \(A\) una matriu de mida \(m\times l\) sobre un cos \(\mathbb{K}\) i \(B\) una matriu de mida \(l\times n\) sobre un cos \(\mathbb{K}\). Aleshores definim el seu producte com una operació \(\cdot\) que satisfà
		\[[A\cdot B]_{i,j}=\sum_{k=0}^{l}[A]_{i,k}[B]_{k,j}.\]
		
		Escriurem \(A\cdot B=AB\).
	\end{definition}
	\subsection{Propietats de les operacions amb matrius}
	\begin{proposition}
		\label{prop:associativitat suma matrius}
		Siguin \(A\), \(B\) i \(C\) tres matrius de mida \(m\times n\) sobre un cos \(\mathbb{K}\). Aleshores
		\[(A+B)+C=A+(B+C).\]
		\begin{proof}
			Per la definició de \myref{def:suma de matrius} tenim que
			\begin{align*}
			[(A+B)+C]_{i,j}&=[A+B]_{i,j}+[C]_{i,j}\\
			&=[A]_{i,j}+[B]_{i,j}+[C]_{i,j}\\
			&=[A]_{i,j}+([B]_{i,j}+[C]_{i,j})\tag{\myref{def:cos}}\\
			&=[A]_{i,j}+[B+C]_{i,j}\\
			&=[A+(B+C)]_{i,j}.\qedhere
			\end{align*}
		\end{proof}
	\end{proposition}
	\begin{proposition}
		\label{prop:commutativitat suma matrius}
		Siguin \(A\) i \(B\) dues matrius de mida \(m\times n\) sobre un cos \(\mathbb{K}\). Aleshores
		\[A+B=B+A.\]
		\begin{proof}
			Per la definició de \myref{def:suma de matrius} tenim que
			\begin{align*}
			[A+B]_{i,j}&=[A]_{i,j}+[B]_{i,j}\\
			&=[B]_{i,j}+[A]_{i,j}\tag{\myref{def:cos}}\\
			&=[B+A]_{i,j}.\qedhere
			\end{align*}
		\end{proof}
	\end{proposition}
	\begin{notation}[Matriu nul·la]
		\labelname{matriu nul·la}\label{notation:matriu nula}\label{notation:matriu zero}
		Denotarem la matriu de mida \(m\times n\)
		\[\left[\begin{matrix}
		0 & \cdots & 0 \\
		\vdots & & \vdots \\
		0 & \cdots & 0
		\end{matrix}\right]
		\in M_{m\times n}(\mathbb{K})\]
		com \(0_{m\times n}\), ó \(0\) si és clar pel context.
	\end{notation}
	\begin{proposition}
		\label{prop:element neutre per la suma de matrius}
		Sigui \(A\) una matriu de mida \(m\times n\) sobre un cos \(\mathbb{K}\). Aleshores
		\[A+0=A.\]
		\begin{proof}
			Per la definició de \myref{def:suma de matrius} tenim que
			\begin{align*}
			[A+0]_{i,j}&=[A]_{i,j}+[0]_{i,j}\\
			&=[A]_{i,j}+0\\
			&=[A]_{i,j},\tag{\myref{def:cos}}
			\end{align*}
			com volíem veure.
		\end{proof}
	\end{proposition}
	\begin{definition}[Producte d'una matriu per escalars]
		\labelname{producte d'una matriu per escalars}\label{def:producte d'una matriu per escalars}
		Sigui \(A\) una matriu de mida \(m\times n\) sobre un cos \(\mathbb{K}\) i \(\alpha\) un escalar de \(\mathbb{K}\). Aleshores definim
		\[[\alpha A]_{i,j}=\alpha[A]_{i,j}.\]
		
		Denotarem \((-1)A\) com \(-A\).
	\end{definition}
	\begin{proposition}
		\label{prop:inverses per la suma de matrius}
		Sigui \(A\) una matriu de mida \(m\times n\) sobre un cos \(\mathbb{K}\). Aleshores
		\[A-A=0.\]
		\begin{proof}
			Per la definició de \myref{def:suma de matrius} tenim que
			\begin{align*}
			[A-A]_{i,j}&=[A]_{i,j}+[-A]_{i,j}\\
			&=[A]_{i,j}-[A]_{i,j}\tag{\myref{def:producte d'una matriu per escalars}}\\
			&=0\tag{\myref{def:cos}} %REF?
			\end{align*}
			i trobem \(A-A=0\).
		\end{proof}
	\end{proposition}
	\begin{proposition}
		\label{prop:associativitat mixta producte escalars per matrius}
		Siguin \(A\) una matriu de mida \(m\times n\) i \(\alpha\) i \(\beta\) dos escalars de \(\mathbb{K}\). Aleshores
		\[(\alpha\beta)A=\alpha(\beta A).\]
		\begin{proof}
			Per la definició de \myref{def:producte d'una matriu per escalars} tenim que
			\begin{align*}
			[(\alpha\beta)A]_{i,j}&=\alpha\beta[A]_{i,j}\\
			&=\alpha(\beta[A_{i,j}])\tag{\myref{def:cos}}\\
			&=\alpha[\beta A]_{i,j}.\qedhere
			\end{align*}
		\end{proof}
	\end{proposition}
	\begin{proposition}
		\label{prop:distributiva respecta la suma d'escalars del producte de matrius}
		Siguin \(A\) una matriu de mida \(m\times n\) i \(\alpha\) i \(\beta\) dos escalars de \(\mathbb{K}\). Aleshores
		\[(\alpha+\beta)A=\alpha A+\beta A.\]
		\begin{proof}
			Per la definició de \myref{def:producte d'una matriu per escalars} tenim que
			\begin{align*}
			[(\alpha+\beta)A]_{i,j}&=(\alpha+\beta)[A]_{i,j}\\
			&=\alpha[A]_{i,j}+\beta[A]_{i,j},\tag{\myref{def:cos}}\\
			&=[\alpha A]_{i,j}+[\beta A]_{i,j}.\qedhere
			\end{align*}
		\end{proof}
	\end{proposition}
	\begin{proposition}
		Sigui \(A\) una matriu de mida \(m\times n\) sobre un cos \(\mathbb{K}\). Aleshores
		\[1A=A.\]
		\begin{proof}
			Per la definició de \myref{def:producte d'una matriu per escalars} tenim que
			\begin{align*}
			[1A]_{i,j}&=1[A]_{i,j}\\
			&=[A]_{i,j},\tag{\myref{def:cos}}
			\end{align*}
			com volíem veure.
		\end{proof}
	\end{proposition}
	\begin{proposition}
		\label{prop:associativitat producte de matrius}
		Siguin \(\mathbb{K}\) un cos i \(A\) una matriu de mida \(m\times l\) sobre \(\mathbb{K}\), \(B\) una matriu de mida \(l\times s\) sobre \(\mathbb{K}\) i \(C\) una matriu de mida \(s\times m\) sobre \(\mathbb{K}\). Aleshores
		\[(AB)C=A(BC).\]
		\begin{proof}
			Per la definició de \myref{def:producte de matrius} tenim que
			\begin{align*}
			[(AB)C]_{i,j}&=\sum_{r=1}^{s}[AB]_{i,r}[C]_{r,j}\\
			&=\sum_{r=1}^{s}\left(\sum_{k=1}^{l}[A]_{i,k}[B]_{k,r}\right)[C]_{r,j}\\
			&=\sum_{r=1}^{s}\sum_{k=1}^{l}[A]_{i,k}[B]_{k,r}[C]_{r,j}\tag{\myref{def:cos}}\\
			&=\sum_{k=1}^{l}\sum_{r=1}^{s}[A]_{i,k}[B]_{k,r}[C]_{r,j}\tag{\myref{def:cos}}\\
			&=\sum_{k=1}^{l}[A]_{i,k}\sum_{r=1}^{s}\left([B]_{k,r}[C]_{r,j}\right)\tag{\myref{def:cos}}\\
			&=\sum_{k=1}^{l}[A]_{i,k}[BC]_{k,j}\\
			&=[A(BC)]_{i,j}.\qedhere
			\end{align*}
		\end{proof}
	\end{proposition}
	\begin{proposition}
		\label{prop:distributiva del producte de matrius respecte la suma}
		Siguin \(A\) i \(A'\) dues matrius de mida \(m\times l\) sobre un cos \(\mathbb{K}\) i \(B\), \(B'\) dues matrius de mida \(l\times n\) sobre un cos \(\mathbb{K}\). Aleshores
		\[(A+A')B=AB+A'B\quad\text{i}\quad A(B+B')=AB+AB'.\]
		\begin{proof}
			Per la definició de \myref{def:producte de matrius} tenim que
			\begin{align*}
			[(A+A')B]_{i,j}&=\sum_{k=1}^{l}[A+A']_{i,k}[B]_{k,j}\\
			&=\sum_{k=1}^{l}\left([A]_{i,k}+[A']_{i,k}\right)[B]_{k,j}\tag{\myref{def:suma de matrius}}\\
			&=\sum_{k=1}^{l}\left([A]_{i,k}[B]_{k,j}+[A']_{i,k}[B]_{k,j}\right)\tag{\myref{def:cos}}\\
			&=\sum_{k=1}^{l}[A]_{i,k}[B]_{k,j}+\sum_{k=1}^{l}[A']_{i,k}[B]_{k,j}\tag{\myref{def:cos}}\\
			&=[AB]_{i,j}+[A'B]_{i,j}\\
			&=[AB+AB']_{i,j},\tag{\myref{def:suma de matrius}}
			\end{align*}
			i
			\begin{align*}
			[A(B+B')]_{i,j}&=\sum_{k=1}^{l}[A]_{i,k}[B+B']_{k,j}\\
			&=\sum_{k=1}^{l}[A]_{i,k}\left([B]_{k,j}+[B']_{k,j}\right)\tag{\myref{def:suma de matrius}}\\
			&=\sum_{k=1}^{l}\left([A]_{i,k}[B]_{k,j}+[A]_{i,k}[B']_{k,j}\right)\tag{\myref{def:cos}}\\
			&=\sum_{k=1}^{l}[A]_{i,k}[B]_{k,j}+\sum_{k=1}^{l}[A]_{i,k}[B]_{k,j}\tag{\myref{def:cos}}\\
			&=[AB]_{i,j}+[AB']_{i,j}\\
			&=[AB+AB']_{i,j},\tag{\myref{def:suma de matrius}}
			\end{align*}
			com volíem veure.
		\end{proof}
	\end{proposition}
	\subsection{Matrius inverses i matrius transposades}
	\begin{notation}[Matriu identitat]
		\labelname{matriu nul·la}\label{notation:matriu identitat}
		Denotarem la matriu d'ordre \(n\)
		\[\left[\begin{matrix}
		1 & 0 & \cdots & 0 \\
		0 & 1 & \ddots & \vdots \\
		\vdots & \ddots & \ddots & 0\\
		0 & \cdots & 0 & 1
		\end{matrix}\right]
		\in M_{n}(\mathbb{K})\]
		com \(I_{n}\) i direm que \(I_{n}\) és la matriu identitat d'ordre \(n\).
	\end{notation}
	\begin{proposition}
		\label{prop:producte per la matriu identitat}
		Sigui \(A\) una matriu de mida \(m\times n\) sobre un cos \(\mathbb{K}\). Aleshores
		\[I_{m}A=A=AI_{n}.\]
		\begin{proof}
			Per la definició de \myref{def:producte de matrius} tenim que
			\[[I_{m}A]_{i,j}=\sum_{k=1}^{m}[I_{m}]_{i,k}[A]_{k,j}.\]
			Ara bé, tenim que
			\[[I_{m}]_{i,j}=
			\begin{cases}
			1 & i=j\\
			0 & i\neq j
			\end{cases}\]
			i per la definició de \myref{def:cos} tenim que
			\[\sum_{k=1}^{m}[I_{m}]_{i,k}[A]_{k,j}=[A]_{i,j}\]
			i per tant \(I_{m}A=A\).
			
			La demostració de l'altre igualtat és anàloga.
		\end{proof}
	\end{proposition}
	\begin{proposition}
		\label{prop:unicitat inverses de matrius pel producte}
		Sigui \(A\) una matriu d'ordre \(n\) sobre un cos \(\mathbb{K}\) tal que existeixin dues matrius \(A_{1}\) i \(A_{2}\) d'ordre \(n\) sobre \(\mathbb{K}\) tal que
		\[AA_{1}=A_{1}A=I_{n}\quad\text{i}\quad AA_{2}=A_{2}A=I_{m}.\]
		Aleshores \(A_{1}=A_{2}\).
		\begin{proof}
			Per la proposició \myref{prop:associativitat producte de matrius} tenim que
			\[A_{1}(AA_{2})=(A_{1}A)A_{2}\]
			i per hipòtesi trobem que \(A_{1}I_{m}=I_{m}A_{2}\), i per la proposició \myref{prop:producte per la matriu identitat} trobem que \(A_{1}=A_{2}\), com volíem veure.
		\end{proof}
	\end{proposition}
	\begin{definition}[Matriu invertible]
		\labelname{matriu invertible}\label{def:matriu invertible}
		\labelname{inversa d'una matriu}\label{def:inversa d'una matriu}
		Sigui \(A\) una matriu d'ordre \(n\) sobre un cos \(\mathbb{K}\) tal que existeixi una matriu \(A'\) d'ordre \(n\) sobre \(\mathbb{K}\) tal que
		\[AA'=A'A=I_{n}.\]
		Aleshores direm que \(A\) és una matriu invertible i que \(A'\) és la inversa de \(A\). També denotarem \(A'=A^{-1}\).
		
		Observem que aquesta definició té sentit per la proposició \myref{prop:unicitat inverses de matrius pel producte}.
	\end{definition}
	\begin{proposition}
		\label{prop:inversa del producte de matrius}
		Siguin \(A\) i \(B\) dues matrius invertibles d'ordre \(n\) sobre un cos \(\mathbb{K}\). Aleshores
		\[(AB)^{-1}=B^{-1}A^{-1}.\]
		\begin{proof}
			Per la definició de \myref{def:matriu invertible} tenim que
			\[(AB)^{-1}AB=I_{n},\]
			i com que, per hipòtesi, les matrius \(A\) i \(B\) són invertibles trobem
			\begin{align*}
			B^{-1}A^{-1}&=(AB)^{-1}ABB^{-1}A^{-1}\\
			&=(AB)^{-1}AA^{-1}\\
			&=(AB)^{-1}.\qedhere
			\end{align*}
		\end{proof}
	\end{proposition}
	\begin{proposition}
		\label{prop:inversa de l'inversa d'una matriu és la matriu}
		Sigui \(A\) una matriu invertible d'ordre \(n\) sobre un cos \(\mathbb{K}\). Aleshores
		\[(A^{-1})^{-1}=A.\]
		\begin{proof}
			Per la proposició \myref{prop:inversa del producte de matrius} tenim que
			\[A^{-1}(A^{-1})^{-1}=A^{-1}A\]
			i com que, per hipòtesi, la matriu \(A\) és invertible trobem
			\[AA^{-1}(A^{-1})^{-1}=AA^{-1}A\]
			i tenim
			\[(A^{-1})^{-1}=A.\qedhere\]
		\end{proof}
	\end{proposition}
	\begin{definition}[Matriu transposada]
		\labelname{matriu transposada}\label{def:matriu transposada}
		\labelname{transposició d'una matriu}\label{def:transposició d'una matriu}
		Sigui \(A\) una matriu de mida \(m\times n\) sobre un cos \(\mathbb{K}\). Definim la transposada de \(A\) com una matriu \(A^{t}\) de mida \(n\times m\) sobre \(\mathbb{K}\) que satisfà
		\[[A^{t}]_{i,j}=[A]_{j,i}.\]
	\end{definition}
	\begin{proposition}
		Siguin \(A\) i \(B\) dues matrius de mida \(m\times n\) sobre un cos \(\mathbb{K}\). Aleshores
		\[(A+B)^{t}=A^{t}+B^{t}.\]
		\begin{proof}
			Per la definició de \myref{def:matriu transposada} tenim que
			\begin{align*}
			[(A+B)^{t}]_{i,j}&=[A+B]_{j,i}\\
			&=[A]_{j,i}+[B]_{j,i}\tag{\myref{def:suma de matrius}}\\
			&=[A^{t}]_{i,j}+[B^{t}]_{i,j}\\
			&=[A^{t}+B^{t}]_{i,j},\tag{\myref{def:suma de matrius}}
			\end{align*}
			com volíem veure.
		\end{proof}
	\end{proposition}
	\begin{proposition}
		Siguin \(A\) una matriu de mida \(m\times l\) sobre un cos \(\mathbb{K}\) i \(B\) una matriu de mida \(l\times n\) sobre \(\mathbb{K}\). Aleshores
		\[(AB)^{t}=B^{t}A^{t}.\]
		\begin{proof}
			Per la definició de \myref{def:matriu transposada} tenim que
			\begin{align*}
			[(AB)^{t}]_{i,j}&=[AB]_{j,i}\\
			&=\sum_{k=1}^{l}[A]_{j,k}[B]_{k,i}\tag{\myref{def:producte de matrius}}\\
			&=\sum_{k=1}^{l}[A^{t}]_{k,j}[B]_{i,k}\\
			&=\sum_{k=1}^{l}[B^{t}]_{i,k}[A^{t}]_{k,j}\tag{\myref{def:cos}}\\
			&=[B^{t}A^{t}]_{i,j},\tag{\myref{def:cos}}
			\end{align*}
			com volíem veure.
		\end{proof}
	\end{proposition}
	\begin{definition}[Matriu simètrica]
		\labelname{matriu simètrica}\label{def:matriu simètrica}
		Sigui \(A\) una matriu d'ordre \(n\) sobre un cos \(\mathbb{K}\) tal que \(A^{t}=A\). Aleshores direm que \(A\) és una matriu simètrica.
	\end{definition}
	\subsection{Producte de matrius en blocs}%FER producte de matrius en blocs
	\begin{definition}[Files i columnes]
		\labelname{fila d'una matriu}\label{def:fila d'una matriu}
		\labelname{columna d'una matriu}\label{def:columna d'una matriu}
		Sigui \(A=(a_{i,j})\) una matriu de mida \(m\times n\) sobre un cos \(\mathbb{K}\). Aleshores definim
		\[F_{i}=\left[\begin{matrix}
		a_{i,1} & \cdots & a_{i,n}
		\end{matrix}\right]\]
		com la \(i\)-èsima fila de \(A\) i
		\[C_{i}=\left[\begin{matrix}
		a_{1,i} \\
		\vdots \\
		a_{m,i}
		\end{matrix}\right]\]
		com la \(i\)-èsima columna de \(A\).
	\end{definition}
	\begin{observation}
		\label{obs:on pertanyen les files i les columnes d'una matriu}
		Si \(A\in M_{m\times n}(\mathbb{K})\) tenim \(F_{i}\in M_{1\times n}(\mathbb{K})\) i \(C_{i}\in M_{m\times 1}(\mathbb{K})\).
	\end{observation}
	\begin{notation}
		Siguin \(A\) una matriu de mida \(m\times n\) sobre un cos \(\mathbb{K}\), \(F_{1},\dots,F_{m}\) les files de \(A\) i \(C_{1},\dots,C_{n}\) les columnes de \(A\). Aleshores denotem
		\[A=\left[\begin{matrix}
		F_{1} \\
		\vdots \\
		F_{m}
		\end{matrix}\right]=\left[\begin{matrix}
		C_{1} & \cdots & C_{n}
		\end{matrix}\right].\]
	\end{notation}
	\begin{notation}
		Siguin \(A=(a_{i,j})\) una matriu de mida \(m\times n\) sobre un cos \(\mathbb{K}\), \(m_{1},\dots,m_{p}\) i \(n_{1},\dots,n_{q}\) nombres naturals tals que \(m_{1}+\dots+m_{p}=m\) i \(n_{1}+\dots+n_{q}=n\), \(m_{0}=n_{0}=1\) i
		\[A_{i,j}=\left[\begin{matrix}
		a_{m_{i-1},n_{i-1}} & \cdots & a_{m_{i-1},n_{i}}\\
		\vdots & & \vdots \\
		a_{m_{i},n_{i-1}} & \cdots & a_{m_{i},n_{i}}
		\end{matrix}\right]\]
		matrius sobre un cos \(\mathbb{K}\). Aleshores denotem
		\[A=\left[\begin{matrix}
		A_{1,1} & \cdots & A_{1,q} \\
		\vdots & & \vdots \\
		A_{p,1} & \cdots & A_{p,q}
		\end{matrix}\right].\]
	\end{notation}
	\section{Rang d'una matriu}
	\subsection{Transformacions elementals}
	\begin{definition}[Transformacions elementals] %REFER per columnes i files
		\labelname{transformacions elementals}\label{def:transformacions elementals}
		Siguin \(A=(a_{i,j})\) una matriu de mida \(m\times n\) sobre un cos \(\mathbb{K}\) i \(i\) i \(j\) dos naturals amb \(i,j\leq m\). Aleshores definim les funcions
		\[P_{i,j}(A)=\left[\begin{matrix}
		a_{1,1} & \cdots & a_{1,n} \\
		\vdots & & \vdots \\
		a_{i-1,1} & \cdots & a_{i-1,n} \\
		a_{j,1} & \cdots & a_{j,n} \\
		a_{i+1,1} & \cdots & a_{i+1,n} \\
		\vdots & & \vdots \\
		a_{j-1,1} & \cdots & a_{j-1,n} \\
		a_{i,1} & \cdots & a_{i,n} \\
		a_{j+1,1} & \cdots & a_{j+1,n} \\
		\vdots & & \vdots \\
		a_{m,1} & \cdots & a_{m,n}
		\end{matrix}\right]\]
		
		\[D_{i,\lambda}(A)=\left[\begin{matrix}
		a_{1,1} & \cdots & a_{1,n} \\
		\vdots & & \vdots \\
		a_{i-1,1} & \cdots & a_{i-1,n} \\
		\lambda a_{i,1} & \cdots & \lambda a_{i,n} \\
		a_{i+1,1} & \cdots & a_{i+1,n} \\
		\vdots & & \vdots \\
		a_{m,1} & \cdots & a_{m,n}
		\end{matrix}\right]\]
		
		\[E_{i,j,\lambda}(A)=\left[\begin{matrix}
		a_{1,1} & \cdots & a_{1,n} \\
		\vdots & & \vdots \\
		a_{i-1,1} & \cdots & a_{i-1,n} \\
		a_{i,1}-\lambda a_{j,1} & \cdots & a_{i,n}-\lambda a_{j,n}\\
		a_{i+1,1} & \cdots & a_{i+1,n} \\
		\vdots & & \vdots \\
		a_{m,1} & & a_{m,n}
		\end{matrix}\right]\]
	\end{definition}
	\begin{definition}[Matrius elementals]
		\labelname{matrius elementals}\label{def:matrius elementals}
		Sigui \(\lambda\) un escalar d'un cos \(\mathbb{K}\). Aleshores definim les matrius de \(M_{n}(\mathbb{K})\)
		\[P_{n}(i,j)=\left[\begin{matrix}
		1 & & & & & & & & & & \\
		& \ddots & & & & & & & & & \\
		& & 1 & & & & & & & & \\
		& & & 0 & \cdots & \cdots & \cdots & 1 & & & \\
		& & & \vdots & 1 & & & \vdots & & & \\
		& & & \vdots & & \ddots & & \vdots & & & \\
		& & & \vdots & & & 1 & \vdots & & & \\
		& & & 1 & \cdots & \cdots & \cdots & 0 & & & \\
		& & & & & & & & 1 & & \\
		& & & & & & & & & \ddots & \\
		& & & & & & & & & & 1
		\end{matrix}\right],\]
		\[D_{n}(i,\lambda)=\left[\begin{matrix}
		1 & & & & & & & \\
		& \ddots & & & & & \\
		& & 1 & & & & \\
		& & & \lambda & & & \\
		& & & & 1 & & \\
		& & & & & \ddots & \\
		& & & & & & 1
		\end{matrix}\right],\]
		i
		\[E_{n}(i,j,\lambda)=\left[\begin{matrix}
		1 & & & & & & \\
		& \ddots & & & & & \\
		& & 1 & \cdots & \lambda & & \\
		& & & \ddots & \vdots & \\
		& & & & 1 & & \\
		& & & & & \ddots & \\
		& & & & & & 1
		\end{matrix}\right],\]
		com les matrius elementals de \(M_{n}(\mathbb{K})\).
	\end{definition}
	
%	\subsection{Matrius esglaonades i mètode de Gauss}
%	\subsection{Teorema de la PAQ}
%	\subsection{Càlcul de la inversa d'una matriu invertible}
%	\section{Sistemes d'equacions lineals}
%	\subsection{Teorema de Rouché-Frobenius}
%	\subsection{Determinant d'una matriu}
%
%\chapter{Espais vectorials i aplicacions lineals}
%	\section{Espais vectorials}
%	
%	\section{Aplicacions lineals}
%	
%\chapter{Classificació d'endomorfismes}
%	\section{Endomorfismes similars}
%	\section{Diagonalització}
%	\section{Forma canònica de Jordan}
%	\section{Forma canònica racional}
%
%\chapter{Espais vectorials euclidians}

\end{document}
