\documentclass[../Apunts.tex]{subfiles}

\begin{document}
\part{Equacions diferencials ordinàries \rom{2}}
\chapter{Sistemes autònoms al pla}
	\emph{\hypersetup{urlcolor=black}\href{https://www.urbandictionary.com/define.php?term=wtf}{wtf} is this assignatura}
	
\begin{comment}
\section{Sistemes autònoms a \ensuremath{\mathbb{R}^{n}}}
%	\subsection{Interpretació geomètrica} % Justificació dels retrats de fase? Això a EDOS I
	\subsection{Estructura de les òrbites}
	\subsection{Integrals primeres}
	\subsection{Superfícies invariants}
	\subsection{Retrat de fase i conjugació}
\section{Sistemes integrables}
	\subsection{Sistemes potencials}
	\subsection{Sistemes Hamiltonians}
	\subsection{El model de Lotka-Volterra}
\section{Sistemes no integrables}
	\subsection{Teorema del flux tubular}
	\subsection{Anàlisi qualitativa dels punts d'equilibri}
	\subsection{Comportament límit de les òrbites}
	\subsection{Teorema de Bendixson-Poincaré}
	\subsection{Funcions de Liapunov}
	\subsection{Cicles límit}
	\subsection{Criteri de Bendixson-Dulac}
%	\subsection{Models a l'ecologia}
	\subsection{Sistema de van der Pol}
\chapter{Equacions en derivades parcials}
\section{Equacions en derivades parcials de primer ordre}
	\subsection{Introducció a les equacions en derivades parcials}
	\subsection{Equacions lineals i quasi-lineals de primer odre}
\section{Equacions en derivades parcials de segon ordre}
	\subsection{Equacions de la corda finita}
	\subsection{Principi d'Alembert}
	\subsection{Problemes de contorn}
	\subsection{L'equació de la calor}
	\subsection{Problema de la barra infinita}
	\subsection{Separació de variables i sèries de Fourier}
	\subsection{L'equació de Laplace}
\end{comment}
\end{document} 
