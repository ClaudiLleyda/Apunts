\documentclass[../Apunts.tex]{subfiles}

\begin{document}
\part{Equacions diferencials ordinàries \rom{2}}
\chapter{Sistemes autònoms al pla}
%	\emph{\hypersetup{urlcolor=black}\href{https://www.urbandictionary.com/define.php?term=wtf}{wtf} is this assignatura}
\section{Trajectòries d'una equació diferencial}
	\subsection{Sistemes autònoms a \ensuremath{\mathbb{R}^{n}}}
	\begin{observation}
		\label{obs:podem entendre les equacions diferencials autònomes com camps vectorials}
		Sigui
		\[\dot{u}(t)=X(u(t))\]
		una equació diferencial autònoma sobre un obert \(\obert{U}\subseteq\mathbb{R}^{n}\) amb \(X\in\mathcal{C}^{1}\). Aleshores la funció \(X\) és un camp vectorial.
	\end{observation}
	\begin{definition}[Trajectòries]
		\labelname{trajectòries}\label{def:trajectòries d'una equació diferencial}
		Sigui \(\varphi\) una solució d'una equació diferencial autònoma
		\[\dot{u}(t)=X(u(t))\]
		on \(X\in\mathcal{C}^{1}\). Aleshores direm que \(\varphi\) és una trajectòria de l'equació diferencial \(\dot{u}(t)=X(u(t))\).
%		
%		Siguin
%		
%		\[\dot{u}(t)=X(u(t))\]
%		una equació diferencial autònoma sobre un obert \(\obert{U}\subseteq\mathbb{R}^{n}\) i \(\varphi\) una solució de l'equació diferencial. Aleshores direm que \(\varphi\) és una trajectòria de l'equació diferencial.
	\end{definition}
	\begin{proposition}
		Siguin
		\[\dot{u}(t)=X(u(t))\]
		una equació diferencial autònoma sobre un obert \(\obert{U}\subseteq\mathbb{R}^{n}\) amb \(X\in\mathcal{C}^{1}\) i \(x\in\obert{U}\) un punt. Aleshores existeix una solució maximal \(\varphi_{x}\) tal que \(\varphi_{x}(0)=x\).
		\begin{proof}
			Considerem el problema de Cauchy
			\[\begin{cases*}
				\displaystyle \dot{u}(t)=X(u(t)) \\
				\displaystyle u(0)=x.
			\end{cases*}\]
			
			Aleshores per la proposició \myref{prop:existeixen solucions improrrogables} tenim que existeix una solució maximal \(\varphi_{x}\) d'aquest problema de Cauchy, i per la definició de \myref{def:problema de Cauchy} trobem que \(\varphi_{x}(0)=x\) i hem acabat.
		\end{proof}
	\end{proposition}
	\begin{definition}[Flux]
		\labelname{flux}\label{def:flux}
		Siguin
		\[\dot{u}(t)=X(u(t))\]
		una equació diferencial autònoma sobre un obert \(\obert{U}\subseteq\mathbb{R}^{n}\) amb \(X\in\mathcal{C}^{1}\), \(x\in\obert{U}\) un punt i \(\varphi_{x}\) la solució maximal tal que \(\varphi_{x}(0)=x\). Aleshores direm que \(\varphi_{x}\) és el flux de \(x\).
	\end{definition}
	
\begin{comment}
\section{Sistemes autònoms a \ensuremath{\mathbb{R}^{n}}}
%	\subsection{Interpretació geomètrica} % Justificació dels retrats de fase? Això a EDOS I
	\subsection{Estructura de les òrbites}
	\subsection{Integrals primeres}
	\subsection{Superfícies invariants}
	\subsection{Retrat de fase i conjugació}
\section{Sistemes integrables}
	\subsection{Sistemes potencials}
	\subsection{Sistemes Hamiltonians}
	\subsection{El model de Lotka-Volterra}
\section{Sistemes no integrables}
	\subsection{Teorema del flux tubular}
	\subsection{Anàlisi qualitativa dels punts d'equilibri}
	\subsection{Comportament límit de les òrbites}
	\subsection{Teorema de Bendixson-Poincaré}
	\subsection{Funcions de Liapunov}
	\subsection{Cicles límit}
	\subsection{Criteri de Bendixson-Dulac}
%	\subsection{Models a l'ecologia}
	\subsection{Sistema de van der Pol}
\chapter{Equacions en derivades parcials}
\section{Equacions en derivades parcials de primer ordre}
	\subsection{Introducció a les equacions en derivades parcials}
	\subsection{Equacions lineals i quasi-lineals de primer odre}
\section{Equacions en derivades parcials de segon ordre}
	\subsection{Equacions de la corda finita}
	\subsection{Principi d'Alembert}
	\subsection{Problemes de contorn}
	\subsection{L'equació de la calor}
	\subsection{Problema de la barra infinita}
	\subsection{Separació de variables i sèries de Fourier}
	\subsection{L'equació de Laplace}
\end{comment}
\end{document} 
