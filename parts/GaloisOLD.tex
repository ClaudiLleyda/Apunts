\documentclass[../Apunts.tex]{subfiles}

\begin{document}
\part{Teoria de Galois}
\chapter[Capítol primer]{Primer}
\section{Extensions de cossos}
\subsection{Elements algebraics i elements transcendents}
	\begin{definition}[Extensió d'un cos]
		\labelname{extensió d'un cos}\label{def:extensió d'un cos}
		Siguin~\(\KK\) i~\(\FF\) dos cossos tals que~\(\KK\subseteq\FF\). Aleshores direm que~\(\FF\) és una extensió de~\(\KK\) i ho denotarem com~\(\FF\extensio\KK\). També direm que~\(\FF\extensio\KK\) és una extensió.
	\end{definition}
	\begin{proposition} %FER Buscar-li un lloc millor. Potser a estructures.
		\label{prop:un cos quocient un irreductible és un cos}
		Siguin~\(\KK\) un cos i~\(p(x)\in\KK[x]\) un polinomi irreductible. Aleshores~\(\KK[x]/(p(x))\) és un cos.
		\begin{proof} % Veure que un cos és un DIP. A estructures lol.
			Per hipòtesi tenim que~\(p(x)\) és irreductible, i per la proposició~\myref{prop:irreductible sii ideal maximal} trobem que l'ideal~\((p(x))\) és maximal. Aleshores per la proposició~\myref{prop:condició equivalent a ideal maximal per R/M cos} tenim que~\(\FF\) és un cos.
		\end{proof}
	\end{proposition}
	\begin{example}
		\label{ex:el cos de polinomis és una extensió}
		Siguin~\(\KK\) un cos i~\(p(x)\in\KK[x]\) un polinomi irreductible. Denotem~\(\FF=\KK[x]/(p(x))\). Aleshores~\(\FF\extensio\KK\) és una extensió.
		\begin{solution}
			Per la proposició~\myref{prop:un cos quocient un irreductible és un cos} tenim que~\(\FF\) és un cos. Veiem que~\(\KK\subseteq\FF\) per l'observació~\myref{obs:un anell està contingut en el seu anell de polinomis}, i per la definició d'\myref{def:extensió d'un cos} hem acabat.
		\end{solution}
	\end{example}
	\begin{example}[Morfisme avaluació]
		\labelname{\empty}
		\label{ex:morfisme d'avaluació}
		\label{ex:morfisme avaluació}
		\label{ex:el morfisme avaluació és un morfisme d'anells}
		Siguin~\(\FF\extensio\KK\) una extensió i~\(\alpha\in\FF\) un element. Volem veure que l'aplicació
%		\begin{align*}
%			\ev_{\alpha}\colon\KK[x]&\longrightarrow\FF \\
%			\KK[x]\setminus\KK\ni x&\longmapsto\alpha \\
%			\KK\ni\lambda&\longmapsto\lambda
%		\end{align*}
		\begin{align*}
			\ev_{\alpha}\colon\KK[x]&\longrightarrow\FF \\
			p(x)&\longmapsto p(\alpha)
		\end{align*}
		és un morfisme d'anells.
		\begin{solution}
			%TODO
		\end{solution}
	\end{example}
	\begin{lemma}
		\label{lema:element algebraic}
		\label{lema:element transcendent}
		Siguin~\(\FF\extensio\KK\) una extensió i~\(\alpha\in\FF\) un element. Aleshores
		\[\ker(\ev_{\alpha})=\begin{cases}
			(0) \\
			(p(x)), & p(x)\text{ és un polinomi mònic irreductible a }\KK[x]
		\end{cases}\]
		\begin{proof} % REVISAR. Pot ser molt més fàcil?
			Suposem que~\(\ker(\ev_{\alpha})\neq(0)\). Per la proposició~\myref{prop:el nucli d'un morfisme entre anells és ideal, la imatge d'un morfisme entre anells és subanell} i l'exemple~\myref{ex:el morfisme avaluació és un morfisme d'anells} tenim que~\(\ker(\ev_{\alpha})\) és un ideal, i com que, per hipòtesi,~\(\KK\) és un cos, tenim que~\(\KK[x]\) és un domini d'ideals principals. %REF
			Per tant, per la definició de~\myref{def:domini d'ideals principals} tenim que~\(\ker(\ev_{\alpha})\) és un ideal principal, i per la definició d'\myref{def:ideal principal} trobem que existeix un~\(a\in\KK[x]\) tal que~\(\ker(\ev_{\alpha})=(a)\).
			
			Tenim que~\(a\) ha de ser un polinomi mònic. Suposem que~\(a\in\KK\). Trobem per la definició~\myref{ex:el morfisme avaluació és un morfisme d'anells} que~\(\ev_{\alpha}(a)=a\), i com que hem suposat que~\(\ker(\ev_{\alpha})\neq(0)\), trobem que~\(a\neq0\), i per tant~\(\ev_{\alpha}(a)=a\neq0\), trobant que~\(a\notin\ker(\ev_{\alpha})\), arribant a contradicció. Per tant~\(a\in\KK[x]\setminus\KK\). Suposem que~\(a=\lambda p(x)\), amb~\(\lambda\in\KK\),~\(\lambda\neq0\) i~\(p\in\KK[x]\setminus\KK\). Per l'exemple~\myref{ex:el morfisme avaluació és un morfisme d'anells} tenim que~\(\ev_{\alpha}\) és un morfisme d'anells, i per la definició de~\myref{def:morfisme entre anells} trobem que
			\begin{align*}
				\ev_{\alpha}(\lambda p(x))&=\ev_{\alpha}(\lambda)\ev_{\alpha}(p(x))\\
				&=\lambda\ev_{\alpha}(p(x)). \tag{\ref{ex:el morfisme avaluació és un morfisme d'anells}}
			\end{align*}
			Ara bé, com que hem suposat que~\(\ker(\ev_{\alpha})=(\lambda p(x))\) tenim~\(\ev_{\alpha}(\lambda p(x))=0\) i, com que hem suposat que~\(\lambda\neq0\), ha de ser~\(\ev_{\alpha}(p(x))=0\), i tenim que
			\[\ker(\ev_{\alpha})=(p(x)).\]
			
			Veiem ara que~\(p(x)\) és irreductible. Pel~\myref{thm:Primer Teorema de l'isomorfisme entre anells} trobem que
			\[\KK[x]/(p(x))=\KK[x]/\ker(\ev_{\alpha})\cong\Ima(\ev_{\alpha}).\]
			
			Com que~\(\FF\) és un cos tenim que~\(\Ima(\ev_{\alpha})\) és un domini. %REF
			Com que~\(\Ima(\ev_{\alpha})\cong\KK[x]/(p(x))\) és un domini, tenim per la proposició~\myref{prop:R/I domini d'integritat sii I ideal primer} que~\((p(x))\) és un ideal primer, i per l'observació~\myref{obs:ideals primer iff primer} trobem que~\(p(x)\) és primer, i per la proposició~\myref{prop:en un DI un primer és un irreductible} tenim que~\(p(x)\) és irreductible.
		\end{proof}
	\end{lemma}
	\begin{definition}[Element algebraic i element transcendent]
		\labelname{element algebraic}\label{def:element algebraic}
		\labelname{element transcendent}\label{def:element transcendent}
%		Siguin~\(\FF\extensio\KK\) una extensió i~\(\alpha\in\FF\) un element tal que existeix un polinomi no nul~\(p(x)\in\KK[x]\) satisfent~\(p(\alpha)=0\). Aleshores direm que~\(\alpha\) és {algebraic} sobre~\(\KK\).
		Siguin~\(\FF\extensio\KK\) una extensió i~\(\alpha\in\FF\) un un element tal que no existeix cap polinomi mònic no nul~\(p(x)\in\KK[x]\) tal que~\(p(\alpha)=0\). Aleshores direm que~\(\alpha\) és {algebraic} sobre~\(\KK\). Denotarem~\(\Irr(\alpha,\KK)=p(x)\).
		
		Si~\(\alpha\) no és algebraic sobre~\(\KK\) direm que és {transcendent} sobre~\(\KK\).
	\end{definition}
	\begin{observation}
		\label{obs:polinomi mínim de descomposició}
		\((\Irr(\alpha,\KK))=\ker(\ev_{\alpha})\)
	\end{observation}
	\begin{example}
		\label{ex:elements algebraics}
		Volem veure que els següents són algebraics.
		\begin{enumerate}
			\item~\(\sqrt{2}\) i~\(\sqrt{3}\) en l'extensió~\(\RR\extensio\QQ\).
			\item~\(\sqrt{2}+\sqrt{3}\) en l'extensió~\(\RR\extensio\QQ\)
		\end{enumerate}
		\begin{solution}
			Tenim que %TODO
			\begin{enumerate}
			\item~\(x^{2}-2\),~\(x^{2}-3\).
			\item~\(x^{4}-10x^{2}+1\).\qedhere
			\end{enumerate}
		\end{solution}
	\end{example}
	\begin{observation}
		\label{obs:condició equivalent a element algebraic}
		\label{obs:condició equivalent a element transcendent}
		Siguin~\(\FF\extensio\KK\) una extensió i~\(\alpha\in\FF\) un element. Aleshores~\(\alpha\) és algebraic si i només si existeix un polinomi no nul~\(p(x)\in\KK[x]\) tal que~\(\ker(\ev_{\alpha})=(p(x))\).
		\begin{proof}
			Si~\(\alpha\) és algebraic aleshores, per la definició d'\myref{def:element algebraic}, existeix un polinomi no nul~\(p(x)\in\KK[x]\) tal que~\(p(\alpha)=0\), i per la definició de~\myref{def:nucli d'un morfisme entre grups} tenim que~\(\ker(\ev_{\alpha})\neq(0)\). Aleshores pel lema~\myref{lema:element algebraic} hem acabat.
		\end{proof}
	\end{observation}
	\subsection{Grau d'una extensió de cossos}
	\begin{definition}[Grau d'una extensió de cossos]
		\labelname{grau d'una extensió}\label{def:grau d'una extensió}
		Sigui~\(\FF\extensio\KK\) una extensió. Aleshores direm que la dimensió del~\(\KK\)-espai vectorial~\(\FF\) és el {grau de l'extensió} i el denotarem com~\(\grauExtensio{\FF}{\KK}\).
		
		Aquesta definició té sentit per la proposició~\myref{prop:un subcòs és un espai vectorial}.
	\end{definition}
	\begin{example}
		\label{ex:el grau de l'extensió d'un cos i el quocient entre el seu anell de polinomis i un polinomi irreductible}
		Siguin~\(\KK\) un cos i~\(p(x)\in\KK[x]\) un polinomi irreductible. Denotem~\(\FF=\KK[x]/p(x)\). Aleshores
		\[\grauExtensio{\FF}{\KK}=\grau(p(x)).\]
		\begin{solution}
			Aquest enunciat té sentit per l'exemple~\myref{ex:el cos de polinomis és una extensió}. %FER
		\end{solution}
	\end{example}
	\begin{theorem}[Fórmula de les Torres]
		\labelname{fórmula de les torres}\label{thm:fórmula de les torres}
		Siguin~\(\EE\extensio\FF\) i~\(\FF\extensio\KK\) dues extensions de cossos. Aleshores
		\[\grauExtensio{\EE}{\KK}=\grauExtensio{\EE}{\FF}\grauExtensio{\FF}{\KK}.\]
		\begin{proof}
			Si~\(\grauExtensio{\EE}{\FF}=\infty\) ó~\(\grauExtensio{\FF}{\KK}=\infty\) tenim que~\(\grauExtensio{\EE}{\KK}=\infty\).
			
			Suposem doncs que~\(\grauExtensio{\EE}{\FF}=n\) i~\(\grauExtensio{\FF}{\KK}=m\). %TODO %FER
		\end{proof}
	\end{theorem}
	\begin{definition}[Torre]
		\labelname{torre}\label{def:torre de cossos}
		Siguin~\(\KK_{1}\subseteq\KK_{2}\subseteq\dots\subseteq\KK_{1}\) cossos. Aleshores direm que
		\[\KK_{1}\subseteq\KK_{2}\subseteq\dots\subseteq\KK_{1}\]
		és una torre.
	\end{definition}
	\begin{corollary}
		\label{cor:fórmula de les torres}
		Sigui~\(\KK_{1}\subseteq\KK_{2}\subseteq\dots\subseteq\KK_{1}\) una torre. Aleshores
		\[\grauExtensio{\KK_{n}}{\KK_{1}}=\grauExtensio{\KK_{n}}{\KK_{n-1}}\cdots\grauExtensio{\KK_{2}}{\KK_{1}}\]
		\begin{proof}
			Conseqüència de la~\myref{thm:fórmula de les torres}.
		\end{proof}
	\end{corollary}
	\subsection{Extensions algebraiques}
	\begin{definition}%[Extensió algebraica]
%		\labelname{extensió algebraica}
		\label{def:mínim anell que conté un element i un cos}
		Siguin~\(\FF\extensio\KK\) una extensió i~\(\alpha\in\FF\) un element. Aleshores denotem
		\[\KK[\alpha]=\{p(\alpha)\in\FF\mid p(x)\in\KK[x]\}.\]
%		és l'{extensió algebraica} de~\(\alpha\) en~\(\KK\).
%		Siguin~\(D\) un domini d'integritat,~\(\FF\) un cos tal que~\(\cosdefraccions(D)\subseteq\FF\) i~\(\alpha\in\FF\) un element. Aleshores direm que
%		\[D[\alpha]=\{p(\alpha)\mid p(x)\in D[x]\}\]
%		és l'{extensió algebraica} de~\(\alpha\) en~\(D\).
	\end{definition}
	\begin{observation}
		\label{obs:un cos està contingut en les seves extensions algebraiques}
		\label{obs:extensió algebraica}
		\(\KK\subseteq\KK[\alpha]\subseteq\FF\).
	\end{observation}
%	\begin{proposition}
%		Siguin~\(D\) un domini d'integritat i~\(\FF\) un cos tal que~\(\cosdefraccions(D)\subseteq\FF\). Aleshores l'extensió algebraica de~\(\alpha\) en~\(D\) és el mínim subanell de~\(\FF\) que conté~\(\alpha\).
%		\begin{proof}
%			
%		\end{proof}
%	\end{proposition}
	\begin{proposition}
		\label{prop:l'extensió algebraica és l'anell de polinomis quocient l'irreductible de l'element}
		Siguin~\(\FF\extensio\KK\) una extensió i~\(\alpha\in\FF\) un element. Aleshores
		\[\KK[\alpha]\cong\KK[x]/(\Irr(\alpha,\KK)).\]
		\begin{proof}
			Tenim que
			\begin{align*}
				\KK[x]/(\Irr(\alpha,\KK))&=\KK[x]/\ker(\ev_{\alpha}) \tag{\ref{obs:polinomi mínim de descomposició}}\\
				&\cong\Ima(\ev_{\alpha}) \tag{\ref{thm:Primer Teorema de l'isomorfisme entre anells}}\\
				&=\{p(\alpha)\in\FF\mid p(x)\in\KK[x]\} \tag{\ref{def:imatge d'un morfisme entre anells} i \ref{ex:morfisme avaluació}}\\
				&= \KK[\alpha], \tag{\ref{def:mínim anell que conté un element i un cos}}
			\end{align*}
			i per tant
			\[\KK[\alpha]\cong\KK[x]/(\Irr(\alpha,\KK)).\qedhere\]
		\end{proof}
	\end{proposition}
	\begin{corollary}
		\label{cor:grau d'una extensió algebraica}
		\(\grauExtensio{\KK[\alpha]}{\KK}=\grau(\Irr(\alpha,\KK))\).
		\begin{proof}
			Per l'exemple~\myref{ex:el grau de l'extensió d'un cos i el quocient entre el seu anell de polinomis i un polinomi irreductible}.
		\end{proof}
	\end{corollary}
	\begin{corollary}
		\label{cor:l'extensió algebraica d'un transcendent és isomorfa a l'anell de polinomis}
		Siguin~\(\FF\extensio\KK\) una extensió i~\(\gamma\in\FF\) un element transcendent sobre~\(\KK\). Aleshores
		\[\KK[\gamma]\cong\KK[x].\]
	\end{corollary}
	\begin{definition}[Extensió algebraica d'un conjunt]
		Siguin~\(\FF\extensio\KK\) una extensió i~\(\Lambda\subseteq\FF\) un conjunt. Aleshores denotem
		\[\KK[\Lambda]=\{p(\alpha)\in\FF\mid p(x)\in\KK[x],\alpha\in\Lambda\}.\]
	\end{definition}
	\begin{definition}[Cos de fraccions d'una extensió algebraica]
		\labelname{cos de fraccions d'una extensió algebraica}\label{def:cos de fraccions d'una extensió algebraica}
		Siguin~\(\FF\extensio\KK\) una extensió i~\(\alpha\in\FF\) un element. Aleshores denotarem
		\[\KK(\alpha)=\cosdefraccions(\KK[\alpha]).\]
		Si~\(\Lambda\subseteq\FF\) és un conjunt denotem
		\[\KK(\Lambda)=\cosdefraccions(\KK[\Lambda]).\]
	\end{definition}
	\begin{observation}
		\label{obs:cos de fraccions d'una extensió algebraica}
		\(\KK\subseteq\KK[\alpha]\subseteq\KK(\alpha)\subseteq\FF\).
	\end{observation}
	\begin{lemma}
		\label{lema:un element és algebraic si i només si l'extensió algebraica és un cos}
		Siguin~\(\FF\extensio\KK\) una extensió i~\(\alpha\in\FF\) un element. Aleshores~\(\alpha\) és algebraic sobre~\(\KK\) si i només si
		\[\KK(\alpha)=\KK[\alpha].\]
		\begin{proof} %FER %Revisar
			Veiem que la condició és suficient~\((\implica)\). Suposem que~\(\alpha\in\FF\) és algebraic sobre~\(\KK\). Per la definició d'\myref{def:element algebraic} trobem que~\(\Irr(\alpha,KK)\) és un polinomi irreductible no nul, i per la proposició~\myref{prop:un cos quocient un irreductible és un cos} tenim que~\(\KK[\alpha]\) és un cos.
			
			Ara bé, pel Teorema~\myref{thm:cos de fraccions} tenim que~\(\KK(\alpha)\subseteq\KK[\alpha]\), però per l'observació~\myref{obs:cos de fraccions d'una extensió algebraica} tenim que~\(\KK[\alpha]\subseteq\KK(\alpha)\), i pel \myref{thm:doble inclusió} tenim que~\(\KK(\alpha)=\KK[\alpha]\).
			
			Acabem veient que la condició és necessària~\((\implicatper)\). Suposem doncs que
			\[\KK(\alpha)=\KK[\alpha].\]
			Per tant~\(\KK[\alpha]\) és un cos, i com que, per l'observació~\myref{obs:extensió algebraica}, tenim que~\(\alpha\in\KK[\alpha]\), i per la definició de~\myref{def:cos per anells} tenim que~\(-1/\alpha\in\KK[\alpha]\). Aleshores, per la definició d'\myref{def:mínim anell que conté un element i un cos} trobem que existeix un polinomi~\(a_{0}+a_{1}x+\dots+a_{k}x^{k}\in\KK[x]\), amb~\(a_{k}\neq0\), tal que
			\[\frac{-1}{\alpha}=a_{0}+a_{1}\alpha+a_{2}\alpha^{2}+\dots+a_{k}\alpha^{k},\]
			i per tant
			\[0=1+a_{0}\alpha+a_{1}\alpha^{2}+\dots+a_{r}\alpha^{r+1},\]
			i trobem que~\(\alpha\) és una arrel del polinomi~\(p(x)=1+a_{0}x+a_{1}x^{2}+\dots+a_{r}x^{r+1}\). Aleshores per la definició de~\myref{ex:morfisme d'avaluació} tenim que
			\[p(x)\in\ker(\ev_{\alpha})\]
			i per tant~\(\ker(\ev_{\alpha})\neq(0)\) i pel lema~\myref{lema:element algebraic} trobem que~\(\alpha\) és algebraic.
		\end{proof}
	\end{lemma}
	\subsection{Extensions simples}
	\begin{definition}[Extensió simple]
		
	\end{definition}
\end{document}
