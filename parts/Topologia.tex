\documentclass[../Apunts.tex]{subfiles}

\begin{document}
\chapter{Les topologies}
	\section{Espais mètrics}
	\subsection{Les funcions contínues i les boles}
	\begin{definition}[Espai mètric]
		\labelname{espai mètric}\label{def:espai mètric}
		\labelname{distància}\label{def:distància}
		Sigui \(X\) un conjunt i \(d\colon X\times X\longrightarrow\mathbb{R}\) una aplicació que per a tot \(x\), \(y\) i \(z\) de \(X\) satisfà
		\begin{enumerate}
			\item \(d(x,y)=0\) si i només si \(x=y\).
			\item \(d(x,y)=d(y,x)\).
			\item \(d(x,y)\leq d(x,z)+d(z,y)\).
			\item \(d(x,y)\leq0\).
		\end{enumerate}
		Aleshores direm que \(X\) amb la distància \(d\) és un espai mètric. També direm que \(d\) és la distància o mètrica de l'espai mètric.
	\end{definition}
	\begin{definition}[Bola]
		\labelname{bola}\label{def:bola}
		Siguin \(X\) amb la distància \(d\) un espai mètric, \(a\) un element de \(X\) i \(r>0\) un nombre real. Aleshores definim
		\[\B(a,r)=\{x\in X\mid d(x,a)<r\}\]
		com la bola de radi \(r\) centrada en \(a\).
	\end{definition}
	\begin{definition}[Obert]
		\labelname{obert}\label{def:obert espai mètric}
		Sigui \(X\) amb la distància \(d\) un espai mètric i \(\obert{U}\) un subconjunt de \(X\) tals que per a tot element \(x\) de \(\obert{U}\) existeix un \(\varepsilon>0\) real tal que \(\B(x,\varepsilon)\subset\obert{U}\). Aleshores direm que \(\obert{U}\) és un obert.
	\end{definition}
	\begin{proposition}
		\label{prop:les boles són oberts}
		Siguin \(X\) amb la distància \(d\) un espai mètric i \(\B(a,r)\) una bola de \(X\). Aleshores \(\B(a,r)\) és un obert.
		\begin{proof}
			Prenem un element \(b\in\B(a,r)\) i definim
			\begin{equation}
			\label{prop:les boles són oberts:eq1}
				\varepsilon=\frac{r-d(a,b)}{2}.
			\end{equation}
			Aleshores considerem la bola \(\B(b,\varepsilon)\) i tenim que \(\B(b,\varepsilon)\subset\B(a,r)\), ja que si prenem un element \(x\) de \(\B(b,\varepsilon)\), per la definició de \myref{def:distància} trobem que
			\begin{align*}
				d(x,a)&\leq d(x,b)+d(b,a)\\
				&<\varepsilon+d(b,a)\tag{\myref{def:bola}}\\
				&=\frac{r-d(a,b)}{2}+d(b,a)\tag{\myref{prop:les boles són oberts:eq1}}\\
				&=\frac{r-d(a,b)}{2}+d(a,b)\tag{\myref{def:distància}}\\
				&=\frac{r+d(a,b)}{2}<r,
			\end{align*}
			ja que, per la definició de \myref{def:bola} tenim que \(d(a,b)<r\) i per tant trobem \(r+d(a,b)<2r\).
		\end{proof}
	\end{proposition}
	\begin{proposition}[Propietat de Hausdorff]
		\labelname{}\label{prop:propietat de Hausdorff}
		Siguin \(X\) amb la distància \(d\) un espai mètric i \(x\) i \(y\) dos elements diferents de \(X\). Aleshores existeixen dos oberts \(\obert{U}\) i \(\obert{V}\) disjunts tals que \(x\) és un element de \(\obert{U}\) i \(y\) és un element de \(\obert{V}\).
		\begin{proof}
			Definim \(r=\frac{d(x,y)}{3}\) i considerem les boles \(\B(x,r)\) i \(\B(y,r)\). Per la definició de \myref{def:distància} i la definició de \myref{def:bola} tenim que \(x\) és un element de \(\B(x,r)\) i \(y\) és un element de \(\B(y,r)\).
			
			També tenim que les boles \(\B(x,r)\) i \(\B(y,r)\) són disjuntes, ja que si prenem un element \(a\) de \(X\) tal que \(a\) pertanyi a \(\B(x,r)\) i a \(\B(y,r)\) aleshores tenim, per la definició de \myref{def:bola}, que \(d(x,a)<r\) i \(d(a,y)<r\), i per la definició de \myref{def:distància} tenim que
			\[d(x,a)+d(a,y)\geq d(x,y).\]
			Ara bé, tenim per hipòtesi que \(r=\frac{d(x,y)}{3}\). Per tant
			\begin{align*}
				d(x,y) &\leq d(x,a)+d(a,y) \\
				&\leq \frac{d(x,y)}{3} + \frac{d(x,y)}{3} \\
				&\leq \frac{2d(x,y)}{3} < d(x,y),
			\end{align*}
			i per tant aquest \(a\) no existeix i trobem que \(\B(x,r)\) i \(\B(y,r)\) són disjunts.
			
			Per acabar, per la proposició \myref{prop:les boles són oberts} tenim que les boles \(\B(x,r)\) i \(\B(y,r)\) són oberts, i hem acabat.
		\end{proof}
	\end{proposition}
	\begin{theorem}
		Siguin \(X\) amb la distància \(d\) i \(X'\) amb la distància \(d'\) dos espais mètrics i \(f\colon X\longrightarrow X'\) una funció. Aleshores \(f\) és contínua si i només si per a tot \(\obert{U}\) obert de \(X'\) tenim que \(\Ima_{\obert{U}}(f^{-1})\) és obert.
		\begin{proof}
			Veiem primer que la condició és necessària (\(\implica\)). Suposem doncs que \(f\) és contínua. Prenem un obert \(\obert{U}\) de \(X'\) i un element \(y\) de \(X'\) tal que \(f^{-1}(y)\in\Ima_{\obert{U}}(f^{-1})\), i denotem \(x=f^{-1}(y)\). Per la definició d'\myref{def:obert espai mètric} tenim que existeix un nombre real \(\varepsilon>0\) tal que \(\B(y,\varepsilon)\subset\Ima_{\obert{U}}(f^{-1})\).
			
			Prenem un element \(x'\) de \(\B(y,\varepsilon)\). Per la definició de \myref{def:bola} tenim que \(d(x,x')<\varepsilon\), i per la definició de \myref{def:funció contínua} tenim que existeix un \(\delta>0\) real tal que \(d'(f(x'),y)<\varepsilon\), i per la definició de \myref{def:bola} això és que \(f(x')\in\B(y,\varepsilon)\). Per tant tenim que \(\Ima_{\B(x,\delta)}(f)\subset\B(y,\varepsilon)\) i trobem que \(\B(x,\delta)\subset\Ima_{\obert{U}}(f^{-1})\), i per la definició de \myref{def:obert espai mètric} trobem que \(\Ima_{\obert{U}}(f^{-1})\) és un obert, com volíem veure.
			
			Veiem ara que la condició és suficient (\(\implicatper\)). Suposem doncs que per a tot \(\obert{U}\) obert de \(X'\) tenim que \(\Ima_{\obert{U}}(f^{-1})\) és obert.
			
			Prenem un element \(xy\) de \(X\) i un real \(\varepsilon>0\). Si denotem \(y=f(x)\) tenim, per la proposició \myref{prop:les boles són oberts}, que la bola \(\B(y,\varepsilon)\) és un obert. Per hipòtesi tenim que \(\Ima_{\B(y,\varepsilon)}(f^{-1})\) és un obert, i per la definició d'\myref{def:obert espai mètric} tenim que existeix un nombre real \(\delta>0\) tal que \(\B(x,\delta)\subset\Ima_{\B(y,\varepsilon)}(f^{-1})\).
			
			Ara bé, per la definició de \myref{def:bola} tenim que això és que per a tot \(x_{0}\) de \(x\) amb \(d'(y,f(x_{0}))<\varepsilon\) tenim que \(d(x,x_{0})<\delta\), i per tant, per la definició de \myref{def:funció contínua} tenim que \(f\) és contínua.
		\end{proof}
	\end{theorem}
	\subsection{L'espai topològic}
	\begin{definition}[Topologia]	% Reescriure amb notació
		\labelname{topologia}\label{def:topologia}
		\labelname{espai topològic}\label{def:espai topològic}
		\labelname{obert}\label{def:obert}
		\labelname{punt}\label{def:punt}
		Sigui \(X\) un conjunt i \(\tau\) una família de subconjunts de \(X\) tals que
		\begin{enumerate}
		\item \(\emptyset\) i \(X\) són elements de \(\tau\).
		\item La intersecció d'una família finita d'elements de \(\tau\) és un element de \(\tau\).
		\item La unió d'una família d'elements de \(\tau\) és un element de \(\tau\).
		\end{enumerate}
		Aleshores direm que \(\tau\) és una topologia de \(X\) o que \(X\) amb la topologia \(\tau\) és un espai topològic.
		
		També direm que els elements de \(\tau\) són oberts i que els elements de \(X\) són punts.
	\end{definition}
	\begin{example}[Topologia induïda per una mètrica]
		\labelname{}\label{ex:topologia induida per una mètrica}
		Siguin \(X\) amb la distància \(d\) un espai mètric i \(\tau=\{\obert{U}\subseteq X\mid \obert{U}\text{ és un obert}\}\). Volem veure que \(X\) amb la topologia \(\tau\) és un espai topològic.
		\begin{solution}
			Per la definició d'\myref{def:obert espai mètric} trobem que \(\emptyset\) és un obert i que \(X\) és un obert, ja que per a tot \(x\) element de \(X\) per la proposició \myref{prop:les boles són oberts} tenim que \(\B(x,\varepsilon)\) és un subconjunt de \(X\) per a tot \(\varepsilon>0\).
			
			Prenem \(\{\obert{U}_{i}\}_{i=1}^{n}\) una família d'elements de \(\tau\) i considerem
			\[\obert{U}=\bigcap_{i=1}^{n}\obert{U}_{i}.\]
			
			Sigui \(x\) un element de \(\obert{U}\). Per la definició d'\myref{def:obert} tenim que per a tot \(i\in\{1,\dots,n\}\) existeix un \(\varepsilon_{i}>0\) real tal que
			\[\B(x,\varepsilon_{i})\subset\obert{U}_{i}.\]
			Per tant, si definim \(\varepsilon=\min_{i\in\{1,\dots,n\}}\varepsilon_{i}\). Aleshores per la definició de \myref{def:bola} tenim que
			\[\B(x,\varepsilon)\subseteq\B(x,\varepsilon_{i})\quad\text{per a tot }i\in\{1,\dots,n\}.\]
			i per tant \(\B(x,\varepsilon)\subset\obert{U}\) i per la definició de \myref{def:obert} tenim que \(\obert{U}\) és un obert.
			
			Prenem \(\{\obert{U}_{i}\}_{i\in I}\) una família d'oberts de \(\tau\) i considerem
			\[\obert{U}=\bigcup_{i\in I}\obert{U}_{i}.\]
			
			Sigui \(x\) un element de \(\obert{U}\). Per la definició d'\myref{def:obert} tenim que per a tot \(i\in\{1,\dots,n\}\) existeix un \(\varepsilon_{i}>0\) real tal que
			\[\B(x,\varepsilon_{i})\subset\obert{U}_{i}.\]
			Per tant, si definim \(\varepsilon=\min_{i\in\{1,\dots,n\}}\varepsilon_{i}\). Aleshores per la definició de \myref{def:bola} tenim que
			\[\B(x,\varepsilon)\subseteq\B(x,\varepsilon_{i})\quad\text{per a tot }i\in\{1,\dots,n\}.\]
			i per tant \(\B(x,\varepsilon)\subset\obert{U}\) i per la definició de \myref{def:obert} tenim que \(\obert{U}\) és un obert, i per la definició de \myref{def:espai topològic} hem acabat.
		\end{solution}
	\end{example}
	\begin{example}[Topologia grollera]
		\labelname{topologia grollera}\label{ex:topologia grollera}
		Siguin \(X\) i \(\tau=\{\emptyset,X\}\) dos conjunts. Aleshores \(\tau\) és una topologia de \(X\).
		\begin{solution}
			Comprovem les condicions de la definició de topologia. Com que \(\tau=\{\emptyset,X\}\) tenim que \(\emptyset\) i \(X\) són elements de \(\tau\). Observem també que
			\[\emptyset\cap X=\emptyset,\quad X\cap X=X\quad\text{i}\quad\emptyset\cup X=X\cup X=X,\]
			I per la definició de \myref{def:topologia}, tenim que \(\tau\) és una topologia de \(X\).
		\end{solution}
	\end{example}
	\begin{example}[Topologia discreta]
		\labelname{topologia discreta}\label{ex:topologia discreta}
		Siguin \(X\) i \(\tau=\mathcal{P}(X)\) dos conjunts. Aleshores \(\tau\) és una topologia de \(X\).
		\begin{solution}
			Tenim per l'\myref{axiom:conjunt potència} que \(\emptyset\) i \(X\) són subconjunts de \(\tau\), ja que per l'\myref{axiom:axioma de regularitat} trobem que \(\emptyset\subseteq X\), i pel \myref{thm:doble inclusió} trobem que \(X\subseteq X\).
			
			Per la definició d'\myref{def:unió de conjunts} trobem que si \(\{\obert{U}_{i}\}_{i\in I}\) és una família de subconjunts de \(X\) aleshores
			\[\bigcup_{i\in I}\obert{U}_{i}\subseteq X,\]
			i per la definició d'\myref{def:intersecció de conjunts} trobem que si \(\{\obert{U}_{i}\}_{i=1}^{n}\) és una família de subconjunts de \(X\) aleshores
			\[\bigcap_{i=1}^{n}\obert{U}_{i}\subseteq X,\]
			i per tant, com que \(\tau=\mathcal{P}(X)\) trobem per la definició de \myref{def:topologia} que \(\tau\) és una topologia de \(X\).
		\end{solution}
	\end{example}
	\subsection{Tancats}
	\begin{definition}[Tancat]
		\labelname{tancat}\label{def:tancat}
		Siguin \(X\) amb la topologia \(\tau\) un espai topològic i \(\tancat{C}\) un subconjunt de \(X\) tal que \(X\setminus\tancat{C}\) sigui obert. Aleshores direm que \(\tancat{C}\) és tancat.
	\end{definition}
	\begin{theorem}
		\label{thm:equivalència obert tancat definició de topologia}
		Sigui \(X\) amb la topologia \(\tau\) un espai topològic. Aleshores
		\begin{enumerate}
			\item\label{thm:equivalència obert tancat definició de topologia:enum 1} \(\emptyset\) i \(X\) són tancats.
			\item\label{thm:equivalència obert tancat definició de topologia:enum 2} La unió de qualsevol família finita de tancats és un tancat.
			\item\label{thm:equivalència obert tancat definició de topologia:enum 3} La intersecció de qualsevol família de tancats és un tancat.
		\end{enumerate}
		\begin{proof}
			Comencem veient el punt \eqref{thm:equivalència obert tancat definició de topologia:enum 1}. Tenim que \(X\setminus\emptyset=X\). Per la definició d'\myref{def:espai topològic} tenim que \(X\) és un obert, i per la definició de \myref{def:tancat} trobem que \(\emptyset\) és un tancat.
			
			També tenim que \(X\setminus X=\emptyset\). Per la definició d'\myref{def:espai topològic} tenim que \(\emptyset\) és un obert, i per la definició de \myref{def:tancat} trobem que \(X\) és un tancat.
			
			Veiem ara el punt \eqref{thm:equivalència obert tancat definició de topologia:enum 2}. Prenem una família \(\{\tancat{C}_{i}\}_{i=1}^{n}\) de tancats de \(X\) i considerem
			\[X\setminus\bigcup_{i=1}^{n}\tancat{C}_{i}=X\cap\left(\bigcup_{i=1}^{n}\tancat{C}_{i}\right)^{\complement}.\]
			Per la \myref{taut:primera llei de De Morgan} trobem que
			\begin{align*}
				X\cap\left(\bigcup_{i=1}^{n}\tancat{C}_{i}\right)^{\complement}&=X\cap\left(\bigcap_{i=1}^{n}\tancat{C}_{i}^{\complement}\right) \\
				&=\bigcap_{i=1}^{n}\left(X\cap\tancat{C}_{i}^{\complement}\right) \\
				&=\bigcap_{i=1}^{n}\left(X\setminus\tancat{C}_{i}\right).
			\end{align*}
			Per hipòtesi tenim que per a tot \(i\in\{1,\dots,n\}\) el conjunt \(\tancat{C}_{i}\) és un tancat, i per la definició de \myref{def:tancat} tenim que per a tot \(i\in\{1,\dots,n\}\) el conjunt \(X\setminus\tancat{C}_{i}\) és un obert, i per la definició d'\myref{def:espai topològic} tenim que \(\bigcap_{i=1}^{n}\left(X\setminus\tancat{C}_{i}\right)\) és un obert. Ara bé, tenim que
			\[X\setminus\bigcup_{i=1}^{n}\tancat{C}_{i}=\bigcap_{i=1}^{n}\left(X\setminus\tancat{C}_{i}\right),\]
			i per tant, per la definició de \myref{def:tancat} trobem que \(\bigcup_{i=1}^{n}\tancat{C}_{i}\) és un tancat.
			
			Veiem per acabat el punt \eqref{thm:equivalència obert tancat definició de topologia:enum 3}. Prenem una família \(\{\tancat{C}_{i}\}_{i\in I}\) de tancats de \(X\) i considerem
			\[X\setminus\bigcap_{i\in I}\tancat{C}_{i}=X\cap\left(\bigcap_{i\in I}\tancat{C}_{i}\right)^{\complement}.\]
			Per la \myref{taut:segona llei de De Morgan} trobem que
			\begin{align*}
				X\cap\left(\bigcap_{i\in I}\tancat{C}_{i}\right)^{\complement}&=X\cap\left(\bigcup_{i\in I}\tancat{C}_{i}^{\complement}\right) \\
				&=\bigcup_{i\in I}\left(X\cap\tancat{C}_{i}^{\complement}\right) \\
				&=\bigcup_{i\in I}\left(X\setminus\tancat{C}_{i}\right).
			\end{align*}
			Ara bé, per la definició de \myref{def:tancat} tenim que per a tot \(i\in I\) el conjunt \(X\setminus\tancat{C}_{i}\) és un obert, i per la definició de \myref{def:espai topològic} trobem que el conjunt \(\bigcup_{i\in I}\left(X\setminus\tancat{C}_{i}\right)\) és un obert. Tenim
			\[X\setminus\bigcap_{i\in I}\tancat{C}_{i}=\bigcup_{i\in I}\left(X\setminus\tancat{C}_{i}\right),\]
			i, de nou per la definició de \myref{def:tancat}, tenim que \(\bigcap_{i\in I}\tancat{C}_{i}\) és un tancat, com volíem veure.
		\end{proof}
	\end{theorem}
	\subsection{Base d'una topologia}
	\begin{definition}[Base d'una topologia]
		\labelname{base d'una topologia}\label{def:base d'una topologia}
		Siguin \(X\) amb la topologia \(\tau\) un espai topològic i \(\base{B}\) una família d'oberts tals que per a tot obert \(\obert{U}\) de \(X\) i per a tot punt \(x\) de \(\obert{U}\) existeix un \(B\in\base{B}\) tal que \(x\in B\subseteq\obert{U}\). Aleshores direm que \(\base{B}\) és una base de la topologia \(\tau\).
	\end{definition}
	\begin{example}
		Siguin \(X\) amb la distància \(d\) un espai mètric i
		\[\base{B}=\{\B(x,\varepsilon)\mid x\in X\text{ i }\varepsilon>0\}\]
		un conjunt. Aleshores \(\base{B}\) és una base de la topologia \(\tau\) induïda per la mètrica.
		\begin{solution}
			Tenim que
			\[\tau=\{\obert{U}\subseteq X\mid\obert{U}\text{ es un obert}\}.\]
			
			Prenem doncs un obert \(\obert{U}\) de \(\tau\) i un punt \(x\) de \(\obert{U}\). Per la definició d'\myref{def:obert espai mètric} tenim que existeix un \(\varepsilon>0\) real tal que \(\B(x,\varepsilon)\) és un subconjunt de \(\obert{U}\), i per la definició de \myref{def:base d'una topologia} hem acabat.
		\end{solution}
	\end{example}
	\begin{definition}[Finor d'una topologia]
		\labelname{finor d'una topologia}\label{def:finor d'una topologia}
		Siguin \(X\) un conjunt i \(\tau\), \(\tau'\) dues topologies de \(X\) tals que \(\tau\subset\tau'\). Aleshores direm que \(\tau'\) és més fina que \(\tau\).
	\end{definition}
%	\begin{proposition}
%		Siguin \(X\) un conjunt i \(\tau\), \(\tau'\) dues topologies de \(X\) tals que \(\tau'\) sigui més fina que \(\tau\). Aleshores l'aplicació \(\Id\colon X\longrightarrow X'\) és una aplicació contínua.
%		\begin{proof}
%			
%		\end{proof}
%	\end{proposition}
	\begin{proposition}
		\label{prop:condició equivalent a base d'una topologia}
		\label{prop:condició per que una topologia sigui la més fina que conté una base}
		Siguin \(X\) un conjunt i \(\base{B}\) una família de subconjunts de \(X\) tals que
		\[\bigcup_{B\in\base{B}}B=X\]
		i tal que per a tot \(\obert{U}\) i \(\obert{V}\) de \(\base{B}\) i per a tot \(x\) de \(\obert{U}\cap\obert{V}\) existeix un \(\obert{W}\) de \(\base{B}\) tal que \(x\) pertanyi a \(\obert{W}\) i \(\obert{W}\subseteq\obert{U}\cap\obert{V}\).	Aleshores existeix una única topologia \(\tau\) de \(X\) tal que \(\base{B}\) és una base de \(\tau\) i \(\tau\) és la topologia menys fina que conté els elements de \(\base{B}\).
		\begin{proof}
			Definim
			\begin{equation}
				\label{prop:condició equivalent a base d'una topologia:eq1}
				\tau=\left\{\obert{U}\subseteq X\mid\obert{U}=\bigcup_{i\in I}B_{i}\subseteq\base{B}\right\}
			\end{equation}
			Observem que \(X\) i \(\emptyset\) pertanyen a \(\tau\).
			
			Siguin \(\obert{U}\) i \(\obert{V}\) dos elements de \(\base{B}\) i prenem un element \(x\) de \(\obert{U}\cap\obert{V}\). Per hipòtesi tenim que existeix un element  \(\obert{W}_{x}\) de \(\base{B}\) tal que \(x\in\obert{W}\subseteq\obert{U}\cap\obert{V}\). Per tant
			\[\obert{U}\cap\obert{V}=\bigcup_{x\in\obert{U}\cap\obert{V}}\obert{W}_{x},\]
			i per la definició del conjunt \(\tau\) tenim que \(\obert{U}\cap\obert{V}\) és un element de \(\tau\).
			
			Prenem dos elements \(\obert{U}\) i \(\obert{V}\) del conjunt \(\tau\). Per la definició \eqref{prop:condició equivalent a base d'una topologia:eq1} trobem que existeixen dues famílies \(\{\obert{U}_{i}\}_{i\in I}\) i \(\{\obert{V}_{j}\}_{j\in J}\) d'elements de \(\base{B}\) tals que
			\[\obert{U}=\bigcup_{i\in I}\obert{U}_{i}\quad\text{i}\quad\obert{V}=\bigcup_{j\in J}\obert{V}_{j}.\]
			
			Per tant tenim que
			\[\obert{U}\cap\obert{V}=\bigcup_{i\in I}\bigcup_{j\in J}(\obert{U}_{i}\cap\obert{V}_{j}),\]
			i per la definició \eqref{prop:condició equivalent a base d'una topologia:eq1} trobem que \(\obert{U}\cap\obert{V}\) pertany a \(\tau\). Per tant per la definició de \myref{def:topologia} trobem que \(\tau\) és una topologia de \(X\).
			
			Veiem ara que \(\base{B}\) és base de la topologia \(\tau\). Prenem un obert \(\obert{U}\) de \(\tau\) i \(x\) un element de \(\obert{U}\). Per la definició \eqref{prop:condició equivalent a base d'una topologia:eq1} tenim que existeix una família \(\{B_{i}\}_{i\in I}\) d'elements de \(\base{B}\) tal que
			\[\obert{U}=\bigcup_{i\in I}B_{i}.\]
			Per tant existeix un element \(B\) de \(\base{B}\) tal que \(x\in B\) i \(B\subseteq\obert{U}\) i per la definició de \myref{def:base d'una topologia} trobem que \(\base{B}\) és una base de la topologia \(\tau\).
			
			Continuem veient que \(\tau\) és la topologia menys fina que conté els elements de \(\base{B}\). Suposem que existeix una topologia \(\tau'\) de \(X\) tal que \(\tau\) és més fina que \(\tau'\) i tal que \(\tau'\) conté els elements de \(\base{B}\). Per la definició de \myref{def:finor d'una topologia} això és que \(\tau'\subset\tau\).
			
			Prenem un obert \(\obert{U}\) de \(\tau\). Per la definició \eqref{prop:condició equivalent a base d'una topologia:eq1} trobem que existeix una família \(\{B_{i}\}_{i\in I}\) d'elements de \(\base{B}\) tals que
			\[\obert{U}=\bigcup_{i\in I}B_{i}.\]
			Ara bé, tenim per hipòtesi que \(\base{B}\) és un subconjunt de la topologia \(\tau'\), i per la definició de \myref{def:topologia} trobem que \(\obert{U}\) pertany a \(\tau'\). Per tant ha de ser \(\tau\subseteq\tau'\), i trobem que \(\tau\) és la topologia menys fina que conté els elements de \(\base{B}\).
			
			Veiem ara que aquesta topologia \(\tau\) és única. Suposem que existeix una altre topologia \(\tau'\) tal que \(\base{B}\) és una base de \(\tau'\) i \(\tau'\) és la topologia menys fina que conté els elements de \(\base{B}\).
			
			Prenem un obert \(\obert{U}\) de \(\tau\). Per la definició \eqref{prop:condició equivalent a base d'una topologia:eq1} trobem que existeix una família \(\{B_{i}\}_{i\in I}\) d'elements de \(\base{B}\) tals que
			\[\obert{U}=\bigcup_{i\in I}B_{i}.\]
			Ara bé, per la definició de \myref{def:topologia} tenim que \(\obert{U}\) és un element de \(\tau'\), ja que per hipòtesi \(\base{B}\) és un subconjunt de \(\tau'\). Per tant tenim que \(\tau\subseteq\tau'\). Ara bé, ja hem vist que \(\tau\) és la topologia menys fina que conté els elements de \(\base{B}\), i per la definició de \myref{def:finor d'una topologia} això és que \(\tau'\subseteq\tau\), i pel \myref{thm:doble inclusió} ha de ser \(\tau=\tau'\), com volíem veure.
		\end{proof}
	\end{proposition}
	\subsection{Entorns, interior i adherència}
	\begin{definition}[Entorn]
		\labelname{entorn}\label{def:entorn}
		\labelname{punt interior}\label{def:punt interior}
		Siguin \(X\) amb la topologia \(\tau\) un espai topològic, \(x\) un punt de \(X\) i \(A\) un conjunt tal que existeix un obert \(\obert{U}\) satisfent \(x\in\obert{U}\) i \(\obert{U}\subseteq A\). Aleshores direm que \(A\) és un entorn de \(x\).
		
		També direm que \(x\) és un punt interior de \(A\).
	\end{definition}
	\begin{definition}[Interior]
		\labelname{interior}\label{def:interior}
		Siguin \(X\) amb la topologia \(\tau\) un espai topològic, \(x\) un punt de \(X\), \(A\) subconjunt de \(X\) i
		\[\Int(A)=\{x\in X\mid A\text{ és un entorn de }x\}.\]
		Aleshores direm que \(\Int(A)\) és l'interior del conjunt \(A\).
	\end{definition}
	\begin{proposition}
		Siguin \(X\) amb la topologia \(\tau\) un espai topològic i \(A\) un subconjunt de \(X\). Aleshores \(\Int(A)\) és un obert.
		\begin{proof}
			Prenem un element \(x\) de \(\Int(A)\). Per la definició de \myref{def:interior} tenim que existeix un obert \(\obert{U}_{x}\) tal que \(x\in\obert{U}_{x}\) i \(\obert{U}_{x}\subseteq A\). Aleshores tenim
			\begin{align*}
			\Int(A)&=\bigcup_{x\in\Int(A)}x \\
			&\subseteq\bigcup_{x\in\Int(A)}\obert{U}_{x} \\
			&\subseteq\bigcup_{x\in\Int(A)}\Int(A)=\Int(A)
			\end{align*}
			i per tant
			\[\Int(A)=\bigcup_{x\in\Int(A)}\obert{U}_{x}\]
			i per la definició de \myref{def:topologia} trobem que \(\Int(A)\) és un obert.
		\end{proof}
	\end{proposition}
	\begin{definition}[Adherència]
		\labelname{adherència}\label{def:adherència}
	\end{definition}
%	\begin{definition}[Aplicació contínua i homeomorfisme]
%		\labelname{aplicació contínua}\label{def:aplicació contínua}
%		\labelname{homeomorfisme}\label{def:homeomorfisme topo}
%		Siguin \(X\) amb \(\tau\) i \(X'\) amb \(\tau'\) dos espais topològics i \(f\colon X\longrightarrow X'\) una aplicació tal que per a tot obert \(\obert{U}\) de \(X'\) tenim que el conjunt
%		\[f^{-1}(\obert{U})=\{x\in X\mid f(x)\in\obert{U}\}\]
%		és un obert de \(X\). Aleshores direm que \(f\) és una aplicació contínua.
%		
%		Si \(f\) és invertible i la seva inversa és una aplicació contínua direm que \(f\) és un homeomorfisme.
%	\end{definition}
%	\begin{observation}
%		\label{obs:les funcions contínues són aplicacions contínues}
%		Sigui \(f\) una funció contínua. Aleshores \(f\) és una aplicació contínua. %Proof?
%	\end{observation}
%	\begin{proposition}
%		Siguin \(X_{1}\) amb la topologia \(\tau_{1}\), \(X_{2}\) amb la topologia \(\tau_{2}\) i \(X_{3}\) amb la topologia \(\tau_{3}\) tres espais topològics i \(f\colon X_{1}\longrightarrow X_{2}\) i \(g\colon X_{2}\longrightarrow X_{3}\) dues aplicacions contínues. Aleshores l'aplicació
%		\begin{align*}
%			h\colon X_{1}&\longrightarrow X_{3} \\
%			x&\longmapsto f(g(x))
%		\end{align*}
%		és una aplicació contínua.
%		\begin{proof}
%			
%		\end{proof}
%	\end{proposition}

%	\subsection{Aplicacions contínues}
%	\subsection{Subespais}
%	\subsection{La topologia producte}
%	\subsection{La topologia quocient}
%	\subsection{Espais compactes}
%	\subsection{Espais de Hausdorff}
%	\subsection{Connexió}
%	\subsection{Varietats}
%	\subsection{Teorema de classificació de les superfícies compactes}
\end{document}
