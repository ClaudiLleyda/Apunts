\documentclass[../Apunts.tex]{subfiles}

\begin{document}
\chapter{Les topologies}
	\section{Espais mètrics}
	\subsection{Boles i oberts}
	\begin{definition}[Espai mètric]
		\labelname{espai mètric}\label{def:espai mètric}
		\labelname{distància}\label{def:distància}
		Sigui \(X\) un conjunt i \(\distancia\colon X\times X\longrightarrow\mathbb{R}\) una aplicació que per a tot \(x\), \(y\) i \(z\) de \(X\) satisfà
		\begin{enumerate}
			\item \(\distancia(x,y)=0\) si i només si \(x=y\).
			\item \(\distancia(x,y)=\distancia(y,x)\).
			\item \(\distancia(x,y)\leq\distancia(x,z)+\distancia(z,y)\).
			\item \(\distancia(x,y)\leq0\).
		\end{enumerate}
		Aleshores direm que \(X\) amb la distància \(\distancia\) és un espai mètric. També direm que \(\distancia\) és la distància o mètrica de l'espai mètric.
	\end{definition}
	\begin{definition}[Bola]
		\labelname{bola}\label{def:bola}
		Siguin \(X\) amb la distància \(\distancia\) un espai mètric, \(a\) un element de \(X\) i \(r>0\) un nombre real. Aleshores definim
		\[\bola(a,r)=\{x\in X\mid\distancia(x,a)<r\}\]
		com la bola de radi \(r\) centrada en \(a\).
	\end{definition}
	\begin{definition}[Obert]
		\labelname{obert}\label{def:obert espai mètric}
		Sigui \(X\) amb la distància \(\distancia\) un espai mètric i \(\obert{U}\) un subconjunt de \(X\) tals que per a tot element \(x\) de \(\obert{U}\) existeix un \(\varepsilon>0\) real tal que \(\bola(x,\varepsilon)\subset\obert{U}\). Aleshores direm que \(\obert{U}\) és un obert.
	\end{definition}
	\begin{proposition}
		\label{prop:les boles són oberts}
		Siguin \(X\) amb la distància \(\distancia\) un espai mètric i \(\bola(a,r)\) una bola de \(X\). Aleshores \(\bola(a,r)\) és un obert.
		\begin{proof}
			Prenem un element \(b\in\bola(a,r)\) i definim
			\begin{equation}
			\label{prop:les boles són oberts:eq1}
				\varepsilon=\frac{r-\distancia(a,b)}{2}.
			\end{equation}
			Aleshores considerem la bola \(\bola(b,\varepsilon)\) i tenim que \(\bola(b,\varepsilon)\subset\bola(a,r)\), ja que si prenem un element \(x\) de \(\bola(b,\varepsilon)\), per la definició de \myref{def:distància} trobem que
			\begin{align*}
				\distancia(x,a)&\leq\distancia(x,b)+\distancia(b,a)\\
				&<\varepsilon+\distancia(b,a)\tag{\myref{def:bola}}\\
				&=\frac{r-\distancia(a,b)}{2}+\distancia(b,a)\tag{\myref{prop:les boles són oberts:eq1}}\\
				&=\frac{r-\distancia(a,b)}{2}+\distancia(a,b)\tag{\myref{def:distància}}\\
				&=\frac{r+\distancia(a,b)}{2}<r,
			\end{align*}
			ja que, per la definició de \myref{def:bola} tenim que \(\distancia(a,b)<r\) i per tant trobem \(r+\distancia(a,b)<2r\).
		\end{proof}
	\end{proposition}
	\begin{proposition}[Propietat de Hausdorff]
		\labelname{}\label{prop:propietat de Hausdorff}
		Siguin \(X\) amb la distància \(\distancia\) un espai mètric i \(x\) i \(y\) dos elements diferents de \(X\). Aleshores existeixen dos oberts \(\obert{U}\) i \(\obert{V}\) disjunts tals que \(x\) és un element de \(\obert{U}\) i \(y\) és un element de \(\obert{V}\).
		\begin{proof}
			Definim \(r=\frac{\distancia(x,y)}{3}\) i considerem les boles \(\bola(x,r)\) i \(\bola(y,r)\). Per la definició de \myref{def:distància} i la definició de \myref{def:bola} tenim que \(x\) és un element de \(\bola(x,r)\) i \(y\) és un element de \(\bola(y,r)\).
			
			També tenim que les boles \(\bola(x,r)\) i \(\bola(y,r)\) són disjuntes, ja que si prenem un element \(a\) de \(X\) tal que \(a\) pertanyi a \(\bola(x,r)\) i a \(\bola(y,r)\) aleshores tenim, per la definició de \myref{def:bola}, que \(\distancia(x,a)<r\) i \(\distancia(a,y)<r\), i per la definició de \myref{def:distància} tenim que
			\[\distancia(x,a)+\distancia(a,y)\geq \distancia(x,y).\]
			Ara bé, tenim per hipòtesi que \(r=\frac{\distancia(x,y)}{3}\). Per tant
			\begin{align*}
				\distancia(x,y) &\leq \distancia(x,a)+\distancia(a,y) \\
				&\leq \frac{\distancia(x,y)}{3} + \frac{\distancia(x,y)}{3} \\
				&\leq \frac{2\distancia(x,y)}{3} < \distancia(x,y),
			\end{align*}
			i per tant aquest \(a\) no existeix i trobem que \(\bola(x,r)\) i \(\bola(y,r)\) són disjunts.
			
			Per acabar, per la proposició \myref{prop:les boles són oberts} tenim que les boles \(\bola(x,r)\) i \(\bola(y,r)\) són oberts, i hem acabat.
		\end{proof}
	\end{proposition}
	\section{L'espai topològic}
	\subsection{Una topologia d'un conjunt i els oberts}
	\begin{definition}[Topologia]	% Reescriure amb notació
		\labelname{topologia}\label{def:topologia}
		\labelname{espai topològic}\label{def:espai topològic}
		\labelname{obert}\label{def:obert}
		\labelname{punt}\label{def:punt}
		Sigui \(X\) un conjunt i \(\tau\) una família de subconjunts de \(X\) tals que
		\begin{enumerate}
		\item \(\emptyset\) i \(X\) són elements de \(\tau\).
		\item La intersecció d'una família finita d'elements de \(\tau\) és un element de \(\tau\).
		\item La unió d'una família d'elements de \(\tau\) és un element de \(\tau\).
		\end{enumerate}
		Aleshores direm que \(\tau\) és una topologia de \(X\) o que \(X\) amb la topologia \(\tau\) és un espai topològic.
		
		També direm que els elements de \(\tau\) són oberts i que els elements de \(X\) són punts.
	\end{definition}
	\begin{example}[Topologia induïda per una mètrica]
		\labelname{}\label{ex:topologia induida per una mètrica}
		Siguin \(X\) amb la distància \(\distancia\) un espai mètric i \(\tau=\{\obert{U}\subseteq X\mid \obert{U}\text{ és un obert}\}\). Volem veure que \(X\) amb la topologia \(\tau\) és un espai topològic.
		\begin{solution}
			Per la definició d'\myref{def:obert espai mètric} trobem que \(\emptyset\) és un obert, ja que per l'\myref{axiom:axioma de regularitat} tenim que \(\emptyset\) és un subconjunt de \(X\), i per a tot \(x\) de \(X\) existeix un \(\varepsilon>0\) tal que \(\bola(x,\varepsilon)\) pertany a \(\emptyset\). També tenim que \(X\) és un obert, ja que pel \myref{thm:doble inclusió} tenim que \(X\subseteq X\) i si prenem un element \(x\) de \(X\) tenim que per a tot \(\varepsilon>0\) la bola \(\bola(x,\varepsilon)\) és un subconjunt de \(X\). Per tant \(\emptyset\) i \(X\) són elements de \(\tau\).
			
			Prenem \(\{\obert{U}_{i}\}_{i=1}^{n}\) una família d'elements de \(\tau\) i considerem
			\[\obert{U}=\bigcap_{i=1}^{n}\obert{U}_{i}.\]
			
			Sigui \(x\) un element de \(\obert{U}\). Per la definició d'\myref{def:obert} tenim que per a tot \(i\in\{1,\dots,n\}\) existeix un \(\varepsilon_{i}>0\) real tal que
			\[\bola(x,\varepsilon_{i})\subset\obert{U}_{i}.\]
			Per tant, si definim \(\varepsilon=\min_{i\in\{1,\dots,n\}}\varepsilon_{i}\). Aleshores per la definició de \myref{def:bola} tenim que
			\[\bola(x,\varepsilon)\subseteq\bola(x,\varepsilon_{i})\quad\text{per a tot }i\in\{1,\dots,n\}.\]
			i per tant \(\bola(x,\varepsilon)\subset\obert{U}\) i per la definició de \myref{def:obert} tenim que \(\obert{U}\) és un obert.
			
			Prenem \(\{\obert{U}_{i}\}_{i\in I}\) una família d'oberts de \(\tau\) i considerem
			\[\obert{U}=\bigcup_{i\in I}\obert{U}_{i}.\]
			
			Sigui \(x\) un element de \(\obert{U}\). Per la definició d'\myref{def:unió de conjunts} tenim que existeix un \(i\in I\) tal que \(\obert{U}_{i}\) conté \(x\). Ara bé, per hipòtesi tenim que \(\obert{U}_{i}\) és un obert, i per la definició d'\myref{def:obert espai mètric} trobem que existeix un \(\varepsilon>0\) tal que \(\bola(x,\varepsilon)\) és un subconjunt de \(\obert{U}_{i}\), i com que \(\obert{U}_{i}\) és un subconjunt de \(\obert{U}\) tenim que \(\bola(x,\varepsilon)\) també és un subconjunt de \(\obert{U}\), i per la definició d'\myref{def:obert espai mètric} trobem que \(\obert{U}\) és un obert, i per la definició de \myref{def:espai topològic} hem acabat.
		\end{solution}
	\end{example}
	\begin{example}[Topologia grollera]
		\labelname{topologia grollera}\label{ex:topologia grollera}
		Siguin \(X\) i \(\tau=\{\emptyset,X\}\) dos conjunts. Aleshores \(\tau\) és una topologia de \(X\).
		\begin{solution}
			Comprovem les condicions de la definició de topologia. Com que \(\tau=\{\emptyset,X\}\) tenim que \(\emptyset\) i \(X\) són elements de \(\tau\). Observem també que
			\[\emptyset\cap X=\emptyset,\quad X\cap X=X\quad\text{i}\quad\emptyset\cup X=X\cup X=X,\]
			I per la definició de \myref{def:topologia}, tenim que \(\tau\) és una topologia de \(X\).
		\end{solution}
	\end{example}
	\begin{example}[Topologia discreta]
		\labelname{topologia discreta}\label{ex:topologia discreta}
		Siguin \(X\) i \(\tau=\mathcal{P}(X)\) dos conjunts. Aleshores \(\tau\) és una topologia de \(X\).
		\begin{solution}
			Tenim per l'\myref{axiom:conjunt potència} que \(\emptyset\) i \(X\) són subconjunts de \(\tau\), ja que per l'\myref{axiom:axioma de regularitat} trobem que \(\emptyset\subseteq X\), i pel \myref{thm:doble inclusió} trobem que \(X\subseteq X\).
			
			Per la definició d'\myref{def:unió de conjunts} trobem que si \(\{\obert{U}_{i}\}_{i\in I}\) és una família de subconjunts de \(X\) aleshores
			\[\bigcup_{i\in I}\obert{U}_{i}\subseteq X,\]
			i per la definició d'\myref{def:intersecció de conjunts} trobem que si \(\{\obert{U}_{i}\}_{i=1}^{n}\) és una família de subconjunts de \(X\) aleshores
			\[\bigcap_{i=1}^{n}\obert{U}_{i}\subseteq X,\]
			i per tant, com que \(\tau=\mathcal{P}(X)\) trobem per la definició de \myref{def:topologia} que \(\tau\) és una topologia de \(X\).
		\end{solution}
	\end{example}
	\subsection{Tancats}
	\begin{definition}[Tancat]
		\labelname{tancat}\label{def:tancat}
		Siguin \(X\) amb la topologia \(\tau\) un espai topològic i \(\tancat{C}\) un subconjunt de \(X\) tal que \(X\setminus\tancat{C}\) sigui obert. Aleshores direm que \(\tancat{C}\) és tancat.
	\end{definition}
	\begin{theorem}
		\label{thm:equivalència obert tancat definició de topologia}
		Sigui \(X\) amb la topologia \(\tau\) un espai topològic. Aleshores
		\begin{enumerate}
			\item\label{thm:equivalència obert tancat definició de topologia:enum 1} \(\emptyset\) i \(X\) són tancats.
			\item\label{thm:equivalència obert tancat definició de topologia:enum 2} La unió de qualsevol família finita de tancats és un tancat.
			\item\label{thm:equivalència obert tancat definició de topologia:enum 3} La intersecció de qualsevol família de tancats és un tancat.
		\end{enumerate}
		\begin{proof}
			Comencem veient el punt \eqref{thm:equivalència obert tancat definició de topologia:enum 1}. Tenim que \(X\setminus\emptyset=X\). Per la definició d'\myref{def:espai topològic} tenim que \(X\) és un obert, i per la definició de \myref{def:tancat} trobem que \(\emptyset\) és un tancat.
			
			També tenim que \(X\setminus X=\emptyset\). Per la definició d'\myref{def:espai topològic} tenim que \(\emptyset\) és un obert, i per la definició de \myref{def:tancat} trobem que \(X\) és un tancat.
			
			Veiem ara el punt \eqref{thm:equivalència obert tancat definició de topologia:enum 2}. Prenem una família \(\{\tancat{C}_{i}\}_{i=1}^{n}\) de tancats de \(X\) i considerem
			\[X\setminus\bigcup_{i=1}^{n}\tancat{C}_{i}=X\cap\left(\bigcup_{i=1}^{n}\tancat{C}_{i}\right)^{\complement}.\]
			Per la \myref{taut:primera llei de De Morgan} trobem que
			\begin{align*}
				X\cap\left(\bigcup_{i=1}^{n}\tancat{C}_{i}\right)^{\complement}&=X\cap\left(\bigcap_{i=1}^{n}\tancat{C}_{i}^{\complement}\right) \\
				&=\bigcap_{i=1}^{n}\left(X\cap\tancat{C}_{i}^{\complement}\right) \\
				&=\bigcap_{i=1}^{n}\left(X\setminus\tancat{C}_{i}\right).
			\end{align*}
			Per hipòtesi tenim que per a tot \(i\in\{1,\dots,n\}\) el conjunt \(\tancat{C}_{i}\) és un tancat, i per la definició de \myref{def:tancat} tenim que per a tot \(i\in\{1,\dots,n\}\) el conjunt \(X\setminus\tancat{C}_{i}\) és un obert, i per la definició d'\myref{def:espai topològic} tenim que \(\bigcap_{i=1}^{n}\left(X\setminus\tancat{C}_{i}\right)\) és un obert. Ara bé, tenim que
			\[X\setminus\bigcup_{i=1}^{n}\tancat{C}_{i}=\bigcap_{i=1}^{n}\left(X\setminus\tancat{C}_{i}\right),\]
			i per tant, per la definició de \myref{def:tancat} trobem que \(\bigcup_{i=1}^{n}\tancat{C}_{i}\) és un tancat.
			
			Veiem per acabat el punt \eqref{thm:equivalència obert tancat definició de topologia:enum 3}. Prenem una família \(\{\tancat{C}_{i}\}_{i\in I}\) de tancats de \(X\) i considerem
			\[X\setminus\bigcap_{i\in I}\tancat{C}_{i}=X\cap\left(\bigcap_{i\in I}\tancat{C}_{i}\right)^{\complement}.\]
			Per la \myref{taut:segona llei de De Morgan} trobem que
			\begin{align*}
				X\cap\left(\bigcap_{i\in I}\tancat{C}_{i}\right)^{\complement}&=X\cap\left(\bigcup_{i\in I}\tancat{C}_{i}^{\complement}\right) \\
				&=\bigcup_{i\in I}\left(X\cap\tancat{C}_{i}^{\complement}\right) \\
				&=\bigcup_{i\in I}\left(X\setminus\tancat{C}_{i}\right).
			\end{align*}
			Ara bé, per la definició de \myref{def:tancat} tenim que per a tot \(i\in I\) el conjunt \(X\setminus\tancat{C}_{i}\) és un obert, i per la definició de \myref{def:espai topològic} trobem que el conjunt \(\bigcup_{i\in I}\left(X\setminus\tancat{C}_{i}\right)\) és un obert. Tenim
			\[X\setminus\bigcap_{i\in I}\tancat{C}_{i}=\bigcup_{i\in I}\left(X\setminus\tancat{C}_{i}\right),\]
			i, de nou per la definició de \myref{def:tancat}, tenim que \(\bigcap_{i\in I}\tancat{C}_{i}\) és un tancat, com volíem veure.
		\end{proof}
	\end{theorem}
	\subsection{Base d'una topologia}
	\begin{definition}[Base d'una topologia]
		\labelname{base d'una topologia}\label{def:base d'una topologia}
		Siguin \(X\) amb la topologia \(\tau\) un espai topològic i \(\base{B}\) una família d'oberts tals que per a tot obert \(\obert{U}\) de \(X\) i per a tot punt \(x\) de \(\obert{U}\) existeix un \(B\in\base{B}\) tal que \(x\in B\subseteq\obert{U}\). Aleshores direm que \(\base{B}\) és una base de la topologia \(\tau\).
	\end{definition}
	\begin{example}
		Siguin \(X\) amb la distància \(\distancia\) un espai mètric i
		\[\base{B}=\{\bola(x,\varepsilon)\mid x\in X\text{ i }\varepsilon>0\}\]
		un conjunt. Aleshores \(\base{B}\) és una base de la topologia \(\tau\) induïda per la mètrica.
		\begin{solution}
			Tenim que
			\[\tau=\{\obert{U}\subseteq X\mid\obert{U}\text{ es un obert}\}.\]
			
			Prenem doncs un obert \(\obert{U}\) de \(\tau\) i un punt \(x\) de \(\obert{U}\). Per la definició d'\myref{def:obert espai mètric} tenim que existeix un \(\varepsilon>0\) real tal que \(\bola(x,\varepsilon)\) és un subconjunt de \(\obert{U}\), i per la definició de \myref{def:base d'una topologia} hem acabat.
		\end{solution}
	\end{example}
	\begin{definition}[Finor d'una topologia]
		\labelname{finor d'una topologia}\label{def:finor d'una topologia}
		Siguin \(X\) un conjunt i \(\tau\), \(\tau'\) dues topologies de \(X\) tals que \(\tau\subset\tau'\). Aleshores direm que \(\tau'\) és més fina que \(\tau\).
	\end{definition}
%	\begin{proposition}
%		Siguin \(X\) un conjunt i \(\tau\), \(\tau'\) dues topologies de \(X\) tals que \(\tau'\) sigui més fina que \(\tau\). Aleshores l'aplicació \(\Id\colon X\longrightarrow X'\) és una aplicació contínua.
%		\begin{proof}
%			
%		\end{proof}
%	\end{proposition}
	\begin{proposition}
		\label{prop:condició equivalent a base d'una topologia}
		\label{prop:condició per que una topologia sigui la més fina que conté una base}
		Siguin \(X\) un conjunt i \(\base{B}\) una família de subconjunts de \(X\) tals que
		\[\bigcup_{B\in\base{B}}B=X\]
		i tal que per a tot \(\obert{U}\) i \(\obert{V}\) de \(\base{B}\) i per a tot \(x\) de \(\obert{U}\cap\obert{V}\) existeix un \(\obert{W}\) de \(\base{B}\) tal que \(x\) pertanyi a \(\obert{W}\) i \(\obert{W}\subseteq\obert{U}\cap\obert{V}\).	Aleshores existeix una única topologia \(\tau\) de \(X\) tal que \(\base{B}\) és una base de \(\tau\) i \(\tau\) és la topologia menys fina que conté els elements de \(\base{B}\).
		\begin{proof}
			Definim
			\begin{equation}
				\label{prop:condició equivalent a base d'una topologia:eq1}
				\tau=\left\{\obert{U}\subseteq X\mid\obert{U}=\bigcup_{i\in I}B_{i}\text{ amb }B_{i}\in\base{B}\right\}
			\end{equation}
			Observem que \(X\) i \(\emptyset\) pertanyen a \(\tau\).
			
			Siguin \(\obert{U}\) i \(\obert{V}\) dos elements de \(\base{B}\) i prenem un element \(x\) de \(\obert{U}\cap\obert{V}\). Per hipòtesi tenim que existeix un element  \(\obert{W}_{x}\) de \(\base{B}\) tal que \(x\in\obert{W}\subseteq\obert{U}\cap\obert{V}\). Per tant
			\[\obert{U}\cap\obert{V}=\bigcup_{x\in\obert{U}\cap\obert{V}}\obert{W}_{x},\]
			i per la definició del conjunt \(\tau\) tenim que \(\obert{U}\cap\obert{V}\) és un element de \(\tau\).
			
			Prenem dos elements \(\obert{U}\) i \(\obert{V}\) del conjunt \(\tau\). Per la definició \eqref{prop:condició equivalent a base d'una topologia:eq1} trobem que existeixen dues famílies \(\{\obert{U}_{i}\}_{i\in I}\) i \(\{\obert{V}_{j}\}_{j\in J}\) d'elements de \(\base{B}\) tals que
			\[\obert{U}=\bigcup_{i\in I}\obert{U}_{i}\quad\text{i}\quad\obert{V}=\bigcup_{j\in J}\obert{V}_{j}.\]
			
			Per tant tenim que
			\[\obert{U}\cap\obert{V}=\bigcup_{i\in I}\bigcup_{j\in J}(\obert{U}_{i}\cap\obert{V}_{j}),\]
			i per la definició \eqref{prop:condició equivalent a base d'una topologia:eq1} trobem que \(\obert{U}\cap\obert{V}\) pertany a \(\tau\). Per tant per la definició de \myref{def:topologia} trobem que \(\tau\) és una topologia de \(X\).
			
			Veiem ara que \(\base{B}\) és base de la topologia \(\tau\). Prenem un obert \(\obert{U}\) de \(\tau\) i \(x\) un element de \(\obert{U}\). Per la definició \eqref{prop:condició equivalent a base d'una topologia:eq1} tenim que existeix una família \(\{B_{i}\}_{i\in I}\) d'elements de \(\base{B}\) tal que
			\[\obert{U}=\bigcup_{i\in I}B_{i}.\]
			Per tant existeix un element \(B\) de \(\base{B}\) tal que \(x\in B\) i \(B\subseteq\obert{U}\) i per la definició de \myref{def:base d'una topologia} trobem que \(\base{B}\) és una base de la topologia \(\tau\).
			
			Continuem veient que \(\tau\) és la topologia menys fina que conté els elements de \(\base{B}\). Suposem que existeix una topologia \(\tau'\) de \(X\) tal que \(\tau\) és més fina que \(\tau'\) i tal que \(\tau'\) conté els elements de \(\base{B}\). Per la definició de \myref{def:finor d'una topologia} això és que \(\tau'\subset\tau\).
			
			Prenem un obert \(\obert{U}\) de \(\tau\). Per la definició \eqref{prop:condició equivalent a base d'una topologia:eq1} trobem que existeix una família \(\{B_{i}\}_{i\in I}\) d'elements de \(\base{B}\) tals que
			\[\obert{U}=\bigcup_{i\in I}B_{i}.\]
			Ara bé, tenim per hipòtesi que \(\base{B}\) és un subconjunt de la topologia \(\tau'\), i per la definició de \myref{def:topologia} trobem que \(\obert{U}\) pertany a \(\tau'\). Per tant ha de ser \(\tau\subseteq\tau'\), i trobem que \(\tau\) és la topologia menys fina que conté els elements de \(\base{B}\).
			
			Veiem ara que aquesta topologia \(\tau\) és única. Suposem que existeix una altre topologia \(\tau'\) tal que \(\base{B}\) és una base de \(\tau'\) i \(\tau'\) és la topologia menys fina que conté els elements de \(\base{B}\).
			
			Prenem un obert \(\obert{U}\) de \(\tau\). Per la definició \eqref{prop:condició equivalent a base d'una topologia:eq1} trobem que existeix una família \(\{B_{i}\}_{i\in I}\) d'elements de \(\base{B}\) tals que
			\[\obert{U}=\bigcup_{i\in I}B_{i}.\]
			Per la definició de \myref{def:topologia} tenim que \(\obert{U}\) és un element de \(\tau'\), ja que per hipòtesi \(\base{B}\) és un subconjunt de \(\tau'\). Per tant tenim que \(\tau\subseteq\tau'\). Ara bé, ja hem vist que \(\tau\) és la topologia menys fina que conté els elements de \(\base{B}\), i per la definició de \myref{def:finor d'una topologia} això és que \(\tau'\subseteq\tau\), i pel \myref{thm:doble inclusió} ha de ser \(\tau=\tau'\), com volíem veure.
		\end{proof}
	\end{proposition}
	\subsection{Entorns, interior i adherència}
	\begin{definition}[Entorn]
		\labelname{entorn}\label{def:entorn}
		\labelname{punt interior}\label{def:punt interior}
		Siguin \(X\) amb la topologia \(\tau\) un espai topològic, \(x\) un punt i \(N\) un conjunt tal que existeix un obert \(\obert{U}\) satisfent que \(x\) és un element de \(\obert{U}\) i \(\obert{U}\) és un subconjunt de \(N\). Aleshores direm que \(N\) és un entorn de \(x\).
		
		També direm que \(x\) és un punt interior de \(N\).
	\end{definition}
	\begin{observation}
		\label{obs:tot punt té un entorn}
		Siguin \(X\) amb la topologia \(\tau\) un espai topològic i \(x\) un punt. Aleshores existeix un subconjunt \(N\) de \(X\) tal que \(N\) és un entorn de \(x\).
		\begin{proof}
			Si \(N=X\) tenim \(x\in X\subseteq N\). Per la definició de \myref{def:topologia} tenim que \(X\) és un obert i per la definició d'\myref{def:entorn} trobem que \(N\) és un entorn de \(x\).
		\end{proof}
	\end{observation}
	\begin{definition}[Interior]
		\labelname{interior}\label{def:interior}
		Siguin \(X\) amb la topologia \(\tau\) un espai topològic, \(x\) un punt de \(X\), \(A\) subconjunt de \(X\) i
		\[\interior(A)=\{x\in X\mid A\text{ és un entorn de }x\}.\]
		Aleshores direm que \(\interior(A)\) és l'interior del conjunt \(A\).
	\end{definition}
	\begin{proposition}
		\label{prop:l'interior d'un conjunt és un obert}
		Siguin \(X\) amb la topologia \(\tau\) un espai topològic i \(A\) un subconjunt de \(X\). Aleshores \(\interior(A)\) és un obert.
		\begin{proof}
			Prenem un element \(x\) de \(\interior(A)\). Per la definició de \myref{def:interior} tenim que existeix un obert \(\obert{U}_{x}\) tal que \(x\in\obert{U}_{x}\) i \(\obert{U}_{x}\subseteq A\). Aleshores tenim
			\begin{align*}
				\interior(A)&=\bigcup_{x\in\interior(A)}x \\
				&\subseteq\bigcup_{x\in\interior(A)}\obert{U}_{x} \\
				&\subseteq\bigcup_{x\in\interior(A)}\interior(A)=\interior(A).
			\end{align*}
			
			Per tant tenim que
			\[\interior(A)\subseteq\bigcup_{x\in\interior(A)}\obert{U}_{x}\quad\text{i}\quad\bigcup_{x\in\interior(A)}\obert{U}_{x}\subseteq\interior(A),\]
			i pel \myref{thm:doble inclusió} trobem
			\[\interior(A)=\bigcup_{x\in\interior(A)}\obert{U}_{x}\]
			i per la definició de \myref{def:topologia} trobem que \(\interior(A)\) és un obert.
		\end{proof}
	\end{proposition}
	\begin{proposition}
		\label{prop:l'interior d'un conjunt és l'unió de tots els oberts continguts en el conjunt}
		Siguin \(X\) amb la topologia \(\tau\) un espai topològic, \(A\) un subconjunt de \(X\) i
		\[Y=\{\obert{U}\in\tau\mid\obert{U}\subseteq A\}.\]
		Aleshores
		\[\interior(A)=\bigcup_{\obert{U}\in Y}\obert{U}.\]
		\begin{proof}
			Sigui \(x\) un element de \(A\) tal que existeixi un obert \(\obert{U}_{x}\) satisfent que \(x\) és un element de \(\obert{U}_{x}\) i \(\obert{U}_{x}\) és un subconjunt de \(A\). Aleshores tenim
			\begin{align*}
				\interior(A)&=\bigcup_{x\in\interior(A)}x \\
				&\subseteq\bigcup_{x\in\interior(A)}\obert{U}_{x} \\
				&\subseteq\bigcup_{x\in\interior(A)}\interior(A)=\interior(A)
			\end{align*}
			i pel \myref{thm:doble inclusió} trobem
			\[\interior(A)=\bigcup_{\obert{U}\in Y}\obert{U}\]
			com volíem veure.
		\end{proof}
	\end{proposition}
	\begin{corollary}
		\label{cor:l'interior d'un conjunt conté tots els seus oberts}
		Siguin \(X\) amb la topologia \(\tau\) un espai topològic, \(A\) un subconjunt de \(X\) i \(\obert{U}\) un obert tal que \(\obert{U}\) sigui un subconjunt de \(A\). Aleshores \(\obert{U}\) és un subconjunt de \(\interior(A)\).
	\end{corollary}
	\begin{corollary}
		\label{cor:un conjunt és obert si i només si el seu interior és obert}
		Siguin \(X\) amb la topologia \(\tau\) un espai topològic i \(A\) un subconjunt de \(X\). Aleshores \(A\) és un obert si i només si \(\interior(A)\) és un obert.
	\end{corollary}
	\begin{definition}[Punt adherent]
		\labelname{punt adherent}\label{def:punt adherent}
		Siguin \(X\) amb la topologia \(\tau\) un espai topològic, \(A\) un subconjunt de \(X\) i \(x\) un punt tal que per a tot entorn \(N\) de \(x\) tenim
		\[A\cap N\neq\emptyset.\]
		Aleshores direm que \(x\) és un punt adherent a \(A\).
	\end{definition}
	\begin{definition}[Clausura]
		\labelname{clausura}\label{def:clausura}
		Siguin \(X\) amb la topologia \(\tau\) un espai topològic i \(A\) un conjunt. Aleshores direm que
		\[\clausura(A)=\{x\in X\mid x\text{ és un punt adherent a }A\}\]
		és la clausura de \(A\).
	\end{definition}
	\begin{observation}
		\label{obs:la clausura d'un conjunt conté el conjunt}
		Siguin \(X\) amb la topologia \(\tau\) un espai topològic i \(A\) un subconjunt de \(X\). Aleshores
		\[A\subseteq\clausura(A).\]
	\end{observation}
	\begin{proposition}
		\label{prop:la clausura d'un conjunt és un tancat}
		Siguin \(X\) amb \(\tau\) un espai topològic i \(A\) un subconjunt de \(X\). Aleshores \(\clausura(A)\) és un tancat.
		\begin{proof}
			 Prenem un element \(x\) de \(X\setminus\clausura(A)\). Per la definició de \myref{def:clausura} tenim que existeix un entorn \(N\) de \(x\) tal que
			 \begin{equation}
			 	\label{prop:la clausura d'un conjunt és un tancat:eq1}
				 A\cap N=\emptyset,
			 \end{equation}
			 i per la definició d'\myref{def:entorn} tenim que existeix un obert \(\obert{U}_{x}\) tal que \(x\) és un element de \(\obert{U}_{x}\) i que \(\obert{U}_{x}\) és un subconjunt de \(N\). Per tant, per \eqref{prop:la clausura d'un conjunt és un tancat:eq1} tenim que \(A\cap\obert{U}_{x}=\emptyset\).
			 
			 Per tant trobem
			 \begin{align*}
				 X\setminus\clausura(A)&=\bigcup_{x\in X\setminus\clausura(A)}x \\
				 &\subseteq\bigcup_{x\in X\setminus\clausura(A)}\obert{U}_{x}\subseteq X\setminus\clausura(A),
			 \end{align*}
			 i pel \myref{thm:doble inclusió} trobem que
			 \[X\setminus\clausura(A)=\subseteq\bigcup_{x\in X\setminus\clausura(A)}\obert{U}_{x},\]
			 i com que \(\obert{U}_{x}\) és un obert, per la definició de \myref{def:topologia} tenim que \(X\setminus\clausura(A)\) és un obert, i per la definició de \myref{def:tancat} trobem que \(\clausura(A)\) és un tancat.
		\end{proof}
	\end{proposition}
	\begin{proposition}
		\label{prop:la clausura d'un conjunt és l'intersecció de tots els tancats que contenen el conjunt}
		Siguin \(X\) amb la topologia \(\tau\) un espai topològic, \(A\) un subconjunt de \(X\) i
		\[Y=\{\tancat{C}\subseteq X\mid A\subseteq\tancat{C}\text{ i }\tancat{C}\text{ és un tancat}\}.\]
		Aleshores
		\[\clausura(A)=\bigcap_{\tancat{C}\in Y}\tancat{C}.\]
		\begin{proof}
			Sigui \(x\) un punt adherent a \(A\) amb \(x\notin A\). Suposem que existeix un tancat \(\tancat{C}\) tal que \(A\) és un subconjunt de \(\tancat{C}\) i tal que \(x\) no pertany a \(\tancat{C}\). Aleshores \(x\) és un element de \(X\setminus\tancat{C}\), i per la definició de \myref{def:tancat} tenim que \(\obert{U}=X\setminus\tancat{C}\) és un obert. Ara bé, per la definició d'\myref{def:entorn} tenim que \(\obert{U}\) és un entorn de \(x\), però \(A\) és un subconjunt de \(\tancat{C}\), i per tant tenim que \(X\setminus\tancat{C}\) és un subconjunt de \(X\setminus A\), i tenim que \(\obert{U}\) és un subconjunt de \(X\setminus A\), i per tant \(A\cap\obert{U}=\emptyset\). Ara bé, per la definició de \myref{def:punt adherent} tenim que \(A\cap\obert{U}\neq\emptyset\). Per tant ha de ser que \(x\) pertany a \(\tancat{C}\) i trobem que
			\begin{equation}
				\label{prop:la clausura d'un conjunt és l'intersecció de tots els tancats que contenen el conjunt:eq1}
				\clausura(A)\subseteq\bigcap_{\tancat{C}\in Y}\tancat{C}.
			\end{equation}
			
			Per la proposició \myref{prop:la clausura d'un conjunt és un tancat} tenim que \(\clausura(A)\) és un tancat, i per tant \(\clausura(A)\) és un element de \(Y\) i tenim que
			\begin{equation}
				\label{prop:la clausura d'un conjunt és l'intersecció de tots els tancats que contenen el conjunt:eq2}
				\bigcap_{\tancat{C}\in Y}\tancat{C}\subseteq\clausura(A).
			\end{equation}
			
			Ara bé, tenim \eqref{prop:la clausura d'un conjunt és l'intersecció de tots els tancats que contenen el conjunt:eq1} i \eqref{prop:la clausura d'un conjunt és l'intersecció de tots els tancats que contenen el conjunt:eq2}, i pel \myref{thm:doble inclusió} trobem que
			\[\clausura(A)=\bigcap_{\tancat{C}\in Y}\tancat{C}.\qedhere\]
		\end{proof}
	\end{proposition}
	\begin{corollary}
		\label{cor:la clausura d'un conjunt és el tancat més petit que el conté}
		Siguin \(X\) amb la topologia \(\tau\) un espai topològic, \(A\) un subconjunt de \(X\) i \(\tancat{C}\) un tancat tal que \(A\) és un subconjunt de \(\tancat{C}\). Aleshores \(\tancat{C}\subseteq\clausura(A)\).
	\end{corollary}
	\begin{corollary}
		\label{cor:un conjunt és tancat si i només si és igual a la seva clausura}
		Siguin \(X\) amb la topologia \(\tau\) un espai topològic i \(A\) un subconjunt de \(X\). Aleshores \(A\) és tancat si i només si \(A=\clausura(A)\).
	\end{corollary}
	\begin{proposition}
		\label{prop:la clausura d'un conjunt és el complementari de l'interior del complementari del conjunt}
		Siguin \(X\) amb la topologia \(\tau\) un espai topològic i \(A\) un subconjunt de \(X\). Aleshores
		\[\clausura(X\setminus A)=X\setminus\interior(A).\]
		\begin{proof}
			Prenem un element \(x\) de \(\clausura(X\setminus A)\). Per la definició de \myref{def:clausura} tenim que per a tot obert \(\obert{U}\) que conté \(x\) tenim que
			\[\obert{U}\cap(X\setminus A)\neq\emptyset.\]
			Ara bé, per la definició d'\myref{def:interior} que \(\interior(A)\) és un subconjunt de \(A\), i per tant \(X\setminus A\) és un subconjunt de \(X\setminus\interior(A)\), i per tant \(\obert{U}\cap(X\setminus\interior(A))\neq\emptyset\). Per tant per la definició d'\myref{def:interior} trobem que
			\begin{equation}
				\label{prop:la clausura d'un conjunt és el complementari de l'interior del complementari del conjunt:eq1}
				\clausura(X\setminus A)\subseteq X\setminus\interior(A)
			\end{equation}
			
			Prenem ara un element \(x\) de \(X\setminus\interior(A)\). Per la definició d'\myref{def:interior} tenim que no existeix cap obert \(\obert{U}\) tal que \(x\) sigui un element de \(\obert{U}\) i \(\obert{U}\) sigui un subconjunt de \(A\), o equivalentment, que \(\obert{U}\cap(X\setminus A)\neq\emptyset\) per a tot obert \(\obert{U}\) tal que \(x\) sigui un element de  \(\obert{U}\) i que \(\obert{U}\) sigui un subconjunt de \(A\). Ara bé, per la definició de \myref{def:clausura} tenim que \(\obert{U}\) és un element de \(\clausura(X\setminus A)\), i per tant trobem que
			\begin{equation}
				\label{prop:la clausura d'un conjunt és el complementari de l'interior del complementari del conjunt:eq2}
				X\setminus\interior(A)\subseteq\clausura(X\setminus A).
			\end{equation}
			
			Per tant, amb \eqref{prop:la clausura d'un conjunt és el complementari de l'interior del complementari del conjunt:eq1} i \eqref{prop:la clausura d'un conjunt és el complementari de l'interior del complementari del conjunt:eq2} i el \myref{thm:doble inclusió} trobem que \(\clausura(X\setminus A)=X\setminus\interior(A)\), com volíem veure.
		\end{proof}
	\end{proposition}
	\section{Aplicacions contínues}
	\subsection{Aplicacions obertes, tancades i contínues}
	\begin{definition}[Aplicació oberta]
		\labelname{aplicació oberta}\label{def:aplicació oberta}
		Siguin \(X\) amb la topologia \(\tau_{X}\) i \(Y\) amb la topologia \(\tau_{Y}\) dos espais topològics i \(f:X\longrightarrow Y\) una aplicació tal que per a tot obert \(\obert{U}\) de \(X\) tenim que el conjunt \(\{f(x)\in Y\mid x\in\obert{U}\}\) és un obert de \(Y\). Aleshores direm que \(f\) és una aplicació oberta.
	\end{definition}
	\begin{observation}
		\label{obs:la composició d'aplicacions obertes és oberta}
		La composició d'aplicacions obertes és oberta.
	\end{observation}
	\begin{definition}[Aplicació tancada]
		\labelname{aplicació tancada}\label{def:aplicació tancada}
		Siguin \(X\) amb la topologia \(\tau_{X}\) i \(Y\) amb la topologia \(\tau_{Y}\) dos espais topològics i \(f:X\longrightarrow Y\) una aplicació tal que per a tot tancat \(\tancat{C}\) de \(X\) tenim que el conjunt \(\{f(x)\in Y\mid x\in\tancat{C}\}\) és un tancat de \(Y\). Aleshores direm que \(f\) és una aplicació tancada.
	\end{definition}
	\begin{observation}
		\label{obs:la composició d'aplicacions tancades és tancada}
		La composició d'aplicacions tancades és tancada.
	\end{observation}
	\begin{definition}[Aplicació contínua]
		\labelname{aplicació contínua}\label{def:aplicació contínua}
		Siguin \(X\) amb la topologia \(\tau_{X}\) i \(Y\) amb la topologia \(\tau_{Y}\) dos espais topològics i \(f:X\longrightarrow Y\) una aplicació tal que per a tot obert \(\obert{V}\) de \(Y\) el conjunt \(\{x\in X\mid f(x)\in\obert{V}\}\) és un obert de \(X\). Aleshores direm que \(f\) és una aplicació contínua.
	\end{definition}
	\begin{observation}
		\label{obs:la composició d'aplicacions contínues és contínua}
		La composició d'aplicacions contínues és contínua.
	\end{observation}
	\subsection{Homeomorfismes entre topologies}
	\begin{definition}[Homeomorfisme]
		\labelname{homeomorfisme entre topologies}\label{def:homeomorfisme entre topologies}
		Siguin \(X\) amb la topologia \(\tau_{X}\) i \(Y\) amb la topologia \(\tau_{Y}\) dos espais topològics i \(f\colon X\longrightarrow Y\) una aplicació bijectiva, oberta i contínua. Aleshores direm que \(f\) és un homeomorfisme.
	\end{definition}
	\begin{definition}[Espais topològics isomorfs]
		\labelname{espais topològics isomorfs}\label{def:espais topològics isomorfs}
		Siguin \(X\) amb la topologia \(\tau_{X}\) i \(Y\) amb la topologia \(\tau_{Y}\) dos espais topològics tals que existeix un homeomorfisme \(f\colon X\longrightarrow Y\). Aleshores direm que \(X\) i \(Y\) són dos espais topològics isomorfs. També denotarem
		\[X\cong Y.\]
	\end{definition}
	\begin{proposition}
			Siguin \(X\) amb la topologia \(\tau_{X}\) i \(Y\) amb la topologia \(\tau_{Y}\) dos espais topològics. Aleshores la relació
			\[X\cong Y\sii X\text{ és isomorf a }Y\]
			és una relació d'equivalència.
			\begin{proof}
				Comprovem les propietats de la definició de relació d'equivalència:
				\begin{enumerate}
					\item Reflexiva: Tenim que l'aplicació \(\Id_{X}\) és bijectiva, oberta i contínua, i per la definició d'\myref{def:espais topològics isomorfs} tenim que \(X\cong X\).
					\item Simètrica: Siguin \(X\) amb la topologia \(\tau_{X}\) i \(Y\) amb la topologia \(\tau_{Y}\) dos espais topològics isomorfs. %FER
					\item Transitiva: Siguin \(X_{1}\) amb la topologia \(\tau_{1}\), \(X_{2}\) amb la topologia \(\tau_{2}\) i \(X_{3}\) amb la topologia \(\tau_{3}\) tres espais topològics tals que \(X_{1}\cong X_{2}\) i \(X_{2}\cong X_{3}\). %FER
				\end{enumerate}
				I per la definició de \myref{def:relació d'equivalència} hem acabat.
			\end{proof}
	\end{proposition}
	
	
%	\begin{definition}[Aplicació contínua i homeomorfisme]
%		\labelname{aplicació contínua}\label{def:aplicació contínua}
%		\labelname{homeomorfisme}\label{def:homeomorfisme topo}
%		Siguin \(X\) amb \(\tau\) i \(X'\) amb \(\tau'\) dos espais topològics i \(f\colon X\longrightarrow X'\) una aplicació tal que per a tot obert \(\obert{U}\) de \(X'\) tenim que el conjunt
%		\[f^{-1}(\obert{U})=\{x\in X\mid f(x)\in\obert{U}\}\]
%		és un obert de \(X\). Aleshores direm que \(f\) és una aplicació contínua.
%		
%		Si \(f\) és invertible i la seva inversa és una aplicació contínua direm que \(f\) és un homeomorfisme.
%	\end{definition}
%	\begin{observation}
%		\label{obs:les funcions contínues són aplicacions contínues}
%		Sigui \(f\) una funció contínua. Aleshores \(f\) és una aplicació contínua. %Proof?
%	\end{observation}
%	\begin{proposition}
%		Siguin \(X_{1}\) amb la topologia \(\tau_{1}\), \(X_{2}\) amb la topologia \(\tau_{2}\) i \(X_{3}\) amb la topologia \(\tau_{3}\) tres espais topològics i \(f\colon X_{1}\longrightarrow X_{2}\) i \(g\colon X_{2}\longrightarrow X_{3}\) dues aplicacions contínues. Aleshores l'aplicació
%		\begin{align*}
%			h\colon X_{1}&\longrightarrow X_{3} \\
%			x&\longmapsto f(g(x))
%		\end{align*}
%		és una aplicació contínua.
%		\begin{proof}
%			
%		\end{proof}
%	\end{proposition}

%	\begin{theorem}
%		Siguin \(X\) amb la distància \(\distancia\) i \(X'\) amb la distància \(\distancia'\) dos espais mètrics i \(f\colon X\longrightarrow X'\) una funció. Aleshores \(f\) és contínua si i només si per a tot \(\obert{U}\) obert de \(X'\) tenim que \(\Ima_{\obert{U}}(f^{-1})\) és obert.
%		\begin{proof}
%			Veiem primer que la condició és necessària (\(\implica\)). Suposem doncs que \(f\) és contínua. Prenem un obert \(\obert{U}\) de \(X'\) i un element \(y\) de \(X'\) tal que \(f^{-1}(y)\in\Ima_{\obert{U}}(f^{-1})\), i denotem \(x=f^{-1}(y)\). Per la definició d'\myref{def:obert espai mètric} tenim que existeix un nombre real \(\varepsilon>0\) tal que \(\bola(y,\varepsilon)\subset\Ima_{\obert{U}}(f^{-1})\).
%			
%			Prenem un element \(x'\) de \(\bola(y,\varepsilon)\). Per la definició de \myref{def:bola} tenim que \(\distancia(x,x')<\varepsilon\), i per la definició de \myref{def:funció contínua} tenim que existeix un \(\delta>0\) real tal que \(\distancia'(f(x'),y)<\varepsilon\), i per la definició de \myref{def:bola} això és que \(f(x')\in\bola(y,\varepsilon)\). Per tant tenim que \(\Ima_{\bola(x,\delta)}(f)\subset\bola(y,\varepsilon)\) i trobem que \(\bola(x,\delta)\subset\Ima_{\obert{U}}(f^{-1})\), i per la definició de \myref{def:obert espai mètric} trobem que \(\Ima_{\obert{U}}(f^{-1})\) és un obert, com volíem veure.
%			
%			Veiem ara que la condició és suficient (\(\implicatper\)). Suposem doncs que per a tot \(\obert{U}\) obert de \(X'\) tenim que \(\Ima_{\obert{U}}(f^{-1})\) és obert.
%			
%			Prenem un element \(xy\) de \(X\) i un real \(\varepsilon>0\). Si denotem \(y=f(x)\) tenim, per la proposició \myref{prop:les boles són oberts}, que la bola \(\bola(y,\varepsilon)\) és un obert. Per hipòtesi tenim que \(\Ima_{\bola(y,\varepsilon)}(f^{-1})\) és un obert, i per la definició d'\myref{def:obert espai mètric} tenim que existeix un nombre real \(\delta>0\) tal que \(\bola(x,\delta)\subset\Ima_{\bola(y,\varepsilon)}(f^{-1})\).
%			
%			Ara bé, per la definició de \myref{def:bola} tenim que això és que per a tot \(x_{0}\) de \(x\) amb \(\distancia'(y,f(x_{0}))<\varepsilon\) tenim que \(\distancia(x,x_{0})<\delta\), i per tant, per la definició de \myref{def:funció contínua} tenim que \(f\) és contínua.
%		\end{proof}
%	\end{theorem}

%	\subsection{Aplicacions contínues}
%	\subsection{Subespais}
%	\subsection{La topologia producte}
%	\subsection{La topologia quocient}
%	\subsection{Espais compactes}
%	\subsection{Espais de Hausdorff}
%	\subsection{Connexió}
%	\subsection{Varietats}
%	\subsection{Teorema de classificació de les superfícies compactes}
	
%	\printbibliography
\end{document}
