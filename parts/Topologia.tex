\documentclass[../Apunts.tex]{subfiles}

\begin{document}
\chapter{Les topologies}
	\section{Espais mètrics}
	\subsection{Les funcions contínues i les boles}
	\begin{definition}[Espai mètric]
		\labelname{espai mètric}\label{def:espai mètric}
		\labelname{distància}\label{def:distància}
		Sigui \(X\) un conjunt i \(d\colon X\times X\longrightarrow\mathbb{R}\) una aplicació que per a tot \(x\), \(y\) i \(z\) de \(X\) satisfà
		\begin{enumerate}
			\item \(d(x,y)=0\) si i només si \(x=y\).
			\item \(d(x,y)=d(y,x)\).
			\item \(d(x,y)\leq d(x,z)+d(z,y)\).
			\item \(d(x,y)\leq0\).
		\end{enumerate}
		Aleshores direm que \((X,d)\) és un espai mètric. També direm que \(d\) és la distància o mètrica de l'espai mètric.
	\end{definition}
	\begin{definition}[Bola]
		\labelname{bola}\label{def:bola}
		Siguin \((X,d)\) un espai mètric, \(a\) un element de \(X\) i \(r>0\) un nombre real. Aleshores definim
		\[\B(a,r)=\{x\in X\mid d(x,a)<r\}\]
		com la bola de radi \(r\) centrada en \(a\).
	\end{definition}
	\begin{definition}[Obert]
		\labelname{obert}\label{def:obert}
		Sigui \((X,d)\) un espai mètric tal que per a tot element \(a\) de \(X\) existeix un \(\varepsilon>0\) real tal que \(\B(a,\varepsilon)\subset A\). Aleshores direm que \(X\) és un obert.
	\end{definition}
	\begin{proposition}
	\label{prop:les boles són oberts}
		Siguin \((X,d)\) un espai mètric i \(\B(a,r)\) una bola de \(X\). Aleshores \(\B(a,r)\) és un obert.
		\begin{proof}
			Prenem un element \(b\in\B(a,r)\) i definim
			\begin{equation}
			\label{prop:les boles són oberts:eq1}
				\varepsilon=\frac{r-d(a,b)}{2}.
			\end{equation}
			Aleshores considerem la bola \(\B(b,\varepsilon)\) i tenim que \(\B(b,\varepsilon)\subset\B(a,r)\), ja que si prenem un element \(x\) de \(\B(b,\varepsilon)\), per la definició de \myref{def:distància} trobem que
			\begin{align*}
				d(x,a)&\leq d(x,b)+d(b,a)\\
				&<\varepsilon+d(b,a)\tag{\myref{def:bola}}\\
				&=\frac{r-d(a,b)}{2}+d(b,a)\tag{\myref{prop:les boles són oberts:eq1}}\\
				&=\frac{r-d(a,b)}{2}+d(a,b)\tag{\myref{def:distància}}\\
				&=\frac{r+d(a,b)}{2}<r,
			\end{align*}
			ja que, per la definició de \myref{def:bola} tenim que \(d(a,b)<r\) i per tant trobem \(r+d(a,b)<2r\).
		\end{proof}
	\end{proposition}
%	\begin{definition}[Funció contínua] % Fer amb Teorema i la definició antiga
%		\labelname{funció contínua}\label{def:funció contínua}
%		Siguin \((X,d)\) i \((X',d')\) dos espais mètrics i \(f\colon X\longrightarrow X'\) una funció tal que per a cada \(x\in X\) i \(\varepsilon>0\) real existeix un \(\delta>0\) real tal que
%		\[\Ima_{\B(x,\delta)}(f)\subset\B(f(x)),\varepsilon).\]
%		Aleshores direm que \(f\) és contínua.
%	\end{definition}
	\begin{theorem}
		Siguin \((X,d)\) i \((X',d')\) dos espais mètrics i \(f\colon X\longrightarrow X'\) una funció. Aleshores \(f\) és contínua si i només si per a tot \(\mathcal{U}\) obert de \(X'\) tenim que \(\Ima_{\mathcal{U}}(f^{-1})\) és obert.
		\begin{proof}
			Veiem primer que la condició és necessària (\(\Rightarrow\)). Suposem doncs que \(f\) és contínua. Prenem un obert \(\mathcal{U}\) de \(X'\) i un element \(y\) de \(X'\) tal que \(f^{-1}(y)\in\Ima_{\mathcal{U}}(f^{-1})\), i denotem \(x=f^{-1}(y)\). Per la definició d'\myref{def:obert} tenim que existeix un nombre real \(\varepsilon>0\) tal que \(\B(y,\varepsilon)\subset\Ima_{\mathcal{U}}(f^{-1})\).
			
			Prenem un element \(x'\) de \(\B(y,\varepsilon)\). Per la definició de \myref{def:bola} tenim que \(d(x,x')<\varepsilon\), i per la definició de \myref{def:funció contínua} tenim que existeix un \(\delta>0\) real tal que \(d'(f(x'),y)<\varepsilon\), i per la definició de \myref{def:bola} això és que \(f(x')\in\B(y,\varepsilon)\). Per tant tenim que \(\Ima_{\B(x,\delta)}(f)\subset\B(y,\varepsilon)\) i trobem que \(\B(x,\delta)\subset\Ima_{\mathcal{U}}(f^{-1})\), i per la definició de \myref{def:obert} trobem que \(\Ima_{\mathcal{U}}(f^{-1})\) és un obert, com volíem veure.
			
			Veiem ara que la condició és suficient (\(\Leftarrow\)). Suposem doncs que per a tot \(\mathcal{U}\) obert de \(X'\) tenim que \(\Ima_{\mathcal{U}}(f^{-1})\) és obert.
			
			Prenem un element \(xy\) de \(X\) i un real \(\varepsilon>0\). Si denotem \(y=f(x)\) tenim, per la proposició \myref{prop:les boles són oberts}, que la bola \(\B(y,\varepsilon)\) és un obert. Per hipòtesi tenim que \(\Ima_{\B(y,\varepsilon)}(f^{-1})\) és un obert, i per la definició d'\myref{def:obert} tenim que existeix un nombre real \(\delta>0\) tal que \(\B(x,\delta)\subset\Ima_{\B(y,\varepsilon)}(f^{-1})\).
			
			Ara bé, per la definició de \myref{def:bola} tenim que això és que per a tot \(x_{0}\) de \(x\) amb \(d'(y,f(x_{0})<\varepsilon\) tenim que \(d(x,x_{0}<\delta)\), i per tant, per la definició de \myref{def:funció contínua} tenim que \(f\) és contínua.
		\end{proof}
	\end{theorem}
	\begin{definition}[Tancat]
		\labelname{tancat}\label{def:tancat}
		Sigui \(X\) un conjunt i \(C\) un subconjunt de \(X\) tal que \(X\setminus C\) sigui obert. Aleshores direm que \(C\) és tancat.
	\end{definition}
	\subsection{Axiomàtica de l'espai topològic}
	\begin{definition}[Topologia]	% Reescriure amb notació
		\labelname{topologia}\label{def:topologia}
		\labelname{oberts topologia}\label{def:oberts topologia}
		Sigui \(X\) un conjunt i \(\tau\) una família de subconjunts de \(X\) tals que
		\begin{enumerate}
		\item \(\emptyset\) i \(X\) són elements de \(\tau\).
		\item La intersecció d'una família finita d'elements de \(\tau\) és un element de \(\tau\).
		\item La unió d'una família d'elements de \(\tau\) és un element de \(\tau\).
		\end{enumerate}
		Aleshores direm que \(\tau\) és una topologia de \(X\). També direm que els elements de \(\tau\) són oberts.
	\end{definition}
%	\subsection{L'espai topològic}
%	\subsection{Entorns, interior i adherència}
%	\subsection{Aplicacions contínues}
%	\subsection{Subespais}
%	\subsection{La topologia producte}
%	\subsection{La topologia quocient}
%	\subsection{Espais compactes}
%	\subsection{Espais de Hausdorff}
%	\subsection{Connexió}
%	\subsection{Varietats}
%	\subsection{Teorema de classificació de les superfícies compactes}
\end{document}