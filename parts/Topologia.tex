\documentclass[../Apunts.tex]{subfiles}

\begin{document}
\part{Topologia}
\chapter{L'espai topològic}
	\section{Espais mètrics}
	\subsection{Boles i oberts}
	\begin{definition}[Espai mètric]
		\labelname{espai mètric}\label{def:espai mètric}
		\labelname{distància}\label{def:distància}
		Sigui \(X\) un conjunt i \(\distancia\colon X\times X\longrightarrow\mathbb{R}\) una aplicació que per a tot \(x\), \(y\) i \(z\) de \(X\) satisfà
		\begin{enumerate}
			\item \(\distancia(x,y)=0\) si i només si \(x=y\).
			\item \(\distancia(x,y)=\distancia(y,x)\).
			\item \(\distancia(x,y)\leq\distancia(x,z)+\distancia(z,y)\).
			\item \(\distancia(x,y)\leq0\).
		\end{enumerate}
		Aleshores direm que \(X\) amb la distància \(\distancia\) és un espai mètric. També direm que \(\distancia\) és la distància o mètrica de l'espai mètric.
	\end{definition}
	\begin{definition}[Bola]
		\labelname{bola}\label{def:bola}
		Siguin \(X\) amb la distància \(\distancia\) un espai mètric, \(a\) un element de \(X\) i \(r>0\) un nombre real. Aleshores definim
		\[\bola(a,r)=\{x\in X\mid\distancia(x,a)<r\}\]
		com la bola de radi \(r\) centrada en \(a\).
	\end{definition}
	\begin{definition}[Obert]
		\labelname{obert}\label{def:obert espai mètric}
		Sigui \(X\) amb la distància \(\distancia\) un espai mètric i \(\obert{U}\) un subconjunt de \(X\) tals que per a tot element \(x\) de \(\obert{U}\) existeix un \(\varepsilon>0\) real tal que \(\bola(x,\varepsilon)\subset\obert{U}\). Aleshores direm que \(\obert{U}\) és un obert.
	\end{definition}
	\begin{proposition}
		\label{prop:les boles són oberts}
		Siguin \(X\) amb la distància \(\distancia\) un espai mètric i \(\bola(a,r)\) una bola de \(X\). Aleshores \(\bola(a,r)\) és un obert.
		\begin{proof}
			Prenem un element \(b\) de \(\bola(a,r)\) i definim
			\begin{equation}
			\label{prop:les boles són oberts:eq1}
				\varepsilon=\frac{r-\distancia(a,b)}{2}.
			\end{equation}
			Aleshores considerem la bola \(\bola(b,\varepsilon)\) i tenim que \(\bola(b,\varepsilon)\subset\bola(a,r)\), ja que si prenem un element \(x\) de \(\bola(b,\varepsilon)\), per la definició de \myref{def:distància} trobem que
			\begin{align*}
				\distancia(x,a)&\leq\distancia(x,b)+\distancia(b,a)\\
				&<\varepsilon+\distancia(b,a)\tag{\myref{def:bola}}\\
				&=\frac{r-\distancia(a,b)}{2}+\distancia(b,a)\tag{\myref{prop:les boles són oberts:eq1}}\\
				&=\frac{r-\distancia(a,b)}{2}+\distancia(a,b)\tag{\myref{def:distància}}\\
				&=\frac{r+\distancia(a,b)}{2}<r,
			\end{align*}
			ja que, per la definició de \myref{def:bola} tenim que \(\distancia(a,b)<r\) i per tant trobem \(r+\distancia(a,b)<2r\).
		\end{proof}
	\end{proposition}
	\begin{proposition}[Propietat de Hausdorff]
		\labelname{}\label{prop:propietat de Hausdorff}
		\label{prop:els espais mètris són Hausdorff}
		Siguin \(X\) amb la distància \(\distancia\) un espai mètric i \(x\) i \(y\) dos elements diferents de \(X\). Aleshores existeixen dos oberts \(\obert{U}\) i \(\obert{V}\) disjunts tals que \(x\) és un element de \(\obert{U}\) i \(y\) és un element de \(\obert{V}\).
		\begin{proof}
			Definim \(r=\frac{\distancia(x,y)}{3}\) i considerem les boles \(\bola(x,r)\) i \(\bola(y,r)\). Per la definició de \myref{def:distància} i la definició de \myref{def:bola} tenim que \(x\) és un element de \(\bola(x,r)\) i \(y\) és un element de \(\bola(y,r)\).
			
			També tenim que les boles \(\bola(x,r)\) i \(\bola(y,r)\) són disjuntes, ja que si prenem un element \(a\) de \(X\) tal que \(a\) pertanyi a \(\bola(x,r)\) i a \(\bola(y,r)\) aleshores tenim, per la definició de \myref{def:bola}, que \(\distancia(x,a)<r\) i \(\distancia(a,y)<r\), i per la definició de \myref{def:distància} tenim que
			\[\distancia(x,a)+\distancia(a,y)\geq \distancia(x,y).\]
			Ara bé, tenim per hipòtesi que \(r=\frac{\distancia(x,y)}{3}\). Per tant
			\begin{align*}
				\distancia(x,y) &\leq \distancia(x,a)+\distancia(a,y) \\
				&\leq \frac{\distancia(x,y)}{3} + \frac{\distancia(x,y)}{3} \\
				&\leq \frac{2\distancia(x,y)}{3} < \distancia(x,y),
			\end{align*}
			i per tant aquest \(a\) no existeix i trobem que \(\bola(x,r)\) i \(\bola(y,r)\) són disjunts.
			
			Per acabar, per la proposició \myref{prop:les boles són oberts} tenim que les boles \(\bola(x,r)\) i \(\bola(y,r)\) són oberts, i hem acabat.
		\end{proof}
	\end{proposition}
	\section{L'espai topològic}
	\subsection{Una topologia d'un conjunt i els oberts}
	\begin{definition}[Topologia]	% Reescriure amb notació
		\labelname{topologia}\label{def:topologia}
		\labelname{espai topològic}\label{def:espai topològic}
		\labelname{obert}\label{def:obert}
		\labelname{punt}\label{def:punt}
		Sigui \(X\) un conjunt i \(\tau\) una família de subconjunts de \(X\) tals que
		\begin{enumerate}
		\item \(\emptyset\) i \(X\) són elements de \(\tau\).
		\item La intersecció d'una família finita d'elements de \(\tau\) és un element de \(\tau\).
		\item La unió d'una família d'elements de \(\tau\) és un element de \(\tau\).
		\end{enumerate}
		Aleshores direm que \(\tau\) és una topologia de \(X\) o que \(X\) amb la topologia \(\tau\) és un espai topològic.
		
		També direm que els elements de \(\tau\) són oberts i que els elements de \(X\) són punts.
	\end{definition}
	\begin{example}[Topologia induïda per una mètrica]
		\labelname{}\label{ex:topologia induida per una mètrica}
		Siguin \(X\) amb la distància \(\distancia\) un espai mètric i \(\tau=\{\obert{U}\subseteq X\mid \obert{U}\text{ és un obert}\}\). Volem veure que \(X\) amb la topologia \(\tau\) és un espai topològic.
		\begin{solution}
			Per la definició d'\myref{def:obert espai mètric} trobem que \(\emptyset\) és un obert, ja que per l'\myref{axiom:axioma de regularitat} tenim que \(\emptyset\) és un subconjunt de \(X\), i per a tot \(x\) de \(X\) existeix un \(\varepsilon>0\) tal que \(\bola(x,\varepsilon)\) pertany a \(\emptyset\). També tenim que \(X\) és un obert, ja que pel \myref{thm:doble inclusió} tenim que \(X\subseteq X\) i si prenem un element \(x\) de \(X\) tenim que per a tot \(\varepsilon>0\) la bola \(\bola(x,\varepsilon)\) és un subconjunt de \(X\). Per tant \(\emptyset\) i \(X\) són elements de \(\tau\).
			
			Prenem \(\{\obert{U}_{i}\}_{i=1}^{n}\) una família d'elements de \(\tau\) i considerem
			\[\obert{U}=\bigcap_{i=1}^{n}\obert{U}_{i}.\]
			
			Sigui \(x\) un element de \(\obert{U}\). Per la definició d'\myref{def:obert} tenim que per a tot \(i\in\{1,\dots,n\}\) existeix un \(\varepsilon_{i}>0\) real tal que
			\[\bola(x,\varepsilon_{i})\subset\obert{U}_{i}.\]
			Per tant, si definim \(\varepsilon=\min_{i\in\{1,\dots,n\}}\varepsilon_{i}\). Aleshores per la definició de \myref{def:bola} tenim que
			\[\bola(x,\varepsilon)\subseteq\bola(x,\varepsilon_{i})\quad\text{per a tot }i\in\{1,\dots,n\}.\]
			i per tant \(\bola(x,\varepsilon)\subset\obert{U}\) i per la definició d'\myref{def:obert} tenim que \(\obert{U}\) és un obert.
			
			Prenem \(\{\obert{U}_{i}\}_{i\in I}\) una família d'oberts de \(\tau\) i considerem
			\[\obert{U}=\bigcup_{i\in I}\obert{U}_{i}.\]
			
			Sigui \(x\) un element de \(\obert{U}\). Per la definició d'\myref{def:unió de conjunts} tenim que existeix un \(i\in I\) tal que \(\obert{U}_{i}\) conté \(x\). Ara bé, per hipòtesi tenim que \(\obert{U}_{i}\) és un obert, i per la definició d'\myref{def:obert espai mètric} trobem que existeix un \(\varepsilon>0\) tal que \(\bola(x,\varepsilon)\) és un subconjunt de \(\obert{U}_{i}\), i com que \(\obert{U}_{i}\) és un subconjunt de \(\obert{U}\) tenim que \(\bola(x,\varepsilon)\) també és un subconjunt de \(\obert{U}\), i per la definició d'\myref{def:obert espai mètric} trobem que \(\obert{U}\) és un obert, i per la definició d'\myref{def:espai topològic} hem acabat.
		\end{solution}
	\end{example}
	\begin{example}[Topologia grollera]
		\labelname{}\label{ex:topologia grollera}
		Siguin \(X\) i \(\tau=\{\emptyset,X\}\) dos conjunts. Aleshores \(\tau\) és una topologia de \(X\).
		\begin{solution}
			Comprovem les condicions de la definició de topologia. Com que \(\tau=\{\emptyset,X\}\) tenim que \(\emptyset\) i \(X\) són elements de \(\tau\). Observem també que \(\emptyset\cap X=\emptyset\), \(X\cap X=X\) i \(\emptyset\cup X=X\cup X=X\), i per la definició de \myref{def:topologia}, tenim que \(\tau\) és una topologia de \(X\).
		\end{solution}
	\end{example}
	\begin{example}[Topologia discreta]
		\labelname{}\label{ex:topologia discreta}
		Siguin \(X\) i \(\tau=\mathcal{P}(X)\) dos conjunts. Aleshores \(\tau\) és una topologia de \(X\).
		\begin{solution}
			Tenim per l'\myref{axiom:conjunt potència} que \(\emptyset\) i \(X\) són subconjunts de \(\tau\), ja que per l'\myref{axiom:axioma de regularitat} trobem que \(\emptyset\subseteq X\), i pel \myref{thm:doble inclusió} trobem que \(X\subseteq X\).
			
			Per la definició d'\myref{def:unió de conjunts} trobem que si \(\{\obert{U}_{i}\}_{i\in I}\) és una família de subconjunts de \(X\) aleshores
			\[\bigcup_{i\in I}\obert{U}_{i}\subseteq X,\]
			i per la definició d'\myref{def:intersecció de conjunts} trobem que si \(\{\obert{U}_{i}\}_{i=1}^{n}\) és una família de subconjunts de \(X\) aleshores
			\[\bigcap_{i=1}^{n}\obert{U}_{i}\subseteq X,\]
			i per tant, com que \(\tau=\mathcal{P}(X)\) trobem per la definició de \myref{def:topologia} que \(\tau\) és una topologia de \(X\).
		\end{solution}
	\end{example}
	\subsection{Tancats}
	\begin{definition}[Tancat]
		\labelname{tancat}\label{def:tancat}
		Siguin \(X\) amb la topologia \(\tau\) un espai topològic i \(\tancat{C}\) un subconjunt de \(X\) tal que \(X\setminus\tancat{C}\) sigui obert. Aleshores direm que \(\tancat{C}\) és tancat.
	\end{definition}
	\begin{theorem}
		\label{thm:equivalència obert tancat definició de topologia}
		Sigui \(X\) amb la topologia \(\tau\) un espai topològic. Aleshores
		\begin{enumerate}
			\item\label{thm:equivalència obert tancat definició de topologia:enum 1} \(\emptyset\) i \(X\) són tancats.
			\item\label{thm:equivalència obert tancat definició de topologia:enum 2} La unió de qualsevol família finita de tancats és un tancat.
			\item\label{thm:equivalència obert tancat definició de topologia:enum 3} La intersecció de qualsevol família de tancats és un tancat.
		\end{enumerate}
		\begin{proof}
			Comencem veient el punt \eqref{thm:equivalència obert tancat definició de topologia:enum 1}. Tenim que \(X\setminus\emptyset=X\). Per la definició d'\myref{def:espai topològic} tenim que \(X\) és un obert, i per la definició de \myref{def:tancat} trobem que \(\emptyset\) és un tancat.
			
			També tenim que \(X\setminus X=\emptyset\). Per la definició d'\myref{def:espai topològic} tenim que \(\emptyset\) és un obert, i per la definició de \myref{def:tancat} trobem que \(X\) és un tancat.
			
			Veiem ara el punt \eqref{thm:equivalència obert tancat definició de topologia:enum 2}. Prenem una família \(\{\tancat{C}_{i}\}_{i=1}^{n}\) de tancats de \(X\) i considerem
			\[X\setminus\bigcup_{i=1}^{n}\tancat{C}_{i}=X\cap\left(\bigcup_{i=1}^{n}\tancat{C}_{i}\right)^{\complement}.\]
			Per la \myref{taut:primera llei de De Morgan} trobem que
			\begin{align*}
				X\cap\left(\bigcup_{i=1}^{n}\tancat{C}_{i}\right)^{\complement}&=X\cap\left(\bigcap_{i=1}^{n}\tancat{C}_{i}^{\complement}\right) \\
				&=\bigcap_{i=1}^{n}\left(X\cap\tancat{C}_{i}^{\complement}\right) \\
				&=\bigcap_{i=1}^{n}\left(X\setminus\tancat{C}_{i}\right).
			\end{align*}
			Per hipòtesi tenim que per a tot \(i\in\{1,\dots,n\}\) el conjunt \(\tancat{C}_{i}\) és un tancat, i per la definició de \myref{def:tancat} tenim que per a tot \(i\in\{1,\dots,n\}\) el conjunt \(X\setminus\tancat{C}_{i}\) és un obert, i per la definició d'\myref{def:espai topològic} tenim que \(\bigcap_{i=1}^{n}\left(X\setminus\tancat{C}_{i}\right)\) és un obert. Ara bé, tenim que
			\[X\setminus\bigcup_{i=1}^{n}\tancat{C}_{i}=\bigcap_{i=1}^{n}\left(X\setminus\tancat{C}_{i}\right),\]
			i per tant, per la definició de \myref{def:tancat} trobem que \(\bigcup_{i=1}^{n}\tancat{C}_{i}\) és un tancat.
			
			Veiem per acabat el punt \eqref{thm:equivalència obert tancat definició de topologia:enum 3}. Prenem una família \(\{\tancat{C}_{i}\}_{i\in I}\) de tancats de \(X\) i considerem
			\[X\setminus\bigcap_{i\in I}\tancat{C}_{i}=X\cap\left(\bigcap_{i\in I}\tancat{C}_{i}\right)^{\complement}.\]
			Per la \myref{taut:segona llei de De Morgan} trobem que
			\begin{align*}
				X\cap\left(\bigcap_{i\in I}\tancat{C}_{i}\right)^{\complement}&=X\cap\left(\bigcup_{i\in I}\tancat{C}_{i}^{\complement}\right) \\
				&=\bigcup_{i\in I}\left(X\cap\tancat{C}_{i}^{\complement}\right) \\
				&=\bigcup_{i\in I}\left(X\setminus\tancat{C}_{i}\right).
			\end{align*}
			Ara bé, per la definició de \myref{def:tancat} tenim que per a tot \(i\in I\) el conjunt \(X\setminus\tancat{C}_{i}\) és un obert, i per la definició d'\myref{def:espai topològic} trobem que el conjunt \(\bigcup_{i\in I}\left(X\setminus\tancat{C}_{i}\right)\) és un obert. Tenim
			\[X\setminus\bigcap_{i\in I}\tancat{C}_{i}=\bigcup_{i\in I}\left(X\setminus\tancat{C}_{i}\right),\]
			i, de nou per la definició de \myref{def:tancat}, tenim que \(\bigcap_{i\in I}\tancat{C}_{i}\) és un tancat, com volíem veure.
		\end{proof}
	\end{theorem}
	\subsection{Base d'una topologia}
	\begin{definition}[Base d'una topologia]
		\labelname{base d'una topologia}\label{def:base d'una topologia}
		Siguin \(X\) amb la topologia \(\tau\) un espai topològic i \(\base{B}\) una família d'oberts tals que per a tot obert \(\obert{U}\) de \(X\) i per a tot punt \(x\) de \(\obert{U}\) existeix un \(B\in\base{B}\) tal que \(x\in B\subseteq\obert{U}\). Aleshores direm que \(\base{B}\) és una base de la topologia \(\tau\).
	\end{definition}
	\begin{example}
		Siguin \(X\) amb la distància \(\distancia\) un espai mètric i
		\[\base{B}=\{\bola(x,\varepsilon)\mid x\in X\text{ i }\varepsilon>0\}\]
		un conjunt. Aleshores \(\base{B}\) és una base de la topologia \(\tau\) induïda per la mètrica.
		\begin{solution}
			Tenim que
			\[\tau=\{\obert{U}\subseteq X\mid\obert{U}\text{ es un obert}\}.\]
			
			Prenem doncs un obert \(\obert{U}\) de \(\tau\) i un punt \(x\) de \(\obert{U}\). Per la definició d'\myref{def:obert espai mètric} tenim que existeix un \(\varepsilon>0\) real tal que \(\bola(x,\varepsilon)\) és un subconjunt de \(\obert{U}\), i per la definició de \myref{def:base d'una topologia} hem acabat.
		\end{solution}
	\end{example}
	\begin{definition}[Finor d'una topologia]
		\labelname{finor d'una topologia}\label{def:finor d'una topologia}
		Siguin \(X\) un conjunt i \(\tau\), \(\tau'\) dues topologies de \(X\) tals que \(\tau\subset\tau'\). Aleshores direm que \(\tau'\) és més fina que \(\tau\).
	\end{definition}
%	\begin{proposition}
%		Siguin \(X\) un conjunt i \(\tau\), \(\tau'\) dues topologies de \(X\) tals que \(\tau'\) sigui més fina que \(\tau\). Aleshores l'aplicació \(\Id\colon X\longrightarrow X'\) és una aplicació contínua.
%		\begin{proof}
%			
%		\end{proof}
%	\end{proposition}
	\begin{proposition}
		\label{prop:condició equivalent a base d'una topologia}
		\label{prop:condició per que una topologia sigui la més fina que conté una base}
		Siguin \(X\) un conjunt i \(\base{B}\) una família de subconjunts de \(X\) tals que
		\[\bigcup_{B\in\base{B}}B=X\]
		i tal que per a tot \(\obert{U}\) i \(\obert{V}\) de \(\base{B}\) i per a tot \(x\) de \(\obert{U}\cap\obert{V}\) existeix un \(\obert{W}\) de \(\base{B}\) tal que \(x\) pertanyi a \(\obert{W}\) i \(\obert{W}\subseteq\obert{U}\cap\obert{V}\).	Aleshores existeix una única topologia \(\tau\) de \(X\) tal que \(\base{B}\) és una base de \(\tau\) i \(\tau\) és la topologia menys fina que conté els elements de \(\base{B}\).
		\begin{proof}
			Definim
			\begin{equation}
				\label{prop:condició equivalent a base d'una topologia:eq1}
				\tau=\left\{\obert{U}\subseteq X\mid\obert{U}=\cup_{i\in I}B_{i}\text{ amb }B_{i}\in\base{B}\right\}
			\end{equation}
			Observem que \(X\) i \(\emptyset\) pertanyen a \(\tau\).
			
			Siguin \(\obert{U}\) i \(\obert{V}\) dos elements de \(\base{B}\) i prenem un element \(x\) de \(\obert{U}\cap\obert{V}\). Per hipòtesi tenim que existeix un element  \(\obert{W}_{x}\) de \(\base{B}\) tal que \(x\in\obert{W}\subseteq\obert{U}\cap\obert{V}\). Per tant
			\[\obert{U}\cap\obert{V}=\bigcup_{x\in\obert{U}\cap\obert{V}}\obert{W}_{x},\]
			i per la definició del conjunt \(\tau\) tenim que \(\obert{U}\cap\obert{V}\) és un element de \(\tau\).
			
			Prenem dos elements \(\obert{U}\) i \(\obert{V}\) del conjunt \(\tau\). Per la definició \eqref{prop:condició equivalent a base d'una topologia:eq1} trobem que existeixen dues famílies \(\{\obert{U}_{i}\}_{i\in I}\) i \(\{\obert{V}_{j}\}_{j\in J}\) d'elements de \(\base{B}\) tals que
			\[\obert{U}=\bigcup_{i\in I}\obert{U}_{i}\quad\text{i}\quad\obert{V}=\bigcup_{j\in J}\obert{V}_{j}.\]
			
			Per tant tenim que
			\[\obert{U}\cap\obert{V}=\bigcup_{i\in I}\bigcup_{j\in J}(\obert{U}_{i}\cap\obert{V}_{j}),\]
			i per la definició \eqref{prop:condició equivalent a base d'una topologia:eq1} trobem que \(\obert{U}\cap\obert{V}\) pertany a \(\tau\). Per tant per la definició de \myref{def:topologia} trobem que \(\tau\) és una topologia de \(X\).
			
			Veiem ara que \(\base{B}\) és base de la topologia \(\tau\). Prenem un obert \(\obert{U}\) de \(\tau\) i \(x\) un element de \(\obert{U}\). Per la definició \eqref{prop:condició equivalent a base d'una topologia:eq1} tenim que existeix una família \(\{B_{i}\}_{i\in I}\) d'elements de \(\base{B}\) tal que
			\[\obert{U}=\bigcup_{i\in I}B_{i}.\]
			Per tant existeix un element \(B\) de \(\base{B}\) tal que \(x\in B\) i \(B\subseteq\obert{U}\) i per la definició de \myref{def:base d'una topologia} trobem que \(\base{B}\) és una base de la topologia \(\tau\).
			
			Continuem veient que \(\tau\) és la topologia menys fina que conté els elements de \(\base{B}\). Suposem que existeix una topologia \(\tau'\) de \(X\) tal que \(\tau\) és més fina que \(\tau'\) i tal que \(\tau'\) conté els elements de \(\base{B}\). Per la definició de \myref{def:finor d'una topologia} això és que \(\tau'\subset\tau\).
			
			Prenem un obert \(\obert{U}\) de \(\tau\). Per la definició \eqref{prop:condició equivalent a base d'una topologia:eq1} trobem que existeix una família \(\{B_{i}\}_{i\in I}\) d'elements de \(\base{B}\) tals que
			\[\obert{U}=\bigcup_{i\in I}B_{i}.\]
			Ara bé, tenim per hipòtesi que \(\base{B}\) és un subconjunt de la topologia \(\tau'\), i per la definició de \myref{def:topologia} trobem que \(\obert{U}\) pertany a \(\tau'\). Per tant ha de ser \(\tau\subseteq\tau'\), i trobem que \(\tau\) és la topologia menys fina que conté els elements de \(\base{B}\).
			
			Veiem ara que aquesta topologia \(\tau\) és única. Suposem que existeix una altre topologia \(\tau'\) tal que \(\base{B}\) és una base de \(\tau'\) i \(\tau'\) és la topologia menys fina que conté els elements de \(\base{B}\).
			
			Prenem un obert \(\obert{U}\) de \(\tau\). Per la definició \eqref{prop:condició equivalent a base d'una topologia:eq1} trobem que existeix una família \(\{B_{i}\}_{i\in I}\) d'elements de \(\base{B}\) tals que
			\[\obert{U}=\bigcup_{i\in I}B_{i}.\]
			Per la definició de \myref{def:topologia} tenim que \(\obert{U}\) és un element de \(\tau'\), ja que per hipòtesi \(\base{B}\) és un subconjunt de \(\tau'\). Per tant tenim que \(\tau\subseteq\tau'\). Ara bé, ja hem vist que \(\tau\) és la topologia menys fina que conté els elements de \(\base{B}\), i per la definició de \myref{def:finor d'una topologia} això és que \(\tau'\subseteq\tau\), i pel \myref{thm:doble inclusió} ha de ser \(\tau=\tau'\), com volíem veure.
		\end{proof}
	\end{proposition}
	\subsection{Entorns, interior i adherència}
	\begin{definition}[Entorn]
		\labelname{entorn}\label{def:entorn}
		\labelname{punt interior}\label{def:punt interior}
		Siguin \(X\) amb la topologia \(\tau\) un espai topològic, \(x\) un punt i \(N\) un conjunt tal que existeix un obert \(\obert{U}\) satisfent que \(x\) és un element de \(\obert{U}\) i \(\obert{U}\) és un subconjunt de \(N\). Aleshores direm que \(N\) és un entorn de \(x\).
		
		També direm que \(x\) és un punt interior de \(N\).
	\end{definition}
	\begin{observation}
		\label{obs:tot punt té un entorn}
		Siguin \(X\) amb la topologia \(\tau\) un espai topològic i \(x\) un punt. Aleshores existeix un subconjunt \(N\) de \(X\) tal que \(N\) és un entorn de \(x\).
		\begin{proof}
			Si \(N=X\) tenim \(x\in X\subseteq N\). Per la definició de \myref{def:topologia} tenim que \(X\) és un obert i per la definició d'\myref{def:entorn} trobem que \(N\) és un entorn de \(x\).
		\end{proof}
	\end{observation}
	\begin{definition}[Interior]
		\labelname{interior}\label{def:interior}
		Siguin \(X\) amb la topologia \(\tau\) un espai topològic, \(x\) un punt de \(X\), \(A\) subconjunt de \(X\) i
		\[\interior(A)=\{x\in X\mid A\text{ és un entorn de }x\}.\]
		Aleshores direm que \(\interior(A)\) és l'interior del conjunt \(A\).
	\end{definition}
	\begin{proposition}
		\label{prop:l'interior d'un conjunt és un obert}
		Siguin \(X\) amb la topologia \(\tau\) un espai topològic i \(A\) un subconjunt de \(X\). Aleshores \(\interior(A)\) és un obert.
		\begin{proof}
			Prenem un punt \(x\) de \(\interior(A)\). Per la definició d'\myref{def:interior} tenim que existeix un obert \(\obert{U}_{x}\) tal que \(x\in\obert{U}_{x}\) i \(\obert{U}_{x}\subseteq A\). Aleshores tenim
			\begin{align*}
				\interior(A)&=\bigcup_{x\in\interior(A)}\{x\} \\
				&\subseteq\bigcup_{x\in\interior(A)}\obert{U}_{x} \\
				&\subseteq\bigcup_{x\in\interior(A)}\interior(A)=\interior(A).
			\end{align*}
			
			Per tant tenim que
			\[\interior(A)\subseteq\bigcup_{x\in\interior(A)}\obert{U}_{x}\quad\text{i}\quad\bigcup_{x\in\interior(A)}\obert{U}_{x}\subseteq\interior(A),\]
			i pel \myref{thm:doble inclusió} trobem
			\[\interior(A)=\bigcup_{x\in\interior(A)}\obert{U}_{x}\]
			i per la definició de \myref{def:topologia} trobem que \(\interior(A)\) és un obert.
		\end{proof}
	\end{proposition}
	\begin{proposition}
		\label{prop:l'interior d'un conjunt és la unió de tots els oberts continguts en el conjunt}
		Siguin \(X\) amb la topologia \(\tau\) un espai topològic, \(A\) un subconjunt de \(X\) i
		\[Y=\{\obert{U}\in\tau\mid\obert{U}\subseteq A\}.\]
		Aleshores
		\[\interior(A)=\bigcup_{\obert{U}\in Y}\obert{U}.\]
		\begin{proof}
			Sigui \(x\) un element de \(A\) tal que existeixi un obert \(\obert{U}_{x}\) satisfent que \(x\) és un element de \(\obert{U}_{x}\) i \(\obert{U}_{x}\) és un subconjunt de \(A\). Aleshores tenim
			\begin{align*}
				\interior(A)&=\bigcup_{x\in\interior(A)}\{x\} \\
				&\subseteq\bigcup_{x\in\interior(A)}\obert{U}_{x} \\
				&\subseteq\bigcup_{x\in\interior(A)}\interior(A)=\interior(A)
			\end{align*}
			i pel \myref{thm:doble inclusió} trobem
			\[\interior(A)=\bigcup_{\obert{U}\in Y}\obert{U}\]
			com volíem veure.
		\end{proof}
	\end{proposition}
	\begin{corollary}
		\label{cor:l'interior d'un conjunt conté tots els seus oberts}
		Siguin \(X\) amb la topologia \(\tau\) un espai topològic, \(A\) un subconjunt de \(X\) i \(\obert{U}\) un obert tal que \(\obert{U}\) sigui un subconjunt de \(A\). Aleshores \(\obert{U}\) és un subconjunt de \(\interior(A)\).
	\end{corollary}
	\begin{corollary}
		\label{cor:un conjunt és obert si i només si és el seu interior}
		Siguin \(X\) amb la topologia \(\tau\) un espai topològic i \(A\) un subconjunt de \(X\). Aleshores \(A\) és un obert si i només si \(A=\interior(A)\).
	\end{corollary}
	\begin{definition}[Punt adherent]
		\labelname{punt adherent}\label{def:punt adherent}
		Siguin \(X\) amb la topologia \(\tau\) un espai topològic, \(A\) un subconjunt de \(X\) i \(x\) un punt tal que per a tot entorn \(N\) de \(x\) tenim
		\[A\cap N\neq\emptyset.\]
		Aleshores direm que \(x\) és un punt adherent a \(A\).
	\end{definition}
	\begin{definition}[Clausura]
		\labelname{clausura}\label{def:clausura}
		Siguin \(X\) amb la topologia \(\tau\) un espai topològic i \(A\) un conjunt. Aleshores direm que
		\[\clausura(A)=\{x\in X\mid x\text{ és un punt adherent a }A\}\]
		és la clausura de \(A\).
	\end{definition}
	\begin{observation}
		\label{obs:la clausura d'un conjunt conté el conjunt}
		Siguin \(X\) amb la topologia \(\tau\) un espai topològic i \(A\) un subconjunt de \(X\). Aleshores
		\[A\subseteq\clausura(A).\]
	\end{observation}
	\begin{proposition}
		\label{prop:la clausura d'un conjunt és un tancat}
		Siguin \(X\) amb \(\tau\) un espai topològic i \(A\) un subconjunt de \(X\). Aleshores \(\clausura(A)\) és un tancat.
		\begin{proof}
			 Prenem un element \(x\) de \(X\setminus\clausura(A)\). Per la definició de \myref{def:clausura} tenim que existeix un entorn \(N\) de \(x\) tal que
			 \begin{equation}
			 	\label{prop:la clausura d'un conjunt és un tancat:eq1}
				 A\cap N=\emptyset,
			 \end{equation}
			 i per la definició d'\myref{def:entorn} tenim que existeix un obert \(\obert{U}_{x}\) tal que \(x\) és un element de \(\obert{U}_{x}\) i que \(\obert{U}_{x}\) és un subconjunt de \(N\). Per tant, per \eqref{prop:la clausura d'un conjunt és un tancat:eq1} tenim que \(A\cap\obert{U}_{x}=\emptyset\).
			 
			 Per tant trobem
			 \begin{align*}
				 X\setminus\clausura(A)&=\bigcup_{x\in X\setminus\clausura(A)}\{x\} \\
				 &\subseteq\bigcup_{x\in X\setminus\clausura(A)}\obert{U}_{x}\subseteq X\setminus\clausura(A),
			 \end{align*}
			 i pel \myref{thm:doble inclusió} trobem que
			 \[X\setminus\clausura(A)=\subseteq\bigcup_{x\in X\setminus\clausura(A)}\obert{U}_{x},\]
			 i com que \(\obert{U}_{x}\) és un obert, per la definició de \myref{def:topologia} tenim que \(X\setminus\clausura(A)\) és un obert, i per la definició de \myref{def:tancat} trobem que \(\clausura(A)\) és un tancat.
		\end{proof}
	\end{proposition}
	\begin{proposition}
		\label{prop:la clausura d'un conjunt és l'intersecció de tots els tancats que contenen el conjunt}
		Siguin \(X\) amb la topologia \(\tau\) un espai topològic, \(A\) un subconjunt de \(X\) i
		\[Y=\{\tancat{C}\subseteq X\mid A\subseteq\tancat{C}\text{ i }\tancat{C}\text{ és un tancat}\}.\]
		Aleshores
		\[\clausura(A)=\bigcap_{\tancat{C}\in Y}\tancat{C}.\]
		\begin{proof}
			Sigui \(x\) un punt adherent a \(A\) amb \(x\notin A\). Suposem que existeix un tancat \(\tancat{C}\) tal que \(A\) és un subconjunt de \(\tancat{C}\) i tal que \(x\) no pertany a \(\tancat{C}\). Aleshores \(x\) és un element de \(X\setminus\tancat{C}\), i per la definició de \myref{def:tancat} tenim que \(\obert{U}=X\setminus\tancat{C}\) és un obert. Ara bé, per la definició d'\myref{def:entorn} tenim que \(\obert{U}\) és un entorn de \(x\), però \(A\) és un subconjunt de \(\tancat{C}\), i per tant tenim que \(X\setminus\tancat{C}\) és un subconjunt de \(X\setminus A\), i tenim que \(\obert{U}\) és un subconjunt de \(X\setminus A\), i per tant \(A\cap\obert{U}=\emptyset\). Ara bé, per la definició de \myref{def:punt adherent} tenim que \(A\cap\obert{U}\neq\emptyset\). Per tant ha de ser que \(x\) pertany a \(\tancat{C}\) i trobem que
			\begin{equation}
				\label{prop:la clausura d'un conjunt és l'intersecció de tots els tancats que contenen el conjunt:eq1}
				\clausura(A)\subseteq\bigcap_{\tancat{C}\in Y}\tancat{C}.
			\end{equation}
			
			Per la proposició \myref{prop:la clausura d'un conjunt és un tancat} tenim que \(\clausura(A)\) és un tancat, i per tant \(\clausura(A)\) és un element de \(Y\) i tenim que
			\begin{equation}
				\label{prop:la clausura d'un conjunt és l'intersecció de tots els tancats que contenen el conjunt:eq2}
				\bigcap_{\tancat{C}\in Y}\tancat{C}\subseteq\clausura(A).
			\end{equation}
			
			Ara bé, tenim \eqref{prop:la clausura d'un conjunt és l'intersecció de tots els tancats que contenen el conjunt:eq1} i \eqref{prop:la clausura d'un conjunt és l'intersecció de tots els tancats que contenen el conjunt:eq2}, i pel \myref{thm:doble inclusió} trobem que
			\[\clausura(A)=\bigcap_{\tancat{C}\in Y}\tancat{C}.\qedhere\]
		\end{proof}
	\end{proposition}
	\begin{corollary}
		\label{cor:la clausura d'un conjunt és el tancat més petit que el conté}
		Siguin \(X\) amb la topologia \(\tau\) un espai topològic, \(A\) un subconjunt de \(X\) i \(\tancat{C}\) un tancat tal que \(A\) és un subconjunt de \(\tancat{C}\). Aleshores \(\clausura(A)\subseteq\tancat{C}\).
	\end{corollary}
	\begin{corollary}
		\label{cor:un conjunt és tancat si i només si és igual a la seva clausura}
		Siguin \(X\) amb la topologia \(\tau\) un espai topològic i \(A\) un subconjunt de \(X\). Aleshores \(A\) és tancat si i només si \(A=\clausura(A)\).
	\end{corollary}
	\begin{proposition}
		\label{prop:la clausura d'un conjunt és el complementari de l'interior del complementari del conjunt}
		Siguin \(X\) amb la topologia \(\tau\) un espai topològic i \(A\) un subconjunt de \(X\). Aleshores
		\[\clausura(X\setminus A)=X\setminus\interior(A).\]
		\begin{proof} % https://math.stackexchange.com/questions/970825/prove-that-the-closure-of-complement-is-the-complement-of-the-interior
			Prenem un element \(x\) de \(\clausura(X\setminus A)\). Per la definició de \myref{def:clausura} tenim que per a tot obert \(\obert{U}\) que conté \(x\) tenim que
			\[\obert{U}\cap(X\setminus A)\neq\emptyset.\]
			Ara bé, per la definició d'\myref{def:interior} que \(\interior(A)\) és un subconjunt de \(A\), i per tant \(X\setminus A\) és un subconjunt de \(X\setminus\interior(A)\), i per tant \(\obert{U}\cap(X\setminus\interior(A))\neq\emptyset\). Per tant per la definició d'\myref{def:interior} trobem que
			\begin{equation}
				\label{prop:la clausura d'un conjunt és el complementari de l'interior del complementari del conjunt:eq1}
				\clausura(X\setminus A)\subseteq X\setminus\interior(A)
			\end{equation}
			
			Prenem ara un element \(x\) de \(X\setminus\interior(A)\). Per la definició d'\myref{def:interior} tenim que no existeix cap obert \(\obert{U}\) tal que \(x\) sigui un element de \(\obert{U}\) i \(\obert{U}\) sigui un subconjunt de \(A\), o equivalentment, que \(\obert{U}\cap(X\setminus A)\neq\emptyset\) per a tot obert \(\obert{U}\) tal que \(x\) sigui un element de  \(\obert{U}\) i que \(\obert{U}\) sigui un subconjunt de \(A\). Ara bé, per la definició de \myref{def:clausura} tenim que \(\obert{U}\) és un element de \(\clausura(X\setminus A)\), i per tant trobem que
			\begin{equation}
				\label{prop:la clausura d'un conjunt és el complementari de l'interior del complementari del conjunt:eq2}
				X\setminus\interior(A)\subseteq\clausura(X\setminus A).
			\end{equation}
			
			Per tant, amb \eqref{prop:la clausura d'un conjunt és el complementari de l'interior del complementari del conjunt:eq1} i \eqref{prop:la clausura d'un conjunt és el complementari de l'interior del complementari del conjunt:eq2} i el \myref{thm:doble inclusió} trobem que \(\clausura(X\setminus A)=X\setminus\interior(A)\), com volíem veure.
		\end{proof}
	\end{proposition}
	\section{Aplicacions contínues}
	\subsection{Aplicacions obertes, tancades i contínues}
	\begin{definition}[Aplicació oberta]
		\labelname{aplicació oberta}\label{def:aplicació oberta}
		Siguin \(X\) amb la topologia \(\tau_{X}\) i \(Y\) amb la topologia \(\tau_{Y}\) dos espais topològics i \(f\colon X\longrightarrow Y\) una aplicació tal que per a tot obert \(\obert{U}\) de \(X\) tenim que el conjunt \(\{f(x)\in Y\mid x\in\obert{U}\}\) és un obert de \(Y\). Aleshores direm que \(f\) és una aplicació oberta.
	\end{definition}
	\begin{observation}
		\label{obs:la composició d'aplicacions obertes és oberta}
		La composició d'aplicacions obertes és oberta.
	\end{observation}
	\begin{definition}[Aplicació tancada]
		\labelname{aplicació tancada}\label{def:aplicació tancada}
		Siguin \(X\) amb la topologia \(\tau_{X}\) i \(Y\) amb la topologia \(\tau_{Y}\) dos espais topològics i \(f\colon X\longrightarrow Y\) una aplicació tal que per a tot tancat \(\tancat{C}\) de \(X\) tenim que el conjunt \(\{f(x)\in Y\mid x\in\tancat{C}\}\) és un tancat de \(Y\). Aleshores direm que \(f\) és una aplicació tancada.
	\end{definition}
	\begin{observation}
		\label{obs:la composició d'aplicacions tancades és tancada}
		La composició d'aplicacions tancades és tancada.
	\end{observation}
	\begin{definition}[Aplicació contínua]
		\labelname{aplicació contínua}\label{def:aplicació contínua}
		Siguin \(X\) amb la topologia \(\tau_{X}\) i \(Y\) amb la topologia \(\tau_{Y}\) dos espais topològics i \(f\colon X\longrightarrow Y\) una aplicació tal que per a tot obert \(\obert{V}\) de \(Y\) el conjunt \(\{x\in X\mid f(x)\in\obert{V}\}\) és un obert de \(X\). Aleshores direm que \(f\) és una aplicació contínua.
	\end{definition}
	\begin{observation}
		\label{obs:la composició d'aplicacions contínues és contínua}
		La composició d'aplicacions contínues és contínua.
	\end{observation}
	\begin{example}
		\label{ex:les esferes són tancats}
		Considerem
		\[\esfera^{n}=\{x\in\mathbb{R}^{n+1}\mid\norm{x}=1\}.\]
		Volem veure que \(\esfera^{n}\) és un tancat de \(\mathbb{R}^{n+1}\).
		\begin{solution}
			Definim l'aplicació
			\begin{align*}
				f:\mathbb{R}^{n+1}&\longrightarrow\mathbb{R} \\
				x&\longmapsto\norm{x}.
			\end{align*}
			\begin{comment}
				Veiem que \(f\) és una aplicació tancada. Sigui \(\tancat{C}\) un tancat de \(\mathbb{R}^{n+1}\) i considerem el conjunt
				\[\tancat{K}=\{f(x)\in\mathbb{R}\mid x\in\tancat{C}\}.\]
				Tenim que \(\tancat{K}\) és un tancat de \(\mathbb{R}\), ja que si \(y\) és un element de \(\tancat{K}\), aleshores \(y\neq f(x)\) per a tot \(x\) de \(\obert{C}\), i per tant existeix un real \(d>0\) tal que
				\[d=\min_{x\in\tancat{C}}\abs{f(x)-y},\]
				i per tant tenim que per a tot \(y\) de \(\mathbb{R}\setminus\tancat{K}\) existeix una bola \(\bola(y,d)\) tal que \(\bola(y,d)\cap\tancat{K}=\emptyset\), i per la definició d'\myref{def:obert espai mètric} trobem que \(\mathbb{R}\setminus\tancat{K}\) és un obert de \(\mathbb{R}\). Aleshores per la definició de \myref{def:tancat} trobem que \(\tancat{K}\) és un tancat, i per la definició d'\myref{def:aplicació tancada} trobem que \(f\) és una aplicació tancada.
				
				Ara bé, tenim que \(\{1\}\) és un tancat de \(\mathbb{R}\)
			\end{comment}
			Veiem que \(f\) és una aplicació contínua. Sigui \(\obert{V}\) un obert de \(\mathbb{R}\) i considerem el conjunt
			\begin{equation}
				\label{ex:les esferes són tancats:eq1}
				\obert{U}=\{x\in\mathbb{R}^{n+1}\mid f(x)\in\obert{V}\}.
			\end{equation}
			Prenem un element \(x\) de \(\mathbb{R}^{n+1}\). Com que, per hipòtesi, tenim que \(\obert{V}\) és un obert de \(\mathbb{R}\), per la definició d'\myref{def:obert espai mètric} tenim que existeix un real \(\delta>0\) tal que la bola \(\bola(f(x),\delta)\) és un subconjunt de \(\mathbb{R}\), i per la definició \eqref{ex:les esferes són tancats:eq1} tenim que existeix un element \(y\) de \(\mathbb{R}^{n+1}\) tal que \(f(y)\) és un element de \(\bola(f(x),\delta)\), i per la definició de \myref{def:bola} trobem que \(\abs{f(x)-f(y)}<\delta\).
		\end{solution}
	\end{example}
	\subsection{Homeomorfismes entre topologies}
	\begin{definition}[Homeomorfisme]
		\labelname{homeomorfisme entre topologies}\label{def:homeomorfisme entre topologies}
		Siguin \(X\) amb la topologia \(\tau_{X}\) i \(Y\) amb la topologia \(\tau_{Y}\) dos espais topològics i \(f\colon X\longrightarrow Y\) una aplicació bijectiva, oberta i contínua. Aleshores direm que \(f\) és un homeomorfisme.
	\end{definition}
	\begin{definition}[Espais topològics homeomorfs]
		\labelname{espais topològics homeomorfs}\label{def:espais topològics homeomorfs}
		Siguin \(X\) amb la topologia \(\tau_{X}\) i \(Y\) amb la topologia \(\tau_{Y}\) dos espais topològics tals que existeix un homeomorfisme \(f\colon X\longrightarrow Y\). Aleshores direm que \(X\) i \(Y\) són dos espais topològics homeomorfs. També denotarem
		\[X\cong Y.\]
	\end{definition}
	\begin{example}
		\label{ex:R és homeomorf a l'interval (0,1)}
		Volem veure que \(\mathbb{R}\) és homeomorf a l'interval \((0,1)\).
		\begin{solution}
			%TODO
		\end{solution}
	\end{example}
	\begin{proposition}
		\label{prop:ser homeomorf és una relació d'equivalència}
		Siguin \(X\) amb la topologia \(\tau_{X}\) i \(Y\) amb la topologia \(\tau_{Y}\) dos espais topològics. Aleshores la relació
		\[X\cong Y\sii X\text{ és homeomorf a }Y\]
		és una relació d'equivalència.
		\begin{proof}
			Comprovem les propietats de la definició de relació d'equivalència:
			\begin{enumerate}
				\item Reflexiva: Tenim que l'aplicació \(\Id_{X}\) és bijectiva, oberta i contínua. Per la definició d'\myref{def:homeomorfisme entre topologies} tenim que \(\Id_{x}\) és un homeomorfisme, i per la definició d'\myref{def:espais topològics homeomorfs} tenim que \(X\cong X\).
				\item Simètrica: Siguin \(X\) amb la topologia \(\tau_{X}\) i \(Y\) amb la topologia \(\tau_{Y}\) dos espais topològics homeomorfs. Per la definició d'\myref{def:espais topològics homeomorfs} tenim que existeix un homeomorfisme \(f\colon X\longrightarrow Y\).
				
				Pel Teorema \myref{thm:bijectiva iff invertible} trobem que existeix l'aplicació inversa \(f^{-1}\) de \(Y\) a \(X\), i pel \corollari{} \myref{cor:la inversa d'una aplicació invertible és invertible} tenim que l'aplicació \(f^{-1}\) és bijectiva.
				
				També trobem que \(f^{-1}\) és oberta, ja que per la definició d'\myref{def:homeomorfisme entre topologies} tenim que \(f\) és contínua, i per la definició d'\myref{def:aplicació oberta} i la definició d'\myref{def:aplicació contínua} trobem que \(f^{1}\) és oberta. De mateixa manera trobem que \(f^{-1}\) és contínua, ja que per la definició d'\myref{def:homeomorfisme entre topologies} tenim que \(f\) és oberta, i per la definició d'\myref{def:aplicació contínua} i la definició d'\myref{def:aplicació oberta} trobem que \(f^{-1}\) és contínua.
				
				Per tant, per la definició d'\myref{def:homeomorfisme entre topologies} trobem que \(f^{-1}\) és un homeomorfisme entre topologies i per tant \(Y\cong X\).
				\item Transitiva: Siguin \(X_{1}\) amb la topologia \(\tau_{1}\), \(X_{2}\) amb la topologia \(\tau_{2}\) i \(X_{3}\) amb la topologia \(\tau_{3}\) tres espais topològics tals que \(X_{1}\cong X_{2}\) i \(X_{2}\cong X_{3}\). Per la definició d'\myref{def:espais topològics homeomorfs} tenim que existeixen dos homeomorfismes \(f\colon X_{1}\longrightarrow X_{2}\) i \(g\colon X_{2}\longrightarrow X_{3}\).
				
				Considerem l'aplicació \(h=g\circ f\). Pel Teorema \myref{thm:conjugació de bijectives bijectiva} trobem que \(h\) és bijectiva, i per l'observació \myref{obs:la composició d'aplicacions obertes és oberta} i l'observació \myref{obs:la composició d'aplicacions contínues és contínua} trobem que \(h\) és oberta i contínua, i per la definició d'\myref{def:homeomorfisme entre topologies} trobem que \(h\) és un homeomorfisme entre \(X_{1}\) i \(X_{3}\), i per la definició d'\myref{def:espais topològics homeomorfs} trobem que \(X_{1}\cong X_{3}\).
			\end{enumerate}
			I per la definició de \myref{def:relació d'equivalència} hem acabat.
		\end{proof}
	\end{proposition}
\chapter{Altres topologies}
	\section{Topologies induïdes}
	\subsection{La topologia induïda per un subconjunt}
	\begin{proposition}
		\label{prop:topologia induida per un subconjunt}
		Siguin \(X\) amb la topologia \(\tau\) un espai topològic i \(A\) un subconjunt de \(X\). Aleshores \(A\) amb la topologia
		\begin{equation}
			\label{prop:topologia induida per un subconjunt:eq1}
			\tau_{A}=\{\obert{U}\subseteq A\mid\text{Existeix un obert }\obert{W}\text{ de }X\text{ tal que }\obert{U}=\obert{W}\cap A\}
		\end{equation}
		és un espai topològic.
		\begin{proof}
			Observem que \(A\) pertany a \(\tau_{A}\), ja que per la definició de \myref{def:topologia} tenim que \(X\) és un obert de \(X\), i trobem \(X\cap A=A\). També tenim que \(\emptyset\) pertany a \(\tau_{A}\), ja que per la definició de \myref{def:topologia} tenim que \(\emptyset\) és un obert de \(X\), i trobem \(\emptyset\cap A=\emptyset\).
			
			Sigui \(\{\obert{U}_{i}\}_{i\in I}\) una família d'elements de \(\tau_{A}\). Per la definició \eqref{prop:topologia induida per un subconjunt:eq1} tenim que existeix una família \(\{\obert{W}_{i}\}_{i\in I}\) d'oberts de \(X\) tals que \(\obert{U}_{i}=\obert{W}_{i}\cap A\), per a tot \(i\) de \(I\). Considerem ara
			\[\obert{U}=\bigcup_{i\in I}\obert{U}_{i}.\]
			Tenim que
			\[\obert{U}=\bigcup_{i\in I}\obert{W}_{i}\cap A,\]
			i tenim
			\[\obert{U}=A\cap\left(\bigcup_{i\in I}\obert{W}_{i}\right),\]
			i per la definició de \myref{def:topologia} tenim que \(\bigcup_{i\in I}\obert{W}_{i}\) és un obert de \(X\), i per la definició \eqref{prop:topologia induida per un subconjunt:eq1} trobem que \(\obert{U}\) pertany a \(\tau_{A}\).
			
			Sigui \(\{\obert{U}\}_{i=1}^{n}\) una família d'elements de \(\tau_{A}\). Per la definició \eqref{prop:topologia induida per un subconjunt:eq1} tenim que existeix una família \(\{\obert{W}_{i}\}_{i=1}^{n}\) d'oberts de \(X\) tals que \(\obert{U}_{i}=\obert{W}_{i}\cap A\), per a tot \(i\) de \(\{1,\dots,n\}\). Considerem ara
			\[\obert{U}=\bigcap_{i=1}^{n}\obert{U}_{i}.\]
			Tenim que
			\[\obert{U}=\bigcap_{i=1}^{n}\obert{W}_{i}\cap A,\]
			i tenim
			\[\obert{U}=A\cap\left(\bigcap_{i=1}^{n}\obert{W}_{i}\right),\]
			i per la definició de \myref{def:topologia} tenim que \(\bigcap_{i=1}^{n}\obert{W}_{i}\) és un obert de \(X\), i per la definició \eqref{prop:topologia induida per un subconjunt:eq1} trobem que \(\obert{U}\) pertany a \(\tau_{A}\). Per tant per la definició d'\myref{def:espai topològic} trobem que \(A\) amb la topologia \(\tau_{A}\) és un espai topològic.
		\end{proof}
	\end{proposition}
	\begin{definition}[Topologia induïda per un subconjunt]
		\labelname{topologia induïda per un subconjunt}\label{def:topologia induida per un subconjunt}
		Siguin \(X\) amb la topologia \(\tau\) un espai topològic i \(A\) un subconjunt de \(X\). Aleshores denotem
		\[\tau_{A}=\{\obert{U}\subseteq A\mid\text{Existeix un obert }\obert{W}\text{ de }X\text{ tal que }\obert{U}=\obert{W}\cap A\}\]
		i direm que \(\tau_{A}\) és la topologia induïda per \(A\).
		
		Aquesta definició té sentit per la proposició \myref{prop:topologia induida per un subconjunt}.
	\end{definition}
	\begin{definition}[Subespai topològic]
		\labelname{subespai topològic}\label{def:subespai topològic}
		Siguin \(X\) amb la topologia \(\tau\) un espai topològic i \(A\) un subconjunt de \(X\). Aleshores direm que \(A\) amb la topologia \(\tau_{A}\) és un subespai topològic de \(X\).
	\end{definition}
	\begin{proposition}
		\label{prop:C és un tancat si i només si existeix un tancat K tal que l'intersecció de A i K és C}
		Siguin \(X\) amb la topologia \(\tau\) un espai topològic i \(\tau_{A}\) la topologia induïda per un subconjunt \(A\) de \(X\). Aleshores \(\tancat{C}\) és un tancat de \(A\) si i només si existeix un tancat \(\tancat{K}\) de \(\tau\) tal que \(\tancat{C}=A\cap\tancat{K}\).
		\begin{proof}
			Comencem veient que la condició és suficient (\(\implica\)). Sigui \(\tancat{C}\) un tancat de \(A\). Per la definició de \myref{def:tancat} tenim que això és equivalent a que \(A\setminus\tancat{C}\) és un obert. Per la definició de \myref{def:topologia induida per un subconjunt} tenim que això és si i només si existeix un obert \(\obert{U}\) de \(X\) tal que \(A\setminus\tancat{C}=\obert{U}\cap A\). Ara bé, tenim que \(\obert{U}=X\setminus(X\setminus\obert{U})\), i per la definició de \myref{def:tancat} tenim que \(\tancat{K}=X\setminus\obert{U}\) és un tancat, i tenim que
			\begin{align*}
				\tancat{C}&=A\setminus A\setminus\tancat{C} \\
				&=A\setminus(\obert{U}\cap A) \\
				&=A\setminus\obert{U} \\
				&=A\cap(X\setminus\obert{U}) \\
				&=A\cap\tancat{K}
			\end{align*}
			i hem acabat.
			
			Veiem ara que la condició és necessària (\(\implicatper\)). Sigui \(\tancat{K}\) un tancat de \(X\). Prenem \(\tancat{C}=A\cap\tancat{K}\). Aleshores tenim
			\begin{align*}
				A\setminus\tancat{C}&=A\setminus(A\cap\tancat{K}) \\
				&=A\cap(X\setminus\tancat{K}).
			\end{align*}
			Ara bé, per hipòtesi, tenim que \(\tancat{K}\) és un tancat de \(X\), i per tant \(X\setminus\tancat{K}\) és un obert, i per la definició de \myref{def:topologia induida per un subconjunt} trobem que \(A\cap(X\setminus\tancat{K})\) és un obert de \(A\), i per la definició de \myref{def:tancat} tenim que \(A\setminus\tancat{C}\) és un tancat de \(A\), com volíem veure.
		\end{proof}
	\end{proposition}
	\begin{proposition}
		\label{prop:un subconjunt d'un subespai topològic obert és un obert si i només si també és un obert de la topologia}
		Siguin \(X\) amb la topologia \(\tau\) una espai topològic, \(\obert{A}\) un obert de \(X\), \(\tau_{A}\) la topologia induïda per \(A\) i \(\obert{U}\) un subconjunt de \(A\). Aleshores \(\obert{U}\) és un obert de \(A\) si i només si \(\obert{U}\) és un obert de \(X\).
		\begin{proof}
			Comencem veient que la condició és suficient (\(\implica\)). Suposem doncs que \(\obert{U}\) és un obert de \(A\). Per la definició de \myref{def:topologia induida per un subconjunt} tenim que existeix un obert \(\obert{W}\) de \(X\) tal que \(\obert{U}=A\cap\obert{W}\). Ara bé, com que, per hipòtesi, \(A\) i \(\obert{W}\) són dos oberts de \(X\) per la definició de \myref{def:topologia} trobem que \(\obert{U}\) és un obert de \(X\).
			
			Veiem ara que la condició és necessària (\(\implicatper\)). Suposem doncs que \(\obert{U}\) és un obert de \(X\). Com que, per la definició de \myref{def:topologia}, \(\obert{U}\) és un subconjunt de \(A\), tenim que \(\obert{U}=A\cap\obert{U}\), i per la definició de \myref{def:topologia induida per un subconjunt} trobem que \(\obert{U}\) és un obert de \(A\).
		\end{proof}
	\end{proposition}
	\begin{proposition}
		\label{prop:un subconjunt d'un subespai topològic és un tancat si i només si també és un tancat de la topologia}
		Siguin \(X\) amb la topologia \(\tau\) una espai topològic, \(\tancat{A}\) un tancat de \(X\), \(\tau_{A}\) la topologia induïda per \(A\) i \(\tancat{C}\) un subconjunt de \(A\). Aleshores \(\tancat{C}\) és un tancat de \(A\) si i només si \(\tancat{C}\) és un tancat de \(X\).
		\begin{proof}
			Comencem veient que la condició és suficient (\(\implica\)). Suposem doncs que \(\tancat{C}\) és un tancat de \(A\). Per la proposició \myref{prop:C és un tancat si i només si existeix un tancat K tal que l'intersecció de A i K és C} tenim que existeix un tancat \(tancat{K}\) de \(X\) tal que \(\tancat{C}=A\cap\tancat{K}\). Ara bé, com que \(A\) i \(\tancat{K}\) són dos tancats de \(X\), pel Teorema \myref{thm:equivalència obert tancat definició de topologia} trobem que \(\tancat{C}\) és un tancat de \(X\).
			
			Veiem ara que la condició és necessària (\(\implicatper\)). Suposem doncs que \(\tancat{C}\) és un tancat de \(X\). Com que, per la definició de \myref{def:topologia}, \(\tancat{C}\) és un subconjunt de \(A\), tenim que \(\tancat{C}=A\cap\tancat{C}\), i per la definició de \myref{def:topologia induida per un subconjunt} trobem que \(\tancat{C}\) és un tancat de \(A\).
		\end{proof}
	\end{proposition}
	\subsection{La topologia producte}
	\begin{proposition}
		\label{prop:la topologia producte}
		Siguin \(X\) amb la topologia \(\tau_{X}\) i \(Y\) amb la topologia \(\tau_{Y}\) dos espais topològics,
		\[\base{B}=\{\obert{V}_{X}\times\obert{V}_{Y}\subseteq X\times Y\mid\obert{V}_{X}\in\tau_{X}\text{ i }\obert{V}_{Y}\in\tau_{Y}\}\]
		i
		\[\tau=\{\obert{U}\subseteq X\times Y\mid\text{Existeix una família }\{\obert{U}_{i}\}_{i\in I}\subseteq\base{B}\text{ tal que }\obert{U}=\cup_{i\in I}\obert{U}_{i}\}.\]
		Aleshores \(\tau\) és una topologia de \(X\times Y\).
		\begin{proof}
			Per l'\myref{axiom:axioma de regularitat} trobem que \(\emptyset\) és un subconjunt de \(\base{B}\), i per la definició de \(\tau\) trobem que \(\emptyset\) és un element de \(\tau\).
			
			Per la definició de \myref{def:topologia} trobem que \(X\) és un obert de \(X\) i \(Y\) és un obert de \(Y\). Per tant \(X\times Y\) és un element de \(\base{B}\) i per la definició de \(\tau\) trobem que \(X\times Y\) és un element de \(\tau\).
			
			Prenem una família \(\{\obert{U}_{i}\}_{i\in I}\) d'elements de \(\tau\) i considerem
			\[\obert{U}=\bigcup_{i\in I}\obert{U}_{i}.\]
			Per la definició de \(\tau\) trobem que per a cada \(i\) de \(I\) existeix una família \(\{\obert{V}_{j}\}_{j\in J_{i}}\) d'elements de \(\base{B}\) tals que
			\[\obert{U}_{i}=\bigcup_{j\in J_{i}}V_{j}.\]
			Ara bé, per la definició de \(\base{B}\) trobem que per a tot \(j\) de \(J_{i}\) existeixen oberts \(\obert{V}_{j,X}\) de \(X\) i oberts \(\obert{V}_{j,Y}\) de \(Y\) tals que
			\[\obert{V}_{j}=\obert{V}_{j,X}\times\obert{V}_{j,Y},\]
			i per tant trobem que
			\[\obert{U}=\bigcup_{i\in I}\bigcup_{j\in J_{i}}\left(\obert{V}_{j,X}\times\obert{V}_{j,Y}\right),\]
			i per la definició de \(\tau\) trobem que \(\obert{U}\) és un element de \(\tau\).
			
			Siguin \(\obert{U}\) i \(\obert{V}\) dos elements de \(\tau\) i considerem
			\[\obert{W}=\obert{U}\cap\obert{V}.\]
			Per la definició de \(\tau\) tenim que existeixen dues famílies \(\{\obert{U}_{i}\}_{i\in I}\) i \(\{\obert{V}_{j}\}_{j\in J}\) de \(\base{B}\) tals que
			\[\obert{U}=\bigcup_{i\in I}\obert{U}_{i}\quad\text{i}\quad\obert{V}=\bigcup_{j\in J}\obert{V}_{j},\]
			i per la definició de \(\base{B}\) trobem que per a cada \(i\) de \(I\) existeixen un obert \(\obert{U}_{i,X}\) de \(X\) i un obert \(\obert{U}_{i,Y}\) de \(Y\) tals que
			\[\obert{U}_{i}=\obert{U}_{i,X}\times\obert{U}_{i,Y}\]
			i per a cada \(j\) de \(J\) existeixen un obert \(\obert{V}_{j,X}\) de \(X\) i un obert \(\obert{V}_{j,Y}\) de \(Y\) tals que
			\[\obert{V}_{j}=\obert{V}_{j,X}\times\obert{V}_{j,Y}.\]
			
			Per tant tenim que
			\[\obert{U}=\left(\bigcup_{i\in I}\left(\obert{U}_{i,X}\times\obert{U}_{i,Y}\right)\right)\cap\left(\bigcup_{j\in J}\left(\obert{V}_{j,X}\times\obert{V}_{j,Y}\right)\right)\]
			i trobem que
			\[\obert{U}=\left(\bigcup_{i\in I}\obert{U}_{i,X}\times\bigcup_{i\in I}\obert{U}_{i,Y}\right)\cap\left(\bigcup_{j\in J}\obert{V}_{j,X}\times\bigcup_{j\in J}\obert{V}_{j,Y}\right)\]
			i per tant
			\[\obert{U}=\left(\bigcup_{i\in I}\obert{U}_{i,X}\cap\bigcup_{j\in J}\obert{V}_{j,X}\right)\times\left(\bigcup_{i\in I}\obert{U}_{i,Y}\cap\bigcup_{j\in J}\obert{V}_{j,Y}\right),\]
			i per la definició de \myref{def:topologia} trobem que
			\[\bigcup_{i\in I}\obert{U}_{i,X}\cap\bigcup_{j\in J}\obert{V}_{j,X}\quad\text{i}\quad\bigcup_{i\in I}\obert{U}_{i,Y}\cap\bigcup_{j\in J}\obert{V}_{j,Y}\]
			són oberts, i per la definició de \(\tau\) trobem que \(\obert{U}\) és un element de \(\tau\), i per la definició de \myref{def:topologia} trobem que \(\tau\) és una topologia de \(X\times Y\), com volíem veure.
		\end{proof}
	\end{proposition}
	\begin{definition}[Topologia producte]
		\labelname{topologia producte}\label{def:topologia producte}
		Siguin \(X\) amb la topologia \(\tau_{X}\) i \(Y\) amb la topologia \(\tau_{Y}\) dos espais topològics i
		\[\base{B}=\{\obert{V}_{X}\times\obert{V}_{Y}\subseteq X\times Y\mid\obert{V}_{X}\in\tau_{X}\text{ i }\obert{V}_{Y}\in\tau_{Y}\}.\]
		Aleshores direm que \(X\times Y\) amb la topologia
		\[\tau=\{\obert{U}\subseteq X\times Y\mid\text{Existeix una família }\{\obert{U}_{i}\}_{i\in I}\subseteq\base{B}\text{ tal que }\obert{U}=\cup_{i\in I}\obert{U}_{i}\}\]
		és l'espai topològic producte de \(X\) i \(Y\) i que \(\tau\) és la topologia producte de \(X\) i \(Y\).
		
		Aquesta definició té sentit per la proposició \myref{prop:la topologia producte}
	\end{definition}
	\begin{example}[Projeccions]
		\labelname{}\label{ex:les projeccions en la topologia producte són contínues i obertes}
		Siguin \(X\times Y\) amb la topologia \(\tau\) la topologia producte de \(X\) i \(Y\) i
		\begin{align*}
			\pi_{X}\colon X\times Y&\longrightarrow X \\
			(x,y)&\longmapsto x
		\end{align*}
		un aplicació. Volem veure que \(\pi_{X}\) és contínua i oberta.
		\begin{solution}
			Comencem veient que \(\pi_{X}\) és contínua. Prenem un obert \(\obert{U}\) de \(X\) i considerem
			\[\{(x,y)\in X\times Y\mid \pi_{X}(x,y)\in\obert{U}\}.\]
			Tenim doncs que
			\[\obert{U}\times Y=\{(x,y)\in X\times Y\mid \pi_{X}(x,y)\in\obert{U}\},\]
			i com que, per hipòtesi, \(Y\) és un espai topològic, tenim que \(Y\) és un obert de \(Y\), i de nou per hipòtesi tenim que \(\obert{U}\) és un obert de \(X\). Per tant trobem que \(\obert{U}\times Y\) és un obert de \(X\times Y\), i per la definició d'\myref{def:aplicació contínua} trobem que \(\pi_{X}\) és contínua.
			
			Veiem ara que \(\pi_{X}\) és oberta. Prenem un obert \(A\) de \(X\times Y\) i un punt \(x\) de \(\{\pi_{X}(x,y)\in X\mid(x,y)\in A\}\). Aleshores tenim que existeix un punt \(a\) de \(A\) tal que \(\pi_{X}(a)=x\). Per la definició de \myref{def:topologia producte} tenim que existeixen un obert \(\obert{U}_{x}\) de \(X\) i un obert \(\obert{V}_{x}\) de \(Y\) tals que
			\[a\in\obert{U}_{x}\times\obert{V}_{x}\subseteq A.\]
			Observem que
			\[\obert{U}=\{\pi_{X}(x,y)\in X\mid(x,y)\in\obert{U}_{x}\times\obert{V}_{x}\}\]
			és un obert de \(X\), i per tant
			\[x\in\obert{U}_{x}\subseteq\{\pi_{X}(x,y)\in X\mid(x,y)\in A\},\]
			i, denotant \(B=\{\pi_{X}(x,y)\in X\mid(x,y)\in A\}\), trobem que
			\begin{align*}
				B&\subseteq\bigcup_{x\in B}\{x\} \\
				&\subseteq\bigcup_{x\in B}\obert{U}_{x}\subseteq B,
			\end{align*}
			i per tant trobem que \(\{\pi_{X}(x,y)\in X\mid(x,y)\in A\}\) és un obert de \(X\), i per la definició d'\myref{def:aplicació oberta} trobem que \(\pi_{X}\) és una aplicació oberta. % REVISAR? Estic medio empanado
		\end{solution}
	\end{example}
	\begin{corollary} %REF
		\label{cor:un espai topològic producte amb un element és homeomorf a l'espai topològic}
		Sigui \(X\) un espai topològic i \(y\) un element. Aleshores
		\[X\times\{y\}\cong X.\]
	\end{corollary}
	\begin{theorem}
		\label{thm:una aplicació és contínua si i només si les seves components són contínues}
		Siguin \(Z\) amb la topologia \(\tau_{Z}\) un espai topològic, \(X\times Y\) amb la topologia producte de \(X\) i \(Y\), \(f\colon Z\longrightarrow X\times Y\) una aplicació. Aleshores \(f\) és contínua si i només si les aplicacions \(\pi_{X}\circ f\) i \(\pi_{Y}\circ f\) són contínues.
		\begin{proof}
			Comencem veient que la condició és suficient (\(\implica\)). Suposem doncs que \(f\) és contínua. Aleshores per l'exemple \myref{ex:les projeccions en la topologia producte són contínues i obertes} trobem que les aplicacions \(\pi_{X}\) i \(\pi_{Y}\) són contínues i per l'observació \myref{obs:la composició d'aplicacions contínues és contínua} hem acabat.
			
			Veiem ara que la condició és necessària (\(\implicatper\)). Suposem doncs que les aplicacions \(\pi_{X}\circ f\) i \(\pi_{Y}\circ f\) són contínues.
			
			Sigui \(\obert{V}\times\obert{W}\) un obert de \(X\times Y\). Per la definició de \myref{def:topologia producte} tenim que \(\obert{V}\) és un obert de \(X\) i \(\obert{W}\) és un obert de \(Y\). Ara bé, com que per hipòtesi les aplicacions \(\pi_{X}\circ f\) i \(\pi_{Y}\circ f\) són contínues, tenim per la definició d'\myref{def:aplicació contínua} que els conjunts
			\[\obert{U}_{X}=\{x\in Z\mid \pi_{X}\circ f(x)\in\obert{V}\}\quad\text{i}\quad
			\obert{U}_{Y}=\{x\in Z\mid \pi_{Y}\circ f(x)\in\obert{W}\}\]
			són oberts de \(Z\), i per la definició de \myref{def:topologia} trobem que el conjunt
			\[\obert{U}=\obert{U}_{X}\cup\obert{U}_{Y}\]
			és un obert de \(Z\). Per tant tenim que
			\[\obert{U}=\{x\in Z\mid f(x)\in\obert{V}\times\obert{W}\},\]
			és un obert de \(Z\), i com que per hipòtesi el conjunt \(\obert{V}\times\obert{W}\) és un obert de \(X\times Y\), per la definició d'\myref{def:aplicació oberta} trobem que \(f\) és una aplicació oberta. %REVISAR
		\end{proof}
	\end{theorem}
	\begin{proposition}
		\label{prop:si dues parelles d'espais topològics són homeomorfs els seus productes cartesians també ho són}
		Siguin \(X_{1}\) amb la topologia \(\tau_{X_{1}}\),  \(X_{2}\) amb la topologia \(\tau_{X_{2}}\), \(Y_{1}\) amb la topologia \(\tau_{Y_{1}}\) i \(Y_{2}\) amb la topologia \(\tau_{Y_{2}}\) quatre espais topològics tals que
		\[X_{1}\cong X_{2}\quad\text{i}\quad Y_{1}\cong Y_{2}.\]
		Aleshores
		\[X_{1}\times Y_{1}\cong X_{2}\times Y_{2}.\]
		\begin{proof}
			Per la definició d'\myref{def:espais topològics homeomorfs} trobem que existeixen dos homeomorfismes \(f\colon X_{1}\longrightarrow X_{2}\) i \(g\colon Y_{1}\longrightarrow Y_{2}\). Considerem doncs l'aplicació
			\begin{align*}
				h\colon X_{1}\times Y_{1}&\longrightarrow X_{2}\times Y_{2} \\
				(x,y)&\longmapsto(f(x),g(y)).
			\end{align*}
			Com que, per hipòtesi, les aplicacions \(f\) i \(g\) són homeomorfismes, tenim per la definició d'\myref{def:homeomorfisme entre topologies} que \(f\) i \(g\) són bijectives. Ara bé, trobem que \(h\) també és bijectiva, ja que té per inversa la funció
			\begin{align*}
				h^{-1}\colon X_{2}\times Y_{2}&\longrightarrow X_{1}\times Y_{1} \\
				(x,y)&\longmapsto(f^{-1}(x),g^{-1}(y)).
			\end{align*}
			
			Continuem veient que \(h\) és una aplicació oberta. Per la definició d'\myref{def:aplicació oberta} i la definició d'\myref{def:aplicació contínua} en tenim prou amb veure que \(h^{-1}\) és una aplicació contínua. Observem que per a tot \((x,y)\) de \(X_{2}\times Y_{2}\) tenim
			\[\pi_{X_{1}}\circ h^{-1}(x,y)=f^{-1}(x)\quad\text{i}\quad\pi_{Y_{1}}\circ h^{-1}(x,y)=g^{-1}(y),\]
			i com que, per hipòtesi, les aplicacions \(f^{-1}\) i \(g^{-1}\) són contínues tenim pel Teorema \myref{thm:una aplicació és contínua si i només si les seves components són contínues} que l'aplicació \(h^{-1}\) és contínua, i per tant \(h\) és una aplicació oberta.
			
			Per acabar veiem ara que \(h\) és una aplicació contínua. Observem que per a tot \((x,y)\) de \(X_{1}\times Y_{1}\) tenim
			\[\pi_{X_{2}}\circ h(x,y)=f(x)\quad\text{i}\quad\pi_{Y_{2}}\circ h(x,y)=g(y),\]
			i com que, per hipòtesi, les aplicacions \(f\) i \(g\) són aplicacions contínues tenim pel Teorema \myref{thm:una aplicació és contínua si i només si les seves components són contínues} que l'aplicació \(h\) és contínua.
			
			Per tant per la definició d'\myref{def:homeomorfisme entre topologies} trobem que \(h\) és un homeomorfisme, i per la definició d'\myref{def:espais topològics homeomorfs} trobem que \(X_{1}\times Y_{1}\cong X_{2}\times Y_{2}\), com volíem veure.
		\end{proof}
	\end{proposition}
	\section{La topologia quocient}
	\subsection{Topologia quocient per una aplicació}
	\begin{proposition}
		\label{prop:topologia quocient}
		Siguin \(X\) amb la topologia \(\tau_{X}\) un espai topològic, \(Y\) un conjunt i \(p\colon X\longrightarrow Y\) una aplicació exhaustiva. Aleshores
		\begin{equation}
			\label{prop:topologia quocient:eq1}
			\tau_{Y}=\{\obert{U}\subseteq Y\mid\Antiima_{\obert{U}}(p)\in\tau_{X}\}
		\end{equation}
		és una topologia de \(Y\).
		\begin{proof} %REFS
			Comprovem que \(\tau_{Y}\) satisfà la definició de \myref{def:topologia}. Tenim que \(\emptyset\) pertany a \(\tau_{Y}\). Per la definició d'\myref{def:antiimatge d'una aplicació} trobem que \(\Ima_{\emptyset}(p)=\emptyset\), i com que per la definició de \myref{def:topologia} tenim que \(\emptyset\) pertany a \(\tau_{X}\) trobem que \(\emptyset\) és un element de \(\tau_{Y}\).
			
			Veiem ara que \(Y\) pertany a \(\tau_{Y}\). Per la definició d'\myref{def:antiimatge d'una aplicació} i com que, per hipòtesi, \(p\) és exhaustiva, per la definició d'\myref{def:aplicació exhaustiva} tenim que \(\Ima_{Y}^{-1}(p)=X\), i per la definició de \myref{def:topologia} tenim que \(X\) és un obert, i per tant \(Y\) pertany a \(\tau_{Y}\).
			
			Sigui \(\{\obert{U}_{i}\}_{i\in I}\) una família d'elements de \(\tau_{Y}\). Aleshores tenim que
			\[\Antiima_{\bigcup_{i\in I}\obert{U}_{i}}(p)=\bigcup_{i\in I}\Antiima_{\obert{U}_{i}}(p),\]
			i com que. per \eqref{prop:topologia quocient:eq1} trobem que \(\Antiima_{\obert{U}_{i}}(p)\) és un obert de \(X\) per a tot \(i\) de \(I\), i per la definició de topologia trobem que
			\[\bigcup_{i\in I}\Antiima_{\obert{U}_{i}}(p)\]
			és un obert de \(X\), i per tant tenim que \(\bigcup_{i\in I}\obert{U}_{i}\) és un element de \(\tau_{Y}\).
			
			Prenem ara una família \(\{\obert{U}_{i}\}_{i=1}^{n}\) d'elements de \(\tau_{Y}\). Tenim que
			\[\Antiima_{\bigcup_{i=1}^{n}\obert{U}_{i}}(p)=\bigcup_{i=1}^{n}\Antiima_{\obert{U}_{i}}(p),\]
			i com que. per \eqref{prop:topologia quocient:eq1} trobem que \(\Antiima_{\obert{U}_{i}}(p)\) és un obert de \(X\) per a tot \(i\) de \(\{1,\dots,n\}\), i per la definició de topologia trobem que
			\[\bigcup_{i=1}^{n}\Antiima_{\obert{U}_{i}}(p)\]
			és un obert de \(X\), i per tant tenim que \(\bigcup_{i=1}^{n}\obert{U}_{i}\) és un element de \(\tau_{Y}\).
		\end{proof}
	\end{proposition}
	\begin{definition}[Topologia quocient]
		\labelname{topologia quocient}\label{def:topologia quocient}
		Siguin \(X\) amb la topologia \(\tau_{X}\) un espai topològic, \(Y\) un conjunt i \(p\colon X\longrightarrow Y\) una aplicació exhaustiva. Aleshores direm que \(Y\) amb la topologia
		\[\tau_{Y}=\{\obert{U}\subseteq Y\mid\Antiima_{\obert{U}}(p)\in\tau_{X}\}\]
		és un espai topològic quocient, o que \(Y\) té la topologia quocient per \(p\).
		
		Aquesta definició té sentit per la proposició \myref{prop:topologia quocient}.
	\end{definition}
	\begin{observation}
		\label{obs:l'aplicació que indueix la topologia en un espai quocient és contínua}
		Sigui \(Y\) un espai topològic amb la topologia quocient per una aplicació \(p\). Aleshores \(p\) és contínua.
	\end{observation} %REF prove?
	\begin{proposition}
		\label{prop:la topologia quocient és equivalent per tancats}
		Sigui \(Y\) un espai topològic amb la topologia quocient per una aplicació \(p\). Aleshores un subconjunt \(\tancat{C}\) de \(Y\) és tancat si i només si \(\Antiima_{\tancat{C}}(p)\) és un tancat.
		\begin{proof} %REFS
			Denotem per \(X\) l'espai topològic sobre el que \(p\) està definida. Sigui \(\tancat{C}\) un tancat de \(Y\). Aleshores, per la definició de \myref{def:tancat}, tenim que \(Y\setminus\tancat{C}\) és un obert de \(Y\). Ara bé, tenim que \(\Antiima_{Y\setminus\tancat{C}}(p)\) és un obert de \(X\). Ara bé, tenim que
			\[\Antiima_{Y\setminus\tancat{C}}(p)=X\setminus\Antiima_{\tancat{C}}(p),\]
			i per la definició de \myref{def:tancat} trobem que \(\Antiima_{\tancat{C}}(p)\) és un tancat de \(X\), com volíem veure.
		\end{proof}
	\end{proposition}
	\begin{theorem}
		\label{thm:la composició d'una aplicació amb l'aplicació que indueix la topologia en un espai quocient és contínua si i només si aquesta aplicació és contínua}
		Siguin \(Y\) un espai topològic amb la topologia quocient per una aplicació \(p\), \(Z\) un espai topològic i \(f\colon Y\longrightarrow Z\) una aplicació. Aleshores \(f\circ p\) és contínua si i només si \(f\) és contínua.
		\begin{proof}
			Denotem per \(X\) l'espai topològic sobre el que \(p\) està definida.
			
			Comencem veient que la condició és suficient (\(\implica\)). Suposem doncs que \(f\circ p\) és contínua. Sigui \(\obert{U}\) un obert de \(Z\). Per la definició d'\myref{def:aplicació contínua} en tenim prou en veure que \(\Antiima_{\obert{U}}(f)\) és un obert.
			
			Tenim per la definició d'\myref{def:aplicació contínua} que \(\Antiima_{\obert{U}}(p\circ f)\) és un obert de \(X\)., i per tant, per la definició de \myref{def:topologia quocient} tenim que \(\Antiima_{\obert{U}}(f)\) és un obert de \(Y\), com volíem.
			
			Veiem ara que la condició és necessària (\(\implicatper\)). Suposem doncs que \(f\) és contínua. Aleshores per l'observació \myref{obs:la composició d'aplicacions contínues és contínua} hem acabat.
		\end{proof}
	\end{theorem}
	\begin{example}
		Volem calcular la topologia de l'espai quocient \(A=\{a,b,c\}\) induït per l'aplicació \(p\colon\mathbb{R}\longrightarrow A\), definida per
		\begin{equation}
			\label{ex:topologia quocient caracteritzacio signe d'un real}
			p(x)=\begin{cases}
				a & \text{si }x>0 \\
				b & \text{si }x<0 \\
				c & \text{si }x=0.
			\end{cases}
		\end{equation}
		\begin{solution}
			Calculem els oberts de \(A\). Tenim per la definició de \myref{def:topologia quocient} que aquests són les antiimatges dels oberts de \(\mathbb{R}\) per \(p\). Prenem un obert \(\obert{U}\) de \(\mathbb{R}\).
			
			Observem primer que si \(\obert{U}=\emptyset\) aleshores
			\[\Antiima_{\obert{U}}(p)=\emptyset,\]
			i si \(0\) pertany a \(\obert{U}\) aleshores, per la definició d'\myref{def:obert espai mètric} tenim que existeixen un \(x\) de \(\obert{U}\) positiu i un \(y\) de \(\obert{U}\) negatiu, i per tant
			\[\Antiima_{\obert{U}}(p)=\{a,b,c\}.\]
			Suposem doncs que \(0\) no pertany a \(\obert{U}\).
			
			Suposem que per a tot \(x\) de \(\obert{U}\) tenim que \(x>0\). Aleshores, per la definició d'\myref{def:antiimatge d'una aplicació} i \eqref{ex:topologia quocient caracteritzacio signe d'un real} tenim que
			\[\Antiima_{\obert{U}}(p)=\{a\}.\]
			
			Suposem ara que per a tot \(x\) detext{ si }0 \(\obert{U}\) tenim que \(x>0\). Aleshores, de nou per la definició d'\myref{def:antiimatge d'una aplicació} i \eqref{ex:topologia quocient caracteritzacio signe d'un real} tenim que
			\[\Antiima_{\obert{U}}(p)=\{b\}.\]
			
			Suposem ara que existeixen un \(x\) de \(\obert{U}\) positiu i un \(y\) de \(\obert{U}\) negatiu. Aleshores, com que tenim que \(0\) no pertany a \(\obert{U}\), trobem que
			\[\Antiima_{\obert{U}}(p)=\{a,b\}.\]
			
			Per tant trobem que els oberts de \(A\) són
			\[\tau=\{\emptyset,\{a\},\{b\},\{a,b\},\{a,b,c\}\}.\qedhere\]
		\end{solution}
	\end{example}
	\subsection{Topologia quocient per una relació d'equivalència}
	\begin{example}[Projecció a un quocient]
		\labelname{}\label{ex:topologia projecció a un quocient}
		Siguin \(X\) un espai topològic i \(\sim\) una relació d'equivalència. Volem veure que \(X/\sim\) té la topologia quocient per l'aplicació
		\begin{align*}
			\pi\colon X&\longrightarrow X/\sim \\
			x&\longmapsto\overline{x}.
		\end{align*}
		\begin{solution}
			Hem de veure que \(\pi\) és exhaustiva. Prenem un element \(\overline{x}\) de \(X/\sim\). Aleshores tenim que \(\pi(x)=\overline{x}\) i per la definició de \myref{def:topologia quocient} hem acabat.
		\end{solution}
	\end{example}
	\begin{definition}
		\label{def:topologia projecció a un quocient}
		Siguin \(X\) un espai topològic, \(\sim\) una relació d'equivalència sobre \(X\) i \(\pi\) la projecció de \(X\) en \(X/\sim\) donada per
		\begin{align*}
			\pi\colon X&\longrightarrow X/\sim \\
			x&\longmapsto\overline{x}.
		\end{align*}
		Aleshores denotem la topologia \(Y\) induïda per \(\pi\) com \(X/\sim\).
		
		Aquesta definició té sentit per l'exemple \myref{ex:topologia projecció a un quocient}.
	\end{definition}
	\begin{example}
		\label{ex:l'esfera en el pla és homeomorfa a la recta [0,1] amb 0sim1}
		Volem veure que l'espai quocient
		\[Y=[0,1]/\{0\sim1\}\]
		amb la projecció \(\pi\) és homeomorf a
		\[\esfera^{1}=\{x\in\mathbb{R}^{2}\mid\norm{x}=1\}.\]
		\begin{solution}
			%TODO
		\end{solution}
	\end{example}
	\subsection{Topologia quocient per un grup}
	\begin{definition}[Acció d'un grup sobre un espai topològic]
		\labelname{acció d'un grup sobre un espai topològic}\label{def:acció d'un grup sobre un espai topològic}
		Siguin \(G\) amb l'operació \(\ast\) un grup amb element neutre \(e\) i \(X\) un espai topològic tals que per a tot \(g\) de \(G\) existeix una aplicació contínua \(\theta_{g}\colon X\longrightarrow X\) tal que per a tot \(x\) de \(X\) es satisfà \(\theta_{e}(x)=x\) i per a tot \(g\) i \(h\) de \(G\) es satisfà \(\theta_{g}\circ\theta_{h}=\theta_{g\ast h}\). Aleshores direm que \(G\) actua sobre \(X\).
	\end{definition}
	\begin{observation}
		\label{obs:els accions de grup en un espai topològic són homeomorfismes}
		Sigui \(G\) un grup que actua sobre \(X\) i \(g\) un element de \(G\). Aleshores \(\theta_{g}\) és un homeomorfisme.
	\end{observation} %FER Prove?
	\begin{definition}[Domini fonamental]
		\labelname{domini fonamental}\label{def:domini fonamental}
		Siguin \(G\) un grup que actua sobre un espai topològic \(X\) i \(D\) un subespai de \(X\) tal que per a tot \(x\) de \(X\) existeixen un \(g\) de \(G\) i \(d\) de \(D\) tals que
		\[x=\theta_{g}(d).\]
		Aleshores direm que \(D\) és un domini fonamental de \(X\).
	\end{definition}
	\begin{example}
		\label{ex:domini fonamental}
		Considerem les aplicacions
		\[S(x,y)=(x,y+1)\quad\text{i}\quad T(x,y)=(x-1,-y)\]
		sobre l'espai topològic \(\mathbb{R}^{2}\) i el grup \(G=\langle\{S,T\}\rangle\).
		Volem veure que \([0,1)\times[0,1)\) és un domini fonamental de \(X\).
		\begin{solution} %REFS (part entera?) % Veure que el grup actua i tal i qual
			Observem que la inversa de \(T\) és \(T^{-1}=(x+1,y)\), i \(T^{-1}\) per la definició de pertany a \(G\).
			
			Prenem un punt \((x,y)\) de \(\mathbb{R}^{2}\). Tenim que existeixen dos enters \(x_{0}\) i \(y_{0}\) tals que \(x_{0}\leq x<x_{0}+1\) i \(y_{0}\leq y<y_{0}+1\). Per tant
			\[(x_{0},y_{0})=S^{x_{0}}(x-x_{0},0)+T^{-y_{0}}(0,y-y_{0}),\]
			i per la definició de \myref{def:domini fonamental} hem acabat.
		\end{solution} % REVISAR
	\end{example}
	\begin{proposition}
		\label{prop:quocient d'un espai per l'acció d'un grup relació}
		Sigui \(G\) amb l'operació \(\ast\) un grup que actua sobre un espai topològic \(X\). Aleshores la relació \(\sim\) definida per a tot \(x\) i \(y\) de \(X\) com
		\[x\sim y\sii\text{existeix un }g\in G\text{ tal que }\theta_{g}(x)=y\]
		és una relació d'equivalència.
		\begin{proof}
			%TODO
		\end{proof}
	\end{proposition}
	\begin{definition}[Quocient d'un espai per l'acció d'un grup]
		\labelname{quocient d'un espai per l'acció d'un grup}\label{def:quocient d'un espai per l'acció d'un grup}
		Sigui \(G\) amb l'operació \(\ast\) un grup que actua sobre un espai topològic \(X\) i \(\sim\) la relació d'equivalència definida per a tot \(x\) i \(y\) de \(X\) com
		\[x\sim y\sii\text{existeix un }g\in G\text{ tal que }\theta_{g}(x)=y.\]
		Aleshores denotarem per
		\[X/\sim=X/G\]
		la topologia quocient de \(X\) per \(G\).
		
		Aquesta definició té sentit per la proposició \myref{prop:quocient d'un espai per l'acció d'un grup relació}.
	\end{definition}
\chapter{Espais topològics}
	\section{Espais compactes}
	\subsection{Recobriments}
	\begin{definition}[Recobriment]
		\labelname{recobriment d'un espai}\label{def:recobriment d'un espai}
		\labelname{recobriment finit d'un espai}\label{def:recobriment finit d'un espai}
		\labelname{recobriment infinit d'un espai}\label{def:recobriment infinit d'un espai}
		Siguin \(X\) un espai topològic i \(\{\obert{U}_{i}\}_{i\in I}\) una família de subespais de \(X\) tals que
		\[X=\bigcup_{i\in I}\obert{U}_{i}.\]
		Aleshores direm que \(\{\obert{U}_{i}\}_{i\in I}\) és un recobriment de \(X\).
		
		Si \(I\) és finit direm que \(\{\obert{U}_{i}\}_{i\in I}\) és un recobriment finit de \(X\), i si \(I\) és infinit direm que \(\{\obert{U}_{i}\}_{i\in I}\) és un recobriment infinit de \(X\).
	\end{definition}
	\begin{definition}[Recobriment obert]
		\labelname{recobriment obert}\label{def:recobriment obert}
		Sigui \(\{\obert{U}_{i}\}_{i\in I}\) un recobriment d'un espai topològic \(X\) tal que per a tot \(i\) de \(I\) tenim que \(\obert{U}_{i}\) és un obert de \(X\). Aleshores direm que \(\{\obert{U}_{i}\}_{i\in I}\) és un recobriment obert de \(X\).
	\end{definition}
	\begin{example}
		\label{ex:un recobriment obert de R}
		Volem trobar un recobriment de \(\mathbb{R}\).
		\begin{solution}
			Prenem la família d'oberts \(\{(-i,i)\}_{i\in\mathbb{N}}\). Aleshores tenim que
			\[\mathbb{R}=\bigcup_{i\in\mathbb{N}}(-i,i),\]
			ja que si \(x\) és un element de \(\mathbb{R}\) trobem que existeix un natural \(m\) tal que \(\abs{x}<m\), i per tant \(x\) pertany a l'interval \((-m,m)\).
		\end{solution}
	\end{example}
	\begin{definition}[Subrecobriment]
		\labelname{subrecobriment}\label{def:subrecobriment}
		\labelname{subrecobriment finit}\label{def:subrecobriment finit}
		\labelname{subrecobriment infinit}\label{def:subrecobriment infinit}
		Siguin \(\{\obert{U}_{i}\}_{i\in I}\) un recobriment d'un espai topològic \(X\) i \(J\) un subconjunt de \(I\) tal que la família \(\{\obert{U}_{j}\}_{j\in J}\) és un recobriment de \(X\). Aleshores direm que \(\{\obert{U}_{j}\}_{j\in J}\) és un subrecobriment de \(\{\obert{U}_{i}\}_{i\in I}\).
		
		Si \(J\) és finit direm que \(\{\obert{U}_{j}\}_{j\in J}\) és un recobriment finit de \(\{\obert{U}_{i}\}_{i\in I}\), i si \(J\) és infinit direm que \(\{\obert{U}_{j}\}_{j\in J}\) és un recobriment infinit de \(\{\obert{U}_{i}\}_{i\in I}\).
	\end{definition}
	\subsection{Compacitat}
	\begin{definition}[Compacte]
		\labelname{espai topològic compacte}\label{def:espai topològic compacte}
		Sigui \(X\) un espai topològic tal que tot recobriment obert de \(X\) admet un subrecobriment finit. Aleshores direm que \(X\) és compacte.
	\end{definition}
	\begin{example}
		\label{ex:R no és compacte}
		Volem veure que \(\mathbb{R}\) no és compacte.
		\begin{solution}
			Per l'exemple \myref{ex:un recobriment obert de R} tenim que \(\{(-i,i)\}_{i\in\mathbb{N}}\) és un recobriment obert de \(\mathbb{R}\).
			
			Suposem que existeix un subrecobriment finit de \(\{(-i,i)\}_{i\in\mathbb{N}}\). Això és que existeix un subconjunt \(I\) de \(\mathbb{N}\) finit tal que \(\{(-i,i)\}_{i\in I}\) és un recobriment de \(\mathbb{R}\). Com que \(I\) és finit tenim que existeix un natural \(m\) tal que \(m=\max\{i\in I\}\), i tenim que \(m+1\) no pertany a cap interval de la família \(\{(-i,i)\}_{i\in I}\). Per tant trobem que
			\[m+1\notin\bigcup_{i\in I}(-i,i).\]
			
			Ara bé, per la definició de \myref{def:recobriment d'un espai} trobem que ha de ser
			\[m+1\in\mathbb{R}=\bigcup_{i\in I}(-i,i),\]
			i arribem a contradicció, trobant que no existeix cap subrecobriment finit de \(\{(-i,i)\}_{i\in\mathbb{N}}\).
		\end{solution}
	\end{example}
	\begin{proposition}
		\label{prop:la compacitat és una propietat topològica}
		Siguin \(X\) i \(Y\) dos espais topològics homeomorfs. Aleshores \(X\) és compacte si i només si \(Y\) és compacte.
		\begin{proof}
			Per la definició de \myref{def:relació d'equivalència} tenim que si \(X\cong Y\) aleshores \(Y\cong X\), i per tant només ens cal veure que si \(X\) és compacte aleshores \(Y\) és compacte.
			
			Suposem doncs que \(X\) és compacte. Per la definició d'\myref{def:espais topològics homeomorfs} tenim que existeix un homeomorfisme \(f\colon X\longrightarrow Y\).
			
			Sigui \(\{\obert{U}_{i}\}_{i\in I}\) un recobriment obert de \(Y\). Per la definició d'\myref{def:homeomorfisme entre topologies} i la definició de \myref{def:recobriment obert} tenim que la família \(\{\Antiima_{\obert{U}_{i}}(f)\}_{i\in I}\) és un recobriment obert de \(X\).
			
			Ara bé, per la definició d'\myref{def:espai topològic compacte} trobem que existeix un subrecobriment finit \(\{\Antiima_{\obert{U}_{j}}(f)\}_{j\in J}\) de \(\{\Antiima_{\obert{U}_{i}}(f)\}_{i\in I}\), i de nou per la definició d'\myref{def:homeomorfisme entre topologies} i la definició de \myref{def:recobriment obert} tenim que la família \(\{\obert{U}_{j}\}_{j\in J}\) és un subrecobriment obert de \(\{\obert{U}_{i}\}_{i\in I}\), i per la definició d'\myref{def:espai topològic compacte} trobem que \(Y\) és compacte.
		\end{proof}
	\end{proposition}
	\begin{example}
		\label{ex:l'interval (0,1) no és compacte}
		Volem veure que l'interval \((0,1)\) no és compacte.
		\begin{solution}
			Per l'exemple \myref{ex:R és homeomorf a l'interval (0,1)} tenim que l'interval \((0,1)\) és homeomorf a \(\mathbb{R}\), per l'exemple \myref{ex:R no és compacte} trobem que \(\mathbb{R}\) no és compacte i per la proposició \myref{prop:la compacitat és una propietat topològica} tenim que \((0,1)\) no és compacte.
		\end{solution}
	\end{example}
	\begin{proposition}
		\label{prop:un espai topològic finit és compacte}
		Sigui \(X\) un espai topològic finit. Aleshores \(X\) és compacte.
		\begin{proof}
			Sigui \(\{\obert{U}_{i}\}_{i\in I}\) un recobriment obert de \(X\), i per la definició de \myref{def:recobriment obert} tenim que \(\{\obert{U}_{i}\}_{i\in I}\subseteq\tau\).
			
			Ara bé, tenim que \(\tau\subseteq\mathcal{P}(X)\), i com que \(X\) és finit, \(\mathcal{P}(X)\) també ho és, i per tant \(\{\obert{U}_{i}\}_{i\in I}\) és finit i per la definició de \myref{def:recobriment finit d'un espai} trobem que \(\{\obert{U}_{i}\}_{i\in I}\) és un recobriment finit de \(X\), i per la definició d'\myref{def:espai topològic compacte} tenim que \(X\) és compacte.
		\end{proof}
	\end{proposition}
	\begin{proposition}
		\label{prop:un espai topològic discret és compacte si i només si és finit}
		Sigui \(X\) un espai topològic amb la topologia discreta. Aleshores \(X\) és compacte si i només si \(X\) és finit.
		\begin{proof}
			Comencem veient que la condició és suficient (\(\implica\)). Veurem que si \(X\) és infinit aleshores \(X\) no és compacte.
			
			Per l'exemple \myref{ex:topologia discreta} tenim que tot subconjunt de \(X\) és un obert, i com que
			\[X=\bigcup_{x\in X}\{x\}\]
			per la definició de \myref{def:recobriment obert} trobem que \(\{\{x\}\}_{x\in X}\) és un recobriment obert de \(X\).
			
			Ara bé, com que \(X\) és infinit, si \(\{\{x\}\}_{x\in I}\) és un subrecobriment finit de \(\{\{x\}\}_{x\in X}\) tenim que existeix algun \(x_{0}\) de \(X\) tal que \(x_{0}\notin I\), i per tant
			\[x_{0}\notin\bigcup_{x\in I}\{x\}\]
			i trobem que el recobriment \(\{\{x\}\}_{x\in X}\) no admet cap subrecobriment finit, i per la definició d'\myref{def:espai topològic compacte} trobem que \(X\) no és compacte.
			
			Per veure que la condició és necessària (\(\implicatper\)) en tenim prou amb la proposició \myref{prop:un espai topològic finit és compacte}.
		\end{proof}
	\end{proposition}
	\begin{definition}[Compacitat per tancats]
		\labelname{compacitat per tancats}\label{def:compacitat per tancats}
		\labelname{compacte per tancats}\label{def:compacte per tancats}
		Sigui \(X\) un espai topològic tal que per a tota família de tancats \(\{\tancat{C}_{i}\}_{i\in I}\) de \(X\) amb
		\[\bigcap_{i\in I}\tancat{C}_{i}=\emptyset\]
		existeix una subfamília finita \(\{\tancat{C}_{i}\}_{i\in J}\) tal que
		\[\bigcap_{i\in J}\tancat{C}_{i}=\emptyset.\]
		Aleshores direm que \(X\) és compacte per tancats.
	\end{definition}
	\begin{proposition}
		\label{prop:compacte si i només si compacte per tancats}
		\label{prop:equivalència entre compacitat i compacitat per tancats}
		Sigui \(X\) un espai topològic. Aleshores \(X\) és compacte si i només si \(X\) és compacte per tancats.
		\begin{proof}
			Siguin \(X\) un espai topològic compacte i \(\{\tancat{C}_{i}\}_{i\in I}\) una família de tancats de \(X\) tals que
			\[\bigcap_{i\in I}\tancat{C}_{i}=\emptyset.\]
			Per la definició de \myref{def:tancat} tenim que per a tot \(i\) de \(I\) el conjunt \(\obert{U}_{i}=X\setminus\tancat{C}_{i}\) és un obert de \(X\), i tenim que
			\begin{align*}
				X&=X\setminus\emptyset \\
				&=X\setminus\left(\bigcap_{i\in I}\tancat{C}_{i}\right) \\
				&=\bigcup_{i\in I}(X\setminus\tancat{C}_{i})=\bigcup_{i\in I}\obert{U}_{i},
			\end{align*}
			i per la definició de \myref{def:recobriment obert} trobem que \(\{\obert{U}_{i}\}_{i\in I}\) és un recobriment obert de \(X\). Ara bé, per la definició d'\myref{def:espai topològic compacte} tenim que existeix un subrecobriment finit \(\{\obert{U}_{i}\}_{i\in J}\) de \(\{\obert{U}_{i}\}_{i\in I}\), i per la definició de \myref{def:subrecobriment} tenim que
			\begin{align*}
				X&=\bigcup_{i\in J}\obert{U}_{i} \\
				&=\bigcup_{i\in J}(X\setminus\tancat{C}_{i})=X\setminus\bigcap_{i\in J}\tancat{C}_{i},
			\end{align*} %REFS
			i per tant trobem que
			\[\bigcap_{i\in J}\tancat{C}_{i}=\emptyset\]
			i per la definició de \myref{def:compacte per tancats} trobem que \(X\) és compacte per tancats.
		\end{proof}
	\end{proposition}
	\begin{definition}[Compacitat de subconjunts]
		\labelname{compacitat d'un subconjunt}\label{def:compacitat d'un subconjunt}
		\labelname{subconjunt compacte}\label{def:subconjunt compacte}
		\labelname{recobriment obert d'un subconjunt}\label{def:recobriment obert d'un subconjunt}
		Sigui \(A\) un subconjunt d'un espai topològic \(X\) tal que per a tota família d'oberts \(\{\obert{U}_{i}\}_{i\in I}\) de \(X\) tal que
		\[A\subseteq\bigcup_{i\in I}\obert{U}_{i}\]
		existeix una subfamília finita \(\{\obert{U}_{i}\}_{i\in J}\) de \(\{\obert{U}_{i}\}_{i\in I}\) tal que
		\[A\subseteq\bigcup_{i\in J}\obert{U}_{i}.\]
		Aleshores direm que \(A\) és un subconjunt compacte de \(X\). També direm que \(\{\obert{U}_{i}\}_{i\in I}\) és un recobriment obert de \(A\).
	\end{definition}
	\subsection{Propietats dels espais compactes}
	\begin{theorem}
		\label{thm:la imatge d'un compacte per una aplicació contínua és un compacte}
		Siguin \(A\) un subconjunt compacte d'un espai topològic \(X\), \(Y\) un espai topològic i \(f\colon X\longrightarrow Y\) una aplicació contínua. Aleshores \(\Ima_{A}(f)\) és un subconjunt compacte de \(Y\).
		\begin{proof} %REFS infinites
			Sigui \(\{\obert{U}_{i}\}_{i\in I}\) un recobriment obert de \(\Ima_{A}(f)\). Per la definició de \myref{def:recobriment obert d'un subconjunt} trobem que
			\[\Ima_{A}(f)\subseteq\bigcup_{i\in I}\obert{U}_{i},\]
			i per tant tenim que
			\[A\subseteq\bigcup_{i\in I}\Antiima_{\obert{U}_{i}}(f).\]
			Com que, per la definició de \myref{def:recobriment obert d'un subconjunt}, per a tot \(i\) de \(I\) el conjunt \(\obert{U}_{i}\) és un obert de \(Y\), per la definició de \myref{def:funció contínua} tenim que \(\Antiima_{\obert{U}_{i}}(f)\) és un obert de \(X\), i trobem per la definició de \myref{def:recobriment obert d'un subconjunt} que \(\{\Antiima_{\obert{U}_{i}}(f)\}_{i\in I}\) és un recobriment obert de \(A\).
			
			Ara bé, tenim per hipòtesi que \(A\) és un subconjunt compacte de \(X\), i per la definició de \myref{def:subconjunt compacte} tenim que existeix una subfamília finita \(\{\Antiima_{\obert{U}_{i}}(f)\}_{i\in J}\) de \(\{\Antiima_{\obert{U}_{i}}(f)\}_{i\in I}\) tal que
			\[A\subseteq\bigcup_{i\in J}\Antiima_{\obert{U}_{i}}(f).\]
			
			Tenim doncs que
			\[\Ima_{A}(f)\subseteq\bigcup_{i\in J}\obert{U}_{i},\]
			i per la definició de \myref{def:compacitat d'un subconjunt} hem acabat.
		\end{proof}
	\end{theorem}
	\begin{corollary}
		\label{cor:el quocient d'un compacte és un compacte}
		Siguin \(X\) un espai topològic compacte i \(p\) una aplicació exhaustiva. Aleshores \(X/p\) és compacte.
		\begin{proof}
			Per l'observació \myref{obs:l'aplicació que indueix la topologia en un espai quocient és contínua} tenim que \(p\colon X\longrightarrow X/p\) és contínua. Per la definició d'\myref{def:imatge d'una aplicació} i la definició d'\myref{def:aplicació exhaustiva} tenim que \(X/p=\Ima(p)\), i pel Teorema \myref{thm:la imatge d'un compacte per una aplicació contínua és un compacte} trobem que \(X/p\) és compacte.
		\end{proof}
	\end{corollary}
	\begin{theorem}
		\label{thm:un tancat en un compacte és compacte}
		Sigui \(\tancat{C}\) un tancat d'un espai topològic compacte \(X\). Aleshores \(\tancat{C}\) és compacte.
		\begin{proof}
			Sigui \(\{\obert{U}_{i}\}_{i\in I}\) un recobriment obert de \(\tancat{C}\). Per la definició de \myref{def:recobriment obert d'un subconjunt} tenim que
			\[\tancat{C}\subseteq\bigcup_{i\in I}\obert{U}_{i}.\]
			Per tant %REF
			\[X=\left(\bigcup_{i\in I}\obert{U}_{i}\right)\cup(X\setminus\tancat{C}),\]
			i per la definició de \myref{def:recobriment obert} tenim que \(\{\obert{U}_{i}\}_{i\in I}\cup(X\setminus\tancat{C})\) és un recobriment obert de \(X\).
			
			Ara bé, per hipòtesi tenim que \(X\) és un compacte, i per la definició d'\myref{def:espai topològic compacte} tenim que existeix un subrecobriment finit de~\(\{\obert{U}_{i}\}_{i\in I}\cup(X\setminus\tancat{C})\), i per tant existeix un subconjunt finit \(J\) de \(I\) tal que
			\[\tancat{C}\subseteq\bigcup_{i\in J}\obert{U}_{i},\]
			i per la definició de \myref{def:compacitat d'un subconjunt} hem acabat.
		\end{proof}
	\end{theorem}
	\begin{theorem}[Teorema de Tychonoff]
		\labelname{Teorema de Tychonoff}\label{thm:Teorema de Tychonoff}
		Siguin \(X\) i \(Y\) dos espais topològics no buits. Aleshores \(X\times Y\) és compacte si i només si \(X\) i \(Y\) són compactes.
		\begin{proof}
			Comencem veient que la condició és suficient (\(\implica\)). Suposem doncs que \(X\times Y\) és un compacte. Per l'exemple \myref{ex:les projeccions en la topologia producte són contínues i obertes} trobem que les projeccions
			\begin{align*}
				\pi_{X}\colon X\times Y&\longrightarrow X \\
				(x,y)&\longmapsto x
			\end{align*}
			i
			\begin{align*}
				\pi_{Y}\colon X\times Y&\longrightarrow Y \\
				(x,y)&\longmapsto y
			\end{align*}
			són aplicacions contínues, i per la proposició \ref{prop:la compacitat és una propietat topològica} trobem que \(X\) i \(Y\) són compactes.
			
			Veiem ara que la condició és necessària (\(\implicatper\)). Suposem doncs que \(X\) i \(Y\) són compactes i sigui \(\{\obert{W}_{i}\}_{i\in I}\) un recobriment obert de \(X\times Y\).
			
			Prenem un punt \(x_{0}\) de \(X\). Aleshores per la definició de \myref{def:recobriment d'un espai} tenim que per a tot \(y\) de \(Y\) existeix un \(i_{x_{0},y}\) de \(I\) tal que el punt~\((x_{0},y)\) pertany a \(\obert{W}_{i_{x_{0},y}}\), i per la definició de \myref{def:topologia quocient} tenim que existeixen oberts \(\obert{U}_{i_{x_{0},y}}\) i \(\obert{V}_{i_{x_{0},y}}\) de \(X\) i \(Y\), respectivament, tals que
			\[(x_{0},y)\in\obert{U}_{i_{x_{0},y}}\times\obert{V}_{i_{x_{0},y}}\subseteq\obert{W}_{i_{x_{0},y}}.\]
			
			Trobem doncs que
			\begin{align*}
				Y&=\bigcup_{y\in Y}\{y\} \\
				&\subseteq\bigcup_{y\in Y}\obert{V}_{i_{x_{0},y}}\subseteq Y,
			\end{align*}
			i pel \myref{thm:doble inclusió} trobem que
			\[\bigcup_{y\in Y}\obert{V}_{i_{x_{0},y}}=Y.\]
			Aleshores per la definició de \myref{def:recobriment obert} trobem que \(\{\obert{V}_{i_{x_{0},y}}\}_{y\in Y}\) és un recobriment obert de \(Y\), i com que, per hipòtesi, \(Y\) és compacte, tenim per la definició d'\myref{def:espai topològic compacte} que existeix un subrecobriment finit \(\{\obert{V}_{i_{x_{0},y}}\}_{y\in Y_{x_{0}}}\) de \(\{\obert{V}_{i_{x_{0},y}}\}_{y\in Y}\).
			
			Definim ara
			\begin{equation}
				\label{thm:Teorema de Tychonoff:eq2}
				\obert{U}_{x_{0}}=\bigcap_{y\in Y_{x_{0}}}\obert{U}_{i_{x_{0},y}}.
			\end{equation}
			Com que \(Y_{x_{0}}\) és finit, trobem per la definició de \myref{def:topologia} que \(\obert{U}_{x_{0}}\) és un obert de \(X\). Tenim doncs que
			\begin{align*}
				X&=\bigcup_{x\in X}\{x\} \\
				&\subseteq\bigcup_{x\in X}\obert{U}_{x}\subseteq X,
			\end{align*}
			i pel \myref{thm:doble inclusió} trobem que
			\[\bigcup_{x\in X}\obert{U}_{x}=X,\]
			i de nou per la definició de \myref{def:recobriment obert} trobem que \(\{\obert{U}_{x}\}_{x\in X}\) és un recobriment obert de \(X\), i per la definició d'\myref{def:espai topològic compacte} trobem que existeix un subrecobriment finit \(\{\obert{U}_{x}\}_{x\in J'}\) de \(\{\obert{U}_{x}\}_{x\in X}\).
			
			Definim el conjunt
			\begin{equation}
				\label{thm:Teorema de Tychonoff:eq1}
				J=\{(x,y)\in X\times Y\mid x\in J'\text{ i }y\in Y_{x}\}
			\end{equation}
			i considerem la subfamília
			\[\{\obert{W}_{i_{x,y}}\}_{(x,y)\in J}\]
			de \(\{\obert{W}_{i}\}_{i\in I}\). Sigui \((x,y)\) un punt de \(X\times Y\). Aleshores, com que la família \(\{\obert{U}_{x}\}_{x\in J'}\) és un recobriment de \(X\) trobem per la definició de \myref{def:recobriment d'un espai} que existeix un \(p\) de \(J'\) tal que \(x\) pertany a \(\obert{U}_{p}\), i com que la família \(\{\obert{V}_{i_{x_{0},y}}\}_{y\in Y_{p}}\) és un recobriment de \(Y\) trobem que existeix un \(q\) de \(Y_{p}\) tal que \(y\) pertany a \(\obert{V}_{q}\).
			
			Ara bé, també tenim per \eqref{thm:Teorema de Tychonoff:eq2} i la definició d'\myref{def:intersecció de conjunts} que \(x\) pertany a \(\obert{U}_{i_{p,q}}\), i per tant
			\[(x,y)\in\obert{U}_{i_{p,q}}\times\obert{V}_{i_{p,q}}\subseteq\obert{W}_{i_{p,q}},\]
			i per tant tenim que \(\{\obert{W}_{i_{x,y}}\}_{(x,y)\in J}\) és un subrecobriment de \(\{\obert{W}_{i}\}_{i\in I}\), i com que els conjunts \(J'\) i \(Y_{x}\) són finits, trobem per \eqref{thm:Teorema de Tychonoff:eq1} que \(J\) és finit, aleshores per la definició de \myref{def:subrecobriment finit} trobem que \(\{\obert{W}_{i_{x,y}}\}_{(x,y)\in J}\) és un subrecobriment finit de \(\{\obert{W}_{i}\}_{i\in I}\) i per la definició d'\myref{def:espai topològic compacte} tenim que \(X\times Y\) és compacte, com volíem.
		\end{proof}
	\end{theorem}
%	\section{Els compactes de \ensuremath{\mathbb{R}^{n}}}
%	\subsection{Teorema de Heine-Borel}
%	\begin{theorem}[Teorema de Heine-Borel]
%		\labelname{Teorema de Heine-Borel}\label{thm:Teorema de Heine-Borel}
%		Sigui \(S\) un subconjunt de \(\mathbb{R}^{n}\). Aleshores \(S\) és compacte si i només si \(S\) és tancat i acotat.
%		\begin{proof}
%			%TODO
%		\end{proof}
%	\end{theorem}
%	\subsection{Compactificació per un punt}
%	\begin{definition}[Compactificació per un punt]
%		\labelname{compactificació per un punt}\label{def:compactificació per un punt}
%		Siguin \(X\) un espai topològic i \(x_{0}\) un element que no pertany a \(X\).
%	\end{definition}
	\section{Espais de Hausdorff}
	\subsection{L'axioma de Hausdorff}
	\begin{definition}[Espai Hausdorff]
		\labelname{espai Hausdorff}\label{def:espai Hausdorff}
		Sigui \(X\) un espai topològic tal que per a tots dos punts diferents \(x\) i \(y\) de \(X\) existeixen dos oberts disjunts \(\obert{U}\) i \(\obert{V}\) de \(X\) tals que \(x\) pertany a \(\obert{U}\) i \(y\) pertany a \(\obert{V}\). Aleshores direm que \(X\) és Hausdorff.
	\end{definition}
	\begin{proposition}
		\label{prop:els espais mètrics són Hausdorff}
		Sigui \(X\) un espai mètric. Aleshores \(X\) és Hausdorff.
		\begin{proof}
			Per l'exemple \myref{ex:topologia induida per una mètrica} trobem que \(X\) és un espai topològic i per la proposició \myref{prop:els espais mètris són Hausdorff} tenim que \(X\) és Hausdorff.
		\end{proof}
	\end{proposition}
	\begin{proposition}
		\label{prop:ser Hausdorff és una propietat topològica}
		Siguin \(X\) i \(Y\) dos espais topològics homeomorfs. Aleshores \(X\) és Hausdorff si i només si \(Y\) és Hausdorff.
		\begin{proof}
			Per la proposició \myref{prop:ser homeomorf és una relació d'equivalència} tenim que si \(X\cong Y\) aleshores \(Y\cong X\), i per tant només ens cal veure que si \(X\) és Hausdorff aleshores \(Y\) és Hausdorff.
			
			Suposem doncs que \(X\) és Hausdorff. Prenem dos punts diferents \(x\) i \(y\) de \(Y\). Per la definició d'\myref{def:espais topològics homeomorfs} tenim que existeix un homeomorfisme \(f\colon X\longrightarrow Y\), i per la definició d'\myref{def:homeomorfisme entre topologies} trobem que \(f\) és bijectiva. Considerem doncs els punts \(f^{-1}(x)\) i \(f^{-1}(y)\). Com que, per hipòtesi, \(X\) és Hausdorff, tenim per la definició d'\myref{def:espai Hausdorff} que existeixen dos oberts disjunts \(\obert{U}\) i \(\obert{V}\) de \(X\) tals que \(f^{-1}(x)\) pertany a \(\obert{U}\) i \(f^{-1}(y)\) pertany a \(\obert{V}\).
			
			Ara bé, per la definició d'\myref{def:homeomorfisme entre topologies} trobem que \(f\) és una aplicació oberta, i per tant trobem que els conjunts \(\Ima_{\obert{U}}(f)\) i \(\Ima_{\obert{V}}(f)\) són oberts de \(Y\), i tenim que \(x\) pertany a \(\Ima_{\obert{U}}(f)\) i \(y\) pertany a \(\Ima_{\obert{V}}(f)\), i per la definició d'\myref{def:imatge d'una aplicació} tenim que aquests són disjunts.
		\end{proof}
	\end{proposition}
	\begin{proposition}
		\label{prop:els punts en un Hausdorff són tancats}
		Sigui \(X\) un espai Hausdorff. Aleshores per a tot \(x\) el conjunt \(\{x\}\) és un tancat.
		\begin{proof}
			Per la definició d'\myref{def:espai Hausdorff} trobem que per a tot punt \(y\) diferent de \(x\) existeixen dos oberts disjunts \(\obert{U}_{y}\) i \(\obert{V}_{y}\) tals que \(x\) pertany a \(\obert{U}_{y}\) i \(y\) pertany a \(\obert{V}_{y}\). Aleshores tenim que
			\begin{align*}
				X\setminus\{x\}&=\bigcup_{y\in X\setminus\{x\}}\{y\} \\
				&\subseteq\bigcup_{y\in X\setminus\{x\}}\obert{U}_{y}\subseteq X\setminus\{x\}
			\end{align*}
			i pel \myref{thm:doble inclusió} trobem que
			\[X\setminus\{x\}=\bigcup_{y\in X\setminus\{x\}}\obert{U}_{y}.\]
			Ara bé, per la definició de \myref{def:topologia} trobem que \(X\setminus\{x\}\) és un obert, i per la definició de \myref{def:tancat} trobem que \(\{x\}\) és un tancat.
		\end{proof}
	\end{proposition}
	\begin{proposition}
		\label{prop:els subespais d'un Hausdorff són Hausdorff}
		Sigui \(X\) un espai Hausdorff i \(A\) un subespai de \(X\). Aleshores \(A\) és Hausdorff.
		\begin{proof}
			Siguin \(x\) i \(y\) dos elements diferents de \(A\). Aleshores per la definició d'espai Hausdorff tenim que existeixen dos oberts disjunts \(\obert{U}\) i \(\obert{V}\) de \(X\) tals que \(x\) pertany a \(\obert{U}\) i \(y\) pertany a \(\obert{V}\). Aleshores per la definició de \myref{def:topologia induida per un subconjunt} trobem que els conjunts \(\obert{U}\cap A\) i \(\obert{V}\cap A\) són oberts disjunts de \(A\) que contenen \(x\) i \(y\), respectivament, i per la definició d'\myref{def:espai Hausdorff} hem acabat.
		\end{proof}
	\end{proposition}
	\subsection{Axiomes de separació de Tychonoff}
	\begin{definition}[Espai de Kolmogorov]
		\labelname{espai de Kolmogorov}\label{def:espai de Kolmogorov}
		Sigui \(X\) un espai topològic tal que per a tots dos punts diferents \(x\) i \(y\) existeix un obert \(\obert{U}\) que o bé conté \(x\) i no \(y\), o bé conté \(y\) i no \(x\). Aleshores direm que \(X\) és un espai de Kolmogorov.
	\end{definition}
	\begin{definition}[Espai de Fréchet]
		\labelname{espai de Fréchet}\label{def:espai de Fréchet}
		Sigui \(X\) un espai topològic tal que per a tots dos punts diferents \(x\) i \(y\) existeixen dos oberts \(\obert{U}\) i \(\obert{V}\) tals que \(x\) pertany a \(\obert{U}\setminus\obert{V}\) i \(y\) pertany a \(\obert{V}\setminus\obert{U}\). Aleshores direm que \(X\) és un espai de Fréchet.
	\end{definition}
	\begin{proposition}
		\label{prop:els espais de Fréchet són de Kolmogorov}
		Sigui \(X\) un espai de Fréchet. Aleshores \(X\) és un espai de Kolmogorov.
		\begin{proof}
			Siguin \(x\) i \(y\) dos punts diferents de \(X\). Aleshores per la definició d'\myref{def:espai de Fréchet} trobem que existeixen dos oberts \(\obert{U}\) i \(\obert{V}\) tals que \(x\) pertany a \(\obert{U}\setminus\obert{V}\) i \(y\) pertany a \(\obert{V}\setminus\obert{U}\). En particular \(x\) pertany a \(\obert{U}\) i \(y\) no pertany a \(\obert{U}\), i per la definició d'\myref{def:espai de Kolmogorov} hem acabat.
		\end{proof}
	\end{proposition}
	\begin{proposition}
		\label{prop:si en un espai topològic tots els punts són tancats aquest és Fréchet}
		Sigui \(X\) un espai tal que per a tot \(x\) de \(X\), el conjunt \(\{x\}\) és un tancat. Aleshores \(X\) és un espai de Fréchet.
		\begin{proof}
			Siguin \(x\) i \(y\) dos punts diferents de \(X\). Per hipòtesi tenim que els conjunts \(\{x\}\) i \(\{y\}\) són tancats i per la definició de \myref{def:tancat} trobem que els conjunts \(X\setminus\{x\}\) i \(X\setminus\{y\}\) són oberts. Ara bé, com que per hipòtesi els punts \(x\) i \(y\) són diferents trobem que \(x\) pertany a \(X\setminus\{y\}\) i \(y\) pertany a \(X\setminus\{x\}\), i per la definició d'\myref{def:espai de Fréchet} trobem que \(X\) és un espai de Fréchet, com volíem veure.
		\end{proof}
	\end{proposition}
	\begin{corollary}
		\label{cor:els espais Hausdorff són Fréchet}
		Sigui \(X\) un espai Hausdorff. Aleshores \(X\) és un espai de Fréchet.
		\begin{proof}
			Per la proposició \myref{prop:els punts en un Hausdorff són tancats} trobem que si \(x\) és un punt de \(X\) aleshores el conjunt \(\{x\}\) és un tancat, i per la proposició \myref{prop:si en un espai topològic tots els punts són tancats aquest és Fréchet} trobem que \(X\) és un espai de Fréchet.
		\end{proof}
	\end{corollary}
	\begin{proposition}
		\label{prop:en un espai de Fréchet els punts són tancats}
		Siguin \(X\) un espai de Fréchet i \(x\) un punt de \(X\). Aleshores \(\{x\}\) és un tancat.
		\begin{proof}
			Per la definició d'\myref{def:espai de Fréchet} trobem que per a tot punt \(y\) de \(X\) diferent de \(x\) existeixen dos oberts \(\obert{U}_{y}\) i \(\obert{V}_{y}\) tals que \(x\) pertany a \(\obert{U}_{y}\setminus\obert{V}_{y}\) i \(y\) pertany a \(\obert{V}_{y}\setminus\obert{U}_{y}\). Considerem
			\begin{equation}
				\label{prop:en un espai de Fréchet els punts són tancats:eq1}
				\obert{V}=\bigcup_{y\in X\setminus\{x\}}\obert{V}_{y}.
			\end{equation}
			Tenim que \(x\) no pertany a \(\obert{V}\), ja que \(x\) no pertany a cap dels \(\obert{V}_{y}\), i tenim que
			\begin{align*}
				X\setminus\{x\}&\subseteq\bigcup_{y\in X\setminus\{x\}}\{y\} \\
				&\subseteq\bigcup_{y\in X\setminus\{x\}}\obert{V}_{y}\\
				&=\obert{V}\subseteq X\setminus\{x\},
			\end{align*}
			i pel \myref{thm:doble inclusió} trobem que
			\[\obert{V}=X\setminus\{x\}.\]
			
			Per la definició de \myref{def:topologia} i \eqref{prop:en un espai de Fréchet els punts són tancats:eq1} trobem que \(\obert{V}\) és un obert, i per la definició de \myref{def:tancat}  trobem que \(\{x\}\) és un tancat.
		\end{proof}
	\end{proposition}
	\begin{definition}[Espai regular]
		\labelname{espai regular}\label{def:espai regular}
		Sigui \(X\) un espai de Fréchet tal que donats un tancat \(\tancat{C}\) i un punt \(x\) que no pertany a \(\tancat{C}\) existeixen dos oberts disjunts \(\obert{U}\) i \(\obert{V}\) tals que \(x\) pertany a \(\obert{U}\) i \(\tancat{C}\) és un subconjunt de \(\obert{V}\). Aleshores direm que \(X\) és un espai regular.
	\end{definition}
	\begin{proposition}
		Sigui \(X\) un espai regular. Aleshores \(X\) és un espai Hausdorff.
		\begin{proof}
			Prenem dos punts diferents \(x\) i \(y\) de \(X\). Com que, per hipòtesi, \(X\) és un espai regular, i per la definició d'\myref{def:espai regular} trobem que \(X\) és un espai de Fréchet, per la proposició \myref{prop:en un espai de Fréchet els punts són tancats} trobem que \(\{y\}\) és un tancat, i per la definició d'\myref{def:espai regular} trobem que existeixen dos oberts disjunts \(\obert{U}\) i \(\obert{V}\) tals que \(x\) pertany a \(\obert{U}\) i \(\{y\}\) és un subconjunt de \(\obert{V}\), i en particular \(y\) pertany a \(\obert{V}\). Aleshores per la definició d'\myref{def:espai Hausdorff} hem acabat.
		\end{proof}
	\end{proposition}
	\begin{definition}[Espai normal]
		\labelname{espai normal}\label{def:espai normal}
		Sigui \(X\) un espai de Fréchet tal que donats dos tancats disjunts \(\tancat{C}\) i \(\tancat{K}\) existeixen dos oberts disjunts \(\obert{U}\) i \(\obert{V}\) tals que \(\tancat{C}\) és un subconjunt de \(\obert{U}\) i \(\tancat{K}\) és un subconjunt de \(\obert{V}\). Aleshores direm que \(X\) és un espai normal.
	\end{definition}
	\begin{proposition}
		\label{prop:els espais normals són espais regulars}
		Sigui \(X\) un espai normal. Aleshores \(X\) és un espai regular.
		\begin{proof}
			Siguin \(\tancat{C}\) un tancat de \(X\) i \(x\) un punt de \(X\setminus\tancat{C}\). Com que per hipòtesi \(X\) és un espai normal, per la definició d'\myref{def:espai normal} trobem que \(X\) és un espai de Fréchet, i per la proposició \myref{prop:en un espai de Fréchet els punts són tancats} trobem que el conjunt \(\{x\}\) és un tancat, i tenim que \(\{x\}\) i \(\tancat{C}\) són disjunts.
			
			Ara bé, per la definició d'\myref{def:espai normal} trobem que existeixen dos oberts disjunts \(\obert{U}\) i \(\obert{V}\) tals que \(\{x\}\) és un subconjunt de \(\obert{U}\) i \(\tancat{C}\) és un subconjunt de \(\obert{V}\). Aleshores tenim que \(x\) pertany a \(\obert{U}\), i per la definició d'\myref{def:espai regular} trobem que \(X\) és un espai regular.
		\end{proof}
	\end{proposition}
	\subsection{Propietats dels espais Hausdorff}
	\begin{proposition}
		\label{prop:els compactes en un Hausdorff són tancats}
		Sigui \(X\) un espai Hausdorff i \(A\) un compacte de \(X\). Aleshores \(A\) és un tancat de \(X\).
		\begin{proof} % REVISAR i arreglar wording
			Si \(A=X\) ó \(A=\emptyset\) aleshores pel Teorema \myref{thm:equivalència obert tancat definició de topologia} trobem que \(A\) és tancat i hem acabat.
			
			Suposem doncs que \(A\) no és \(X\) ni \(\emptyset\). Tenim que existeix un punt \(x\) de \(X\setminus A\). Per la definició d'\myref{def:espai Hausdorff} trobem que per a tot \(a\) de \(A\) existeixen dos oberts disjunts \(\obert{U}_{a,x}\) i \(\obert{V}_{a,x}\) tals que \(x\) pertany a \(\obert{U}_{a,x}\) i \(a\) pertany a \(\obert{V}_{a,x}\). Aleshores trobem que
			\begin{align*}
				A&=\bigcup_{a\in A}\{a\} \\
				&\subseteq\bigcup_{a\in A}\obert{V}_{a,x},
			\end{align*}
			i per tant trobem que
			\[\bigcup_{a\in A}\obert{V}_{a,x}\subseteq A.\]
			Aleshores per la definició de \myref{def:recobriment obert} trobem que la família \(\{\obert{V}_{a,x}\}_{a\in A}\) és un recobriment obert de \(A\), i com que, per hipòtesi, \(A\) és compacte trobem per la definició de \myref{def:subconjunt compacte} que existeix un subrecobriment finit \(\{\obert{V}_{a,x}\}_{a\in A'}\) de \(\{\obert{V}_{a,x}\}_{a\in A}\), i per la definició de \myref{def:subrecobriment} trobem que
			\begin{equation}
				\label{prop:els compactes en un Hausdorff són tancats:eq1}
				\bigcup_{a\in A'}\obert{V}_{a,x}\subseteq A.
			\end{equation}
			
			Per la definició de topologia trobem que el conjunt
			\[\obert{U}_{x}=\bigcap_{a\in A'}\obert{U}_{a,x}\]
			és un obert de \(X\) i com que, per hipòtesi, per a tot \(a\) de \(A\) tenim que
			\[\obert{U}_{a,x}\cap\obert{V}_{a,x}=\emptyset\]
			trobem per \eqref{prop:els compactes en un Hausdorff són tancats:eq1} que per a tot \(x\) de \(X\setminus A\) es satisfà
			\[A\cap\obert{U}_{x}=\emptyset,\]
			i per tant
			\[\bigcup_{x\in X\setminus A}\obert{U}_{x}\subseteq X\setminus A.\]
			
			Ara bé, tenim que
			\begin{align*}
				X\setminus A&=\bigcup_{x\in X\setminus A}\{x\} \\
				&\subseteq\bigcup_{x\in X\setminus A}\obert{U}_{x} \\
				&\subseteq X\setminus A,
			\end{align*}
			i pel \myref{thm:doble inclusió} trobem que
			\[X\setminus A=\bigcup_{x\in X\setminus A}\obert{U}_{x}.\]
			Aleshores per la definició de \myref{def:topologia} trobem que \(X\setminus A\) és un obert i per la definició de \myref{def:tancat} trobem que \(A\) és un tancat.
		\end{proof}
	\end{proposition}
	\begin{theorem}
		\label{thm:dos espais són Hausdorff si i només si el seu producte és Hausdorff}
		Siguin \(X\) i \(Y\) dos espais topològics no buits. Aleshores \(X\times Y\) és Hausdorff si i només si \(X\) i \(Y\) són Hausdorff.
		\begin{proof}
			Comencem veient que la condició és suficient (\(\implica\)). Suposem doncs que \(X\times Y\) és Hausdorff. Prenem un element \(y\) de \(Y\). Tenim pel \corollari{} \myref{cor:un espai topològic producte amb un element és homeomorf a l'espai topològic} que \(X\times\{y\}\cong X\) i tenim que \(X\times\{y\}\subseteq X\times Y\). Per hipòtesi tenim que \(X\times Y\) és Hausdorff. Aleshores per la proposició \myref{prop:els subespais d'un Hausdorff són Hausdorff} trobem que \(X\times\{y\}\) és Hausdorff i per la proposició \myref{prop:ser Hausdorff és una propietat topològica} trobem que \(X\) és Hausdorff.
			
			La demostració per veure que \(Y\) és Hausdorff és anàloga.
			
			Veiem ara que la condició és necessària. Suposem doncs que \(X\) i \(Y\) són Hausdorff. Prenem dos punts diferents \((x_{1},y_{1})\) i \((x_{2},y_{2})\) de \(X\times Y\). Si \(y_{1}=y_{2}\) tenim per la proposició \myref{prop:parelles ordenades} que \(x_{1}\neq x_{2}\). Com que, per hipòtesi, \(X\) és Hausdorff tenim per la definició d'\myref{def:espai Hausdorff} que existeixen dos oberts disjunts \(\obert{U}\) i \(\obert{V}\) tals que \(x_{1}\) pertany a \(\obert{U}\) i \(x_{2}\) pertany a \(\obert{V}\). Aleshores per la definició de \myref{def:topologia producte} i la definició de \myref{def:topologia} trobem que els conjunts \((\obert{U},Y)\) i \((\obert{V},Y)\) són oberts disjunts de \(X\times Y\) i per la definició d'\myref{def:espai Hausdorff} tenim que \(X\times Y\) és Hausdorff.
			
			Si \(x_{1}=x_{2}\) tenim de nou per la proposició \myref{prop:parelles ordenades} que \(y_{1}\neq y_{2}\) i l'argument per veure que \(X\times Y\) és Hausdorff és anàleg.
		\end{proof}
	\end{theorem}
	\begin{theorem}
		Siguin \(X\) un espai compacte, \(Y\) un espai Hausdorff i \(f\colon X\longrightarrow Y\) una aplicació contínua i bijectiva. Aleshores \(X\cong Y\).
		\begin{proof}
			per la definició d'\myref{def:espais topològics homeomorfs} en tenim prou amb veure que \(f\) és un homeomorfisme, i per la definició d'\myref{def:homeomorfisme entre topologies} en tenim prou amb veure que \(f\) és una aplicació tancada. Sigui \(\tancat{C}\) un tancat de \(X\). Com que, per hipòtesi, \(X\) és un compacte, pel Teorema \myref{thm:un tancat en un compacte és compacte} trobem que \(\tancat{C}\) és un compacte. Per hipòtesi tenim també que \(f\) és contínua, i pel teorema \myref{thm:la imatge d'un compacte per una aplicació contínua és un compacte} trobem doncs que \(\Ima_{\tancat{C}}(f)\) és un compacte de \(Y\). De nou per hipòtesi tenim que \(Y\) és Hausdorff i per la proposició \myref{prop:els compactes en un Hausdorff són tancats} trobem que \(\Ima_{\tancat{C}}(f)\) és un tancat de \(Y\). Per tant per la definició d'\myref{def:aplicació tancada} trobem que \(f\) és tancada i hem acabat.
		\end{proof}
	\end{theorem}
	\begin{theorem}
		\label{thm:els espais Hausdorff compactes són espais normals}
		Sigui \(X\) un espai Hausdorff compacte. Aleshores \(X\) és un espai normal.
		\begin{proof} % Llegir. L'he copiat com un loro
			Siguin \(\tancat{C}\) i \(\tancat{K}\) dos tancats disjunts i fixem un punt \(x\) de \(\tancat{C}\). Com que, per hipòtesi, \(X\) és Hausdorff, per la definició d'\myref{def:espai Hausdorff} trobem que per a tot \(y\) de \(\tancat{K}\) existeixen dos oberts disjunts \(\obert{U}_{x,y}\) i \(\obert{V}_{x,y}\) tals que \(x\) pertany a \(\obert{U}_{x,y}\) i \(y\) pertany a \(\obert{V}_{x,y}\). Tenim doncs que
			\begin{align*}
				\tancat{K}&=\bigcup_{y\in\tancat{K}}\{y\} \\
				&\subseteq\bigcup_{y\in\tancat{K}}\obert{V}_{x,y},
			\end{align*}
			i per la definició de \myref{def:recobriment obert} trobem que la família \(\{\obert{V}_{x,y}\}_{y\in\tancat{K}}\) és un recobriment obert de \(\tancat{K}\). Aleshores, com que per hipòtesi \(X\) és un espai compacte, pel Teorema \myref{thm:un tancat en un compacte és compacte} trobem que \(\tancat{K}\) és un compacte i per la definició de \myref{def:subconjunt compacte} tenim que existeix un subrecobriment finit \(\{\obert{V}_{x,y}\}_{y\in I_{x}}\) de \(\{\obert{V}_{x,y}\}_{y\in\tancat{K}}\). Considerem els conjunts
			\[\obert{U}_{x}=\bigcap_{y\in I_{x}}\obert{U}_{x,y}\quad\text{i}\quad\obert{V}_{x}=\bigcup_{y\in I_{x}}\obert{V}_{x,y}.\]
			Com que \(I_{x}\) és finit, per la definició de \myref{def:topologia} trobem que \(\obert{U}_{x}\) i \(\obert{V}_{x}\) són oberts de \(X\) i tenim que
			\[\obert{U}_{x}\cap\obert{V}_{x}=\emptyset\]
			i \(x\) pertany a \(\obert{U}_{x}\) i \(y\) pertany a \(\obert{V}_{x}\).
			
			Ara bé, tenim que
			\begin{align*}
				\tancat{C}&=\bigcup_{x\in\tancat{C}}\{x\} \\
				&\subseteq\bigcup_{x\in\tancat{C}}\obert{U}_{x},
			\end{align*}
			i per la definició de \myref{def:recobriment obert} trobem que la família \(\{\obert{U}_{x}\}_{y\in\tancat{C}}\) és un recobriment obert de \(\tancat{C}\). Aleshores, com que per hipòtesi \(X\) és un espai compacte, pel Teorema \myref{thm:un tancat en un compacte és compacte} trobem que \(\tancat{C}\) és un compacte i per la definició de \myref{def:subconjunt compacte} tenim que existeix un subrecobriment finit \(\{\obert{U}_{x}\}_{y\in I}\) de \(\{\obert{U}_{x}\}_{y\in\tancat{X}}\). Definim
			\[\obert{U}=\bigcap_{x\in I}\obert{U}_{x}\quad\text{i}\quad\obert{V}=\bigcup_{x\in I}\obert{V}_{x}.\]
			Com que \(I\) és finit, per la definició de \myref{def:topologia} trobem que \(\obert{U}\) i \(\obert{V}\) són oberts de \(X\) i tenim que
			\[\obert{U}\cap\obert{V}=\emptyset\]
			i \(\tancat{C}\) és un subconjunt de \(\obert{U}\) i \(\tancat{K}\) és un subconjunt de \(\obert{V}\), i per la definició d'\myref{def:espai normal} hem acabat.
		\end{proof}
	\end{theorem}
	\begin{lemma}
		\label{lema:els elements del quocient d'un Hausdorff compacte per un grup finit són tancats en l'espai original}
		Siguin \(G\) un grup finit que actua sobre un espai Hausdorff compacte \(X\) i \(x\) un punt de \(X\). Aleshores \(\overline{x}\) és un tancat de \(X\).
		\begin{proof}
			Per la definició de \myref{def:classe d'equivalència} trobem que
			\[\overline{y}=\{x\in X\mid x\sim y\},\]
			i per la definició de \myref{def:quocient d'un espai per l'acció d'un grup} trobem que
			\[\overline{y}=\{x\in X\mid\text{ existeix un }g\in G\text{ tal que }y=\theta_{g}(x)\}.\]
			
			Com que per hipòtesi \(G\) és finit, trobem que \(\overline{y}\) és finit. Tenim també que
			\[\overline{y}=\bigcup_{x\in\overline{y}}\{x\},\]
			i com que \(\overline{y}\) és finit aquesta és una unió finita. Ara bé, per la proposició \myref{prop:els punts en un Hausdorff són tancats} trobem que els conjunts \(\{x\}\) són tancats i pel Teorema \myref{thm:equivalència obert tancat definició de topologia} hem acabat.
		\end{proof}
	\end{lemma}
	\begin{theorem}
		\label{thm:el quocient d'un Hausdorff compacte per un grup finit és Hausdorff compacte}
		Sigui \(G\) un grup finit que actua sobre un espai Hausdorff compacte \(X\). Aleshores \(X/G\) és un espai Hausdorff compacte.
		\begin{proof} % http://math.ucr.edu/~res/math205C-2011/freeactions.pdf % Revisar
			Siguin \(x\) i \(y\) dos punts de \(X\) tals que \(\overline{x}\) és diferent de \(\overline{y}\). Pel lema \myref{lema:els elements del quocient d'un Hausdorff compacte per un grup finit són tancats en l'espai original} trobem que \(\overline{y}\) és un tancat, i per la proposició \myref{prop:els punts en un Hausdorff són tancats} trobem que el conjunt \(\{x\}\) és un tancat.
			
			Per hipòtesi tenim que \(X\) és un espai Hausdorff compacte, i pel Teorema \myref{thm:els espais Hausdorff compactes són espais normals} trobem que \(X\) és un espai normal. Aleshores per la definició d'\myref{def:espai normal} trobem que existeixen dos oberts disjunts \(\obert{U}'\) i \(\obert{V}'\) tals que \(\{x\}\) és un subconjunt de \(\obert{U}'\) i \(\overline{y}\) és un subconjunt de \(\obert{V}'\).
			
			Sigui
			\[\obert{V}=\bigcap_{g\in G}\Ima_{\obert{V}'}(\theta_{g}).\]
			Per la definició de \myref{def:topologia quocient} trobem que \(\Ima_{\obert{V}'}(\theta_{g})\) és un obert de \(X/G\), i com que per hipòtesi \(G\) és finit, pel Teorema \myref{thm:equivalència obert tancat definició de topologia} trobem que \(\obert{V}\) és un obert de \(X/G\). També tenim per la definició de \myref{def:quocient d'un espai per l'acció d'un grup} que \(\overline{y}\) és un element de \(\obert{V}\). Per l'observació{obs:els accions de grup en un espai topològic són homeomorfismes} tenim que \(\theta_{g}\) és un homeomorfisme, i per la definició d'\myref{def:homeomorfisme entre topologies} trobem que \(\theta_{g}\) és bijectiva. Aleshores, com que tenim que \(\obert{U}'\) i \(\obert{V}'\) són disjunts tenim que \(\obert{U}'\) i \(\obert{V}\) són disjunts. Sigui ara
			\[\obert{U}=\bigcup_{g\in G}\Ima_{\obert{U}'}(\theta_{g}).\]
			Per la definició de \myref{def:topologia quocient} trobem que \(\Ima_{\obert{U}'}(\theta_{g})\) és un obert de \(X/G\), i per la definició de \myref{def:topologia} trobem que \(\obert{U}\) és un obert de \(X/G\). Per l'observació{obs:els accions de grup en un espai topològic són homeomorfismes} tenim que \(\theta_{g}\) és un homeomorfisme, i per la definició d'\myref{def:homeomorfisme entre topologies} trobem que \(\theta_{g}\) és bijectiva. Aleshores, com que tenim que \(\obert{U}'\) i \(\obert{V}\) són disjunts tenim que \(\obert{U}\) i \(\obert{V}\) són disjunts.
			
			Sigui \(\pi\) la projecció
			\begin{align*}
				\pi\colon X&\longrightarrow X/G \\
				x&\longmapsto\overline{x}.
			\end{align*}
			Per la definició de \myref{def:topologia quocient} trobem que els conjunts \(\Ima_{\obert{U}}(\pi)\) i \(\Ima_{\obert{V}}(\pi)\) són oberts de \(X/G\), i tenim que aquests són disjunts ja que si \(\pi(z)\) pertany a la seva intersecció, per la definició de \myref{def:quocient d'un espai per l'acció d'un grup} trobem que \(\overline{z}\) ha de ser un element de \(\obert{U}\cap\obert{V}\), i tenim que aquests són disjunts. Per tant per la definició d'\myref{def:espai Hausdorff} tenim que \(X/G\) és Hausdorff.
			
			Per l'observació \myref{obs:l'aplicació que indueix la topologia en un espai quocient és contínua} tenim que \(\pi\) és contínua, i pel Teorema \myref{thm:la imatge d'un compacte per una aplicació contínua és un compacte} trobem que \(X/G\) és compacte i hem acabat.
		\end{proof}
	\end{theorem}
%	\section{La topologia de Zariski}
%	\subsection{Propietats de la topologia de Zariski}
\chapter{Espais connexos}
	\section{Connexió}
	\subsection{Els espais connexos}
	\begin{proposition}
		\label{prop:topologia a la unió disjunta}
		Siguin \(X\) amb la topologia \(\tau_{1}\) i \(Y\) amb la topologia \(\tau_{2}\) dos espais topològics. Aleshores el conjunt \(X\uniodisjunta Y\) amb la topologia
		\begin{equation}
			\label{prop:topologia a la unió disjunta:eq1}
			\tau=\{\obert{U}\uniodisjunta\obert{V}\mid\obert{U}\in\tau_{1}\text{ i }\obert{V}\in\tau_{2}\}
		\end{equation}
		és un espai topològic.
		\begin{proof}
			Per la definició de \myref{def:topologia} tenim que \(X\) és un obert de \(X\) i \(Y\) és un obert de \(Y\), i per \eqref{prop:topologia a la unió disjunta:eq1} tenim que \(X\uniodisjunta Y\) pertany a \(\tau\). Tenim també per la definició de \myref{def:topologia} que \(\emptyset\) és un obert de \(X\) i \(Y\), i tenim que \(\emptyset\times\{0\}=\emptyset\) i \(\emptyset\times\{1\}=\emptyset\). Per tant de nou per \eqref{prop:topologia a la unió disjunta:eq1} tenim que \(\emptyset\) pertany a \(\tau\).
			
			Prenem ara una família \(\{\obert{U}_{i}\uniodisjunta\obert{V}_{i}\}_{i\in I}\) d'elements de \(\tau\) i considerem
			\[\obert{U}=\bigcup_{i\in I}\obert{U}_{i}\uniodisjunta\obert{V}_{i}.\]
			Tenim que
			\begin{align*}
				\obert{U}&=\bigcup_{i\in I}\obert{U}_{i}\uniodisjunta\obert{V}_{i} \\
				&=\bigcup_{i\in I}\big((\obert{U}_{i}\times\{0\})\cup(\obert{V}_{i}\times\{1\})\big) \tag{\ref{def:unió disjunta}} \\
				&=\left(\bigcup_{i\in I}(\obert{U}_{i}\times\{0\})\right)\cup\left(\bigcup_{i\in I}(\obert{V}_{i}\times\{1\})\right) \\
				&=\left(\bigcup_{i\in I}\obert{U}_{i}\times\{0\}\right)\cup\left(\bigcup_{i\in I}\obert{V}_{i}\times\{1\}\right).
			\end{align*}
			Per la definició de \myref{def:topologia} trobem que els conjunts
			\[\bigcup_{i\in I}\obert{U}_{i}\quad\text{i}\quad\bigcup_{i\in I}\obert{V}_{i}\]
			són oberts de \(X\) i \(Y\), respectivament; i per \eqref{prop:topologia a la unió disjunta:eq1} trobem que \(\obert{U}\) pertany a \(\tau\).
			
			Prenem ara una família finita \(\{\obert{U}_{i}\uniodisjunta\obert{V}_{i}\}_{i=1}^{n}\) d'elements de \(\tau\) i considerem
			\[\obert{U}=\bigcap_{i=1}^{n}\obert{U}_{i}\uniodisjunta\obert{V}_{i}.\]
			Tenim que
			\begin{align*}
				\obert{U}&=\bigcap_{i=1}^{n}\obert{U}_{i}\uniodisjunta\obert{V}_{i} \\
				&=\bigcap_{i=1}^{n}\big((\obert{U}_{i}\times\{0\})\cup(\obert{V}_{i}\times\{1\})\big) \tag{\ref{def:unió disjunta}} \\
				&=\left(\bigcap_{i=1}^{n}(\obert{U}_{i}\times\{0\})\right)\cup\left(\bigcap_{i=1}^{n}(\obert{V}_{i}\times\{1\})\right) \\
				&=\left(\bigcap_{i=1}^{n}\obert{U}_{i}\times\{0\}\right)\cup\left(\bigcap_{i=1}^{n}\obert{V}_{i}\times\{1\}\right).
			\end{align*}
			Per la definició de \myref{def:topologia} trobem que els conjunts
			\[\bigcap_{i=1}^{n}\obert{U}_{i}\quad\text{i}\quad\bigcap_{i=1}^{n}\obert{V}_{i}\]
			són oberts de \(X\) i \(Y\), respectivament; i per \eqref{prop:topologia a la unió disjunta:eq1} trobem que \(\obert{U}\) pertany a \(\tau\).
			
			Per tant per la definició de \myref{def:topologia} trobem que \(\tau\) és una topologia de \(X\uniodisjunta Y\), com volíem veure.
		\end{proof}
	\end{proposition}
	\begin{definition}[Unió disconnexa]
		\labelname{unió disconnexa}\label{def:unió disconnexa}
		Siguin \(X\) i \(Y\) dos espais topològics. Aleshores direm que l'espai topològic \(X\uniodisjunta Y\) amb la topologia
		\[\tau=\{\obert{U}\uniodisjunta\obert{V}\mid\obert{U}\text{ és un obert de }X\text{ i }\obert{V}\text{ és un obert de }Y\}\]
		és la unió disconnexa de \(X\) i \(Y\).
		
		Aquesta definició té sentit per la proposició \myref{prop:topologia a la unió disjunta}.
	\end{definition}
	\begin{definition}[Espai connex]
		\labelname{espai connex}\label{def:espai connex}
		Sigui \(X\) un espai topològic tal que no existeixen dos espais no buits \(Y_{1}\) i \(Y_{2}\) tals que \(X\cong Y_{1}\uniodisjunta Y_{2}\). Aleshores direm que \(X\) és connex.
	\end{definition}
	\begin{proposition}
		\label{prop:condicions equivalents a espai connex}
		Sigui \(X\) un espai topològic. Aleshores són equivalents
		\begin{enumerate}
			\item\label{prop:condicions equivalents a espai connex:eq1} \(X\) és connex.
			\item\label{prop:condicions equivalents a espai connex:eq2} No existeixen dos oberts no buits disjunts \(\obert{U}\) i \(\obert{V}\) tals que \(X=\obert{U}\cup\obert{V}\).
			\item\label{prop:condicions equivalents a espai connex:eq3} No existeixen dos tancats no buits disjunts \(\tancat{C}\) i \(\tancat{K}\) tals que \(X=\tancat{C}\cup\tancat{K}\).
			\item\label{prop:condicions equivalents a espai connex:eq4} Si \(A\) és un subconjunt de \(X\) tal que \(A\) és obert i tancat aleshores \(A=\emptyset\) ó \(A=X\).
		\end{enumerate}
		\begin{proof}
			%TODO
%			Veiem que \eqref{prop:condicions equivalents a espai connex:eq1}\(\implica\)\eqref{prop:condicions equivalents a espai connex:eq2}. Suposem que \(X\) és connex i prenem dos oberts no buits disjunts \(\obert{U}\) i \(\obert{V}\). Pel \corollari{} \myref{cor:un espai topològic producte amb un element és homeomorf a l'espai topològic} tenim que \(\obert{U}\times\{0\}\cong\obert{U}\) i \(\obert{V}\times\{1\}\cong\obert{V}\), i com que per hipòtesi \(\obert{U}\) i \(\obert{V}\) són disjunts tenim que
%			\[\obert{U}\cup\obert{V}\cong\obert{U}\uniodisjunta\obert{V},\] %REF
%			i aleshores per la definició d'\myref{def:espai connex} trobem que no existeixen dos oberts no buits disjunts \(\obert{U}\) i \(\obert{V}\) tals que \(X=\obert{U}\cup\obert{V}\).
		\end{proof}
	\end{proposition}
	\begin{example}
		\label{ex:l'espai amb la topologia grollera és connex}
		Volem veure que un espai topològic \(X\) amb la topologia grollera. Aleshores \(X\) és connex.
		\begin{solution}
			Tenim per l'exemple \myref{ex:topologia grollera} i la definició de \myref{def:tancat} que els tancats i els oberts de \(X\) són \(\emptyset\) i \(X\). Aleshores per la proposició \myref{prop:condicions equivalents a espai connex} trobem que \(X\) és connex.
		\end{solution}
	\end{example}
	\subsection{Propietats dels espais connexos}
	\begin{proposition}
		\label{prop:la unió d'una família no disjunta de connexos és connexa}
		Siguin \(X\) un espai topològic i \(\{\obert{U}_{i}\}_{i\in I}\) una família de subespais topològics connexos de \(X\) tals que
		\[\bigcap_{i\in I}\obert{U}_{i}\neq\emptyset.\]
		Aleshores tenim que
		\[\obert{U}=\bigcup_{i\in I}\obert{U}_{i}\]
		és un espai topològic connex.
		\begin{proof}
%			Sigui \(A\) un subespai obert i tancat de \(\obert{U}\). Si \(A=\emptyset\) tenim per la proposició \myref{prop:condicions equivalents a espai connex} hem acabat. Suposem doncs que \(A\neq\emptyset\). Aleshores
		\end{proof}
	\end{proposition}
	\begin{corollary}
		\label{cor:la unió d'una família numerable de connexos no disjunts dos a dos és connexa}
		Siguin \(X\) un espai topològic i \(\{\obert{U}_{i}\}_{i\in\mathbb{N}}\) una família de subespais topològics connexos de \(X\) tals que
		\[\obert{U}_{i}\cap\obert{U}_{i+1}\neq\emptyset\quad\text{per a tot }i\in\mathbb{N}.\]
		Aleshores tenim que
		\[\obert{U}=\bigcup_{i\in I}\obert{U}_{i}\]
		és un espai topològic connex.
		\begin{proof}
			%TODO
		\end{proof}
	\end{corollary}
	\begin{example}
		\label{ex:els connexos en R són els intervals}
		Volem veure que un subconjunt de \(\mathbb{R}\) és connex si i només si és un interval.
		\begin{solution}
			%TODO
		\end{solution}
	\end{example}
	\begin{proposition}
		\label{prop:la connexió és conserva per aplicacions contínues}
		Siguin \(A\) un subespai connex d'un espai topològic \(X\) i \(f\colon X\longrightarrow Y\) una aplicació contínua. Aleshores \(\Ima_{A}(f)\) és connex.
		\begin{proof}
			%TODO
%			Considerem l'aplicació
%			\begin{align*}
%				g\colon A&\longrightarrow\Ima_{A}(f) \\
%				g(x)&\longmapsto f(x).
%			\end{align*}
%			Per la definició d'\myref{def:imatge d'una aplicació} i la definició d'\myref{def:aplicació contínua} tenim que \(g\) és contínua. Prenem un subespai \(S\) de \(\Ima_{A}(g)\) tal que \(S\) sigui obert i tancat en \(\Ima_{A}(g)\).
		\end{proof}
	\end{proposition}
	\begin{proposition}
		\label{prop:dos espais topològics són connexos si i només si el seu producte cartesià ho és}
		Siguin \(X\) i \(Y\) dos espais topològics. Aleshores \(X\times Y\) és un espai connex si i només si \(X\) i \(Y\) són espais connexos.
		\begin{proof}
			Comencem veient que la condició és suficient (\(\implica\)). Suposem doncs que \(X\times Y\) és un espai connex. Tenim per l'exemple \myref{ex:les projeccions en la topologia producte són contínues i obertes} que les projeccions \(\pi_{X}\) i \(\pi_{Y}\) són contínues, i per la proposició \myref{prop:la connexió és conserva per aplicacions contínues} tenim que \(X\) i \(Y\) són connexos.
			
			Veiem ara que la condició és necessària (\(\implicatper\)). Suposem doncs que \(X\) i \(Y\) són espais connexos i fixem un \(y\) de \(Y\). Aleshores tenim que
			\[X\times Y=\bigcup_{x\in X}\left((X\times\{y\})\cup(\{x\}\times Y)\right)\]
			i per la proposició \myref{prop:la unió d'una família no disjunta de connexos és connexa} hem acabat.
		\end{proof}
	\end{proposition}
	\begin{proposition}
		\label{prop:un conjunt inclós entre un connex i la seva clausura és connex}
		Sigui \(A\) un subespai connex d'un espai topològic \(X\) i \(B\) un conjunt tal que
		\[A\subseteq B\subseteq\clausura(A).\]
		Aleshores \(B\) és connex.
		\begin{proof}
			%TODO
		\end{proof}
	\end{proposition}
	\subsection{Connexió per camins}
	\begin{definition}[Camí]
		\labelname{camí}\label{de:camí}
		\labelname{origen d'un camí}\label{def:orígen d'un camí}
		\labelname{final d'un camí}\label{def:final d'un camí}
		Siguin \(X\) un espai topològic i \(\omega\colon[0,1]\longrightarrow X\) una aplicació contínua. Aleshores direm que \(\omega\) és un camí.
		
		També direm que \(\omega(0)\) és l'origen del camí i \(\omega(1)\) el final del camí.
	\end{definition}
	\begin{definition}[Connex per camins]
		\label{connexió per camins}\label{def:connexió per camins}
		Sigui \(X\) un espai topològic tal que per a tots dos punts \(x\) i \(y\) de \(X\) existeix un camí \(\omega\) tal que \(x\) és l'origen de \(\omega\) i \(y\) és el final de \(\omega\). Aleshores direm que \(X\) és connex per camins.
	\end{definition}
	\begin{proposition}
		\label{prop:els connexos per camins són connexos}
		Sigui \(X\) un espai topològic connex per camins. Aleshores \(X\) és connex.
		\begin{proof}
			%TODO
		\end{proof}
	\end{proposition}
	\begin{example}
		\label{ex:no tots els espais connexos són connexos per camins}
		Volem veure que no tots els espais connexos són connexos per camins.
		\begin{solution}
			%TODO
		\end{solution}
	\end{example}
	\subsection{Components connexos d'un espai}
	\begin{proposition}
		\label{prop:components connexos}
		Sigui \(X\) un espai topològic i \(\sim\) una relació sobre \(X\) tal que \(x\sim y\) si i només si existeix un subespai \(C\) connex de \(X\) tal que \(x\) i \(y\) pertanyen a \(C\). Aleshores \(\sim\) és una relació d'equivalència.
		\begin{proof}
			%TODO
		\end{proof}
	\end{proposition}
	\begin{definition}[Components connexos]
		\labelname{components connexos}\label{def:compontents connexos}
		Siguin \(X\) un espai topològic i \(\sim\) una relació sobre \(X\) tal que \(x\sim y\) si i només si existeix un subespai \(C\) connex de \(X\) tal que \(x\) i \(y\) pertanyen a \(C\). Aleshores direm que les classes d'equivalència de \(\sim\) són components connexos de \(X\).
		
		Denotarem \(\overline{x}\) com \(\componentconnex(x)\).
		
		Aquesta definició té sentit per la proposició \myref{prop:components connexos}.
	\end{definition}
	\begin{proposition}
		\label{prop:el component connex d'un punt és la unió dels connexos que el contenen}
		Sigui \(x\) un punt d'un espai topològic \(X\). Aleshores
		\[\componentconnex(x)=\bigcup_{\substack{x\in C\\C\text{ és connex}}}C.\] % Kern
		\begin{proof}
			%TODO
		\end{proof}
	\end{proposition}
	\begin{proposition}
		\label{prop:el component connex d'un punt és connex}
		Sigui \(x\) un punt d'un espai topològic \(X\). Aleshores \(\componentconnex(x)\) és connex.
		\begin{proof}
			%TODO
		\end{proof}
	\end{proposition}
	\begin{proposition}
		\label{prop:el component connex d'un punt conté tots els connexos que contenen el punt}
		Siguin \(x\) un punt d'un espai topològic \(X\) i \(C\) un connex que conté \(x\). Aleshores \(C\) és un subconjunt de \(\componentconnex(x)\).
		\begin{proof}
			%TODO
		\end{proof}
	\end{proposition}
	\begin{proposition}
		\label{prop:els components connexos són disjunts}
		Siguin \(x\) i \(y\) dos punts d'un espai topològic \(X\) tals que \(\componentconnex(x)\neq\componentconnex(y)\). Aleshores \(\componentconnex(x)\cap\componentconnex(y)=\emptyset\).
		\begin{proof}
			%TODO
		\end{proof}
	\end{proposition}
	\begin{proposition}
		\label{prop:un component connex és un tancat}
		Sigui \(x\) un punt d'un espai topològic \(X\). Aleshores \(\componentconnex(x)\) és un tancat.
		\begin{proof}
			%TODO
		\end{proof}
	\end{proposition}
	
	
	\begin{comment}
	\section{Espai d'Alexandroff}
	\subsection{Propietats bàsiques}
	\begin{definition}[Topologia d'Alexandroff]
		\labelname{topologia d'Alexandroff}\label{def:topologia d'Alexandroff}
		Siguin \(X\) amb la topologia \(\tau\) un espai topològic tal que per a tota família d'oberts \(\{\obert{U}_{i}\}_{i\in I}\) de \(X\) tenim que
		\[\bigcap_{i\in I}\obert{U}_{i}\]
		és un obert de \(X\). Aleshores direm que \(\tau\) és una topologia d'Alexandroff o que \(X\) amb \(\tau\) és un espai topològic d'Alexandroff.
	\end{definition}
	\begin{proposition}
		\label{prop:els espais topològics finits són d'Alexandroff}
		Siguin \(X\) amb la topologia \(\tau\) un espai topològic amb \(X\) finit. Aleshores \(\tau\) és una topologia d'Alexandroff.
		\begin{proof}
			Com que \(X\) és finit tenim que \(\mathcal{P}(X)\) és finit, i com que \(\tau\) és, per la definició de \myref{def:topologia}, un subconjunt de \(\mathcal{P}(X)\) trobem que \(\tau\) és finit. %REF
			
			Sigui \(\{\obert{U}_{i}\}_{i\in I}\) una família d'oberts de \(X\) i definim
			\[\obert{U}=\bigcap_{i\in I}\obert{U}_{i}.\]
			Tenim que \(\{\obert{U}_{i}\}_{i\in I}\) és un subconjunt de \(\tau\), i tenim també que \(\tau\) és finit, per tant \(\{\obert{U}_{i}\}_{i\in I}\) ha de ser finit i per tant existeix un natural \(n\) tal que \(\{\obert{U}_{i}\}_{i\in I}=\{\obert{U}_{i}\}_{i=1}^{n}\), i per la definició de \myref{def:topologia} hem acabat.
		\end{proof}
	\end{proposition}
	\end{comment}
%	
%	\begin{definition}[Aplicació contínua i homeomorfisme]
%		\labelname{aplicació contínua}\label{def:aplicació contínua}
%		\labelname{homeomorfisme}\label{def:homeomorfisme topo}
%		Siguin \(X\) amb \(\tau\) i \(X'\) amb \(\tau'\) dos espais topològics i \(f\colon X\longrightarrow X'\) una aplicació tal que per a tot obert \(\obert{U}\) de \(X'\) tenim que el conjunt
%		\[f^{-1}(\obert{U})=\{x\in X\mid f(x)\in\obert{U}\}\]
%		és un obert de \(X\). Aleshores direm que \(f\) és una aplicació contínua.
%		
%		Si \(f\) és invertible i la seva inversa és una aplicació contínua direm que \(f\) és un homeomorfisme.
%	\end{definition}
%	\begin{observation}
%		\label{obs:les funcions contínues són aplicacions contínues}
%		Sigui \(f\) una funció contínua. Aleshores \(f\) és una aplicació contínua. %Proof?
%	\end{observation}
%	\begin{proposition}
%		Siguin \(X_{1}\) amb la topologia \(\tau_{1}\), \(X_{2}\) amb la topologia \(\tau_{2}\) i \(X_{3}\) amb la topologia \(\tau_{3}\) tres espais topològics i \(f\colon X_{1}\longrightarrow X_{2}\) i \(g\colon X_{2}\longrightarrow X_{3}\) dues aplicacions contínues. Aleshores l'aplicació
%		\begin{align*}
%			h\colon X_{1}&\longrightarrow X_{3} \\
%			x&\longmapsto f(g(x))
%		\end{align*}
%		és una aplicació contínua.
%		\begin{proof}
%			
%		\end{proof}
%	\end{proposition}

%	\begin{theorem}
%		Siguin \(X\) amb la distància \(\distancia\) i \(X'\) amb la distància \(\distancia'\) dos espais mètrics i \(f\colon X\longrightarrow X'\) una funció. Aleshores \(f\) és contínua si i només si per a tot \(\obert{U}\) obert de \(X'\) tenim que \(\Ima_{\obert{U}}(f^{-1})\) és obert.
%		\begin{proof}
%			Veiem primer que la condició és necessària (\(\implica\)). Suposem doncs que \(f\) és contínua. Prenem un obert \(\obert{U}\) de \(X'\) i un element \(y\) de \(X'\) tal que \(f^{-1}(y)\in\Ima_{\obert{U}}(f^{-1})\), i denotem \(x=f^{-1}(y)\). Per la definició d'\myref{def:obert espai mètric} tenim que existeix un nombre real \(\varepsilon>0\) tal que \(\bola(y,\varepsilon)\subset\Ima_{\obert{U}}(f^{-1})\).
%			
%			Prenem un element \(x'\) de \(\bola(y,\varepsilon)\). Per la definició de \myref{def:bola} tenim que \(\distancia(x,x')<\varepsilon\), i per la definició de \myref{def:funció contínua} tenim que existeix un \(\delta>0\) real tal que \(\distancia'(f(x'),y)<\varepsilon\), i per la definició de \myref{def:bola} això és que \(f(x')\in\bola(y,\varepsilon)\). Per tant tenim que \(\Ima_{\bola(x,\delta)}(f)\subset\bola(y,\varepsilon)\) i trobem que \(\bola(x,\delta)\subset\Ima_{\obert{U}}(f^{-1})\), i per la definició d'\myref{def:obert espai mètric} trobem que \(\Ima_{\obert{U}}(f^{-1})\) és un obert, com volíem veure.
%			
%			Veiem ara que la condició és suficient (\(\implicatper\)). Suposem doncs que per a tot \(\obert{U}\) obert de \(X'\) tenim que \(\Ima_{\obert{U}}(f^{-1})\) és obert.
%			
%			Prenem un element \(xy\) de \(X\) i un real \(\varepsilon>0\). Si denotem \(y=f(x)\) tenim, per la proposició \myref{prop:les boles són oberts}, que la bola \(\bola(y,\varepsilon)\) és un obert. Per hipòtesi tenim que \(\Ima_{\bola(y,\varepsilon)}(f^{-1})\) és un obert, i per la definició d'\myref{def:obert espai mètric} tenim que existeix un nombre real \(\delta>0\) tal que \(\bola(x,\delta)\subset\Ima_{\bola(y,\varepsilon)}(f^{-1})\).
%			
%			Ara bé, per la definició de \myref{def:bola} tenim que això és que per a tot \(x_{0}\) de \(x\) amb \(\distancia'(y,f(x_{0}))<\varepsilon\) tenim que \(\distancia(x,x_{0})<\delta\), i per tant, per la definició de \myref{def:funció contínua} tenim que \(f\) és contínua.
%		\end{proof}
%	\end{theorem}

%	\begin{definition}[El conjunt de Cantor]
%		\labelname{conjunt de Cantor}\label{def:conjunt de Cantor}
%		Siguin
%		\[X_{1}=\left(\frac{1}{3},\frac{2}{3}\right)\quad\text{i}\quad I_{1}=[0,1]\setminus X_{1}.\]
%		Definim
%		\[X_{n+1}=X_{n}\cup\left(\bigcup_{i=0}^{3^{n}-1}\left(\frac{1+3i}{3^{n+1}},\frac{2+3i}{3^{n+1}}\right)\right)\]
%		i \(I_{n+1}=[0,1]\setminus X_{n+1}\). Aleshores direm que
%		\[C=\bigcap_{i=0}^{\infty}I_{n}\]
%		és el conjunt de Cantor.
%	\end{definition}

%	\subsection{Varietats}
%	\subsection{Teorema de classificació de les superfícies compactes}
	
	\printbibliography
	La majoria del contingut està escrit seguint \cite{ACTEAguade}, que també s'utilitza per les classes de teoria de l'assignatura. He extret la demostració d'un Teorema de \cite{SchultzFreeActionsOnFiniteGroupsOnHausdorffSpaces}.
	
	La bibliografia del curs inclou els textos \cite{AFirstCourseInAlgebraicTopologyKosniowski,ABasicCourseInAlgebraicTopologyKosniowski,TopologyKlaus,ACTEAguade}.
\end{document}

% Treure la merda de unió disjunta. Fer-ho com a \cite{AFirstCourseInAlgebraicTopologyKosniowski}. Pàgina 54 del llibre, abans del teorema 8.11, per X/A i el tema de connexitat

% https://arxiv.org/pdf/0708.2136.pdf (L) Sorprenentment elegant (Espais d'Alexandroff)
% http://math.ucr.edu/~res/math205C-2011/freeactions.pdf El quocient d'un espai Hausdorff compacte per un grup és Hausdorff compacte