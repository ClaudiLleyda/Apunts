\documentclass[../Apunts.tex]{subfiles}

\begin{document}
\part{Probabilitat i modelització estocàstica}
\chapter{Models probabilístics}
\section{L'espai de probabilitat}
\subsection{Espais mostrals}
	\begin{definition}[Experiment aleatori]
		\labelname{experiment aleatori}\label{def:experiment aleatori}
		\labelname{espai mostral}\label{def:espai mostral}
		Definirem de manera informal els fenòmens que estudiarem, els \emph{experiments aleatoris}. Aquests tenen les següents propietats:
		\begin{enumerate}
			\item Coneixem tots els possibles resultats de l'experiment, però no el que sortirà. El conjunt de possibles resultats~\(\Omega\) serà l'\emph{espai mostral}.
			\item Tenim alguna manera d'assignar probabilitats als resultats, o a conjunts de resultats.
		\end{enumerate}
	\end{definition}
	\begin{example}
		\label{ex:espais mostrals}
		Volem determinar l'espai mostral dels següents experiments:
		\begin{enumerate}
			\item\label{ex:espais mostrals:eq1} Llençar un dau de sis cares.
			\item\label{ex:espais mostrals:eq2} Un partit de bàsquet.
			\item\label{ex:espais mostrals:eq3} Llençar una moneda.
		\end{enumerate}
		\begin{solution}
			Així, els seus espais mostrals són, respectivament,
			\begin{enumerate}
				\item \(\Omega=\{1,2,3,4,5,6\}\).
				\item \(\Omega=\NN\times\NN\).
				\item \(\Omega=\{\heads,\tails\}\).\qedhere
			\end{enumerate}
			%TODO (?)
		\end{solution}
	\end{example}
	\begin{definition}[Esdeveniment]
		\labelname{esdeveniment}\label{def:esdeveniment}
		\labelname{esdeveniment segur}\label{def:esdeveniment segur}
		\labelname{esdeveniment contrari}\label{def:esdeveniment contrari}
		\labelname{esdeveniment impossible}\label{def:esdeveniment impossible}
		Sigui~\(A\subseteq\Omega\) un subconjunt d'un espai mostral~\(\Omega\). Aleshores direm que~\(A\) és un esdeveniment.
		
		Si el resultat~\(\omega\in A\subseteq\Omega\) s'ha realitzat direm que l'esdeveniment~\(A\) s'ha realitzat. També definim
		\begin{enumerate}
			\item Direm que~\(A=\Omega\) és l'esdeveniment segur i~\(A=\emptyset\) és l'esdeveniment impossible.
			\item Direm que~\(B=\Omega\setminus A\) és l'esdeveniment contrari a l'esdeveniment~\(A\).
%			\item Direm que~\(A=\emptyset\) és l'esdeveniment impossible.
		\end{enumerate}
	\end{definition}
	\begin{example}
		Volem determinar els conjunts d'esdeveniments següents:
		\begin{enumerate}
			\item Tirar un dau de sis cares i que surti un nombre parell.
			\item Una partit de bàsquet on guanya el visitant.
			\item Un món on acabo aquests apunts abans que la carrera.
		\end{enumerate}
		\begin{solution}
			Els conjunts d'esdeveniments són, respectivament,
			\begin{enumerate}
				\item \(A=\{2,4,6\}\).
				\item \(A=\{(m,n)\in\NN\times\NN\mid n>m\}\).
				\item \(A=\emptyset\).\qedhere
			\end{enumerate}
			%TODO (?)
		\end{solution}
	\end{example}
\subsection{Àlgebres i \ensuremath{\sigma}-àlgebres}
	\begin{definition}[Àlgebra]
		\labelname{àlgebra}\label{def:àlgebra}
		Sigui~\(\algebra{A}\) una co{\lgem}ecció de subconjunts~\(\Omega\) tal que
		\begin{enumerate}
			\item \(\Omega\in\algebra{A}\).
			\item Si~\(A\in\algebra{A}\), aleshores~\(\Omega\setminus A\in\algebra{A}\).
			\item Si~\(A\), \(B\in\algebra{A}\), aleshores~\(A\cup B\in\algebra{A}\).
		\end{enumerate}
		Aleshores direm que~\(\algebra{A}\) és un àlgebra de conjunts sobre~\(\Omega\).
	\end{definition}
	\begin{observation}
		\label{obs:el conjunt buit pertany a qualsevol àlgebra}
		Si~\(\algebra{A}\) és un àlgebra, aleshores~\(\emptyset\in\algebra{A}\).
	\end{observation}
	\begin{example}
		\label{ex:exemple d'àlgebra}
		Sigui~\(\Omega=\{1,2,\dots,6\}\). Volem veure que
		\[\algebra{A}=\{\emptyset,\Omega,\{1,2\},\{3,4,5,6\}\}\]
		és un àlgebra.
		\begin{solution}
			Comprovem la definició~d'\myref{def:àlgebra}. Tenim que~\(\Omega\in\algebra{A}\). Prenem un esdeveniment~\(A\in\algebra{A}\) i calculem~\(\Omega\setminus A\). Podem veure tots els casos fent
			\[
				\Omega\setminus\emptyset=\Omega,\qquad
				\Omega\setminus\Omega=\emptyset,\qquad
				\Omega\setminus\{1,2\}=\{3,4,5,6\},\quad\text{i}\quad
				\Omega\setminus\{3,4,5,6\}=\{1,2\}.
			\]
			
			Prenen ara~\(A\),~\(B\in\algebra{A}\). Si~\(A=B\) aleshores~\(A\cup B=A\), i per tant~\(A\cup B\in\algebra{A}\), i si tenim~\(A\cup B\in\algebra{A}\) aleshores~\(B\cup A=A\cup B\in\algebra{A}\) i hem acabat. També tenim que si~\(A\in\algebra{A}\), aleshores~\(\emptyset\cup A=A\in\algebra{A}\) i~\(\Omega\cup A=\Omega\in\algebra{A}\). Per tant només ens cal veure el cas
			\[
				\{1,2\}\cup\{3,4,5,6\}=\{1,2,3,4,5,6\}=\Omega\in\algebra{A},
			\]
			i per la definició~d'\myref{def:àlgebra} trobem que~\(\algebra{A}\) és un àlgebra sobre~\(\Omega\)
		\end{solution}
	\end{example}
	\begin{proposition}
		\label{prop:les àlgebras són tancades per interseccions}
		Siguin~\(A\),~\(B\) dos elements d'un àlgebra~\(\algebra{A}\). Aleshores
		\[A\cap B\in\algebra{A}.\]
		\begin{proof}
			Per la definició~d'\myref{def:àlgebra} tenim que~\(A^{\complement}\),~\(B^{\complement}\in\algebra{A}\) i, de nou per la definició~d'\myref{def:àlgebra}, tenim que~\(A^{\complement}\cup B^{\complement}\in\algebra{A}\). Ara bé,%REF
			tenim que~\(A^{\complement}\cup B^{\complement}=A\cap B\), i per tant~\(A\cap B\in\algebra{A}\).
		\end{proof}
	\end{proposition}
	\begin{definition}[\(\sigma\)-àlgebra]
		\labelname{\ensuremath{\sigma}-àlgebra}\label{def:sigma àlgebra}
		Sigui~\(\Algebra{A}\) un àlgebra tal que per a tota família~\(\{A_{n}\}_{n\in\NN}\) d'elements~d'\(\Algebra{A}\) tenim que
		\[\bigcup_{n=0}^{\infty}A_{n}\in\Algebra{A}.\]
		Aleshores direm que~\(\Algebra{A}\) és una~\(\sigma\)-àlgebra.
	\end{definition}
	\begin{observation}
		Sigui~\(\Algebra{A}\) un àlgebra sobre un conjunt~\(\Omega\) finit. Aleshores~\(\Algebra{A}\) és una~\(\sigma\)-àlgebra.
	\end{observation}
	\begin{proposition}
		\label{prop:les sigma àlgebras són tancades per interseccions numerables}
		Sigui~\(\{A_{n}\}_{n\in\NN}\) una família d'elements d'un~\(\sigma\)-àlgebra~\(\Algebra{A}\). Aleshores
		\[A=\bigcap_{n=0}^{\infty}A_{n}\]
		és un element de~\(\Algebra{A}\).
		\begin{proof}
			Tenim que
			\[A^{\complement}=\Big(\bigcap_{n=0}^{\infty}A_{n}\Big)^{\complement}=\Big(\bigcup_{n=0}^{\infty}A_{n}^{\complement}\Big).\]
			Ara bé, per la definició~d'\myref{def:àlgebra} tenim que~\(A_{n}^{\complement}\in\Algebra{A}\) per a tot~\(n\in\NN\), i per la definició de~\myref{def:sigma àlgebra} trobem que~\(A^{\complement}\in\Algebra{A}\), i de nou per la definició~d'\myref{def:àlgebra} tenim que~\(A\in\Algebra{A}\).
		\end{proof}
	\end{proposition}
\subsection{L'àlgebra de Borel}
	\begin{proposition}
		\label{prop:existeix una sigma àlgebra mínima}
		Sigui~\(\{\Algebra{A}_{i}\}_{i\in I}\) una família de~\(\sigma\)-àlgebres sobre un conjunt~\(\Omega\). Aleshores
		\[\Algebra{A}=\bigcap_{i\in I}\Algebra{A}_{i}\]
		és una~\(\sigma\)-àlgebra sobre~\(\Omega\).
		\begin{proof}
			Comprovem que~\(\Algebra{A}\) satisfà la definició~\myref{def:sigma àlgebra}. Veiem que~\(\Omega\in\Algebra{A}\). Tenim per la definició de~\myref{def:àlgebra} que~\(\Omega\in\Algebra{A}_{i}\) per a tot~\(i\in I\), i per la definició d'\myref{def:intersecció de conjunts} trobem que~\(\Omega\in\Algebra{A}\).
			
			Prenem ara un~\(A\in\Algebra{A}\) i considerem~\(A^{\complement}\). Si~\(A\in\Algebra{A}\) tenim que per a tot~\(i\in I\) es satisfà~\(A\in\Algebra{A}_{i}\), i per la definició~d'\myref{def:àlgebra} trobem que~\(A^{\complement}\in\Algebra{A}_{i}\) per a tot~\(i\in I\), i per la definició d'\myref{def:intersecció de conjunts} tenim que es satisfà~\(A^{\complement}\in\Algebra{A}\).
			
			Per acabar premen~\(A\),~\(B\in\Algebra{A}\). Tenim que~\(A\),~\(B\in\Algebra{A}_{i}\) per a tot~\(i\in I\), i per la definició~d'\myref{def:àlgebra} trobem que~\(A\cup B\in\Algebra{A}_{i}\) per a tot~\(i\in I\), i per la definició~d'\myref{def:intersecció de conjunts} trobem que~\(A\cup B\in\Algebra{A}\).
			
			Per tant per la definició~d'\myref{def:àlgebra} tenim que~\(\Algebra{A}\) és un àlgebra sobre~\(\Omega\).
			
			Prenem ara una família~\(\{A_{n}\}_{n\in\mathbb{N}}\) d'elements~d'\(\Algebra{A}\) i denotem
			\[A=\bigcup_{n=0}^{\infty}A_{n}.\]
			Tenim de nou que~\(\{A_{n}\}_{n\in\mathbb{N}}\) també és una família d'elements~d'\(\Algebra{A}_{i}\) per a tot~\(i\in I\), i per la definició~d'\myref{def:sigma àlgebra} tenim que~\(A\in\Algebra{A}_{i}\) per a tot~\(i\in I\), i per la definició~d'\myref{def:intersecció de conjunts} trobem que~\(A\in\Algebra{A}\), i per la definició~d'\myref{def:sigma àlgebra} trobem que~\(\Algebra{A}\) és una~\(\sigma\)-àlgebra sobre~\(\Omega\).
		\end{proof}
	\end{proposition}
	\begin{definition}[\ensuremath{\sigma}-àlgebra generada per un conjunt]
		\labelname{\ensuremath{\sigma}-àlgebra generada per un conjunt}\label{def:sigma àlgebra generada per un conjunt}
		Sigui~\(\Algebra{C}\subseteq\parts(\Omega)\) una família de subconjunts d'un conjunt~\(\Omega\). Denotarem el mínim~\(\sigma\)-àlgebra generada per~\(\Algebra{C}\) com~\(\sigma(\Algebra{C})\) i direm que és la~\(\sigma\)-àlgebra generada per~\(\Algebra{C}\).
		
		Aquesta definició té sentit per la proposició~\myref{prop:existeix una sigma àlgebra mínima}.
	\end{definition}
	\begin{example}[\ensuremath{\sigma}-àlgebra de singletons]
		\labelname{\ensuremath{\sigma}-àlgebra de singletons}\label{ex:sigma àlgebra de singletons}
		Denotem~\(\Algebra{C}=\{\{x\}\subseteq\RR\mid x\in\RR\}\). Volem veure que
		\[\sigma(\Algebra{C})=\{A\subseteq\RR\mid A\text{ ó }A^{\complement}\text{ és finit o numerable}\}.\]
		\begin{solution}
			%TODO % Fer a topo i sorprendre als nens amb això. O al revès potser molaria més. Fer aquí i referenciar a topo com un boss. Per allà on axiomes de separació pot ser útil.
		\end{solution}
	\end{example}
	\begin{definition}[\(\sigma\)-àlgebra de Borel]
		\labelname{\ensuremath{\sigma}-àlgebra}\label{def:sigma àlgebra de Borel}
		Sigui
		\[\borel(\RR)=\{\obert{U}\subseteq\RR\mid\obert{U}\text{ és un obert de }\RR\}.\]
		Direm que~\(\borel(\RR)\) és la~\(\sigma\)-àlgebra de Borel. Direm que els elements de~\(\borel(\RR)\) són conjunts borelians.
	\end{definition}
\subsection{Probabilitats}
	\begin{definition}[Probabilitat]
		\labelname{probabilitat}\label{def:probabilitat}
		Siguin~\(\Algebra{A}\) una~\(\sigma\)-àlgebra sobre un conjunt~\(\Omega\) i~\(\Probabilitat\colon\Algebra{A}\longrightarrow[0,1]\) una aplicació tal que
		\begin{enumerate}
			\item \(\Probabilitat(\Omega)=1\).
			\item Per a tota família~\(\{A_{n}\}_{n\in\NN}\subseteq\Algebra{A}\) d'esdeveniments disjunts dos a dos tenim que
			\begin{equation}
				\Probabilitat\Big(\bigcup_{n=0}^{\infty}A_{n}\Big)=\sum_{n=0}^{\infty}\Probabilitat(A_{n}).
				\tag{\(\sigma\)-additivitat}
			\end{equation}
		\end{enumerate}
		Aleshores direm que~\(\Probabilitat\) és una probabilitat.
		
		També direm que~\((\Omega,\Algebra{A},\Probabilitat)\) és un espai de probabilitat.
	\end{definition}
	\begin{notation}[Unió disjunta]
		Sigui~\(\{A_{n}\}_{n\in\NN}\) una família de conjunts disjunts dos a dos. Aleshores denotarem
		\[\bigcup_{n=0}^{\infty}A_{n}=\biguplus_{n=0}^{\infty}A_{n}\]
		i direm que és la unió disjunta.
	\end{notation}
	\begin{proposition}
		\label{prop:la probabilitat del buit és nulla}
		Sigui~\(\Probabilitat\) una probabilitat. Aleshores~\(\Probabilitat(\emptyset)=0\).
		\begin{proof}
			Considerem la família~\(\{A_{n}\}_{n\in\NN}\) amb~\(A_{n}=\emptyset\) per a tot~\(n\in\NN\). Tenim que per a tot~\(i\),~\(j\in\NN\) es satisfà~\(A_{i}\cap A_{j}=\emptyset\) i
			\[\emptyset=\biguplus_{n=0}^{\infty}\emptyset.\]
			Aleshores per la definició de~\myref{def:probabilitat} tenim que
			\[\Probabilitat(\emptyset)=\sum_{n=0}^{\infty}\Probabilitat(\emptyset),\]
			i obtenim que~\(\Probabilitat(\emptyset)=0\).
		\end{proof}
	\end{proposition}
	\begin{proposition}
		\label{prop:les funcions probabilitat són creixents monòtonament}
		Siguin~\(\Probabilitat\) una probabilitat sobre una~\(\sigma\)-àlgebra~\(\Algebra{A}\) i~\(A\),~\(B\in\Algebra{A}\) dos esdeveniments tals que~\(A\subseteq B\). Aleshores
		\[\Probabilitat(A)\leq\Probabilitat(B).\]
		\begin{proof}
			Tenim que
			\[B=A\uplus(B\setminus A).\]
			Aleshores per la definició de~\myref{def:probabilitat} trobem que
			\[\Probabilitat(B)=\Probabilitat(A\uplus(B\setminus A))=\Probabilitat(A)+\Probabilitat(B\setminus A),\]
			i com que, per la definició de~\myref{def:probabilitat}, tenim que~\(\Probabilitat(B\setminus A)\geq0\), trobem que
			\[\Probabilitat(A)\leq\Probabilitat(B).\qedhere\]
		\end{proof}
	\end{proposition}
	\begin{proposition}
		\label{prop:probabilitat de la resta de esdeveniments}
		Siguin~\(\Probabilitat\) una probabilitat sobre una~\(\sigma\)-àlgebra~\(\Algebra{A}\) i~\(A\),~\(B\in\Algebra{A}\) dos esdeveniments tals que~\(A\subseteq B\). Aleshores
		\[\Probabilitat(B\setminus A)=\Probabilitat(B)-\Probabilitat(A).\]
		\begin{proof}
			Tenim que
			\[B=A\uplus(B\setminus A).\]
			Aleshores per la definició de~\myref{def:probabilitat} trobem que
			\[\Probabilitat(B)=\Probabilitat(A\uplus(B\setminus A))=\Probabilitat(A)+\Probabilitat(B\setminus A),\]
			i trobem que
			\[\Probabilitat(B\setminus A)=\Probabilitat(B)-\Probabilitat(A).\]
		\end{proof}
	\end{proposition}
	\begin{corollary}
		\label{cor:probabilitat del complementari}
		Siguin~\(\Probabilitat\) una probabilitat sobre una~\(\sigma\)-àlgebra~\(\Algebra{A}\) i~\(A\in\Algebra{A}\) un esdeveniment. Aleshores
		\[\Probabilitat(A^{\complement})=1-\Probabilitat(A).\qedhere\]
	\end{corollary}
	\begin{proposition}
		Siguin~\(\Probabilitat\) una probabilitat sobre una~\(\sigma\)-àlgebra~\(\Algebra{A}\) i~\(\{A_{i}\}_{i=1}^{n}\subseteq\Algebra{A}\) una família d'esdeveniments. Aleshores
		\begin{multline*}
			\Probabilitat\Big(\bigcup_{i=1}^{n}A_{i}\Big)=\sum_{i=1}^{n}\Probabilitat(A_{i})-\sum_{i=1}^{n}\sum_{j=i}^{n}\Probabilitat(A_{i}\cap A_{j})+\\
			+\sum_{i=1}^{n}\sum_{j=i}^{n}\sum_{k=j}^{n}\Probabilitat(A_{i}\cap A_{j}\cap A_{k})+\dots+(-1)^{n+1}\Probabilitat\Big(\bigcap_{i=1}^{n}A_{i}\Big).
		\end{multline*}
		\begin{proof}
			Ho veiem per inducció sobre~\(n\). Si~\(n=1\) tenim que
			\[\Probabilitat(A)=\Probabilitat(A).\]
			Suposem doncs que la hipòtesi és certa per a~\(m-1\) fixe i veiem-ho pel cas~\(m\). Tenim que
			\begin{align*}
				\Probabilitat\Big(\bigcup_{i=1}^{m}A_{i}\Big)&=\Probabilitat\Big(A_{m} \cup\bigcup_{i=1}^{m-1}A_{i}\Big) \\
				&=\Probabilitat(A_{m})+\Probabilitat\Big(\bigcup_{i=1}^{m-1}A_{i}\Big)-\Probabilitat\Big(A_{m}\cap\bigcup_{i=1}^{m-1}A_{i}\Big) \\
				&=\Probabilitat(A_{m})+\Probabilitat\Big(\bigcup_{i=1}^{m-1}A_{i}\Big)-\Probabilitat\Big(\bigcup_{i=1}^{m-1}(A_{i}\cap A_{m})\Big) \\
				&=\sum_{i=1}^{m-1}\Probabilitat(A_{i})+\Probabilitat(A_{m})-\sum_{i=1}^{m-1}\sum_{j=1}^{m-1}\Probabilitat(A_{i}\cap A_{j})-\sum_{i=1}^{m-1}\Probabilitat(A_{i}\cap A_{m}) \\
				&\phantom{=}+\sum_{i=1}^{m-1}\sum_{j=i}^{m-1}\sum_{k=j}^{m-1}\Probabilitat(A_{i}\cap A_{j}\cap A_{k})+\sum_{i=1}^{m-1}\sum_{j=i}^{m-1}\Probabilitat(A_{i}\cap A_{j}\cap A_{m})+ \\
				&\phantom{=}+\dots+(-1)^{m}\Probabilitat(A_{1}\cap\dots\cap A_{m-1})+ \\
				&\phantom{=}+(-1)^{m}\sum_{i=1}^{m-2}P(A_{1}\cap\dots\cap A_{i-1}\cap A_{i+1}\cap\dots\cap A_{m-2})+ \\
				&\phantom{=}+(-1)^{m}\Probabilitat(A_{1}\cap\dots\cap A_{m}) \\
				&=\sum_{i=1}^{m}\Probabilitat(A_{i})-\sum_{i=1}^{m}\sum_{j=i}^{m}\Probabilitat(A_{i}\cap A_{j})+\dots+(-1)^{m+1}\Probabilitat\Big(\bigcap_{i=1}^{m}A_{i}\Big),
			\end{align*}
			i pel \myref{thm:principi d'inducció} hem acabat.
		\end{proof}
	\end{proposition}
	\begin{corollary}
		\label{cor:subadditivitat finita}
		Siguin~\(\Probabilitat\) una probabilitat sobre una~\(\sigma\)-àlgebra~\(\Algebra{A}\) i~\(\{A_{i}\}_{i=1}^{n}\subseteq\Algebra{A}\) una família d'esdeveniments. Aleshores
		\[\Probabilitat\Big(\bigcup_{i=1}^{n}A_{i}\Big)\leq\sum_{i=1}^{n}\Probabilitat(A_{i}).\]
	\end{corollary}
	\begin{theorem}[Definició de Laplace]
		\labelname{}\label{thm:definició de Laplace}
		Sigui~\((\Omega,\Algebra{A},\Probabilitat)\) un espai de probabilitat amb~\(\Omega=\{\omega_{1},\dots,\omega_{n}\}\) i~\(\Algebra{A}=\parts(\Omega)\) tals que per a tot~\(i\),~\(j\in\{1,\dots,n\}\) tenim que~\(\Probabilitat(\{\omega_{i}\})=\Probabilitat(\{\omega_{j}\})\). Aleshores per a tot esdeveniment~\(A\in\Algebra{A}\) tenim
		\[\Probabilitat(A)=\frac{\abs{A}}{\abs{\Omega}}.\]
		\begin{proof}
			Observem que~\(\abs{\Omega}=1\), i que
			\[\Omega=\biguplus_{i=1}^{n}\{\omega_{i}\}\]
			Per tant tenim que
			\[1=\Probabilitat(\Omega)=\Probabilitat\Big(\bigcup_{i=1}^{n}\{\omega_{i}\}\Big)=\sum_{i=1}^{n}\Probabilitat(\{\omega_{i}\}),\]
			i com que, per hipòtesi, per a tot~\(i\),~\(j\in\{1,\dots,n\}\) tenim que~\(\Probabilitat(\{\omega_{i}\})=\Probabilitat(\{\omega_{j}\})\), trobem que per a tot~\(i\in\{1,\dots,n\}\) es satisfà
			\[1=\sum_{i=1}^{n}\Probabilitat(\{\omega_{i}\})=n\Probabilitat(\{\omega_{i}\}),\]
			i per tant, per a tot~\(i\in\{1,\dots,n\}\) tenim
			\[\Probabilitat(\{\omega_{i}\})=\frac{1}{n}.\]
			Aleshores, si~\(A=\{\omega_{i_{1}},\dots,\omega_{i_{r}}\}\) tenim que~\(\abs{A}=r\) i que
			\[A=\biguplus_{j=1}^{r}\{\omega_{i_{j}}\},\]
			i per tant
			\begin{align*}
				\Probabilitat(A)&=\Probabilitat\Big(\biguplus_{j=1}^{r}\{\omega_{i_{j}}\}\Big) \\
				&=\sum_{j=1}^{r}\Probabilitat(\{\omega_{i_{j}}\}) \\
				&=\frac{r}{n}=\frac{\abs{A}}{\abs{\Omega}}.\qedhere
			\end{align*}
%			\[\Probabilitat(A)=\Probabilitat\Big(\biguplus_{j=1}^{r}\{\omega_{i_{j}}\}\Big)=\sum_{j=1}^{r}\Probabilitat(\{\omega_{i_{j}}\})=\frac{r}{n}=\frac{\abs{A}}{\abs{\Omega}}.\qedhere\]
		\end{proof}
	\end{theorem}
	\begin{proposition}[Continuïtat seqüencial per successions creixents]
		\labelname{continuïtat seqüencial per successions creixents}\label{prop:continuitat sequencial per successions creixents}
		Siguin~\((\Omega,\Algebra{A},\Probabilitat)\) un espai de probabilitat i~\(\{A_{n}\}_{n\in\NN}\subseteq\Algebra{A}\) una família d'esdeveniments tals que
		\[A_{0}\subseteq A_{1}\subseteq\dots\subseteq A_{n}\subseteq\cdots\]
		amb
		\[A=\bigcup_{n=0}^{\infty}A_{n}.\]
		Aleshores
		\[\lim_{n\to\infty}\Probabilitat(A_{n})=\Probabilitat(A).\]
		\begin{proof}
			Considerem els conjunts
			\begin{equation}
				\label{prop:continuitat sequencial per successions creixents:eq1}
				B_{n}=A_{n}\cap A_{n-1}^{\complement}=A_{n}\setminus A_{n-1},\qquad\text{amb}\qquad B_{0}=\emptyset.
			\end{equation}
			Aleshores tenim que % REVISAR
			\[\bigcup_{n=0}^{\infty}A{n}=\biguplus_{n=0}^{\infty}B_{n}.\]
			Així veiem que
			\begin{align*}
				\Probabilitat(A)&=\Probabilitat\Big(\bigcup_{n=0}^{\infty}A_{n}\Big) \\
				&=\Probabilitat\Big(\biguplus_{n=0}^{\infty}B_{n}\Big) \\
				&=\sum_{n=0}^{\infty}\Probabilitat(B_{n}) \\
				&=\sum_{n=0}^{\infty}\Probabilitat(A_{n}\setminus A_{n-1}) \tag{\ref{prop:continuitat sequencial per successions creixents:eq1}}\\
				&=\sum_{n=0}^{\infty}\big(\Probabilitat(A_{n})-\Probabilitat(A_{n-1})\big)\tag{\ref{prop:probabilitat de la resta de esdeveniments}} \\
				&=\lim_{n\to\infty}\Probabilitat(A_{n})-\Probabilitat(A_{0}) \\
				&=\lim_{n\to\infty}\Probabilitat(A_{n})-\Probabilitat(\emptyset)=\lim_{n\to\infty}\Probabilitat(A_{n}), \tag{\ref{prop:la probabilitat del buit és nulla}}
			\end{align*}
			com volíem veure.
		\end{proof}
	\end{proposition}
	\begin{proposition}[Continuïtat seqüencial per successions decreixents]
		\label{prop:continuitat sequencial per successions decreixents}
		Siguin~\((\Omega,\Algebra{A},\Probabilitat)\) un espai de probabilitat i~\(\{A_{n}\}_{n\in\NN}\subseteq\Algebra{A}\) una família d'esdeveniments tals que
		\[A_{0}\supseteq A_{1}\supseteq\dots\supseteq A_{n}\supseteq\cdots\]
		amb
		\[A=\bigcap_{n=0}^{\infty}A_{n}.\]
		Aleshores
		\[\lim_{n\to\infty}\Probabilitat(A_{n})=\Probabilitat(A).\]
		\begin{proof}
			Considerem les conjunts
			\[B_{n}=A_{n}^{\complement}.\]
			Aleshores tenim que
			\[B_{0}\subseteq B_{1}\subseteq\dots\subseteq B_{n}\subseteq\cdots,\]
			i per la proposició de \myref{prop:continuitat sequencial per successions creixents} trobem que
			\[\lim_{n\to\infty}\Probabilitat(B_{n})=\Probabilitat\Big(\bigcup_{n=0}^{\infty}B_{n}\Big).\]
			Ara bé, tenim que
			\begin{align*}
				\Probabilitat(A)&=\Probabilitat\bigg(\bigcap_{n=0}^{\infty}A_{n}\bigg) \\
				&=\Probabilitat\bigg(\bigcap_{n=0}^{\infty}B_{n}^{\complement}\bigg) \\
				&=\Probabilitat\bigg(\Big(\bigcup_{n=0}^{\infty}B_{n}\Big)^{\complement}\bigg) \\
				&=1-\Probabilitat\Big(\bigcup_{n=0}^{\infty}B_{n}\Big) \\
				&=1-\lim_{n\to\infty}\Probabilitat(B_{n}) \\
				&=1-1+\lim_{n\to\infty}\Probabilitat(A_{n})=\lim_{n\to\infty}\Probabilitat(A_{n}).\qedhere
			\end{align*}
		\end{proof}
	\end{proposition}
	%TODO Exemples
\section{Probabilitat condicionada i independència}
\subsection{Probabilitat condicionada}
	\begin{definition}[Probabilitat condicionada]
		\labelname{probabilitat condicionada}\label{def:probabilitat condicionada}
		Siguin~\(A\),~\(B\) dos esdeveniments amb~\(\Probabilitat(A)\neq0\). Aleshores definim
		\[\Probabilitat(B\sota A)=\frac{\Probabilitat(A\cap B)}{\Probabilitat(A)}\]
		com la probabilitat condicionada de~\(B\) sobre~\(A\), o la probabilitat de~\(B\) donat~\(A\).
	\end{definition}
	\begin{observation}[Regla de les probabilitats compostes]
		\labelname{regla de les probabilitats compostes}\label{obs:regla de les probabilitats compostes}
		\[\Probabilitat(A\cap B)=\Probabilitat(A)\Probabilitat(B\sota A).\]
	\end{observation}
	\begin{proposition}
		\label{prop:la probabilitat condicionada és una probabilitat}
		Siguin~\((\Omega,\Algebra{A},\Probabilitat)\) un espai de probabilitat i~\(A\in\Algebra{A}\) un esdeveniment amb~\(\Probabilitat(A)\neq0\). Aleshores l'aplicació
		\begin{align*}
			\Probabilitat(\cdot\sota A)\colon\algebra{A}&\longrightarrow[0,1] \\
			B&\longmapsto\Probabilitat(B\sota A)
		\end{align*}
		és una probabilitat.
		\begin{proposition}
			%TODO
		\end{proposition}
	\end{proposition}
	\begin{corollary}
		\label{cor:probabilitat condicionada del complementari}
		\label{cor:la probabilitat condicionada és una probabilitat}
		Sigui~\(A\) un esdeveniment amb~\(\Probabilitat(A)\neq0\). Aleshores
		\[\Probabilitat(C^{\complement}\sota A)=1-\Probabilitat(C\sota A).\]
	\end{corollary}
	\begin{example}
		%TODO
	\end{example}
	\begin{theorem}[Fórmula de les probabilitats compostes]
		\labelname{fórmula de les probabilitats compostes}\label{thm:fórmula de les probabilitats compostes}
		Siguin~\(A_{1},\dots,A_{n}\) esdeveniments tals que~\(\Probabilitat(A_{1}\cap\dots\cap A_{n-1})\neq0\). Aleshores
		\[\Probabilitat(A_{1}\cap\dots\cap A_{n})=\Probabilitat(A)\Probabilitat(A_{2}\sota A_{1})\Probabilitat(A_{3}\sota A_{2}\cap A_{1})\dots\Probabilitat(A_{n}\sota A_{n-1}\cap\dots\cap A_{1}).\]
		\begin{proof}
			%TODO
		\end{proof}
	\end{theorem}
	\begin{example}%[Problema de Chicago]
		%TODO
	\end{example}
\subsection{La fórmula de les probabilitats totals i la fórmula de Bayes}
	\begin{theorem}[Teorema de les probabilitats totals]
		\labelname{Teorema de les probabilitats totals}\label{thm:Teorema de les probabilitats totals}
		Siguin~\((\Omega,\Algebra{A},\Probabilitat)\) un espai de probabilitat,~\(\{A_{n}\}_{n\in I}\subseteq\Algebra{A}\) una família numerable d'esdeveniments tals que
		\[\Omega=\biguplus_{n\in I}A_{n}\]
		amb~\(\Probabilitat(A_{n})\neq0\) per a tot~\(n\in I\) i~\(A\in\Algebra{A}\) un esdeveniment. Aleshores
		\[\Probabilitat(A)=\sum_{n\in I}\Probabilitat(A_{n})\Probabilitat(A\sota A_{n}).\]
		\begin{proof}
			%TODO
		\end{proof}
	\end{theorem}
	\begin{theorem}[Fórmula de Bayes]
		\labelname{fórmula de Bayes}\label{thm:fórmula de Bayes}
		Siguin~\((\Omega,\Algebra{A},\Probabilitat)\) un espai de probabilitat,~\(\{A_{n}\}_{n=1}^{N}\subseteq\Algebra{A}\) una família d'esdeveniments tals que
		\[\Omega=\biguplus_{n\in I}A_{n}\]
		amb~\(\Probabilitat(A_{n})\neq0\) per a tot~\(n\in\{1,\dots,N\}\) i~\(A\in\Algebra{A}\) un esdeveniment tal que~\(\Probabilitat(A)\neq0\). Aleshores per a tot~\(j\in\{1,\dots,N\}\) tenim
		\[\Probabilitat(A_{j}\sota A)=\frac{\Probabilitat(A_{j})\Probabilitat(A\sota A_{j})}{\sum_{n=1}^{N}\Probabilitat(A_{n})\Probabilitat(A\sota A_{n})}.\]
		\begin{proof}
			%TODO
		\end{proof}
	\end{theorem}
	\begin{example}
		%TODO
	\end{example}
\subsection{Independència d'esdeveniments}
	\begin{definition}[Esdeveniments independents]
		\labelname{esdeveniments independents}\label{def:esdeveniments independents}
		Siguin~\(A\) i~\(B\) dos esdeveniments tals que
		\[\Probabilitat(A\cap B)=\Probabilitat(A)\Probabilitat(B).\]
		Aleshores direm que els esdeveniments~\(A\) i~\(B\) són independents.
	\end{definition}
	\begin{observation}
		\label{obs:condició equivalent a esdeveniments independents}
		Siguin~\(A\) i~\(B\) dos esdeveniments. Aleshores~\(A\) i~\(B\) són independents si i només si~\(\Probabilitat(B\sota A)=\Probabilitat(B)\).
		\begin{proof}
			%TODO
		\end{proof}
	\end{observation}
	\begin{example}
		%TODO
	\end{example}
	\begin{proposition}
		\label{prop:el buit i el total són esdeveniments independents de la resta}
		\label{prop:el buit és un esdeveniment independent de la resta}
		\label{prop:el total és un esdeveniment independent de la resta}
		Sigui~\((\Omega,\Algebra{A},\Probabilitat)\) un espai de probabilitat. Aleshores per a tot esdeveniment~\(A\in\Algebra{A}\) tenim que~\(\Omega\) i~\(A\) són esdeveniments independents i~\(\emptyset\) i~\(A\) són esdeveniments independents.
		\begin{proof}
			%TODO
		\end{proof}
	\end{proposition}
	\begin{proposition}
		\label{prop:el complementari d'un esdeveniment conserva la independència}
		Siguin~\(A\) i~\(B\) dos esdeveniments. Aleshores són equivalents
		\begin{enumerate}
			\item \(A\) i~\(B\) són esdeveniments independents.
			\item \(A^{\complement}\) i~\(B\) són esdeveniments independents.
			\item \(A\) i~\(B^{\complement}\) són esdeveniments independents.
			\item \(A^{\complement}\) i~\(B^{\complement}\) són esdeveniments independents.
		\end{enumerate}
		\begin{proof}
			%TODO
		\end{proof}
	\end{proposition}
	\begin{definition}[Esdeveniments independents]
		\labelname{esdeveniments independents}\label{def:esdeveniments independents cas general}
		Sigui~\(\{A_{i}\}_{i\in I}\) una família d'esdeveniments tals que per a qualsevol~\(i_{1},\dots,i_{k}\in I\) diferents dos a dos tenim que
		\[\Probabilitat(\bigcap_{j=1}^{k}A_{i_{j}})=\prod_{j=1}^{k}\Probabilitat(A_{i_{j}}).\]
		aleshores direm que els esdeveniments~\(A_{1},\dots,A_{n}\) són independents.
	\end{definition}
\chapter{Variables i vectors aleatoris}
\section{Distribució i funció de distribució d'una variable aleatòria}
\subsection{Variables aleatòries}
	\begin{definition}[Variable aleatòria]
		\labelname{variable aleatòria}\label{def:variable aleatòria}
		Siguin~\(\Algebra{A}\) una~\(\sigma\)-àlgebra sobre un conjunt~\(\Omega\) i~\(X\colon\Omega\longrightarrow\RR\) una aplicació tal que per a tot conjunt~\(B\in\borel(\RR)\) tenim
		\[X^{-1}(B)\in\Algebra{A}.\]
		Aleshores direm que~\(X\) és una variable aleatòria sobre~\(\Algebra{A}\).
	\end{definition}
	\begin{example}
		%TODO
	\end{example}
	\begin{notation}
		\label{not:variables aleatòries}
		Sigui~\(X\colon\Omega\longrightarrow\RR\) una variable aleatòria sobre un~\(\sigma\)-àlgebra~\(\Algebra{A}\). Aleshores denotem
		\begin{enumerate}
			\item \(\{X\in B\}=\{\omega\in\Omega\mid X(\omega)\in B\}\).
			\item \(\{X=a\}=\{X\in\{a\}\}\).
			\item \(\{a< X<b\}=\{X\in(a,b)\}\).
			\item \(\{a\leq X<b\}=\{X\in[a,b)\}\).
			\item \(\{a< X\leq b\}=\{X\in(a,b]\}\).
			\item \(\{a\leq X\leq b\}=\{X\in[a,b]\}\).
		\end{enumerate}
	\end{notation}
	\begin{proposition}
		\label{prop:condició suficient per ser una variable aleatòria}
		Siguin~\(\Algebra{A}\) una~\(\sigma\)-àlgebra sobre un conjunt~\(\Omega\) i~\(X\colon\Omega\longrightarrow\RR\) una aplicació tal que per a tot~\(x\in\RR\) tenim que
		\[X^{-1}((-\infty,x])\in\Algebra{A}.\]
		Aleshores~\(X\) és una variable aleatòria.
		\begin{proof}
			%TODO
		\end{proof}
	\end{proposition}
	\begin{proposition}
		\label{prop:les variables aleatòries formen un anell}
		Siguin~\(X\) i~\(Y\) dues variables aleatòries i~\(\lambda\in\mathbb{R}\) un escalar. Aleshores~\(X+y\),~\(XY\) i~\(\lambda X\) són variables aleatòries.
		\begin{proof}
			%TODO
		\end{proof}
	\end{proposition}
	\begin{proposition}
		\label{prop:les variables aleatòries formen un cos}
		Sigui~\(X\colon\Omega\longrightarrow\RR\) una variable aleatòria sobre una~\(\sigma\)-àlgebra tal que per a tot~\(\omega\in\Omega\) tenim que~\(X(\omega)\neq0\). Aleshores~\(1/X\) és una variable aleatòria.
		\begin{proof}
			%TODO
		\end{proof}
	\end{proposition}
	\begin{proposition}
		\label{prop:el cos de variables aleatòries és tancat}
		Sigui~\((X_{n})_{n\in\NN}\) una successió de variables aleatòries sobre una~\(\sigma\)-àlgebra sobre un conjunt~\(\Omega\) tals que per a tot~\(\omega\in\Omega\) tenim que la successió~\((X_{n}(\omega))_{n\in\NN}\) és convergent. Aleshores la funció
		\begin{align*}
			X\colon\Omega&\longrightarrow\mathbb{R} \\
			\omega&\longmapsto\lim_{n\to\infty}X_{n}(\omega)
		\end{align*}
		és una variable aleatòria.
		\begin{proof}
			%TODO
		\end{proof}
	\end{proposition}
\subsection{Distribució o llei d'una variable aleatòria}
	\begin{proposition}
		\label{prop:distribució d'una variable aleatòria}
		Siguin~\(X\colon\Omega\longrightarrow\RR\) una variable aleatòria i
		\begin{align*}
			\Probabilitat_{X}\colon\borel(\RR)&\longrightarrow\RR \\
			B&\longmapsto\Probabilitat(X^{-1}(B)).
		\end{align*}
		Aleshores~\((\RR,\borel(\RR),\Probabilitat_{X})\) és un espai de probabilitat.
		\begin{proof}
			%TODO
		\end{proof}
	\end{proposition}
	\begin{definition}[Distribució d'una variable aleatòria]
		\labelname{distribució d'una variable aleatòria}\label{def:distribució d'una variable aleatòria}
		Sigui~\(X\colon\Omega\longrightarrow\RR\) una variable aleatòria. Aleshores direm que la funció
		\begin{align*}
			\Probabilitat_{X}\colon\borel(\RR)&\longrightarrow\RR \\
			B&\longmapsto\Probabilitat(X^{-1}(B))
		\end{align*}
		és la distribució de~\(X\). També direm que és la llei de~\(X\).
	\end{definition}
	\begin{example}
		%TODO
	\end{example}
\subsection{Igualtat i igualtat quasi segura de variables aleatòries}
	\begin{definition}[Igualtat de variables aleatòries]
		\labelname{variables aleatòries amb la mateixa distribució}\label{def:variables aleatòries amb la mateixa distribució}
		\labelname{igualtat de variables aleatòries}\label{def:igualtat de variables aleatòries}
		Siguin~\(X\) i~\(Y\) dues variables aleatòries tals que per a tot~\(B\in\borel(\RR)\) tenim que
		\[\Probabilitat_{X}(B)=\Probabilitat_{Y}(B).\]
		Aleshores direm que~\(X\) i~\(Y\) tenen la mateixa distribució, i ho denotarem com
		\[X\igualsendistribucio Y.\]
	\end{definition}
	\begin{example}
		%TODO
	\end{example}
	\begin{definition}[Igualtat quasi segura de variables aleatòries]
		\labelname{igualtat quasi segura de variables aleatòries}\label{def:igualtat quasi segura de variables aleatòries}
		Siguin~\(X\) i~\(Y\) dues variables aleatòries sobre una~\(\sigma\)-àlgebra~\(\Algebra{A}\) tals que~\(\Probabilitat\{X=Y\}=1\). Aleshores direm que~\(X\) i~\(Y\) són iguals quasi segurament i ho denotarem com
		\[X\qs Y.\]
	\end{definition}
	\begin{observation}
		\label{obs:dues variables iguals quasi segurament són iguals en distribució}
		Siguin~\(X\) i~\(Y\) dues variables aleatòries tals que~\(X\qs Y\). Aleshores~\(X\igualsendistribucio Y\).
	\end{observation}
	\begin{example}
		%TODO
	\end{example}
\subsection{Funció de distribució d'una variable aleatòria}
	\begin{definition}[Funció de distribució]
		\labelname{funció de distribució d'una variable aleatòria}\label{def:funció de distribució d'una variable aleatòria}
		Siguin~\(X\) una variable aleatòria i
		\begin{align*}
			F\colon\RR&\longrightarrow[0,1] \\
			x&\longmapsto\Probabilitat\{X\leq x	\}
		\end{align*}
		una funció. Aleshores direm que~\(F\) és la funció de distribució de~\(X\).
	\end{definition}
	\begin{proposition}
		\label{prop:les funcions de distribució són creixents}
		Sigui~\(F\) una funció de distribució. Aleshores~\(F\) és creixent.
		\begin{proof}
			%TODO
		\end{proof}
	\end{proposition}
	\begin{proposition} %TODO REVISAR
		\label{prop:les funcions de distribució són contínues per la dreta}
		\label{prop:les funcions de distribució tenen límit per l'esquerra}
		Sigui~\(F\) una funció de distribució. Aleshores per a tot~\(x\in\RR\) tenim que
		\[\lim_{\substack{t\to x\\t<x}}F(t)=F(x)\qquad\text{i}\qquad\lim_{\substack{t\to x\\t>x}}F(t)=L\in\RR.\]
		\begin{proof}
			%TODO
		\end{proof}
	\end{proposition}
	\begin{proposition}
		\label{prop:límits laterals d'una funció de distribució}
		Sigui~\(F\) una funció de distribució. Aleshores
		\[\lim_{x\to-\infty}F(x)=0\qquad\text{i}\qquad\lim_{x\to\infty}F(x)=1.\]
		\begin{proof}
			%TODO
		\end{proof}
	\end{proposition}
	\begin{proposition}
		\label{prop:les funcions de distribució tienen com a màxim un nombre numerable de punts de discontinuitat}
		Sigui~\(F\) una funció de distribució. Aleshores el conjunt de punts de discontinuïtat de~\(F\) és com a màxim numerable.
		\begin{proof}
			%TODO % Fer desprès de FVR
		\end{proof}
	\end{proposition}
	\begin{proposition}
		\label{prop:funcions de distribució per calcular variables aleatòries en intervals tancats}
		Sigui~\(F\) una funció de distribució d'una variable aleatòria~\(X\). Aleshores per a tot~\(s\),~\(t\in\RR\) tenim que
		\[\Probabilitat\{s<X\leq t\}=F(t)-F(s).\]
	\end{proposition}
	\begin{proposition}
		\label{prop:funcions de distribució per calcular variables aleatòries en semirectes}
		Sigui~\(F\) una funció de distribució d'una variable aleatòria~\(X\). Aleshores per a tot~\(x\in\RR\) tenim que
		\[\Probabilitat\{X<x\}=\lim_{\substack{t\to x\\t<x}}F(t).\]
	\end{proposition}
	\begin{proposition}
		\label{prop:funcions de distribució per calcular variables aleatòries en punts}
		Sigui~\(F\) una funció de distribució d'una variable aleatòria~\(X\). Aleshores
		\[\Probabilitat\{X=x\}=F(x)-\lim_{\substack{t\to x\\t<x}}F(t).\]
	\end{proposition}
\section{Variables aleatòries discretes}
\subsection{Variables aleatòries discretes}
	\begin{definition}[Variable aleatòria discreta]
		\labelname{variable aleatòria discreta}\label{def:variable aleatòria discreta}
		\labelname{suport d'una variable aleatòria discreta}\label{def:suport d'una variable aleatòria discreta}
		Sigui~\(X\) una variable aleatòria tal que existeix un conjunt finit o numerable~\(S\subseteq\RR\) tal que
		\[\Probabilitat\{X\in S\}=1,\]
		i tal que per a tot~\(x\in S\) tenim
		\[P\{X=x\}\neq0.\]
		Aleshores direm que~\(X\) és una variable aleatòria discreta. També direm que~\(S\) és el suport de~\(X\).
	\end{definition}
	\begin{example}
		%TODO
	\end{example}
	\begin{observation}
		\label{obs:els conjunts finits o numerables són borelians}
		Sigui~\(S\subseteq\RR\) un conjunt finit o numerable. Aleshores~\(S\) és borelià.
		\begin{proof}
			%TODO
		\end{proof}
	\end{observation}
\subsection{Funció de probabilitat d'una variable aleatòria}
	\begin{definition}[Funció de probabilitat]
		\labelname{funció de probabilitat d'una variable aleatòria discreta}\label{def:funció de probabilitat d'una variable aleatòria discreta}
		\labelname{repartiment de massa d'una variable aleatòria discreta}\label{def:repartiment de massa d'una variable aleatòria discreta}
		Siguin~\(X\) una variable aleatòria discreta amb suport~\(S\) i
		\begin{align*}
			p\colon S&\longrightarrow[0,1] \\
			x&\longmapsto\Probabilitat\{X=x\}.
		\end{align*}
		una funció. Aleshores direm que~\(p\) és la funció de probabilitat de~\(X\). També direm que~\(p\) és el repartiment de massa de~\(X\).
	\end{definition}
	\begin{observation}
		\label{prop:la funció de probabilitat d'una variable aleatòria discreta suma 1 en el suport}
		Sigui~\(X\) una variable aleatòria discreta amb suport~\(S\) i funció de probabilitat~\(p\). Aleshores
		\[\sum_{x\in S}p(s)=1.\]
		\begin{proof}
			%TODO
		\end{proof}
	\end{observation}
	\begin{proposition}
		\label{prop:probabilitat d'un conjunt en una variable aleatòria discreta}
		Siguin~\(X\) una variable aleatòria discreta amb suport~\(S\) i funció de probabilitat~\(p\), i~\(B\in\borel(\RR)\) un borelià. Aleshores
		\[\Probabilitat\{X=B\}=\sum_{\substack{x\in S\\x\in B}}p(x).\]
		\begin{proof}
			%TODO
		\end{proof}
	\end{proposition}
	%Potser caldrà reescriure aquestes coses per fer-les formals. I anar-les arrosegant per fer les solucions poc a poc? Però aleshores es pot fer molt llarg. Decidiré quan em posi a fer-les.
	\begin{example}[Variable aleatòria degenerada]
		\labelname{variable aleatòria degenerada}\label{ex:variable aleatòria degenerada}
		Siguin~\((\Omega,\Algebra{A},\Probabilitat)\) un espai de probabilitat i~\(a\in\RR\) un real. Volem veure que la funció
		\begin{align*}
			X\colon\Omega&\longrightarrow\RR \\
			x&\longmapsto a
		\end{align*}
		és una variable aleatòria. També volem calcular i dibuixar la seva funció de probabilitat i de distribució.
		\begin{solution}
			%TODO
		\end{solution}
	\end{example}
	\begin{example}[Llei de Bernoulli]
		\labelname{llei de Bernoulli}\label{def:llei de Bernoulli}
		Sigui~\(X\) una variable aleatòria discreta amb suport~\(S=\{0,1\}\). Volem calcular i dibuixar la seva funció de probabilitat i de distribució.
		\begin{solution}
			%TODO
		\end{solution}
	\end{example}
	\begin{example}[Llei uniforme sobre~\(n\) punts]
		\labelname{llei uniforme sobre~\(n\) punts}\label{def:llei uniforme sobre n punts}
		Sigui~\(X\) una variable aleatòria discreta amb suport~\(S=\{x_{1},\dots,x_{n}\}\) tal que per a tot~\(i\in\{1,\dots,n\}\) tenim que
		\[P\{X=x_{i}\}=\frac{1}{n}.\]
		Aleshores escriurem~\(X\sim\Unif(x_{1},\dots,x_{n})\). Volem calcular i dibuixar la seva funció de probabilitat i de distribució.
		\begin{solution}
			%TODO
		\end{solution}
	\end{example}
	% Fer la resta de distribucions/lleis
	%TODO Binomial
	%TODO Poisson
	\begin{example}
		%Aplicació d'alguna de les lleis
	\end{example}
\section{Variables aleatòries absolutament contínues} % i mixtes
\subsection{Funcions de densitat i de distribució}
	\begin{definition}[Variable aleatòria absolutament contínua]
		\labelname{variable aleatòria absolutament contínua}\label{def:variable aleatòria absolutament contínua}
		\labelname{funció de densitat d'una variable aleatòria absolutament contínua}\label{def:funció de densitat d'una variable aleatòria abolsutament contínua}
		\labelname{funció de distribució d'una variable aleatòria absolutament contínua}\label{def:funció de distribució d'una variable aleatòria absolutament contínua}
		Sigui~\(X\) una variable aleatòria tal que existeix una funció~\(f\colon\RR\longleftarrow\RR\) satisfent
		\begin{enumerate}
			\item Per a tot~\(x\in\RR\) tenim que~\(f(x)\geq0\).
			\item La funció~\(f\) és integrable sobre tot~\(RR\).
			\item \(\int_{-\infty}^{\infty}f(x)dx=1\).
			\item Per a tot~\(a\),~\(b\in\RR\cup\{\pm\infty\}\) tenim que
			\[\Probabilitat\{a\leq X\leq b\}=\int_{a}^{b}f(x)dx.\]
		\end{enumerate}
		Aleshores direm que~\(X\) és una variable aleatòria absolutament contínua. També direm que~\(f\) és la funció densitat de~\(X\) i que la funció
		\[F(x)=\Probabilitat\{X=x\}=\int_{-\infty}^{x}f(t)dt\]
		és la funció de distribució de~\(X\).
	\end{definition}
	\begin{observation}
		\label{obs:la funció distribució d'una variable aleatòria absolutament contínua és contínua}
		Sigui~\(F\) la funció de distribució d'una variable aleatòria absolutament contínua~\(X\). Aleshores~\(F\) és contínua. % i absolutament contínua
	\end{observation}
%	\begin{example}
%		%TODO % Veure si ho ha un exemple previ a les lleis i distribucions
%	\end{example}
	\begin{example}[Llei uniforme]	
	\labelname{llei uniforme}\label{ex:llei uniforme}
		Siguin~\(a\),~\(b\in\RR\) dos reals amb~\(a<b\) i~\(X\) una variable aleatòria tal que la seva funció de densitat sigui
		\begin{align*}
			f(x)=\frac{1}{b-a}\ind_{(a,b)}(x).
		\end{align*}
		Volem calcular la seva funció de distribució i dibuixar-les.
		
		Direm que~\(X\sim\Unif(a,b)\).
		\begin{solution}
			%TODO
		\end{solution}
	\end{example}
	\begin{example}[Llei exponencial]
		\labelname{llei exponencial}\label{ex:llei exponencial}
		Siguin~\(\lambda\in\RR\) un real amb~\(\lambda>0\) i~\(X\) una variable aleatòria tal que la seva funció de densitat sigui
		\begin{align*}
			f(x)=\lambda\e^{-\lambda x}\ind_{(0,\infty)}(x).
		\end{align*}
		Volem calcular la seva funció de distribució i dibuixar-les.
		
		Direm que~\(X\sim\Exp(\lambda)\).
		\begin{solution}
			%TODO
		\end{solution}
	\end{example}
	% Propietat de la falta de memòria
	\begin{example}[Distribució gaussiana]
		\labelname{distribució gaussiana}\label{ex:distribució gaussiana}
		Siguin~\(\mu\),~\(\sigma\in\RR\) un real amb~\(\sigma>0\) i~\(Z\) una variable aleatòria tal que la seva funció de densitat sigui
		\begin{align*}
			f(x)=\frac{1}{\sigma\sqrt{2\pi}}\e^{-\frac{(x-\mu)^{2}}{2\sigma^{2}}}.
		\end{align*}
		Volem comprovar que aquesta és una funció de densitat i dibuixar-la per~\(\mu=2\) i~\(\sigma=1\).
		
		Direm que~\(Z\sim\Gauss(\mu,\sigma^{2})\).
		\begin{solution}
			%TODO
		\end{solution}
	\end{example}
	\begin{example}[Distribució Gamma]
		\labelname{distribució Gamma}\label{ex:distribució Gamma}
		Siguin~\(r\),~\(\alpha\in\RR\) dos reals amb~\(r>0\) i~\(\alpha>0\). Volem veure que existeix una variable aleatòria~\(X\) amb funció de distribució
		\[f(x)=\frac{\alpha^{r}}{\Gamma(r)}x^{r-1}\e^{-\alpha x}\ind_{(0,\infty)}(x).\]
		També volem dibuixar~\(f(x)\). % Determinar valors guays.
		
		Sigui una variable aleatòria~\(X\) amb funció de distribució~\(f(x)\). Aleshores direm que~\(X\sim\Gamm(r,\alpha)\).
		\begin{solution}
			%TODO
		\end{solution}
	\end{example}
%	\begin{example}[Distribució beta]
%		\labelname{distribució beta}\label{ex:distribució beta}
%	\end{example}
\subsection{Criteri per calcular densitats de variables aleatòries}
	\begin{proposition}
		\label{prop:criteri per calcular densitats de variables aleatòries}
		Sigui~\(X\) una variable aleatòria amb funció de distribució~\(F\colon\RR\longrightarrow\RR\) tal que
		\begin{enumerate}
			\item La funció~\(F\) és contínua~\(\RR\).
			\item La funció~\(F\) és derivable en tot~\(\RR\) excepte en una quantitat finita numerable de punts.
			\item La funció~\(F'\) és contínua en tot~\(\RR\) excepte en una quantitat finita de punts.
		\end{enumerate}
		Aleshores per a tot~\(x\in\RR\) tenim que
		\[F(x)=\int_{-\infty}^{x}F'(t)\diff t.\]
		\begin{proof}
			%TODO
		\end{proof}
	\end{proposition}
	\begin{example}
		%TODO
	\end{example}
%\subsection{Funcions de variables aleatòries amb densitat}
	% Exercicis xungos del final (pàg. 59)
\section{Vectors aleatoris}
\subsection{Vectors de variables aleatòries}
%\chapter{Esperança matemàtica}
%\chapter{Convergència de successions de variables aleatòries}
%\chapter{Els teoremes límit de la Teoria de la Probabilitat}
\end{document}
