\documentclass[../Apunts.tex]{subfiles}

\begin{document}
\part{Anàlisi complexa i de Fourier}
\chapter{Els nombres complexos}
\section{Estructura de \ensuremath{\mathbb{C}}}
	\subsection{Cos de nombres complexos}
	\begin{notation}[Conjunt de nombres complexos]
		\label{notation:cos de nombres complexos}
		Denotarem
		\[\mathbb{C}=\{a+b\iu\mid a,b\in\mathbb{R}\}.\]
%		\[\mathbb{C}=\{\alpha+\beta\iu\mid\alpha,\beta\in\mathbb{R}\}.\]
	\end{notation}
	\begin{definition}[Suma de nombres complexos]
		\labelname{suma de nombres complexos}\label{def:suma de nombres complexos}
		Siguin \(a+b\iu\) i \(c+d\iu\) dos nombres complexos. Aleshores definim la seva suma com
		\[(a+b\iu)+(c+d\iu)=(a+c)+(b+d)\iu.\]
	\end{definition}
	\begin{observation}
		\label{obs:els nombres complexos estan tancats per la suma}
		Siguin \(a+b\iu\) i \(c+d\iu\) dos nombres complexos. Aleshores
		\[(a+b\iu+c+d\iu)\in\mathbb{C}\]
	\end{observation}
	\begin{proposition}
		\label{prop:els nombres complexos commuten per la suma}
		Siguin \(a+b\iu\) i \(c+d\iu\) dos nombres complexos. Aleshores
		\[(a+b\iu)+(c+d\iu)=(c+d\iu)+(a+b\iu).\]
		\begin{proof}
			Per la definició de \myref{def:suma de nombres complexos} tenim
			\begin{align*}
				a+b\iu+c+d\iu&=(a+c)+(b+d)\iu \\
				&=(c+a)+(d+b)\iu=c+d\iu+a+b\iu.\qedhere
			\end{align*}
		\end{proof}
	\end{proposition}
	\begin{proposition}
		\label{prop:els nombres complexos són associatius per la suma}
		Siguin \(a+b\iu\), \(c+d\iu\) i \(u+v\iu\) tres nombres complexos. Aleshores
		\[(a+b\iu)+\big((c+d\iu)+(u+v\iu))\big)=\big((a+b\iu)+(c+d\iu)\big)+(u+v\iu).\]
		\begin{proof}
			Per la definició de \myref{def:suma de nombres complexos} tenim
			\begin{align*}
				(a+b\iu)+\big((c+d\iu)+(u+v\iu))\big)&=(a+b\iu)+\big((c+u)+(d+v)\iu\big) \\
				&=\big(a+(c+u)\big)+\big(b+(d+v)\big)\iu \\
				&=\big((a+c)+u\big)+\big((b+d)+v\big)\iu \\
				&=\big((a+c)+(b+d)\iu\big)+(u+v\iu) \\
				&=\big((a+b\iu)+(c+d\iu)\big)+(u+v\iu).\qedhere
			\end{align*}
		\end{proof}
	\end{proposition}
	\begin{proposition}
		\label{prop:element neutre per la suma dels complexos}
		Sigui \(a+b\iu\) un nombre complex. Aleshores
		\[(a+b\iu)+0=a+b\iu.\]
		\begin{proof}
			Per la definició de \myref{def:suma de nombres complexos} tenim
			\[(a+b\iu)+0=(a+0)+(b+0)\iu=a+b\iu.\qedhere\]
		\end{proof}
	\end{proposition}
	\begin{proposition}
		\label{prop:element invers per la suma dels complexos}
		Sigui \(a+b\iu\) un nombre complex. Aleshores
		\[(a+b\iu)+(-a-b\iu)=0.\]
		\begin{proof}
			Per la definició de \myref{def:suma de nombres complexos} tenim
			\[(a+b\iu)+(-a-b\iu)=(a-a)+(b-b)\iu=0.\qedhere\]
		\end{proof}
	\end{proposition}
	\begin{definition}[Producte de nombres complexos]
		\labelname{producte de nombres complexos}\label{def:producte de nombres complexos}
		Siguin \(a+b\iu\) i \(c+d\iu\) dos nombres complexos. Aleshores definim el seu producte com
		\[(a+b\iu)\cdot(c+d\iu)=(ac-bd)+(ad+bc)\iu.\]
	\end{definition}
	\begin{observation}
		\label{obs:els nombres complexos estan tancats pel producte}
		Siguin \(a+b\iu\) i \(c+d\iu\) dos nombres complexos. Aleshores
		\[(a+b\iu)(c+d\iu)\in\mathbb{C}.\]
	\end{observation}
	\begin{proposition}
		\label{prop:els nombres complexos commuten pel producte}
		Siguin \(a+b\iu\) i \(c+d\iu\) dos nombres complexos. Aleshores
		\[(a+b\iu)(c+d\iu)=(c+d\iu)(a+b\iu).\]
		\begin{proof}
			Per la definició de \myref{def:producte de nombres complexos} tenim
			\begin{align*}
				(a+b\iu)(c+d\iu)&=(ac-bd)+(ad+bc)\iu \\
				&=(ca-bd)+(da+cb)\iu=(c+d\iu)(a+b\iu).\qedhere
			\end{align*}
		\end{proof}
	\end{proposition}
	\begin{proposition}
		\label{prop:els nombres complexos són associatius pel producte}
		Siguin \(a+b\iu\), \(c+d\iu\) i \(u+v\iu\) tres nombres complexos. Aleshores
		\[(a+b\iu)\big((c+d\iu)(u+v\iu))\big)=\big((a+b\iu)(c+d\iu)\big)(u+v\iu).\]
		\begin{proof}
			Prenem \(\alpha=a+b\iu\), \(\beta=c+d\iu\) i \(\gamma=u+v\iu\). Per la definició de \myref{def:producte de nombres complexos} tenim
			\begin{align*}
				\alpha\big(\beta\gamma\big)&=(a+b\iu)\big((c+d\iu)(u+v\iu)\big) \\
				&=(a+b\iu)\big((cu-dv)+(cv+du)\iu\big) \\
				&=\big(a(cu-dv)-b(cv+du)\big)+\big(a(cv+du)+b(cu-dv)\big)\iu \\
				&=(acu-adv-bcv-bdu)+(acv+adu+bcu-bdv)\iu \\
				&=(acu-bdu-adv-bcv)+(acv-bdv+adu+bcu)\iu \\
				&=\big((ac-bd)u-(ad+bc)v\big)+\big((ac-bd)v+(ad+bc)u\big)\iu \\
				&=\big((ac-bd)+(ad+bc)\iu\big)(u+v\iu) \\
				&=\big((a+b\iu)(c+d\iu)\big)(u+v\iu)=(\alpha\beta)\gamma.\qedhere
			\end{align*}
		\end{proof}
	\end{proposition}
	\begin{proposition}
		\label{prop:element neutre pel producte dels complexos}
		Sigui \(a+b\iu\) un nombre complex. Aleshores
		\[(a+b\iu)\cdot1=a+b\iu.\]
		\begin{proof}
			Per la definició de \myref{def:producte de nombres complexos} tenim
			\[(a+b\iu)\cdot1=(a\cdot1)+(b\cdot1)\iu=a+b\iu.\qedhere\]
		\end{proof}
	\end{proposition}
	\begin{proposition}
		\label{prop:element invers pel producte de nombres complexos}
		Sigui \(a+b\iu\) un nombre complex. Aleshores
		\[(a+b\iu)\Big(\frac{a}{a^{2}+b^{2}}+\frac{-b}{a^{2}+b^{2}}\iu\Big)=1.\]
		\begin{proof}
			Per la definició de \myref{def:producte de nombres complexos} tenim
			\begin{align*}
				(a+b\iu)\Big(\frac{a}{a^{2}+b^{2}}+\frac{-b}{a^{2}+b^{2}}\iu\Big)&=\Big(\frac{a^{2}}{a^{2}+b^{2}}-\frac{-b^{2}}{a^{2}+b^{2}}\Big)+\Big(\frac{ab}{a^{2}+b^{2}}+\frac{-ba}{a^{2}+b^{2}}\Big)\iu \\
				&=\Big(\frac{a^{2}+b^{2}}{a^{2}+b^{2}}\Big)+\Big(\frac{ab-ab}{a^{2}+b^{2}}\Big)\iu=1.\qedhere
			\end{align*}
		\end{proof}
	\end{proposition}
	\begin{proposition}
		\label{prop:distribuitva del producte respecte la suma de nombres complexos}
		Siguin \(a+b\iu\), \(c+d\iu\) i \(u+v\iu\) tres nombres complexos. Aleshores
		\[(a+b\iu)\big((c+d\iu)+(u+v\iu)\big)=(a+b\iu)(c+d\iu)+(a+b\iu)(u+v\iu).\]
		\begin{proof}
			Per la definició de \myref{def:suma de nombres complexos} i \myref{def:producte de nombres complexos} tenim
			\begin{align*}
				(a+b\iu)\big((c+d\iu)+(u+v\iu)\big)&=(a+b\iu)\big((c+u)+(d+v)\iu\big) \\
				&=\big(a(c+u)-b(d+v)\big)+\big(a(d+v)+b(c+u)\big)\iu \\
				&=(ac+au-bd-bv)+(ad+av+bc+bu)\iu \\
				&=(ac-bd+au-bv)+(ad+bc+av+bu)\iu\\
				&=(ac-bd)+(ad+bc)\iu+(au-bv)+(av+bu)\iu \\
				&=(a+b\iu)(c+d\iu)+(a+b\iu)(u+v\iu).\qedhere
			\end{align*}
		\end{proof}
	\end{proposition}
	\begin{corollary}
		\label{cor:els complexos formen un cos}
		El conjunt \(\mathbb{C}\) amb la suma \(+\) i el producte \(\cdot\) és un cos.
	\end{corollary}
	
	
	\subsection{Propietats de nombres complexos}
	\begin{definition}[Norma d'un nombre complex]
		\labelname{norma d'un nombre complex}\label{def:norma d'un nombre complex}
		Sigui \(z=a+b\iu\) un nombre complex. Aleshores definim la seva norma com
		\[\abs{z}=\sqrt{a^{2}+b^{2}}.\]
	\end{definition}
	\begin{definition}[Conjugat d'un nombre complex]
		\labelname{conjugat d'un nombre complex}\label{def:nombre d'un nombre complex}
		Sigui \(z=a+b\iu\) un nombre complex. Aleshores definim
		\[\conjugat{z}=a-b\iu\]
		com el conjugat de \(z\).
	\end{definition}
	\begin{proposition}
		\label{prop:la nomra d'un nombre complex és l'arrel del nombre pel seu conjugat}
		Sigui \(z\) un nombre complex. Aleshores es satisfà
		\[\abs{z}=\sqrt{z\conjugat{z}}.\]
		\begin{proof}
			%TODO
		\end{proof}
	\end{proposition}
	
	\subsection{Topologia de nombres complexos}
\end{document} 
