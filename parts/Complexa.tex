\documentclass[../Apunts.tex]{subfiles}

\begin{document}
\part{Anàlisi complexa i de Fourier}
\chapter{Els nombres complexos}
\section{Estructura dels nombres complexos}
    \subsection{Cos de nombres complexos}
    \begin{notation}[Conjunt de nombres complexos]
        \label{notation:cos de nombres complexos}
        Denotarem
        \[\field{C}=\{a+\iu b\mid a,b\in\field{R}\}.\]
        %\[\field{C}=\{\alpha+\iu\beta\mid\alpha,\beta\in\field{R}\}.\]
    \end{notation}
    \begin{definition}[Suma de nombres complexos]
        \labelname{suma de nombres complexos}
        \label{def:suma de nombres complexos}
        Siguin~\(a+\iu b\) i~\(c+\iu d\) dos nombres complexos.
        Aleshores definim la seva suma com
        \[(a+\iu b)+(c+\iu d)=(a+c)+\iu(b+d).\]
    \end{definition}
    \begin{observation}
        \label{obs:els nombres complexos estan tancats per la suma}
        Siguin~\(a+\iu b\) i~\(c+\iu d\) dos nombres complexos.
        Aleshores
        \[(a+\iu b+c+\iu d)\in\field{C}\]
    \end{observation}
    \begin{proposition}
        \label{prop:els nombres complexos commuten per la suma}
        \label{prop:el producte de nombres complexos és commutatiu}
        Siguin~\(a+\iu b\) i~\(c+\iu d\) dos nombres complexos.
        Aleshores
        \[(a+\iu b)+(c+\iu d)=(c+\iu d)+(a+\iu b).\]
    \end{proposition}
    \begin{proof}
        Per la definició de~\myref{def:suma de nombres complexos} tenim
        \begin{align*}
            a+\iu b+c+\iu d&=(a+c)+\iu(b+d) \\
            &=(c+a)+(d+b)\iu=c+\iu d+a+\iu b.\qedhere
        \end{align*}
    \end{proof}
    \begin{proposition}
        \label{prop:els nombres complexos són associatius per la suma}
        Siguin~\(a+\iu b\),~\(c+\iu d\) i~\(u+\iu v\) tres nombres complexos.
        Aleshores
        \[
            (a+\iu b)+\big((c+\iu d)+(u+\iu v)\big) = 
            \big((a+\iu b)+(c+\iu d)\big)+(u+\iu v).
        \]
    \end{proposition}
    \begin{proof}
        Per la definició de~\myref{def:suma de nombres complexos} tenim
        \begin{align*}
            (a+\iu b)+\big((c+\iu d)+(u+\iu v)\big)
            &=(a+\iu b)+\big((c+u)+\iu(d+v)\big) \\
            &=\big(a+(c+u)\big)+\iu\big(b+(d+v)\big) \\
            &=\big((a+c)+u\big)+\iu\big((b+d)+v\big) \\
            &=\big((a+c)+\iu(b+d)\big)+(u+\iu v) \\
            &=\big((a+\iu b)+(c+\iu d)\big)+(u+\iu v).\qedhere
        \end{align*}
    \end{proof}
    \begin{proposition}
        \label{prop:element neutre per la suma dels complexos}
        Sigui~\(a+\iu b\) un nombre complex.
        Aleshores
        \[(a+\iu b)+0=a+\iu b.\]
    \end{proposition}
    \begin{proof}
        Per la definició de~\myref{def:suma de nombres complexos} tenim
        \[(a+\iu b)+0=(a+0)+\iu(b+0)=a+\iu b.\qedhere\]
    \end{proof}
    \begin{proposition}
        \label{prop:element invers per la suma dels complexos}
        Sigui~\(a+\iu b\) un nombre complex.
        Aleshores
        \[(a+\iu b)+(-a-\iu b)=0.\]
    \end{proposition}
    \begin{proof}
        Per la definició de~\myref{def:suma de nombres complexos} tenim
        \[(a+\iu b)+(-a-\iu b)=(a-a)+\iu(b-b)=0.\qedhere\]
    \end{proof}
    \begin{definition}[Producte de nombres complexos]
        \labelname{producte de nombres complexos}
        \label{def:producte de nombres complexos}
        Siguin~\(a+\iu b\) i~\(c+\iu d\) dos nombres complexos.
        Aleshores definim el seu producte com
        \[(a+\iu b)\cdot(c+\iu d)=(ac-bd)+\iu(ad+bc).\]
    \end{definition}
    \begin{observation}
        \label{obs:els nombres complexos estan tancats pel producte}
        Siguin~\(a+\iu b\) i~\(c+\iu d\) dos nombres complexos.
        Aleshores
        \[(a+\iu b)(c+\iu d)\in\field{C}.\]
    \end{observation}
    \begin{proposition}
        \label{prop:els nombres complexos commuten pel producte}
        Siguin~\(a+\iu b\) i~\(c+\iu d\) dos nombres complexos.
        Aleshores
        \[(a+\iu b)(c+\iu d)=(c+\iu d)(a+\iu b).\]
    \end{proposition}
    \begin{proof}
        Per la definició de~\myref{def:producte de nombres complexos} tenim
        \begin{align*}
            (a+\iu b)(c+\iu d)&=(ac-bd)+\iu(ad+bc) \\
            &=(ca-bd)+\iu(da+cb)=(c+\iu d)(a+\iu b).\qedhere
        \end{align*}
    \end{proof}
    \begin{proposition}
        \label{prop:els nombres complexos són associatius pel producte}
        Siguin~\(a+\iu b\),~\(c+\iu d\) i~\(u+\iu v\) tres nombres complexos.
        Aleshores
        \[
            (a+\iu b)\big((c+\iu d)(u+\iu v))\big) =
            \big((a+\iu b)(c+\iu d)\big)(u+\iu v).
        \]
    \end{proposition}
    \begin{proof}
        Prenem~\(\alpha=a+\iu b\),~\(\beta=c+\iu d\) i~\(\gamma=u+\iu v\)
        Per la definició de~\myref{def:producte de nombres complexos} tenim
        \begin{align*}
            \alpha\big(\beta\gamma\big)
            &=(a+\iu b)\big((c+\iu d)(u+\iu v)\big) \\
            &=(a+\iu b)\big((cu-dv)+\iu(cv+du)\big) \\
            &=\big(a(cu-dv)-b(cv+du)\big)+\iu\big(a(cv+du)+b(cu-dv)\big) \\
            &=(acu-adv-bcv-bdu)+\iu(acv+adu+bcu-bdv) \\
            &=(acu-bdu-adv-bcv)+\iu(acv-bdv+adu+bcu) \\
            &=\big((ac-bd)u-(ad+bc)v\big)+\iu\big((ac-bd)v+(ad+bc)u\big) \\
            &=\big((ac-bd)+\iu(ad+bc)\big)(u+\iu v) \\
            &=\big((a+\iu b)(c+\iu d)\big)(u+\iu v)=(\alpha\beta)\gamma.
            \qedhere
        \end{align*}
    \end{proof}
    \begin{proposition}
        \label{prop:element neutre pel producte dels complexos}
        Sigui~\(a+\iu b\) un nombre complex.
        Aleshores
        \[(a+\iu b)\cdot1=a+\iu b.\]
    \end{proposition}
    \begin{proof}
        Per la definició de~\myref{def:producte de nombres complexos} tenim
        \[(a+\iu b)\cdot1=(a\cdot1)+\iu(b\cdot1)=a+\iu b.\qedhere\]
    \end{proof}
    \begin{proposition}
        \label{prop:element invers pel producte de nombres complexos}
        Sigui~\(a+\iu b\) un nombre complex.
        Aleshores
        \[
            (a+\iu b)\Big(\frac{a}{a^{2}+b^{2}}+\iu\frac{-b}{a^{2}+b^{2}}\Big)
            =1.
        \]
    \end{proposition}
    \begin{proof}
        Per la definició de~\myref{def:producte de nombres complexos} tenim
        \begin{align*}
            (a+\iu b)\Big(\frac{a}{a^{2}+b^{2}}+\iu\frac{-b}{a^{2}+b^{2}}\Big)
            &=\Big(\frac{a^{2}}{a^{2}+b^{2}}-\frac{-b^{2}}{a^{2}+b^{2}}\Big)
            +\iu\Big(\frac{ab}{a^{2}+b^{2}}+\frac{-ba}{a^{2}+b^{2}}\Big) \\
            &=\Big(\frac{a^{2}+b^{2}}{a^{2}+b^{2}}\Big)
            +\iu\Big(\frac{ab-ab}{a^{2}+b^{2}}\Big)=1.
            \qedhere
        \end{align*}
    \end{proof}
    \begin{proposition}
        \label{prop:distribuitva del producte respecte la suma de nombres complexos}
        Siguin~\(a+\iu b\),~\(c+\iu d\) i~\(u+\iu v\) tres nombres complexos.
        Aleshores
        \[
            (a+\iu b)\big((c+\iu d)+(u+\iu v)\big)
            =(a+\iu b)(c+\iu d)+(a+\iu b)(u+\iu v).
        \]
    \end{proposition}
    \begin{proof}
        Per la definició de~\myref{def:suma de nombres complexos}
        i~\myref{def:producte de nombres complexos} tenim
        \begin{align*}
            (a+\iu b)\big((c+\iu d)+(u+\iu v)\big)
            &=(a+\iu b)\big((c+u)+\iu(d+v)\big) \\
            &=\big(a(c+u)-b(d+v)\big)+\iu\big(a(d+v)+b(c+u)\big) \\
            &=(ac+au-bd-bv)+\iu(ad+av+bc+bu) \\
            &=(ac-bd+au-bv)+\iu(ad+bc+av+bu)\\
            &=(ac-bd)+\iu(ad+bc)+(au-bv)+\iu(av+bu) \\
            &=(a+\iu b)(c+\iu d)+(a+\iu b)(u+\iu v).
            \qedhere
        \end{align*}
    \end{proof}
    \begin{corollary}
        \label{cor:els complexos formen un cos}
        El conjunt~\(\field{C}\) amb la suma~\(+\)
        i el producte~\(\cdot\) és un cos.
    \end{corollary}
    \subsection{Propietats de nombres complexos}
    \begin{definition}[Conjugat d'un nombre complex]
        \labelname{conjugat d'un nombre complex}
        \label{def:conjugat d'un nombre complex}
        Sigui~\(z=a+\iu b\) un nombre complex.
        Aleshores definim
        \[\conjugat{z}=a-\iu b\]
        com el conjugat de~\(z\).
    \end{definition}
    \begin{proposition}
        \label{prop:el conjugat del conjugat d'un nombre complex és ell mateix}
        Sigui~\(z\) un nombre complex.
        Aleshores
        \[\conjugat{\conjugat{z}}=z.\]
    \end{proposition}
    \begin{proof}
        Per la definició de~\myref{def:nombre complex} tenim que
        existeixen~\(a\),~\(b\in\field{R}\) tals que~\(z=a+\iu b\).
        Aleshores per la definició de~\myref{def:conjugat d'un nombre complex}
        tenim que
        \begin{align*}
            \conjugat{\conjugat{z}}
            &=\conjugat{\conjugat{a+\iu b}} \\
            &=\conjugat{a-\iu b} \\
            &=a+\iu b
            =z.
            \qedhere
        \end{align*}
    \end{proof}
    \begin{proposition}
        \label{prop:el conjugat de la suma és la suma de conjugats}
        Siguin~\(z\) i~\(w\) dos nombres complexos.
        Aleshores
        \[\conjugat{z+w}=\conjugat{z}+\conjugat{w}.\]
    \end{proposition}
    \begin{proof}
        Per la definició de~\myref{def:nombre complex} tenim que
        existeixen~\(a\),~\(b\),~\(c\),~\(d\in\field{R}\) tals que
        \[z=a+\iu b\qquad\text{i}\qquad w=c+\iu d.\]
        Aleshores per la definició de~\myref{def:suma de nombres complexos}
        i la definició de~\myref{def:conjugat d'un nombre complex} trobem que
        \begin{align*}
            \conjugat{z+w}
            &=\conjugat{(a+\iu b)+(c+\iu d)} \\
            &=\conjugat{(a+c)+\iu(b+d)} \\
            &=(a+c)-\iu(b+d) \\
            &=(a-\iu b)+(c-d\iu) \\
            &=\conjugat{a+\iu b}+\conjugat{c+\iu d}
            =\conjugat{z}+\conjugat{w}.
            \qedhere
        \end{align*}
    \end{proof}
    \begin{proposition}
        \label{prop:el conjugat del producte és el producte de conjugats}
        Siguin~\(z\) i~\(w\) dos nombres complexos.
        Aleshores
        \[\conjugat{zw}=\conjugat{z}\,\conjugat{w}.\]
    \end{proposition}
    \begin{proof}
        Per la definició de~\myref{def:nombre complex} tenim que
        existeixen~\(a\),~\(b\),~\(c\),~\(d\in\field{R}\) tals que
        \[z=a+\iu b\qquad\text{i}\qquad w=c+\iu d.\]
        Aleshores per la definició de~\myref{def:producte de nombres complexos}
        i la definició de~\myref{def:conjugat d'un nombre complex} trobem que
        \begin{align*}
            \conjugat{zw}
            &=\conjugat{(a+\iu b)(c+\iu d)} \\
            &=\conjugat{(ac-bd)+\iu(ad+bc)} \\
            &=(ac-bd)-\iu(ad+bc) \\
            &=(ac-bd)+\iu(-ad-bc) \\
            &=(a-\iu b)(c-\iu d) \\
            &=\conjugat{a+\iu b}\,\conjugat{c+\iu d}
            =\conjugat{z}\,\conjugat{w}.
            \qedhere
        \end{align*}
    \end{proof}
    \begin{proposition}
        \label{prop:el producte d'un nombre complex pel seu conjugat és la %
        suma dels quadrats de la seva part real i imaginaria}
        Sigui~\(z=a+\iu b\) un nombre complex.
        Aleshores
        \[z\conjugat{z}=a^{2}+b^{2}.\]
    \end{proposition}
    \begin{proof}
        Per la definició de~\myref{def:producte de nombres complexos} i la
        definició de~\myref{def:conjugat d'un nombre complex} trobem que
        \begin{align*}
            z\conjugat{z}
            &=(a+\iu b)\conjugat{a+\iu b} \\
            &=(a+\iu b)(a-\iu b)
            =a^{2}+b^{2}.
            \qedhere
        \end{align*}
    \end{proof}
    \begin{proposition}
        \label{prop:un nombre complex és igual al seu conjugat %
        si i només si és un real}
        Sigui~\(z\) un nombre complex.
        Aleshores~\(z=\conjugat{z}\) si i només si~\(z\in\field{R}\).
    \end{proposition}
    \begin{proof}
        Per la definició de~\myref{def:nombre complex} tenim que
        existeixen~\(a\),~\(b\in\field{R}\) tals que~\(z=a+\iu b\).
        Aleshores si tenim~\(z=\conjugat{z}\), per la definició
        de~\myref{def:conjugat d'un nombre complex} trobem que
        \[a+\iu b=a-\iu b,\]
        que és equivalent a~\(b=-b\), i per tant ha de ser~\(b=0\)
        i trobem que~\(z=a\in\field{R}\).
    \end{proof}
    \begin{proposition}
        \label{prop:inversa d'un nombre complex en funció del seu conjugat}
        Sigui~\(z\neq0\) un nombre complex.
        Aleshores
        \[z^{-1}=\frac{\conjugat{z}}{z\conjugat{z}}.\]
    \end{proposition}
    \begin{proof}
        Per la definició de~\myref{def:nombre complex} tenim que
        existeixen~\(a\),~\(b\in\field{R}\) tals que~\(z=a+\iu b\).
        Aleshores tenim que
        \begin{align*}
            z\frac{\conjugat{z}}{z\conjugat{z}}
            &=\frac{z\conjugat{z}}{z\conjugat{z}} \\
            &=\frac{a^{2}+b^{2}}{a^{2}+b^{2}}=1
            \tag{\ref{prop:el producte d'un nombre complex pel seu conjugat %
            és la suma dels quadrats de la seva part real i imaginaria}}
        \end{align*}
        i per la definició de~\myref{def:l'invers d'un element d'un anell}
        hem acabat.
    \end{proof}
    \begin{definition}[Part real i part imaginària d'un nombre complex]
        \labelname{part real i part imaginària d'un nombre complex}
        \label{def:part real i part imaginària d'un nombre complex}
        \labelname{part real d'un nombre complex}
        \label{def:part real d'un nombre complex}
        \labelname{part imaginària d'un nombre complex}
        \label{def:part imaginària d'un nombre complex}
        Sigui~\(z=a+\iu b\) un nombre complex.
        Aleshores definim
        \[\Re(z)=a\]
        com la part real de~\(z\) i
        \[\Im(z)=b\]
        com la part imaginària de~\(z\).
    \end{definition}
    \begin{proposition}
        \label{prop:fórmules per la part real i part imaginària d'un nombre complex}
        \label{prop:fórmula per la part real d'un nombre complex}
        \label{prop:fórmula per la part imaginària d'un nombre complex}
        Sigui~\(z\) un nombre complex.
        Aleshores
        \[
            \Re(z)=\frac{z+\conjugat{z}}{2}
            \qquad\text{i}\qquad
            \Im(z)=\frac{z-\conjugat{z}}{2\iu}.
        \]
    \end{proposition}
    \begin{proof}
        Per la definició de~\myref{def:nombre complex} tenim que
        existeixen~\(a\),~\(b\in\field{R}\) tals que~\(z=a+\iu b\).
        Aleshores tenim que
        \[
            \frac{z+\conjugat{z}}{2}
            =\frac{a+\iu b+a-\iu b}{2}
            =\frac{2a}{2}
            =a
            =\Re(z),
        \]
        i
        \[
            \frac{z-\conjugat{z}}{2\iu}
            =\frac{a+\iu b-a+\iu b}{2\iu}
            =\frac{2\iu b}{2\iu}
            =b
            =\Im(z),
        \]
        i per la definició
        de~\myref{def:part real i part imaginària d'un nombre complex}
        hem acabat.
    \end{proof}
    \subsection{Topologia de nombres complexos}
    \begin{definition}[Mòdul d'un nombre complex]
        \labelname{mòdul d'un nombre complex}
        \label{def:mòdul d'un nombre complex}
        Sigui~\(z=a+\iu b\) un nombre complex.
        Aleshores definim el seu mòdul com
        \[\modul{z}=\sqrt{a^{2}+b^{2}}.\]
    \end{definition}
    \begin{observation}
        \label{obs:les parts real i imaginàries d'un complex són menors %
        que el seu mòdul}
        \label{obs:la part real d'un complex és menor que el seu mòdul}
        \label{obs:la part imaginària d'un complex és menor que el seu mòdul}
        \(\Re(z)\leq\modul{z}\),~\(\Im(z)\leq\modul{z}\).
    \end{observation}
    \begin{observation}
        \label{prop:el mòdul d'un nombre complex és no negatiu}
        Sigui~\(z\) un nombre complex.
        Aleshores~\(\modul{z}\geq0\).
    \end{observation}
    \begin{proposition}
        \label{prop:el mòdul d'un nombre complex és zero %
        si i només si aquest és zero}
        Sigui~\(z\) un nombre complex.
        Aleshores
        \[\modul{z}=0\sii z=0.\]
    \end{proposition}
    \begin{proof}
        Per la definició de~\myref{def:nombre complex} tenim que~\(z=a+\iu b\).
        Suposem que~\(\modul{z}=0\).
        Per la definició de~\myref{def:mòdul d'un nombre complex} tenim
        que~\(\sqrt{a^{2}+b^{2}}=0\), i per tant ha de ser~\(a^{2}+b^{2}=0\),
        d'on trobem que~\(a=b=0\) i per tant~\(z=0\).
    \end{proof}
    \begin{proposition}
        \label{prop:el mòdul d'un nombre complex és l'arrel del nombre %
        pel seu conjugat}
        Sigui~\(z\) un nombre complex.
        Aleshores es satisfà
        \[\modul{z}=\sqrt{z\conjugat{z}}.\]
    \end{proposition}
    \begin{proof}
        Per la definició de~\myref{def:nombre complex} tenim que~\(z=a+\iu b\),
        i per la proposició~\myref{prop:el producte d'un nombre complex pel seu
        conjugat és la suma dels quadrats de la seva part real i imaginaria}
        trobem que
        \begin{align*}
            z\conjugat{z}&=a^{2}+b^{2} \\
            &=\big(\sqrt{a^{2}+b^{2}}\big)^{2}=\modul{z}^{2}, \tag{\ref{def:mòdul d'un nombre complex}}
        \end{align*}
        i per tant 
        \[\modul{z}=\sqrt{z\conjugat{z}}.\qedhere\]
    \end{proof}
    \begin{proposition}[Desigualtat triangular]
        \labelname{desigualtat triangular}
        \label{prop:desigualta triangular nombres complexos}
        Siguin~\(z\) i~\(w\) dos nombres complexos.
        Aleshores
        \[\modul{z+w}\leq\modul{z}+\modul{w}.\]
    \end{proposition}
    \begin{proof}
        Per la definició de~\myref{def:mòdul d'un nombre complex} tenim que
        \begin{align*}
            \modul{z+w}^{2}&=(z+w)\conjugat{(z+w)} \\
            &=(z+w)(\conjugat{z}+\conjugat{w})
            \tag{\ref{prop:el conjugat de la suma és la suma de conjugats}} \\
            &=z\conjugat{z}+z\conjugat{w}+\conjugat{z}w+w\conjugat{w} \\
            &=\modul{z}^{2}+\modul{w}^{2}+\conjugat{z}w+z\conjugat{w}
            \tag{\ref{prop:el mòdul d'un nombre complex és l'arrel del nombre %
            pel seu conjugat}} \\
            &=\modul{z}^{2}+\modul{w}^{2}+\conjugat{\conjugat{\conjugat{z}w}}
            +z\conjugat{w} \tag{\ref{prop:el conjugat del conjugat d'un nombre %
            complex és ell mateix}} \\
            &=\modul{z}^{2}+\modul{w}^{2}+\conjugat{\conjugat{\conjugat{z}}\,
            \conjugat{w}}+z\conjugat{w} \tag{\ref{prop:el conjugat del %
            producte és el producte de conjugats}} \\
            &=\modul{z}^{2}+\modul{w}^{2}+\conjugat{z\conjugat{w}}+z\conjugat{w}
            \tag{\ref{prop:el conjugat del conjugat d'un nombre complex és %
            ell mateix}} \\
            &=\modul{z}^{2}+\modul{w}^{2}+2\Re(zw)
            \tag{\ref{prop:fórmula per la part real d'un nombre complex}} \\
            &\leq\modul{z}^{2}+\modul{w}^{2}+2\modul{zw}
            \tag{\ref{obs:la part real d'un complex és menor que el seu mòdul}}\\
            &\leq\modul{z}^{2}+\modul{w}^{2}+2\modul{z}\modul{w} \\
            &=(\modul{z}+\modul{w})^{2},
        \end{align*}
        i per tant~\(\modul{z+w}\leq\modul{z}+\modul{w}\).
    \end{proof}
    \begin{proposition}
        \label{prop:el producte de mòduls és el mòdul del producte}
        Siguin~\(z\) i~\(w\) dos nombres complexos.
        Aleshores
        \[\modul{zw}=\modul{z}\modul{w}.\]
    \end{proposition}
    \begin{proof}
        Per la definició de~\myref{def:mòdul d'un nombre complex} tenim que
        \begin{align*}
            \modul{zw}^{2}&=\sqrt{zwz\conjugat{zw}}^{2} \\
            &=zw\conjugat{zw} \\
            &=zw\conjugat{z}\,\conjugat{w}
            \tag{\ref{prop:el conjugat del producte és el producte de conjugats}} \\
            &=z\conjugat{z}w\conjugat{w}
            \tag{\ref{prop:el producte de nombres complexos és commutatiu}} \\
            &=\sqrt{z\conjugat{z}}^{2}\sqrt{w\conjugat{w}}^{2}=\modul{z}^{2}\modul{w}^{2}
        \end{align*}
        i per tant~\(\modul{zw}=\modul{z}\modul{w}\).
    \end{proof}
    \begin{proposition}
        \label{prop:el valor absolut de la resta de mòduls és més petit o igual que el mòdul de la suma}
        Siguin~\(z\) i~\(w\) dos nombres complexos.
        Aleshores
        \[\abs{\modul{z}-\modul{w}}\leq\modul{z+w}.\]
    \end{proposition}
    \begin{proof}
        Tenim que
        \begin{align*}
            \modul{z}&=\modul{z+w-w} \\
            &\leq\modul{z+w}+\modul{w}
            \tag{\ref{prop:desigualta triangular nombres complexos}}
        \end{align*}
        i tenim que~\(\abs{\modul{z}-\modul{w}}\leq\modul{z+w}\).
    \end{proof}
    \begin{proposition}
        \label{prop:els complexos són un espai mètric}
        El conjunt~\(\CC\) amb
        \begin{align*}
            \distancia\colon\CC\times\CC&\longrightarrow\CC \\
            (z,w)&\longmapsto\modul{z-w}
        \end{align*}
        és un espai mètric.
    \end{proposition}
    \begin{proof}
        Veiem que satisfà les condicions de la definició de~\myref{def:espai mètric}.
        \begin{enumerate}
            \item Siguin~\(z\) i~\(w\) dos nombres complexos tals
                que~\(\distancia(z,w)=0\).
                Aleshores tenim que~\(\modul{z-w}=0\), i per la
                proposició~\myref{prop:el mòdul d'un nombre complex és zero
                si i només si aquest és zero} tenim que ha de ser~\(z-w=0\),
                i per tant~\(z=w\).
            \item Siguin~\(z\) i~\(w\) dos nombres complexos.
                Aleshores
                \[
                    \distancia(z,w)
                    =\modul{z-w}
                    =\modul{-(z-w)}
                    =\modul{w-z}
                    =\distancia(w,z).
                \]
            \item Siguin~\(z\),~\(w\) i~\(t\) tres nombres complexos.
                Aleshores
                \begin{align*}
                    \distancia(z,w)&=\modul{z-w} \\
                    &\leq\modul{z}-\modul{w}
                    \tag{\ref{prop:el valor absolut de la resta de mòduls és %
                    més petit o igual que el mòdul de la suma}}\\
                    &=\modul{z}-\modul{t}+\modul{t}-\modul{w} \\
                    &\leq\modul{z-t}+\modul{w-t}
                    \tag{\ref{prop:el valor absolut de la resta de mòduls és %
                    més petit o igual que el mòdul de la suma}}\\
                    &=\modul{z-t}+\modul{t-w} \\
                    &=\distancia(z,t)+\distancia(t,w)
                \end{align*}
            \item Siguin~\(z\) i~\(w\) dos nombres complexos.
                Aleshores per l'observació~\myref{prop:el mòdul d'un nombre %
                complex és no negatiu} tenim que
            \[\distancia(z,w)=\modul{z-w}\geq0.\]
        \end{enumerate}
        i per la definició de~\myref{def:espai mètric} tenim que~\(\distancia\)
        és una distància i que~\(\CC\) amb la distància~\(\distancia\) és un
        espai mètric.
    \end{proof}
    \begin{proposition}
        \label{prop:el pla complex és homeomorf al pla real}
        Tenim que~\(\CC\) és homeomorf a~\(\RR^{2}\).
    \end{proposition}
    \begin{proof}
        %TODO
    \end{proof}
    %\subsection{Representació de nombres complexos}
    %\begin{proposition}
        %\label{prop:nombre complex com a sinus i cosinus}
        %Sigui~\(z\) un nombre complex no nul.
        %Aleshores existeix un~\(\theta\in\RR\) tal que
        %\[z=\modul{z}\bigl(\cos(\theta)+\iu\sin(\theta)\bigr).\]
    %\end{proposition}
    %\begin{proof}
        
    %\end{proof}
    %\begin{proposition}
        %\label{prop:nombre complex com a exponencial}
        %Sigui~\(z\) un nombre complex no nul.
        %Aleshores existeix un~\(\theta\in\RR\) tal que
        %\[z=\modul{z}\e^{\iu\theta}.\]
    %\end{proposition}
    %\begin{proof}
        
    %\end{proof}

    %\subsection{Argument d'un nombre complex}
    %\begin{definition}[Argument]
        %\labelname{argument}\label{def:argument}
        %Sigui~\(z\) un nombre complex no nul.
        %Definim l'argument de~\(z\) com
        %\[
            %\arg(z)
            %=\{\theta\in\RR\mid z
            %=\modul{z}\bigl(\cos(\theta)+\iu\sin(\theta)\bigr)\}.
        %\]
    %\end{definition}
    %\begin{observation}
        %Siguin~\(z\) un nombre complex no nul i~\(\theta\) un real tal
        %que~\(\theta\in\arg(z)\).
        %Aleshores
        %\[\arg(z)=\{\theta+2\uppi k\in\RR\mid k\in\ZZ\}.\]
    %\end{observation}
    %\begin{definition}[Argument principal]
        %\labelname{argument principal}
        %\label{def:argument principal}
        %Sigui~\(z\) un nombre complex no nul.
        %Definim l'argument principal de~\(z\) com
        %\[
            %\Arg(z)=\theta\in(-\uppi,\uppi]
            %\qquad\text{tal que}\qquad
            %z=\modul{z}\bigl(\cos(\theta)+\iu\sin(\theta)\bigr).
        %\]
    %\end{definition}
    %\begin{observation}
        %\label{obs:continuitat de l'argument principal}
        %La funció~\(\Arg\) és contínua en~\(\CC\setminus(-\infty,0]\).
    %\end{observation}
\section{Funcions de variable complexa}
    \subsection{Continuïtat i \ensuremath{\CC}-derivabilitat}
    \begin{observation}
        Siguin~\(\Omega\subseteq\CC\) un subconjunt
        i~\(f\colon\Omega\longrightarrow\CC\) una funció.
        Aleshores existeixen dues funcions~\(u,v\colon\Omega\longrightarrow\RR\)
        tals que~\(f(z)=u(z)+\iu(z)\) per a tot~\(z\in\Omega\).
    \end{observation}
    \begin{definition}[Continuïtat]
        \labelname{funció contínua}\label{def:funció complexa contínua}
        Siguin~\(\Omega\subseteq\CC\) un obert,~\(z_{0}\in\Omega\) un punt
        i~\(f\colon\Omega\longrightarrow\CC\) una funció tal que
        \[\lim_{z\to z_{0}}f(z)=f(z_{0}).\]
        Aleshores direm que~\(f\) és contínua en~\(z_{0}\).
        
        Si~\(f\) és contínua per a tot~\(z\in\Omega\) direm que~\(f\) és
        contínua en~\(\Omega\).
    \end{definition}
    \begin{observation}
        \label{obs:una funció complexa és contínua si i només si %
        ho són les seves components}
        Siguin~\(\Omega\subseteq\CC\) un obert i
        \begin{align*}
            f\colon\Omega&\longrightarrow\CC \\
            z&\longmapsto u(z)+\iu v(z)
        \end{align*}
        una funció.
        Aleshores~\(f\) és contínua si i només si~\(u\) i~\(v\) són contínues.
    \end{observation}
    \begin{definition}[\ensuremath{\CC}-derivable]
        \labelname{funció derivable complexa}
        \label{def:C-derivable}\label{def:funció C-derivable}
        Siguin~\(\Omega\subseteq\CC\) un obert,~\(z_{0}\in\Omega\) un punt
        i~\(f\colon\Omega\longrightarrow\CC\) una funció tal que
        \[
            \lim_{\substack{h\to0\\ h\in\CC}}\frac{f(z_{0}+h)-f(z_{0})}{h}
            =f'(z_{0}).
        \]
        Aleshores direm que~\(f\) és~\(\CC\)-derivable en~\(z_{0}\).
    \end{definition}
    \begin{observation}
        \label{obs:C-derivable implica contínua}
        Sigui~\(f\) una funció~\(\CC\)-derivable en un punt~\(z_{0}\).
        Aleshores~\(f\) és contínua en~\(z_{0}\).
    \end{observation}
    \begin{example}
        \label{ex:la funció identitat és entera}
        Volem veure que la funció~\(f(z)=z\) és contínua i~\(\CC\)-derivable,
        i calcular~\(f'\).
    \end{example}
    \begin{solution}
        Observem que
        \begin{align*}
            f\colon\CC&\longrightarrow\CC \\
            a+\iu b&\longmapsto \Re(a)+\iu\Im(b),
        \end{align*}
        i per la proposició~\myref{prop:fórmules per la part real i part %
        imaginària d'un nombre complex} trobem que~\(\Re\) i~\(\Im\) són
        funcions contínues ne~\(\CC\).
        Per tant per l'observació~\myref{obs:una funció complexa és contínua %
        si i només si ho són les seves components} tenim que~\(f\) és contínua
        en~\(\CC\).
        
        Calculem ara
        \begin{align*}
            f'(z_{0})&=\lim_{h\to0}\frac{f(z_{0}+h)-f(z_{0})}{h} \\
            &=\lim_{h\to0}\frac{z_{0}+h-z_{0}}{h} \\
            &=\lim_{h\to0}\frac{h}{h}=1,
        \end{align*}
        i per tant la funció~\(f\) és~\(\CC\)-derivable en~\(\CC\) i~\(f'(z)=1\).
    \end{solution}
    \begin{example}
        Volem veure que l'aplicació~\(f(z)=\conjugat{z}\) no és~\(\CC\)-derivable.
        Calculem
        \begin{align*}
            \lim_{h\to0}\frac{f(z_{0}+h)-f(z_{0})}{h}
            &=\lim_{h\to0}\frac{\conjugat{z_{0}+h}-\conjugat{z_{0}}}{h} \\
            &=\lim_{h\to0}\frac{\conjugat{z_{0}}+\conjugat{h}-\conjugat{z_{0}}}{h}
            =\lim_{h\to0}\frac{\conjugat{h}}{h}.
            \tag{\ref{prop:el conjugat de la suma és la suma de conjugats}}
        \end{align*}
        Ara bé, si~\(h=1\) trobem que~\(\conjugat{h}/h=1\), però si~\(h=\iu\)
        trobem que~\(\conjugat{h}/h=-\iu/\iu=-1\), i per tant~\(f\)
        no és~\(\CC\)-derivable en cap punt de~\(\CC\).
    \end{example}
\section{Funcions holomorfes i funcions enteres}
    \subsection{Funcions holomorfes}
    \begin{definition}[Funció holomorfa]
        \labelname{funció holomorfa}\label{def:funció holomorfa}
        \labelname{funció entera}%\label{def:funció entera}
        Siguin~\(\Omega\subseteq\CC\) un obert
        i~\(f\colon\Omega\longrightarrow\CC\) una funció~\(\CC\)-derivable per
        a tot punt~\(z\in\Omega\).
        Aleshores direm que~\(f\) és holomorfa en~\(\Omega\).
        
        Si~\(f\) és holomorfa en tot~\(\CC\) direm que~\(f\) és entera.
    \end{definition}
    \begin{notation}[Funcions holomorfes]
        Sigui~\(\Omega\subseteq\CC\) un obert.
        Denotarem
        \[
            \Hol(\Omega)
            =\{f\colon\Omega\longrightarrow\CC\mid f\text{ és holomorfa en }\Omega\}.
        \]
    \end{notation}
    \begin{proposition}
        \label{prop:derivació de funcions holomorfes}
        Siguin~\(\Omega\subseteq\CC\) un obert i~\(f,g\in\Hol(\Omega)\) dues
        funcions.
        Aleshores
        \begin{enumerate}
            \item \(f+g\in\Hol(\Omega)\) i~\((f+g)'(z)=f'(z)+g'(z)\).
            \item \(fg\in\Hol(\Omega)\) i~\((fg)'(z)=f'(z)g(z)+g'(z)f(z)\).
            \item Si~\(g(z)\neq0\) per a tot~\(z\in\Omega\),
                aleshores~\(f/g\in\Hol(\Omega)\) i
                \[(f/g)'(z)=\frac{f'(z)g(z)-g'(z)f(z)}{g(z)^{2}}.\]
        \end{enumerate}
    \end{proposition}
    \begin{proof}
        %TODO
    \end{proof}
    \begin{definition}[Funció entera]
        \labelname{funció entera}\label{def:funció entera}
        Sigui~\(f\) una funció holomorfa en tot~\(\CC\).
        Aleshores direm que~\(f\) és entera.
    \end{definition}
    \begin{example}
        Els polinomis són funcions enteres.
    \end{example}
    \begin{solution}
        Per l'exemple~\myref{ex:la funció identitat és entera} tenim que la
        funció~\(g(z)=z\) és entera
        Aleshores per la proposició~\myref{prop:derivació de funcions holomorfes}
        trobem que les funcions~\(g_{n}(z)=a_{n}z^{n}\) són enteres, i de nou
        per la proposició trobem que les
        funcions~\(f(z)=a_{0}+a_{1}z+\dots+a_{n}z^{n}\) són enteres.
    \end{solution}
    \subsection{Sèries de potències complexes}
    \begin{definition}[Sèries de potències]
        \labelname{sèrie de potències}\label{def:sèrie de potències complexes}
        Sigui~\((a_{n})_{n\in\mathbb{N}}\) una successió de nombres complexos.
        Aleshores direm que la sèrie
        \[\sum_{n=0}^{\infty}a_{n}z^{n}\]
        és una sèrie de potències complexa.
        
        Si existeix el límit
        \[\lim_{N\to\infty}\sum_{n=0}^{N}a_{n}z^{n}=L\]
        existeix direm que la sèrie és convergent i que
        \[\sum_{n=0}^{\infty}a_{n}z^{n}=L.\]
        
    \end{definition}
    \begin{theorem}
        \label{thm:radi de convergència d'una sèrie de potències complexa}
        Sigui~\(\sum_{n=0}^{\infty}a_{n}z^{n}\) una sèrie de potències complexa
        i~\(R\) un real tal que
        \[R^{-1}=\limsup_{n\to\infty}\sqrt[n]{\modul{a_{n}}}.\]
        Aleshores
        \begin{enumerate}
            \item\label{thm:radi de convergència d'una sèrie de potències complexa:enum1}
                la sèrie~\(\sum_{n=0}^{\infty}a_{n}z^{n}\) és absolutament
                convergent per a tot~\(z\) satisfent~\(\modul{z}<R\).
            \item\label{thm:radi de convergència d'una sèrie de potències complexa:enum2}
                per a tot real~\(r<R\) la sèrie de
                funcions~\(\sum_{n=0}^{\infty}a_{n}z^{n}\) convergeix
                uniformement per a tot~\(z\) satisfent~\(\modul{z}\leq r\).
            \item\label{thm:radi de convergència d'una sèrie de potències complexa:enum3}
                la sèrie~\(\sum_{n=0}^{\infty}a_{n}z^{n}\) és divergent per a
                tot~\(z\) satisfent~\(\modul{z}>R\).
        \end{enumerate}
    \end{theorem}
    \begin{proof}
        %TODO
    \end{proof}
    \begin{definition}[Radi de convergència]
%       \labelname{radi de convergència d'una sèrie de potències}
%       \label{def:radi de convergència d'una sèrie de potències}
        Siguin~\(\sum_{n=0}^{\infty}a_{n}z^{n}\) una sèrie de potències i~\(R\)
        un real satisfent
        \begin{enumerate}
            \item la sèrie~\(\sum_{n=0}^{\infty}a_{n}z^{n}\) és absolutament
                convergent per a tot~\(z\) satisfent~\(\modul{z}<R\).
            \item per a tot real~\(r<R\) la sèrie de
                funcions~\(\sum_{n=0}^{\infty}a_{n}z^{n}\) convergeix
                uniformement per a tot~\(z\) satisfent~\(\modul{z}\leq r\).
            \item La sèrie~\(\sum_{n=0}^{\infty}a_{n}z^{n}\) és divergent
                per a tot~\(z\) satisfent~\(\modul{z}>R\).
        \end{enumerate}
        Aleshores direm que~\(R\) és el radi de convergència de la sèrie de
        potències.
        
        Aquesta definició té sentit pel Teorema
        \myref{thm:radi de convergència d'una sèrie de potències complexa}.
    \end{definition}
    \begin{definition}[Disc]
        \labelname{disc}
        \label{def:disc}
        Sigui~\(a\) un nombre complex i~\(R>0\) un nombre real.
        Aleshores definim
        \[\Disc(a,R)=\{z\in\CC\mid\modul{z-a}<R\}\]
        com el disc centrat en~\(a\) de radi~\(R\).
    \end{definition}
    \begin{proposition}
        Sigui~\(f(z)=\sum_{n=0}^{\infty}a_{n}z^{n}\) una sèrie de potències
        complexa amb radi de convergència~\(R\).
        Aleshores~\(f(z)\) és holomorfa sobre~\(\Disc(0,R)\).
    \end{proposition}
    \begin{proof}
        %TODO
    \end{proof}
    \begin{example}
        \label{ex:exponencial complexa}
        La funció
        \[\e^{z}=\sum_{n=0}^{\infty}\frac{z^{n}}{n!}\]
        és entera.
    \end{example}
    \begin{notation}
        Denotarem
        \begin{gather*}
            \begin{split}
                \cos(z)&=\frac{\e^{\iu z}+\e^{-\iu z}}{2} \\
                \cosh(z)&=\frac{\e^{z}+e^{-z}}{2}
            \end{split}\qquad
            \begin{split}
                \sin(z)&=\frac{\e^{\iu z}-\e^{-\iu z}}{2\iu} \\
                \sinh(z)&=\frac{\e^{z}-e^{-z}}{2}
            \end{split} \\
        \end{gather*}
        per a tot~\(z\in\CC\).
    \end{notation}
    \begin{observation}
        Les funcions~\(\cos(z)\),~\(\sin(z)\),~\(\cosh(z)\)
        i~\(\sinh(z)\) són enteres.
    \end{observation}
%   \begin{observation}
%       Sigui~\(z\) un nombre complex.
%       Aleshores
%       \[\cos(z+2\uppi k)=\cos(z)\qquad\text{i}\qquad\sin(z+2\uppi k)=\sin(z)\]
%       per a tot~\(k\) enter.
%   \end{observation}
    \begin{observation}
        Sigui~\(z\) un nombre complex.
        Aleshores
        \begin{equation*}
            \cosh(z)=\cos(\iu z)\qquad\text{i}\qquad\sinh(z)=-\iu\sin(\iu z).
        \end{equation*}
    \end{observation}
    \begin{theorem}[Teorema d'Abel]
        \labelname{Teorema d'Abel}
        \label{thm:Teorema d'Abel per sèries de potències complexes}
        Sigui~\(f(z)=\sum_{n=0}^{\infty}a_{n}z^{n}\) amb radi de
        convergència~\(R=1\) tal que~\(\sum_{n=0}^{\infty}a_{n}\) sigui convergent.
        Aleshores
        \[\lim_{\substack{z\to1\\z\in A}}f(z)=\sum_{n=0}^{\infty}a_{n}\]
        on
        \begin{equation}
            \label{eq:1:thm:Teorema d'Abel per sèries de potències complexes}
            A=\{z\in\CC\mid\abs{1-z}<k(1-\modul{z})\text{ per a cert }k\in\RR\}.
        \end{equation}
    \end{theorem}
    \begin{proof}
        %TODO
%       Podem suposar, sense pèrdua de generalitat, que~\(S=0\)
%       redefinint~\(a_{0}\) com~\(a_{0}-S\).
%       
%       Definim
%       \[S_{N}=\sum_{n=0}^{N}a_{n}.\]
%       Aleshores~\(\lim_{N\to\infty}S_{N}=0\).
%       Per tant, per la definició de~\myref{def:límit} tenim que per a
%       tot~\(\varepsilon>0\) existeix un~\(n_{0}\in\NN\) tal que per a
%       tot~\(N>n_{0}\) tenim que~\(\abs{S_{N}}<\varepsilon\).
%       Per tant si prenem un~\(\varepsilon>0\) tenim que existeix un~\(n_{0}\)
%       tal que per a tot~\(N>n_{0}\) trobem
%       \begin{align*}
%           \modul{S_{N}z^{N}}&=\modul{S_{N}}\modul{z^{N}}
%           \tag{\ref{def:mòdul d'un nombre complex}} \\
%           &<\varepsilon\modul{z^{N}}
%           \tag{\ref{def:límit}} \\
%           &=\varepsilon\modul{z}^{N}
%           \tag{\ref{prop:el producte de mòduls és el mòdul del producte}} \\
%           &\leq\varepsilon.
%           \tag{\ref{eq:1:thm:Teorema d'Abel per sèries de potències complexes}}
%       \end{align*}
%       i per tant~\(\modul{S_{N}z^{N}}<\varepsilon\).
%       També tenim pel~\myref{thm:criteri M de Weierstrass} que la
%       sèrie~\(\sum_{n=0}^{N}\) és convergent quan~\(z\) tendeix a~\(1\),
%       i per tant~\(\modul{1-z}\)
    \end{proof}
    \subsection{Equacions de Cauchy-Riemann}
    \begin{lemma}[Equacions de Cauchy-Riemann]
        \labelname{equacions de Cauchy-Riemann}
        \label{lema:equacions de Cauchy-Riemann}
        Denotant~\(x=\Re(z)\) i~\(y=\Im(z)\), sigui
        \begin{equation*}
            \label{eq:1:lema:equaciosn de Cauchy-Riemann}
            f(z,\conjugat{z})
            =f(x,y)
            =u(z,\conjugat{z})+\iu v(z,\conjugat{z})
            =u(x,y)+\iu v(x,y)
        \end{equation*}
        una funció complexa.
        Aleshores
        \begin{equation*}
            \label{eq:3:lema:equacions de Cauchy-Riemann}
            \frac{\partial f}{\partial z}
            =\frac{1}{2}\biggl(\frac{\partial f}{\partial x}
            -\iu\frac{\partial f}{\partial y}\biggr)
            \qquad\text{i}\qquad
            \frac{\partial f}{\partial\conjugat{z}}
            =\frac{1}{2}\biggl(\frac{\partial f}{\partial x}
            +\iu\frac{\partial f}{\partial y}\biggr)
    \end{equation*}
    \end{lemma}
    \begin{proof}
        Per la proposició~\myref{prop:fórmules per la part real i part %
        imaginària d'un nombre complex} tenim que
        \[
            x=\frac{z+\conjugat{z}}{2}
            \qquad\text{i}\qquad
            y=\frac{z-\conjugat{z}}{2\iu}.
        \]
        Per tant
        \begin{gather}
            \label{eq:2:lema:equacions de Cauchy-Riemann}
            \begin{split}
                \frac{\partial x}{\partial z}&=\frac{1}{2}\\
                \frac{\partial x}{\partial\conjugat{z}}&=\frac{1}{2}
            \end{split}\qquad\text{i}\qquad
            \begin{split}
                \frac{\partial y}{\partial z}&=\frac{1}{2\iu}\\
                \frac{\partial y}{\partial\conjugat{z}}&=\frac{-1}{2\iu}
            \end{split}
        \end{gather}
        Aleshores calculem
        \begin{align*}
            \frac{\partial f}{\partial z}
            &=\frac{\partial u}{\partial x}\frac{\partial x}{\partial z}
            +\frac{\partial u}{\partial y}\frac{\partial y}{\partial z}
            +\iu\biggl(\frac{\partial v}{\partial x}\frac{\partial x}{\partial z}
            +\frac{\partial v}{\partial y}\frac{\partial y}{\partial z}\biggr)
            \tag{\ref{eq:1:lema:equaciosn de Cauchy-Riemann}}\\
            &=\frac{1}{2}\frac{\partial u}{\partial x}+
            \frac{1}{2\iu}\frac{\partial u}{\partial y}+
            \iu\biggl(\frac{1}{2}\frac{\partial v}{\partial x}
            +\frac{1}{2\iu}\frac{\partial v}{\partial y}\biggr)
            \tag{\ref{eq:2:lema:equacions de Cauchy-Riemann}}\\
            &=\frac{1}{2}\Biggl(\frac{\partial u}{\partial x}
            +\iu\frac{\partial v}{\partial x}
            -\iu\biggl(\frac{\partial u}{\partial y}
            +\iu\frac{\partial v}{\partial y}\biggr)\Biggr)
            =\frac{1}{2}\biggl(\frac{\partial f}{\partial x}
            -\iu\frac{\partial f}{\partial y}\biggr),
            \tag{\ref{eq:3:lema:equacions de Cauchy-Riemann}}
        \end{align*}
        i
        \begin{align*}
            \frac{\partial f}{\partial\conjugat{z}}
            &=\frac{\partial u}{\partial x}\frac{\partial x}{\partial\conjugat{z}}
            +\frac{\partial v}{\partial y}\frac{\partial y}{\partial\conjugat{z}}
            +\iu\biggl(\frac{\partial v}{\partial x}\frac{\partial x}{\partial\conjugat{z}}
            +\frac{\partial v}{\partial y}\frac{\partial y}{\partial\conjugat{z}}\biggr)
            \tag{\ref{eq:1:lema:equaciosn de Cauchy-Riemann}}\\
            &=\frac{1}{2}\frac{\partial u}{\partial x}
            -\frac{1}{2\iu}\frac{\partial u}{\partial y}
            +\iu\biggl(\frac{1}{2}\frac{\partial v}{\partial x}
            -\frac{1}{2\iu}\frac{\partial v}{\partial y}\biggr)
            \tag{\ref{eq:2:lema:equacions de Cauchy-Riemann}}\\
            &=\frac{1}{2}\Biggl(\frac{\partial u}{\partial x}
            +\iu\frac{\partial v}{\partial x}+\iu\biggl(\frac{\partial u}{\partial y}
            +\iu\frac{\partial v}{\partial y}\biggr)\Biggr)
            =\frac{1}{2}\biggl(\frac{\partial f}{\partial x}
            +\iu\frac{\partial f}{\partial y}\biggr),
            \tag{\ref{eq:3:lema:equacions de Cauchy-Riemann}}
        \end{align*}
        com volíem veure.
    \end{proof}
    \begin{observation}
        Sigui~\(f(z,\conjugat{z})\) una funció complexa.
        Aleshores
        \[
            \frac{\partial f}{\partial z}=\frac{\partial f}{\partial x}
            \qquad\text{i}\qquad
            \frac{\partial f}{\partial\conjugat{z}}=0.
        \]
    \end{observation}
    \begin{example}
        Exemple de les equacions de Cauchy-Riemann.
    \end{example}
    \begin{solution}
        %TODO
    \end{solution}
    \begin{definition}[\ensuremath{\CC}-diferenciable]
        \labelname{diferenciable complexa}
        \label{def:C-diferenciable}
        Sigui~\(f(x,y)=y(x,y)+\iu v(x,y)\) una funció sobre un
        obert~\(\Omega\subseteq\CC\) tal que l'aplicació%
        ~\((u,v)\colon\RR^{2}\longrightarrow\RR^{2}\) és diferencable.
        Aleshores direm que~\(f\) és~\(\CC\)-diferenciable.
    \end{definition}
    \begin{proposition}
        \label{prop:les funcions holomorfes són funcions diferenciables}
        Sigui~\(f\colon\Omega\longrightarrow\CC\) una funció holomorfa.
        Aleshores~\(f\) és diferenciable en~\(\Omega\).
    \end{proposition}
    \begin{proof}
        Prenem~\(z\in\Omega\).
        Per la definició de~\myref{def:funció holomorfa} tenim que~\(f\)
        és~\(\CC\)-derivable, i per la definició de~\myref{def:C-derivable}
        el límit
        \begin{equation*}
            \lim_{h\to0}\frac{f(z+h)-f(z)}{h} = f'(z).
        \end{equation*}
        existeix.
        Aleshores existeixen~\(\alpha_{1}+\iu\alpha_{2}\in\CC\) tals
        que~\(f'(z)=\alpha_{1}+\iu\alpha_{2}\).
        Per tant, si~\(f(z)=u(z)+\iu v(z)\) i~\(h=h_{1}+\iu h_{2}\) tenim que
        \begin{equation*}
            \lim_{h\to0}\frac{\Bigl(\bigl(u(z+h)-u(z)\bigr)
            +\iu\bigl(v(z+h)-v(z)\bigr)\Bigr)(\alpha_{1}+\iu\alpha_{2})
            (h_{1}+\iu h_{2})}{h} = 0,
        \end{equation*}
        que és equivalent a
        \begin{equation*}
            \lim_{h\to0}\frac{1}{h}\biggl(\Bigl(\begin{matrix}
                u(z+h)-u(z) \\
                v(z+h)-v(z)
            \end{matrix}
            \Bigr)+\Bigl(\begin{matrix}
                \alpha_{1} & -\alpha_{2} \\
                \alpha_{2} & \alpha_{1}
            \end{matrix}\Bigr)\Bigl(\begin{matrix}
                h_{1} \\
                h_{2}
            \end{matrix}\Bigr)\biggr)
            =\Bigl(\begin{matrix}
                0 \\
                0
            \end{matrix}\Bigr).
        \end{equation*}
        i per tant, per la definició de~\myref{def:C-diferenciable} tenim que~\(f\) és~\(\CC\)-diferenciable.
    \end{proof}
\end{document}
