\documentclass[../Apunts.tex]{subfiles}

\begin{document}
\part{Teoria de Galois}
\chapter[Capítol primer]{Primer}
\section{Extensions de cossos}
\subsection{Elements algebraics i elements transcendents}
	\begin{definition}[Extensió d'un cos]
		\labelname{extensió d'un cos}\label{def:extensió d'un cos}
		Siguin~\(\KK\) i~\(\FF\) dos cossos tals que~\(\KK\subseteq\FF\). Aleshores direm que~\(\FF\) és una extensió de~\(\KK\) i ho denotarem com~\(\FF\extensio\KK\). També direm que~\(\FF\extensio\KK\) és una extensió.
	\end{definition}
	\begin{example}
		\label{ex:el cos de polinomis és una extensió}
		Siguin~\(\KK\) un cos i~\(p(x)\in\KK[x]\) un polinomi irreductible. Denotem~\(\FF=\KK[x]/(p(x))\). Aleshores~\(\FF\extensio\KK\) és una extensió.
		\begin{solution} % Veure que un cos és un DIP
			Veiem que~\(\FF\) és un cos. Per hipòtesi tenim que~\(p(x)\) és irreductible, i per la proposició \myref{prop:irreductible sii ideal maximal} trobem que l'ideal~\((p(x))\) és maximal. Aleshores per la proposició \myref{prop:condició equivalent a ideal maximal per R/M cos} tenim que~\(\FF\) és un cos.
			
			Per acabar veiem que~\(\KK\subseteq\FF\) per l'observació \myref{obs:un anell està contingut en el seu anell de polinomis}, i per la definició d'\myref{def:extensió d'un cos} hem acabat.
		\end{solution}
	\end{example}
	\begin{example}[Morfisme avaluació]
		\labelname{\empty}
		\label{ex:el morfisme avaluació és un morfisme d'anells}
		Siguin~\(\FF\extensio\KK\) una extensió i~\(\alpha\in\FF\) un element. Volem veure que l'aplicació
		\begin{align*}
			\ev_{\alpha}\colon\KK[x]&\longrightarrow\FF \\
			\KK[x]\setminus\KK\ni x&\longmapsto\alpha \\
			\KK\ni\lambda&\longmapsto\lambda
		\end{align*}
		és un morfisme d'anells
		\begin{solution}
			%TODO
		\end{solution}
	\end{example}
	\begin{lemma}
		Siguin~\(\FF\extensio\KK\) una extensió i~\(\alpha\in\FF\) un element. Aleshores
		\[\ker(\ev_{\alpha})=\begin{cases}
			(0) \\
			(p(x)), & p(x)\text{ és un polinomi mònic irreductible a }\KK[x]
		\end{cases}\]
		\begin{proof}
			Suposem que~\(\ker(\ev_{\alpha})\neq(0)\). Per la proposició \myref{prop:el nucli d'un morfisme entre anells és ideal, la imatge d'un morfisme entre anells és subanell} i l'exemple \myref{ex:el morfisme avaluació és un morfisme d'anells} tenim que~\(\ker(\ev_{\alpha})\) és un ideal, i com que, per hipòtesi,~\(\KK\) és un cos, tenim que~\(\KK[x]\) és un domini d'ideals principals. %REF
			Per tant, per la definició de \myref{def:domini d'ideals principals} tenim que~\(\ker(\ev_{\alpha})\) és un ideal principal, i per la definició d'\myref{def:ideal principal} trobem que existeix un~\(a\in\KK[x]\) tal que~\(\ker(\ev_{\alpha})=(a)\).
			
			Tenim que~\(a\) ha de ser un polinomi mònic. Suposem que~\(a\in\KK\). Trobem per la definició \myref{ex:el morfisme avaluació és un morfisme d'anells} que~\(\ev_{\alpha}(a)=a\), i com que hem suposat que~\(\ker(\ev_{\alpha})\neq(0)\), trobem que~\(a\neq0\), i per tant~\(\ev_{\alpha}(a)=a\neq0\), trobant que~\(a\notin\ker(\ev_{\alpha})\), arribant a contradicció. Per tant~\(a\in\KK[x]\setminus\KK\). Suposem que~\(a=\lambda p(x)\), amb~\(\lambda\in\KK\),~\(\lambda\neq0\) i~\(p\in\KK[x]\setminus\KK\). Per l'exemple \myref{ex:el morfisme avaluació és un morfisme d'anells} tenim que~\(\ev_{\alpha}\) és un morfisme d'anells, i per la definició de \myref{def:morfisme entre anells} trobem que
			\begin{align*}
				\ev_{\alpha}(\lambda p(x))&=\ev_{\alpha}(\lambda)\ev_{\alpha}(p(x))\\
				&=\lambda\ev_{\alpha}(p(x)). \tag{\ref{ex:el morfisme avaluació és un morfisme d'anells}}
			\end{align*}
			Ara bé, com que hem suposat que~\(\ker(\ev_{\alpha})=(\lambda p(x))\) tenim~\(\ev_{\alpha}(\lambda p(x))=0\) i, com que hem suposat que~\(\lambda\neq0\), ha de ser~\(\ev_{\alpha}(p(x))=0\), i tenim que
			\[\ker(\ev_{\alpha})=(p(x)).\]
			
			Veiem ara que~\(p(x)\) és irreductible. Pel \myref{thm:Primer Teorema de l'isomorfisme entre anells} trobem que
			\[\KK[x]/(p(x))=\KK[x]/\ker(\ev_{\alpha})\cong\Ima(\ev_{\alpha}).\]
			
			Com que~\(\FF\) és un cos tenim que~\(\Ima(\ev_{\alpha})\) és un domini. %REF
			Com que~\(\Ima(\ev_{\alpha})\cong\KK[x]/(p(x))\) és un domini, tenim per la proposició \myref{prop:R/I domini d'integritat sii I ideal primer} que~\((p(x))\) és un ideal primer, i per l'observació \myref{obs:ideals primer iff primer} trobem que~\(p(x)\) és primer, i per la proposició \myref{prop:en un DI un primer és un irreductible} tenim que~\(p(x)\) és irreductible.
		\end{proof}
	\end{lemma}
	\begin{definition}[Element algebraic i element transcendent]
		\labelname{element algebraic}\label{def:element algebraic}
		\labelname{element transcendent}\label{def:element transcendent}
		
	\end{definition}
	\begin{definition}[Grau d'una extensió]
		\labelname{grau d'una extensió}\label{def:grau d'una extensió}
		Sigui~\(\KK\extensio\FF\) una extensió. Aleshores direm que la dimensió del \(\KK\)-espai vectorial~\(\FF\) és el grau de l'extensió i el denotarem com~\(\grauExtensio{\KK}{\FF}\).
		
		Aquesta definició té sentit per la proposició~\myref{prop:un subcòs és un espai vectorial}.
	\end{definition}
	\begin{example}
		Siguin~\(\KK\) un cos i~\(p(x)\in\KK[x]\) un polinomi irreductible. Denotem~\(\FF=\KK[x]/p(x)\). Aleshores
		\[\grauExtensio{\KK}{\FF}=\grau(p(x)).\]
		\begin{solution}
			Aquest enunciat té sentit per l'exemple \myref{ex:el cos de polinomis és una extensió}. %FER
		\end{solution}
	\end{example}
	\begin{theorem}[Fórmula de les Torres]
		\labelname{fórmula de les torres}\label{thm:fórmula de les torres}
		Siguin~\(\EE\extensio\FF\) i~\(\FF\extensio\KK\) dues extensions de cossos. Aleshores
		\[\grauExtensio{\EE}{\KK}=\grauExtensio{\EE}{\FF}\grauExtensio{\FF}{\KK}.\]
		\begin{proof}
			Si~\(\grauExtensio{\EE}{\FF}=\infty\) ó~\(\grauExtensio{\FF}{\KK}=\infty\) tenim que~\(\grauExtensio{\EE}{\KK}=\infty\).
			
			Suposem doncs que~\(\grauExtensio{\EE}{\FF}=n\) i~\(\grauExtensio{\FF}{\KK}=m\). %TODO
		\end{proof}
	\end{theorem}
	\begin{definition}[Torre]
		\labelname{torre}\label{def:torre de cossos}
		Siguin~\(\KK_{1}\subseteq\KK_{2}\subseteq\dots\subseteq\KK_{1}\) cossos. Aleshores direm que
		\[\KK_{1}\subseteq\KK_{2}\subseteq\dots\subseteq\KK_{1}\]
		és una torre.
	\end{definition}
	\begin{corollary}
		\label{cor:fórmula de les torres}
		Sigui~\(\KK_{1}\subseteq\KK_{2}\subseteq\dots\subseteq\KK_{1}\) una torre. Aleshores
		\[\grauExtensio{\KK_{n}}{\KK_{1}}=\grauExtensio{\KK_{n}}{\KK_{n-1}}\cdots\grauExtensio{\KK_{2}}{\KK_{1}}\]
		\begin{proof}
			Conseqüència de la~\myref{thm:fórmula de les torres}.
		\end{proof}
	\end{corollary}
	\subsection{Extensions algebraiques}
	\begin{definition}[Extensió algebraica]
		Sigui~\(\FF\extensio\KK\) una extensió de cossos tal que per a tot~\(\alpha\in\FF\), tenim que~\(\alpha\) és algebraic sobre~\(\KK\). Aleshores direm que l'extensió~\(\FF\extensio\KK\) és algebraica.
	\end{definition}
\end{document}
