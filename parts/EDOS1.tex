\documentclass[../Apunts.tex]{subfiles}

\begin{document}
\chapter{Equacions diferencials de primer ordre en una variable}
	\section{Espai de funcions contínues i acotades}
	\subsection{L'espai de funcions contínues i acotades és complet}
	\begin{notation}[Espai de funcions contínues i acotades]
		\labelname{espai de funcions contínues i acotades}\label{notation:espai de funcions contínues i acotades}
		Siguin \(X\) amb la topologia \(\tau\) un espai topològic i \(T\) un compacte de \(\mathbb{R}^{n}\). Aleshores denotem
		\[\mathcal{C}_{b}(X,T)=\{f(x)\mid f(x)\text{ és una funció contínua i acotada de }X\text{ a }T\}.\]
	\end{notation}
	\begin{definition}[Norma d'una funció funció]
		\labelname{norma d'una funció funció}\label{def:norma d'una funció funció}
		Siguin \(X\) amb la topologia \(\tau\) un espai topològic, \(T\) un compacte de \(\mathbb{R}^{n}\) i \(f(x)\) una funció de \(\mathcal{C}_{b}(X,T)\). Aleshores definim
		\[\norm{f}=\sup_{x\in X}\norm{f(x)}\]
		com la norma de \(f\).
	\end{definition}
%	\begin{proposition}
%		\labelname{l'espai de funcions contínues i acotades és un espai vectorial}\label{prop:l'espai de funcions contínues i acotades és un espai vectorial}
%		Sigui \(X\) amb la topologia \(\tau\) un espai topològic. Aleshores \(\mathcal{C}_{b}(X)\) és un espai vectorial.
%		\begin{proof}
%			%TODO
%		\end{proof}
%	\end{proposition}
%	\begin{proposition}
%		\labelname{l'espai de funcions contínues i acotades és un espai mètric}\label{prop:l'espai de funcions contínues i acotades és un espai mètric}
%		Sigui \(X\) amb la topologia \(\tau\) un espai topològic. Aleshores \(\mathcal{C}_{b}(X)\) és un espai mètric.
%		\begin{proof}
%			%TODO
%		\end{proof}
%	\end{proposition}
	\begin{definition}[Successió de Cauchy]
		\labelname{successió de Cauchy}\label{def:successió de Cauchy amb normes}
		Sigui \((a_{n})_{n\in\mathbb{N}}\) una successió tal que per a tots naturals \(n\) i \(m\) existeix un real \(\varepsilon>0\) tal que
		\[\norm{a_{n}-a_{m}}<\varepsilon.\]
		Aleshores direm que \((a_{n})_{n\in\mathbb{N}}\) és de Cauchy.
	\end{definition}
	\begin{definition}[Espai mètric complet]
		\labelname{espai mètric complet}\label{def:espai mètric complet}
		Siguin \(X\) amb la distància \(\distancia\) un espai mètric tal que per a tota successió \((a_{n})_{n\in\mathbb{N}}\) de Cauchy tenim que
		\[a=\lim_{n\to\infty}a_{n}\]
		és un element de \(X\). Aleshores direm que \(X\) és complet.
	\end{definition}
	\begin{definition}[Distància entre funcions]
		\labelname{distància entre funcions}\label{def:distància entre funcions acotades}
		Siguin \(X\) amb la topologia \(\tau\) un espai topològic, \(T\) un compacte de \(\mathbb{R}^{n}\) i \(f\) i \(g\) dues funcions de \(\mathcal{C}_{b}(X,T)\). Aleshores definim
		\[\distancia(f,g)=\norm{f(x)-g(x)}\]
		com la distància entre \(f\) i \(g\) en \(\mathcal{C}_{b}(X,T)\).
	\end{definition}
	\begin{lemma}
		\label{thm:l'espai de funcions contínues i acotades és un espai mètric}
		Siguin \(X\) amb la topologia \(\tau\) un espai topològic i \(T\) un compacte de \(\mathbb{R}^{n}\). Aleshores \(\mathcal{C}_{b}(X,T)\) amb la distància \(\distancia\) és espai mètric.
		\begin{proof}
			%TODO
		\end{proof}
	\end{lemma}
	\begin{notation}
		Siguin \(X\) amb la topologia \(\tau\) un espai topològic i \(T\) un compacte de \(\mathbb{R}^{n}\). Aleshores denotarem per \(\mathcal{C}_{b}(X,T)\) l'espai mètric de amb la distància entre funcions \(d\).
		
		Observem que això té sentit pel lemma \myref{thm:l'espai de funcions contínues i acotades és un espai mètric}.
	\end{notation}
	\begin{theorem}
		\label{thm:l'espai de funcions contínues i acotades és complet}
		Siguin \(X\) amb la topologia \(\tau\) un espai topològic i \(T\) un compacte de \(\mathbb{R}^{n}\). Aleshores \(\mathcal{C}_{b}(X,T)\) és espai mètric complet.
		\begin{proof}
			Sigui \((f_{n})_{n\in\mathbb{N}}\) una successió de Cauchy de \(\mathcal{C}_{b}(X,T)\). Per la definició de \myref{def:successió de Cauchy amb normes} tenim que per a tots naturals \(n\) i \(m\) existeix un real \(\varepsilon>0\) tal que
			\[\norm{f_{n}-f_{m}}<\varepsilon.\]
			Per la definició de \myref{def:distància entre funcions acotades} trobem que
			\[\sup_{x\in X}\abs{f_{n}(x)-f_{m}(x)}<\varepsilon\]
			i trobem que per a tot \(x\) de \(X\) es satisfà
			\[\abs{f_{n}(x)-f_{m}(x)}<\varepsilon\]
			i per la definició de \myref{def:successió de Cauchy amb normes} trobem que la successió \((f_{n}(x))_{n\in\mathbb{N}}\) és de Cauchy, i per la definició de \myref{def:límit} trobem que la successió \((f_{n}(x))_{n\in\mathbb{N}}\) és convergent. Denotem
			\begin{equation}
				\label{thm:l'espai de funcions contínues i acotades és complet:eq1}
				\lim_{n\to\infty}f_{n}(x)=f(x).
			\end{equation}
			
			Sigui \(n_{0}\) un natural. Per la definició de \myref{def:límit} tenim que existeix un real \(\varepsilon>0\) tal que per a tot \(n\) i \(m\) enters amb \(n,m\geq n_{0}\)
			\[\abs{f_{n}(x)-f_{m}(x)}<\frac{\varepsilon}{2}.\]
			Ara bé, per \eqref{thm:l'espai de funcions contínues i acotades és complet:eq1} tenim que
			\[\lim_{m\to\infty}\abs{f_{n}(x)-f_{m}(x)}=\abs{f_{n}(x)-f(x)}.\]
			Per tant tenim que
			\[\abs{f_{n}(x)-f(x)}\leq\frac{\varepsilon}{2}<\varepsilon\]
			i per la definició de \myref{def:convergència uniforme} tenim que \((f_{n})_{n\in\mathbb{N}}\) convergeix uniformement a \(f(x)\), i per la definició de \myref{def:espai mètric complet} tenim que \(\mathcal{C}_{b}(X,T)\) és complet.
		\end{proof}
	\end{theorem}
	\subsection{Aplicacions contractives i punts fixes}
	\begin{definition}[Aplicació contractiva]
		\labelname{aplicació contractiva}\label{def:aplicació contractiva}
		Siguin \(X\) amb la distància \(\distancia\) un espai mètric i \(T\colon X\longrightarrow X\) una aplicació tal que existeixi un real \(0<k<1\) satisfent, per a tots \(x\) i \(y\) de \(X\),
		\[\distancia(T(x),T(y))\leq k\distancia(x,y).\]
		Aleshores direm que \(T\) és una aplicació contractiva sobre X.
	\end{definition}
	\begin{observation}
		\label{obs:les aplicacions contractives són uniformement contínues}
		Siguin \(X\) amb la distància \(\distancia\) un espai mètric i \(T\) una aplicació contractiva sobre \(X\). Aleshores \(T\) és uniformement contínua.
	\end{observation}
	\begin{definition}[Punt fix]
		\labelname{punt fix d'una aplicació contractiva}\label{def:punt fix d'un aplicació contractiva}
		Siguin \(X\) amb la distància \(\distancia\) un espai mètric, \(T\) una aplicació contractiva sobre \(X\) i \(a\) un punt tal que \(\distancia(a,T(a))=0\). Aleshores direm que \(a\) és un punt fix de \(T\).
	\end{definition}
	\begin{theorem}
		\label{thm:les aplicacions contractives tenen un únic punt fix}
		Siguin \(X\) amb la distància \(\distancia\) un espai mètric complet i \(T\) una aplicació contractiva sobre \(X\). Aleshores \(T\) té un únic punt fix.
		\begin{proof}
			Per la definició d'\myref{def:aplicació contractiva} tenim que existeix un real \(0<k<1\) tal que per a tots \(x\) i \(y\) de \(X\) es satisfà
			\begin{equation}
				\label{thm:les aplicacions contractives tenen un únic punt fix:eq1}
				\distancia(T(x),T(y))\leq k\distancia(x,y).
			\end{equation}
			
			Veiem que si existeix un punt fix aquest és únic. Suposem que existeixen dos punts \(a\) i \(b\) de \(X\) tals que \(a\) i \(b\) són punts fixos de \(T\). Aleshores tenim que
			\[\distancia(a,b)=\distancia(T(a),T(b)),\]
			i per \eqref{thm:les aplicacions contractives tenen un únic punt fix:eq1} tenim que
			\[\distancia(a,b)\leq k\distancia(a,b).\]
			Ara bé, com que \(0<k<1\), ha de ser \(\distancia(a,b)=0\), i per tant trobem que \(a=b\).
			
			Veiem ara que existeix un punt fix. Prenem un element \(x\) de \(X\) i considerem la successió \(\{T^{n}(x)\}_{n\in\mathbb{N}}\). Sigui \(r\)un natural. Aleshores tenim que per a tot \(n\) natural
			\begin{align*}
				\distancia(T^{n+r}(x),T^{n}(x))&\leq k\distancia(T^{n+r-1}(x),T^{n-1}(x))\tag{\myref{thm:les aplicacions contractives tenen un únic punt fix:eq1}}\\
				&\leq k^{2}\distancia(T^{n+r-2}(x),T^{n-2}(x))\tag{\myref{thm:les aplicacions contractives tenen un únic punt fix:eq1}}\\
				&~\vdots \\
				&\leq k^{n}\distancia(T^{r}(x),x)\tag{\myref{thm:les aplicacions contractives tenen un únic punt fix:eq1}}\\
				&\leq
				 k^{n}(\distancia(T^{r}(x),T^{r-1}(x))+\distancia(T^{r-1}(x),T^{r-2}(x))+\dots\\
				&\quad\dots+\distancia(T(x),x))\\
				&k^{n}\left(k^{r-1}\distancia(T(x),x)+k^{r-2}\distancia(T(x),x)+\dots+\distancia(T(x),x)\right)\\
				&=k^{n}(1+k+k^{2}+\dots+k^{r+1})\distancia(T(x),x)\\
				&<\frac{k^{n}}{1-k}\distancia(T(x),x).\tag{\myref{ex:sèries geomètriques}}
			\end{align*}
			Ara bé, com que \(0<k<1\) tenim que existeix un real \(\varepsilon>0\) tal que
			\[\distancia(T^{n+r}(x),T^{n}(x))<\varepsilon,\]
			i per la definició de \myref{def:successió de Cauchy amb normes} tenim que la successió \(\{T^{n}(x)\}_{n\in\mathbb{N}}\) és una successió de Cauchy, i com que, per hipòtesi, \(X\) és complet, tenim que existeix un element \(p\) de \(X\) tal que
			\begin{equation}
				\label{thm:les aplicacions contractives tenen un únic punt fix:eq2}
				p=\lim_{n\to\infty}T^{n}(x).
			\end{equation}
			
			Considerem el límit
			\[\lim_{n\to\infty}T(T^{n}(x)).\]
			Com que, per hipòtesi, \(T\) és contínua, tenim que %REF
			\[\lim_{n\to\infty}T(T^{n}(x))=T\left(\lim_{n\to\infty}T^{n}(x)\right)\]
			i per \eqref{thm:les aplicacions contractives tenen un únic punt fix:eq2} trobem que
			\[\lim_{n\to\infty}T(T^{n}(x))=T(p).\]
			
			Ara bé, tenim que 
			\[\lim_{n\to\infty}T(T^{n}(x))=\lim_{n\to\infty}T^{n+1}(x)\]
			i per tant trobem que
			\[T(p)=p,\]
			i per la definició de \myref{def:punt fix d'un aplicació contractiva} hem acabat.
		\end{proof}
	\end{theorem}
	\begin{corollary}
		\label{cor:si una funció té una potència contractiva aleshores aquesta té un únic punt fix}
		Siguin \(X\) amb la distància \(\distancia\) un espai mètric complet i \(T\) una aplicació de \(X\) en \(X\) tal que existeix un natural \(k\) satisfent que \(T^{k}\) sigui una aplicació contractiva sobre \(X\). Aleshores \(T\) té un únic punt fix.
		\begin{proof}
			Suposem que existeixen dos punts \(x\) i \(y\) de \(X\) tals que \(x\) i \(y\) són punts fixos de \(T\). Aleshores per la definició de \myref{def:punt fix d'un aplicació contractiva} tenim que
			\[T^{k}(a)=a\quad\text{i}\quad T^{k}(b)=b.\]
			Ara bé, pel Teorema \myref{thm:les aplicacions contractives tenen un únic punt fix} tenim que ha de ser \(a=b\), i per tant aquest punt és únic. % CONTINUAR
			
			Veiem ara que
		\end{proof}
	\end{corollary}
	\section{Teoremes d'existència i unicitat}
	\subsection{Equacions diferencials ordinàries}
	\begin{definition}[Funció Lipschitz respecte la segona variable]
		\labelname{funció Lipschitz respecte la segona variable}\label{def:funció Lipschitz respecte la segona variable}
		Siguin \(\Omega=\mathbb{R}\times\mathbb{R}^{n}\) un conjunt, \(K>0\) un real i \(f\colon\Omega\longrightarrow\mathbb{R}^{n}\) una funció tal que
		\[\norm{f(t,x)-f(t,y)}\leq K\norm{x-y}\]
		per a tot \((t,x)\) i \((t,y)\) de \(\Omega\). Aleshores direm que \(f\) és Lipschitz respecte la segona variable.
	\end{definition}
	\begin{theorem}[Teorema de Picard]
		\labelname{Teorema de Picard}\label{thm:Teorema de Picard}
		Siguin \(a\), \(b\) i \(t_{0}\) dos reals, \(x_{0}\) un punt de \(\mathbb{R}^{n}\), \(\Omega=(t_{0}-a,t_{0}+a)\times\bola(x_{0},b)\) un conjunt, \(f\colon\Omega\longrightarrow\mathbb{R}^{n}\) una funció contínua i Lipschitz respecte la segona variable i
		\[M=\max_{(t,x)\in\Omega}\norm{f(t,x)}.\]
		Aleshores el problema de Cauchy
		\[\begin{cases*}
			\displaystyle\frac{\partial u(x)}{\partial x}=f(t,x) \\
			\displaystyle u(t_{0})=x_{0}
		\end{cases*}\]
		té una única solució sobre l'interval
		\[\left(t_{0}-\min\left\{a,\frac{b}{M}\right\},t_{0}+\min\left\{a,\frac{b}{M}\right\}\right).\]
		\begin{proof}
			Denotem \(X=\mathcal{C}_{b}((t_{0}-a,t_{0}+a),\bola(x_{0},b))\) i definim una aplicació
			\begin{align*}
				T\colon X&\longrightarrow X \\
				u(x)&\longmapsto x_{0}+\int_{t_{0}}^{t}f(s,u(s))\diff s.
			\end{align*}
			Observem que una funció \(u(x)\) és solució del problema de Cauchy de l'enunciat si i només si %REVISAR
			\[T(u(x))=u(x).\]
			
			Tenim que si \(u(x)\) és un element de \(X\), aleshores \(T(u(x))\) també pertany a \(X\), ja que si prenem un \(t\) de \((t_{0}-a,t_{0}+a)\) trobem que
			\begin{align*}
				\norm{T(u(x))(t)-x_{0}}&=\norm{x_{0}-x_{0}+\int_{t_{0}}^{t}f(s,u(s))\diff s}\\
				&=\norm{\int_{t_{0}}^{t}f(s,u(s))\diff s}\\
				&\leq\abs{\int_{t_{0}}^{t}\norm{f(s,u(s))}\diff s}\tag{\myref{thm:la norma d'una integral és menos que l'integral de la norma}}\\
				&\leq M\abs{t-t_{0}}\leq b.
			\end{align*}
			i per tant tenim que
			\[\norm{T(u(x))(t)-x_{0}}\leq b.\]
			Per la definició de \myref{def:bola} trobem que \(T(u(x))(t)\) és un element de \(\bola(x_{0},b)\), i per tant \(T(u(x))\) és un element de \(X\) per a tot \(u(x)\) de \(X\).
			
			Per hipòtesi tenim que \(f\) és Lipschitz respecte la segona variable, i per la definició de \myref{def:funció Lipschitz respecte la segona variable} trobem que existeix un real \(K>0\) tal que
			\begin{equation}
				\label{thm:Teorema de Picard:eq1}
				\norm{f(t,x)-f(t,y)}\leq K\norm{x-y}.
			\end{equation}
			Volem veure que per a tot \(m\) natural, \(u(x)\) i \(v(x)\) funcions de \(X\) i \(t\) de \((t_{0}-a,t_{0}+a)\) tenim
			\[\norm{T^{m}(u(x))(t)-T^{m}(v(x))(t)}\leq\frac{K^{m}}{m!}\abs{t-t_{0}}^{m}\distancia(u(x),v(x)).\]
			Ho fem per inducció. Veiem el cas \(m=1\). Tenim
			\begin{align*}
				\norm{T(u(x))(t)-T(v(x))(t)}&=\norm{\int_{t_{0}}^{t}{f(s,u(s))}\diff s-\int_{t_{0}}^{t}{f(s,v(s))}\diff s}\\
				&=\norm{\int_{t_{0}}^{t}\left({f(s,u(s))}-{f(s,v(s))}\right)\diff s}\tag{\ref{prop:propietats basiques multiple integrals Riemann definides}}\\
				&\leq\abs{\int_{t_{0}}^{t}\norm{f(s,u(s))-f(s,v(s))}\diff s}\tag{\myref{thm:la norma d'una integral és menos que l'integral de la norma}}\\
				&\leq\abs{\int_{t_{0}}^{t}K\norm{u(s)-v(s)}\diff s}\tag{\ref{thm:Teorema de Picard:eq1}}\\
				&=K\abs{\int_{t_{0}}^{t}\norm{u(s)-v(s)}\diff s}\tag{\ref{prop:propietats basiques multiple integrals Riemann definides}}\\
				&\leq K\abs{t-t_{0}}\distancia(u(x),v(x)).
			\end{align*}
			Suposem ara que l'hipòtesi és certa per a \(m-1\) fix. És a dir, suposem que
			\begin{equation}
				\label{thm:Teorema de Picard:eq2}
				\norm{T^{m-1}(u(x))(t)-T^{m-1}(v(x))(t)}\leq\frac{K^{m-1}}{(m-1)!}\abs{t-t_{0}}^{m-1}\distancia(u(x),v(x))
			\end{equation}
			i veiem que també és cert per \(m\). Tenim pel Teorema \myref{thm:la norma d'una integral és menos que l'integral de la norma} que
			\begin{align*}
				\norm{T^{m}(u)(t)-T^{m}(v)(t)}&=\norm{\int_{t_{0}}^{t}\left(f(s,T^{m-1}(u)(s))-f(s,T^{m-1}(v)(s))\right)\diff s}\\
				&\leq\abs{\int_{t_{0}}^{t}\norm{f(s,T^{m-1}(u)(s))-f(s,T^{m-1}(v)(s))}\diff s}\\
				&\leq\abs{\int_{t_{0}}^{t}K\norm{T^{m-1}(u)(s)-T^{m-1}(v)(s)}\diff s}\tag{\ref{thm:Teorema de Picard:eq1}}\\
				&\leq\abs{\int_{t_{0}}^{t}\frac{K^{m}}{(m-1)!}\abs{s-t_{0}}^{m-1}\distancia(u(x),v(x))\diff s}\tag{\ref{thm:Teorema de Picard:eq2}}\\
				&=\frac{K^{m}}{(m-1)!}\distancia(u(x),v(x))\int_{t_{0}}^{t}\abs{s-t_{0}}^{m-1}\diff s\tag{\ref{prop:propietats basiques multiple integrals Riemann definides}}\\
				&=\frac{K^{m}}{m!}\abs{t-t_{0}}^{m}\distancia(u(x),v(x)).%REF integrar polinomis
			\end{align*}
			Per tant, pel \myref{thm:principi d'inducció} tenim que per a tot \(m\) natural, \(u(x)\) i \(v(x)\) funcions de \(X\) i \(t\) de \((t_{0}-a,t_{0}+a)\) es satisfà
			\[\norm{T^{m}(u(x))(t)-T^{m}(v(x))(t)}\leq\frac{K^{m}}{m!}\abs{t-t_{0}}^{m}\distancia(u(x),v(x)),\]
			com volíem veure.
			
			Ara bé, tenim que
			\[\lim_{m\to\infty}\frac{K^{m}}{m!}\abs{t-t_{0}}^{m}=0,\]%REF
			i per la definició de \myref{def:límit} tenim que existeix un natural \(n_{0}\) tal que per a tot \(n>n_{0}\) es satisfà
			\[0<{\frac{K^{m}}{m!}\abs{t-t_{0}}^{m}}<1.\]
			
			Per tant, per la definició d'\myref{def:aplicació contractiva} tenim que per a tot \(n>n_{0}\) l'aplicació \(T^{n}\) és una aplicació contractiva, i pel coro{\lgem}ari \myref{cor:si una funció té una potència contractiva aleshores aquesta té un únic punt fix} tenim que existeix una única funció \(u(x)\) tal que
			\[T(u(x))=u(x).\qedhere\]
		\end{proof}
	\end{theorem}
	
	
	
	
	
	
	
	
	
	\begin{definition}[Equació diferencial ordinària d'ordre \(n\)]
		\labelname{equació diferencial ordinària d'odre \ensuremath{n}}\label{def:equació diferencial ordinària d'ordre n}
		\labelname{solució de l'equació diferencial ordinària}\label{def:solució de l'equació diferencial ordinària}
		Siguin \(\Omega=\mathbb{R}\times\mathbb{R}^{n}\) un conjunt i \(F\colon\Omega\longrightarrow\mathbb{R}^{m}\), \(u\colon\mathbb{R}\longrightarrow\mathbb{R}^{m}\) dues funcions tals que
		\[\frac{\partial^{n}u(x)}{\partial x^{n}}=F\left(t,u(x),\frac{\partial u(x)}{\partial x},\dots,\frac{\partial^{n-1}u(x)}{\partial x^{n-1}}\right).\]
		Aleshores direm que
		és una equació diferencial ordinària. També direm que \(t\) és la variable temporal, \(u(x)\) la variable espacial.
	\end{definition}
	\begin{proposition}
		\label{prop:edos1:1}
		Siguin \(I\) un interval tancat i acotat de \(\mathbb{R}\) i \(f\colon I\longrightarrow\mathbb{R}^{m}\) una funció contínua. Aleshores l'equació diferencial ordinària d'ordre \(1\)
		\begin{equation}
			\label{edos1:prop1:eq1}
			\frac{\partial u(x)}{\partial x}=f(u(x)).
		\end{equation}
		té una única solució de la forma
		\[u(x)=\int_{x_{0}}^{x}f(u(t))\diff t.\]
		\begin{proof}
			Observem primer que pel \myref{thm:Weierstrass màxims i mínims múltiples variables} la funció \(f(x)\) està acotada en \(I\), i pel Teorema \myref{thm:Contínua + acotada implica integrable Riemann} tenim que la funció \(f(x)\) és integrable Riemann, i per tant, pel \myref{thm:Teorema Fonamental del Càlcul}, trobem que per a tot \(x_{0}\) de \(I\)
			\[u(x)=\int_{x_{0}}^{x}f(u(t))\diff t\]
			és una solució de l'equació diferencial ordinària \eqref{edos1:prop1:eq1}.
			
			Aquesta demostració també ens serveix per veure la unicitat.
		\end{proof}
	\end{proposition}
	\begin{observation} %REVISAR
		Observem que, en la proposició \myref{prop:edos1:1}, si la funció \(f\) no és integrable en \(I\) aleshores l'equació diferencial \eqref{edos1:prop1:eq1} no té solució. %Integrable Riemann?
	\end{observation}
	\begin{proposition}[Problema de Cauchy]
		\labelname{}\label{prop:problema de Cauchy}
		Siguin \((a,b)\) un interval obert de \(\mathbb{R}\), \(f\colon(a,b)\longrightarrow\mathbb{R}\) una funció contínua tal que \(f(x)\neq0\) per a tot \(x\) de \((a,b)\), \(x_{0}\) un element de \((a,b)\) i \(t_{0}\) un real. Aleshores el sistema
		\begin{equation}
			\label{prop:problema de Cauchy:eq1}
			\begin{cases}
				\displaystyle \frac{\partial u(x)}{\partial x}=f(x) \\
				\displaystyle u(t_{0})=x_{0}
			\end{cases}
		\end{equation}
		té una única solució de la forma
		\[u(t)=F^{-1}(t-t_{0})\]
		on
		\[F(x)=\int_{x_{0}}^{x}\frac{\diff t}{f(t)}.\]
		\begin{proof}
			Observem que, per la definició de \myref{def:equació diferencial ordinària d'ordre n} l'equació
			\[\frac{\partial u(x)}{\partial x}=f(x)\]
			és una equació ordinària d'orde \(1\).
			
			Com que, per hipòtesi, tenim \(f(x)\neq0\) per a tot \(x\) de \((a,b)\) trobem que la funció \(\frac{1}{f(x)}\) és derivable %REF
			i per la proposició \myref{prop:Diferenciable implica contínua} trobem que \(\frac{1}{f(x)}\) és contínua. També tenim pel \myref{thm:Weierstrass màxims i mínims múltiples variables} que \(\frac{1}{f(x)}\) està acotada en \((a,b)\) i pel Teorema \myref{thm:Contínua + acotada implica integrable Riemann} trobem que \(\frac{1}{f(x)}\) és integrable en \((a,b)\). Definim doncs %NO és en (a,b). És en (x_{0},x)
			\[F(x)=\int_{x_{0}}^{x}\frac{\diff t}{f(t)}.\]
			Pel \myref{thm:Teorema Fonamental del Càlcul} tenim que
			\begin{equation}
				\label{prop:problema de Cauchy:eq2}
				\frac{\partial F(x)}{\partial x}=\frac{1}{f(x)},
			\end{equation}                
			i com que \(\frac{1}{f(x)}\neq0\) per a tot \(x\) de \((a,b)\) tenim que \(F(x)\) és invertible en \((a,b)\). Denotem amb \(F^{-1}\) la seva inversa.
			
			Ara bé, tenim que
			\begin{equation}
				\label{prop:problema de Cauchy:eq3}
				\frac{\partial F(u(x))}{\partial x}=1,
			\end{equation}
			ja que
			\begin{align*}
				\frac{\partial F(u(x))}{\partial x}&=\frac{\partial F(u(x))}{\partial u(x)}\frac{\partial u(x)}{\partial x}\tag{\myref{thm:regla de la cadena}}\\
				&=\frac{1}{f(u(x))}\frac{\partial u(x)}{\partial x}\tag{\ref{prop:problema de Cauchy:eq2}}\\
				&=\frac{1}{f(u(x))}f(u(x))=1.\tag{\ref{prop:problema de Cauchy:eq1}}\\
			\end{align*}
			I per tant, pel \myref{thm:Teorema Fonamental del Càlcul} trobem que
			\[\int_{t_{0}}^{t}\frac{\partial F(u(x))}{\partial x}\diff x=F(u(t))-F(u(t_{0})),\]
			i per \eqref{prop:problema de Cauchy:eq2} trobem \(F(u(t))-F(u(t_{0}))=t-t_{0}\)
			
			Ara bé, per hipòtesi, tenim que \(u(t_{0})=x_{0}\), i trobem
			\[F(x_{0})=\int_{x_{0}}^{x_{0}}\frac{1}{f(t)}\diff t=0.\]
			Per tant
			\[F(u(t))=t-t_{0},\]
			i com que, com ja hem vist, \(F(x)\) és invertible en \((F(a),F(b))\) trobem
			\[u(t)=F^{-1}(t-t_{0})\]
			per a tot \(t\) de \((t_{0}+F(a), t_{0}+F(b))\).
			
			Amb aquesta demostració també podem veure que aquesta és la única solució de l'equació diferencial.
		\end{proof}
	\end{proposition}
	\begin{example}
		\label{ex:edos problema de Cauchy 1}
		Sigui \(\alpha\) un real. Volem trobar una solució al sistema
		\begin{equation}
			\label{ex:edos problema de Cauchy 1:eq1}
			\begin{cases*}
				\frac{\partial u(x)}{\partial x}=\alpha x^{2} \\
				u(1)=1.
			\end{cases*}
		\end{equation}
		\begin{solution}
			Observem que, per la definició de \myref{def:equació diferencial ordinària d'ordre n} que l'equació \(\frac{\partial u(x)}{\partial x}=\alpha x^{2}\) és una equació diferencial ordinària d'odre \(1\). Per la proposició \myref{prop:problema de Cauchy} trobem que les solucions de \eqref{ex:edos problema de Cauchy 1:eq1} són de la forma
			\[u(x)=F^{-1}(x-1)\]
			on
			\[F(x)=\int_{x_{0}}^{x}\frac{\diff t}{\alpha t^{2}}.\]
			
			Per tant, tenim que
			\[F(x)=\frac{x-1}{\alpha x}\]
			i trobem
			\[u(x)=\frac{1}{1-\alpha (x-1)}.\qedhere\]
		\end{solution}
	\end{example}
	\begin{proposition}[Equacions de variables separades]
		\labelname{}\label{prop:equacions de variables separades}
		Siguin \((a,b)\) i \((t_{1},t_{2})\) dos intervals oberts de \(\mathbb{R}\), \(f\colon(a,b)\longrightarrow\mathbb{R}\) i \(g\colon(t_{1},t_{2})\longrightarrow\mathbb{R}\) dues funcions contínues amb \(f(x)\neq0\) per a tot \(x\) de \((a,b)\), \(x_{0}\) un element de \((a,b)\) i \(t_{0}\) un element de \((t_{1},t_{2})\). Aleshores el problema de Cauchy
		\begin{equation}
			\begin{cases}
				\displaystyle \frac{\partial u(x)}{\partial x}=g(x)f(u(x)) \\
				\displaystyle u(t_{0})=x_{0}.
			\end{cases}
		\end{equation}
		té una única solució de la forma
		\[u(x)=F^{-1}\left(\int_{t_{0}}^{x}g(t)\diff t\right)\]
		on
		\[F(x)=\int_{x_{0}}^{x}\frac{\diff t}{f(t)}.\]
		\begin{proof}
			Com que, per hipòtesi, \(f(x)\neq0\) per a tot \(x\) de \((a,b)\) definim \(h=\frac{1}{f(u(x))}\) i observem que
			\[h(u(x))\frac{\partial u(x)}{\partial x}=g(x)\]
			i per tant
			\[\int h(u(x))\frac{\partial u(x)}{\partial x}\diff {x}=\int g(x)\diff x.\]
			
			Ara bé, per la \myref{thm:regla de la cadena} i el \myref{thm:Teorema Fonamental del Càlcul} tenim que això és
			\[H(x)=G(x)+C\]
			on \(C\) és un real i
			\[H(x)=\int_{t_{0}}^{x}h(t)\diff t\quad\text{i}\quad G(x)=\int_{t_{0}}^{x}g(t)\diff t\]
			i per tant
			\[H(u(x))=G(x)+C\]
%			Podem seguir la solució de la proposició \myref{prop:problema de Cauchy} per trobar que les úniques solucions són
%			\[u(x)=F^{-1}\left(\int_{t_{0}}^{x}g(t)\diff t\right)\]
%			on
%			\[F(x)=\int_{x_{0}}^{x}\frac{\diff t}{f(t)}.\qedhere\]
		\end{proof}
	\end{proposition}
	\begin{example}
		\label{ex:equacions de variables separades}
		Volem trobar una solució al sistema
		\begin{equation}
			\label{ex:equacions de variables separades:eq1}
			\begin{cases*}
				\displaystyle \frac{\partial u(x)}{\partial x}=\frac{u(x)^{3}x}{\sqrt{1+x^{2}}} \\
				u(0)=1.
			\end{cases*}
		\end{equation}
		\begin{solution}
			Definim
			\[f(u(x))=u(x)^{-3}\quad\text{i}\quad g(x)=\frac{x}{\sqrt{1+x^{2}}}.\]
			Per tant tenim que
			\[f(u(x))\frac{\partial u(x)}{\partial x}=g(x)\]
			
			Observem que
			\[u(x)=\int_{x_{0}}^{x}\frac{yu(x)}{\sqrt{1+y^{2}}}\diff x\]
			Per la proposició \myref{prop:equacions de variables separades} tenim que les solucions al sistema \eqref{ex:equacions de variables separades:eq1} són de la forma
			\[u(t)=F^{-1}\left(\int_{t_{0}}^ {t}\frac{y}{\sqrt{1+y^{2}}}\diff y\right)\]
			on
			\[F(x)=\int_{x_{0}}^{x}\frac{\diff t}{u(t)^{3}}.\]
			Per tant tenim que
			\[\] %TODO ACABAR
		\end{solution}
	\end{example}
	\begin{proposition}[Equacions diferencials lineals]
		\labelname{}\label{prop:equacions diferencials lineals}
		Siguin \((t_{1},t_{2})\) un interval obert de \(\mathbb{R}\) i \(a\colon(t_{1},t_{2})\longrightarrow\mathbb{R}\), \(b\colon(t_{1},t_{2})\longrightarrow\mathbb{R}\) dues funcions contínues. Aleshores el problema de Cauchy
		\begin{equation}
			\label{prop:equacions diferencials lineals:eq1}
			\begin{cases}
				\displaystyle \frac{\partial u(x)}{\partial x}=a(t)u(x)+b(t)\\
				\displaystyle u(t_{0})=x_{0}.
			\end{cases}
		\end{equation}
		té una única solució de la forma
		\[u(x)=\left(x_{0}+\int_{t_{0}}^{t}b(s)e^{-\int_{t_{0}}^{s}a(\tau)\diff \tau}\diff s\right)e^{\int_{t_{0}}^{x}a(s)\diff s}.\]
		\begin{proof}
			Considerem el canvi de variable
			\begin{equation}
				\label{prop:equacions diferencials lineals:eq2}
				u(x)=c(x)e^{\int_{t_{0}}^{x}a(s)\diff s}.
			\end{equation}
			Aleshores pel \myref{thm:Teorema Fonamental del Càlcul} tenim %I ref de la derivada del producte
			\[\frac{\partial u(x)}{\partial x}=\frac{\partial c(x)}{\partial x}e^{\int_{t_{0}}^{x}a(s)\diff s}+c(x)a(x)e^{\int_{t_{0}}^{x}a(s)\diff s}\]
			i per \eqref{prop:equacions diferencials lineals:eq1} i \eqref{prop:equacions diferencials lineals:eq2} tenim
			\[a(t)u(x)+b(t)=\frac{\partial c(x)}{\partial x}e^{\int_{t_{0}}^{x}a(s)\diff s}+a(x)u(x).\]
			Per tant trobem
			\[\frac{\partial c(x)}{\partial x}=b(x)e^{\int_{t_{0}}^{x}a(s)\diff s}\]
			i tenim
			\begin{equation}
				\label{prop:equacions diferencials lineals:eq3}
				\begin{cases}
					\displaystyle \frac{\partial c(x)}{\partial x}=b(x)e^{\int_{t_{0}}^{x}a(s)\diff s}\\
					\displaystyle c(t_{0})=x_{0}.
				\end{cases}
			\end{equation}
			
			Aleshores per la proposició \myref{prop:problema de Cauchy} trobem que aquest problema té una única solució, i és de la forma
			\[c(t)=x_{0}+\int_{t_{0}}^{t}b(s)e^{-\int_{t_{0}}^{s}a(\tau)\diff \tau}\diff s\]
			i desfent el canvi de variables \eqref{prop:equacions diferencials lineals:eq2} trobem que les solucions són de la forma
			\[u(x)=\left(x_{0}+\int_{t_{0}}^{t}b(s)e^{-\int_{t_{0}}^{s}a(\tau)\diff \tau}\diff s\right)e^{\int_{t_{0}}^{x}a(s)\diff s}.\qedhere\]
		\end{proof}
	\end{proposition}
%	\printbibliography
\end{document}

% http://issc.uj.ac.za/downloads/problems/ordinary.pdf