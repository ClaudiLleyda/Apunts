\documentclass[../Apunts.tex]{subfiles}

\begin{document}
	\chapter{Teoria de grups}
	\section{Grups}
	\subsection{Propietats bàsiques dels grups}
	\begin{definition}[Grup]
		\labelname{grup}
		\label{def:grup}
		Siguin \(G\neq\emptyset\) un conjunt i \(\ast\colon G\times G\to G\) una operació que satisfà
		\begin{enumerate}
			\item Per a tot \(x,y,z\in G\)
			\[x\ast(y\ast z)=(x\ast y)\ast z.\]
			\item Existeix un \(e\in G\) tal que per a tot \(x\in G\)
			\[x\ast e=e\ast x=x.\]
			\item Per a cada \(x\in G\) existeix \(x'\) tal que
			\[x\ast x'=x'\ast x=e.\]
		\end{enumerate}
		Aleshores \(G\) és un grup amb la l'operació \(\ast\). També direm \(\ast\) dota al conjunt \(G\) d'estructura de grup.
	\end{definition}
	\begin{proposition}
		\label{prop:unicitat neutre del grup}
		Siguin \(G\) un grup amb l'operació \(\ast\) i \(e\in G\) tal que \(x\ast e=e\ast x=x\) per a tot \(x\in G\). Aleshores \(e\) és únic.
		\begin{proof}
			Suposem que existeix un altre element de \(G\) amb aquesta propietat, diguem-ne \(\hat{e}\in G\). Aleshores hauria de ser
			\[e\ast\hat{e}=e,\]
			però per hipòtesi
			\[e\ast\hat{e}=\hat{e}.\]
			Per tant, ha de ser \(e=\hat{e}\).
		\end{proof}
	\end{proposition}
	\begin{definition}[Element neutre d'un grup]
		\labelname{l'element neutre d'un grup}
		\label{def:l'element neutre del grup}
		Siguin \(G\) un grup amb l'operació \(\ast\) i \(e\) un element de \(G\) tal que \(x\ast e=e\ast x=x\) per a tot \(x\in G\). Aleshores direm que \(e\) és l'element neutre de \(G\).
		
		Aquesta definició té sentit per la proposició \myref{prop:unicitat neutre del grup}.
	\end{definition}
	\begin{notation}
		\label{notation:potencies per l'operació repetida en un grup}
		Donat un grup \(G\) amb l'operació \(\ast\) escriurem
		\[(x_{1}\ast x_{2})\ast x_{3}=x_{1}\ast x_{2}\ast x_{3}.\]
		
		També denotarem
		\[x^{n}=x\ast\overset{n)}{\dots}\ast x.\]
		
		Si denotem la conjugació del grup per \(+\) usarem la notació additiva i escriurem
		\[x_{1}+\dots+x_{n}\]
		per referir-nos a la conjugació de \(+\) amb si mateix \(n\) vegades.
		
		També denotarem
		\[nx=x+\overset{n)}{\cdots}+ x.\]
	\end{notation}
	\begin{proposition}
		\label{prop:podem tatxar pels costats en grups}
		Siguin \(G\) un grup amb l'operació \(\ast\) i \(a,b,c\) tres elements de \(G\). Aleshores
		\begin{enumerate}
			\item\label{enum:podem tatxar pels costats en grups 1} \(a\ast c=b\ast c\implica a=b\).
			\item\label{enum:podem tatxar pels costats en grups 2} \(c\ast a=c\ast b\implica a=b\).
		\end{enumerate}
		\begin{proof}
			Farem només la demostració del punt \eqref{enum:podem tatxar pels costats en grups 1} ja que l'altre és anàloga.
			
			Com que per hipòtesi \(G\) és un grup, per la definició de \myref{def:grup} tenim que existeix \(c'\) tal que \(c\ast c'=e\), on \(e\) és l'element neutre \(G\), i tenim
			\[a\ast c\ast c'=b\ast c\ast c',\]
			el que significa que
			\[a\ast e=b\ast e,\]
			i ens queda \(a=b\).
		\end{proof}
	\end{proposition}
	\begin{proposition}
		\label{prop:unicitat inversa en grups}
		Siguin \(G\) un grup amb l'operació \(\ast\) i element neutre \(e\) i \(a\) un element de \(G\). Aleshores existeix un únic \(a'\in G\) tal que
		\[a\ast a'=a'\ast a=e.\]
		\begin{proof}
			Notem que existeix un \(a'\in G\) que satisfà l'equació per la definició de \myref{def:grup}, i per tant la proposició té sentit.
			
			Suposem doncs que existeix \(a''\in G\) tal que
			\[a\ast a''=a''\ast a=e.\]
			Però aleshores tenim
			\[a\ast a''=e=a\ast a',\]
			i per la proposició \myref{prop:podem tatxar pels costats en grups} ha de ser \(a'=a''\), com volíem demostrar.
		\end{proof} 
	\end{proposition}
	\begin{definition}[Invers d'un element]
		\labelname{l'invers d'un element d'un grup}
		\label{def:l'invers d'un element d'un grup}
		Siguin \(G\) un grup amb l'operació \(\ast\) i element neutre \(e\) i \(a\) un element de \(G\). Per la definició de grup tenim que existeix un \(a'\in G\) tal que
		\[a\ast a'=a'\ast a=e.\]
		Aleshores direm que \(a'\) és l'invers de \(a\) en \(G\), i el denotarem per \(a^{-1}\).
		
		Aquesta definició té sentit per la proposició \myref{prop:unicitat inversa en grups} i la notació introduïda en \myref{notation:potencies per l'operació repetida en un grup}.
	\end{definition}
	\begin{proposition}
		\label{prop:l'invers de l'element neutre d'un grup és ell mateix}
		Sigui \(G\) un grup amb l'operació \(\ast\) i element neutre \(e\). Aleshores
		\[e^{-1}=e.\]
		\begin{proof}
			Per la definició de \myref{def:grup} tenim que
			\[e\ast e^{-1}=e^{-1}\ast e=e,\]
			i per ha de ser \(e^{-1}=e\).
		\end{proof}
	\end{proposition}
	\begin{proposition}
		\label{prop:grups:l'invers de l'invers d'un element es l'element}
		Siguin \(G\) un grup amb l'operació \(\ast\) i \(a\) un element de \(G\). Aleshores
		\[\left(a^{-1}\right)^{-1}=a.\]
		\begin{proof}
			Sigui \(e\) l'element neutre de \(G\). Com que \(\left(a^{-1}\right)^{-1}\) és l'invers de \(a^{-1}\) tenim
			\[\left(a^{-1}\right)^{-1}\ast a^{-1}=e\]
			però també tenim que
			\[a\ast a^{-1}=e.\]
			Per tant és
			\[a\ast a^{-1}=(a^{-1})^{-1}\ast a^{-1},\]
			i per la proposició \myref{prop:podem tatxar pels costats en grups} ha de ser
			\[a=\left(a^{-1}\right)^{-1}.\qedhere\]
		\end{proof}
	\end{proposition}
	\begin{proposition}
		\label{prop:invers de a b = b invers a invers}
		Siguin \(G\) un grup amb l'operació \(\ast\) i \(a,b\) dos elements de \(G\). Aleshores
		\[(a\ast b)^{-1}=b^{-1}\ast a^{-1}.\]
		\begin{proof}
			Sigui \(e\) l'element neutre de \(G\). Considerem
			\begin{align*}
			(b^{-1}\ast a^{-1})\ast(a\ast b)&=b^{-1}\ast a^{-1}\ast a\ast b\\
			&=b^{-1}\ast e\ast b\\
			&=b^{-1}\ast b=e.
			\end{align*}
			i de manera anàloga trobem
			\[(a\ast b)\ast(b^{-1}\ast a^{-1})=e.\]
			Així doncs, per la proposició \myref{prop:unicitat inversa en grups} tenim que \(a\ast b\) és l'inversa de \(b^{-1}\ast a^{-1}\), és a dir
			\[(a\ast b)^{-1}=b^{-1}\ast a^{-1}.\qedhere\]
		\end{proof}
	\end{proposition}
	\begin{lemma}
		\label{lema:solucions úniques en grups a equacions}
		Siguin \(G\) un grup amb l'operació \(\ast\) i \(a,b\) dos elements de \(G\). Aleshores existeixen \(x,y\in G\) únics tals que
		\[b\ast x=a\quad\text{i}\quad y\ast b=a.\]
		\begin{proof}
			Fem només una de les demostracions, ja que l'altre és anàloga. Com que per hipòtesi \(G\) és un grup, per la definició de \myref{def:grup} tenim que existeix \(b^{-1}\in G\) tal que \(b^{-1}\ast b=e\), on \(e\) és l'element neutre de \(G\). Per tant considerem
			\[b^{-1}\ast(b\ast x)=b^{-1}\ast a\]
			i per la definició de \myref{def:grup} tenim que això és equivalent a
			\[(b^{-1}\ast b)\ast x=b^{-1}\ast a,\]
			i de nou per la definició de grup, i per la definició de \myref{def:l'element neutre del grup},
			\[e\ast x=x=b^{-1}\ast a ;\]
			i la unicitat ve donada per la proposició \myref{prop:unicitat neutre del grup}.
		\end{proof}
	\end{lemma}
	\begin{theorem}
		\label{thm:definicions alternatives de grup}
		Siguin \(G\) un conjunt i \(\ast\colon G\times G\to G\) una operació binària que satisfà
		\(x\ast(y\ast z)=(x\ast y)\ast z\) per a tot \(x,y,z\in G\). Aleshores els següents enunciats són equivalents:
		\begin{enumerate}
			\item\label{enum:definicions alternatives de grup 1} \(G\) és un grup amb l'operació \(\ast\).
			\item\label{enum:definicions alternatives de grup 2} \(G\neq\emptyset\) i per a tot \(a,b\in G\) existeix uns únics \(x,y\in G\) tals que
			\[b\ast x=a\quad\text{i}\quad y\ast b=a.\]
			\item\label{enum:definicions alternatives de grup 3} Existeix \(e\in G\) tal que per a tot \(x\in G\) tenim \(x\ast e=x\) i existeix un \(x^{-1}\in G\) tal que \(x\ast x^{-1}=e\).
		\end{enumerate}
		\begin{proof}
			Comencem demostrant \eqref{enum:definicions alternatives de grup 1}\(\implica\)\eqref{enum:definicions alternatives de grup 2}. Suposem que \(G\) és un grup amb l'operació. Veiem que \(G\) no és buit per la definició de \myref{def:grup}, i la segona part és el lema \myref{lema:solucions úniques en grups a equacions}.
			
			Demostrem ara \eqref{enum:definicions alternatives de grup 2}\(\implica\)\eqref{enum:definicions alternatives de grup 3}. La primera part es pot veure fixant \(x\in G\). Pel punt \eqref{enum:definicions alternatives de grup 2} tenim que per a cada \(a\in G\) existeix un únic \(b\in G\) tal que
			\[a\ast b=x,\]
			i podem fer
			\[a\ast b\ast e=x\ast e\]
			i substituint ens queda
			\[x\ast e=x.\]
			Per veure la segona part notem que pel punt \eqref{enum:definicions alternatives de grup 2} tenim que per a tot \(x\in G\) existeix un \(a\in G\) tal que
			\[x\ast a=e,\]
			i aleshores \(a=x^{-1}\).
			
			Ara només ens queda veure \eqref{enum:definicions alternatives de grup 3}\(\implica\)\eqref{enum:definicions alternatives de grup 1}. Tenim que per a tot \(x\in G\) existeix un \(x^{-1}\) tal que \(x\ast x^{-1}=e\), i de la mateixa manera, existeix un \(y\in G\) tal que \(x^{-1}\ast y=e\). Per tant
			\begin{align*}
			e&=x^{-1}\ast y\\
			&=x^{-1}\ast e\ast y\\
			&=x^{-1}\ast x\ast x^{-1}\ast y\\
			&=x^{-1}\ast x\ast e=x^{-1}\ast x.
			\end{align*}
			Així tenim que per a tot \(x\in G\) es compleix \(x\ast x^{-1}=x^{-1}\ast x=e\), d'on podem veure que \(e\ast x=x\ast e\), i com que, per hipòtesi, l'operació \(\ast\) satisfà \(x\ast(y\ast z)=(x\ast y)\ast z\) per a tot \(x,y,z\in G\) es compleix la definició de \myref{def:grup} i tenim que \(G\) és un grup amb l'operació \(\ast\).
			
			Així tenim \eqref{enum:definicions alternatives de grup 1}\(\implica\)\eqref{enum:definicions alternatives de grup 2}\(\implica\)\eqref{enum:definicions alternatives de grup 3}\(\implica\)\eqref{enum:definicions alternatives de grup 1}, com volíem veure.
		\end{proof}
	\end{theorem}
	\subsection{Subgrups i subgrups normals}
	\begin{definition}[Subgrup]
		\labelname{subgrup}
		\label{def:subgrup}
		Siguin \(G\) un grup amb l'operació \(\ast\) i \(H\subseteq G\) un subconjunt de \(G\) tal que \(H\) sigui un grup amb l'operació \(\ast\). Aleshores diem que \(H\) és un subgrup de \(G\).
		
		També ho denotarem com \(H\leq G\).
	\end{definition}
	\begin{observation}
		\label{obs:l'element neutre d'un grup pertany als subgrups}
		\(e\in H\).
	\end{observation}
	\begin{proposition}
		\label{prop:condició equivalent a subgrup}
		Siguin \(G\) un grup amb l'operació \(\ast\) i element neutre \(e\) i \(H\) un subconjunt de \(G\). Aleshores \(H\) és un subgrup de \(G\) si i només si per a tot \(x,y\in H\) tenim que \(x\ast y^{-1}\in H\).
		\begin{proof}
			Demostrem primer que la condició és necessària (\(\implica\)). Això ho podem veure per la definició de \myref{def:grup}, ja que tenim que \(y^{-1}\) existeix i pertany a \(H\), i per tant \(x\ast y^{-1}\) també pertany a \(H\).
			
			Demostrem ara que la condició és suficient (\(\implicatper\)). Tenim que per a tot \(x\in H\) es compleix
			\[x\ast x^{-1}=e,\]
			i per tant \(e\in H\). També tenim que per a tot \(x\in H\) es compleix
			\[e\ast x^{-1}=x^{-1},\]
			i per tant \(x^{-1}\in H\).
			
			Ara només ens queda veure que \(\ast\) és tancat en \(H\); és a dir, que per a tot \(x,y\in H\) tenim \(x\ast y\in H\). Com que ja hem vist que \(y^{-1}\) existeix i pertany a \(H\), per la proposició \myref{prop:grups:l'invers de l'invers d'un element es l'element} tenim
			\[x\ast y^{-1}=x\ast y\]
			i per hipòtesi \(x\ast y\in H\).
			
			Per tant, per la definició de \myref{def:grup} tenim que \(H\) és un grup amb l'operació \(\ast\), i com que per hipòtesi \(H\subseteq G\) per la definició de \myref{def:subgrup} tenim que \(H\) és un subgrup de \(G\).
		\end{proof}
	\end{proposition}
	\begin{proposition}
		\label{prop:intersecció de subgrups es subgrup}
		Siguin \(G\) un grup amb l'operació \(\ast\) i element neutre \(e\), \(\{H_{i}\}_{i\in I}\) una família de subgrups de \(G\) i \(H=\bigcap_{i\in I}H_{i}\). Aleshores \(H\) és un subgrup de \(G\).
		\begin{proof}
			Ho demostrarem amb la proposició \myref{prop:condició equivalent a subgrup}. Tenim \(H\subseteq G\) i \(H\neq\emptyset\), ja que \(e\in H\). Comprovem ara que per a tot \(x,y\in H\) tenim \(x\ast y^{-1}\in H\). Tenim que si \(x,y\in H\), per la definició de \(H\), \(x,y\in H_{i\in I}\); i com que \(H_{i\in I}\) és un subgrup de \(G\), \(x\ast y^{-1}\in H_{i\in I}\) per la proposició \myref{prop:condició equivalent a subgrup}, i per tant \(x\ast y^{-1}\in H\), com volíem veure.
		\end{proof}
	\end{proposition}
	\begin{proposition}
		\label{prop:els subgrups generats per un conjunt existeixen}
		Siguin \(G\) un grup amb l'operació \(\ast\) i element neutre \(e\), \(S\neq\emptyset\) un subconjunt de \(G\), \(\{H_{i}\}_{i\in I}\) una família de subgrups de \(G\) tals que \(S\subseteq H_{i}\) per a tot \(i\in I\) i \(H=\bigcap_{i\in I}H_{i}\). Aleshores \(H\) i és un subgrup de \(G\).
		\begin{proof}
			Comprovem que \(H\) existeix, és a dir, que \(H\neq\emptyset\) tenim prou amb veure que \(e\in H\). Veiem que \(H\) és un subgrup de \(G\) ho podem veure per la proposició \myref{prop:intersecció de subgrups es subgrup}.
		\end{proof}
	\end{proposition}
	\begin{definition}[Mínim subgrup generat per un conjunt]
		\labelname{mínim subgrup generat per un conjunt}
		\label{def:mínim subgrup generat per un subconjunt}
		Siguin \(G\) un grup amb l'operació \(\ast\), \(S\) un subconjunt de \(G\), \(\{H_{i}\}_{i\in I}\) una família de subgrups de \(G\) tals que \(S\subseteq H_{i}\) per a tot \(i\in I\) i \(H=\bigcap_{i\in I}H_{i}\). Aleshores direm que el subgrup \(H\leq G\) és el mínim subgrup generat per \(S\) i ho denotarem amb \(\langle S\rangle\).
		
		Aquesta definició té sentit per la proposició \myref{prop:els subgrups generats per un conjunt existeixen}.
	\end{definition}
	\begin{proposition}
		\label{prop:forma grups cíclics}
		Siguin \(G\) un grup amb l'operació \(\ast\) i \(g\) un element de \(G\). Aleshores \(\langle\{g\}\rangle=\{g^{i}\}_{i\in\mathbb{Z}}\).
		\begin{proof}
			Ho demostrem per doble inclusió.
			
			Comencem veient que \(\{g^{i}\}_{i\in\mathbb{Z}}\subseteq\langle\{g\}\rangle\). Per la definició de \myref{def:mínim subgrup generat per un subconjunt} tenim que existeix una família de subconjunts de \(G\) que denotarem per \(\{H_{i}\}_{i\in I}\), amb \(\{g\}\subseteq H=\bigcap_{i\in I}H_{i}\). Com que \(\{H_{i}\}_{i\in I}\) són subgrups de \(G\) tenim que, donat que \(g\in H_{i}\), \(g^{n}\in H_{i}\) per a tot \(i\in I\) i tot \(n\in\mathbb{Z}\) per la definició de \myref{def:grup}, i per tant \(g^{n}\in H\), el que és equivalent a dir que \(g^{n}\in \langle g\rangle\) per a tot \(n\in\mathbb{Z}\)
			
			Ara veiem que \(\langle\{g\}\rangle\subseteq\{g^{i}\}_{i\in\mathbb{Z}}\). Denotarem \(H_{g}=\{g^{i}\}_{i\in\mathbb{Z}}\). Hem de veure que \(H_{g}\) és un grup amb l'operació \(\ast\). Observem que per a tot \(g^{i},g^{j}\in H_{g}\), \(g^{i}\ast g^{-j}=g^{i-j}\in H_{g}\), i per tant \(H_{g}\leq G\). Ara bé, com que \(\{g\}\subseteq H_{g}\), tenim que \(H_{g}\in\{H_{i}\}_{i\in I}\), és a dir, que \(H_{g}\) pertany a la família de subconjunts de \(G\) que contenen \(\{g\}\); el que significa que \(\langle\{g\}\rangle\leq H_{g}\), i per tant \(\langle\{g\}\rangle\subseteq\{g^{i}\}_{i\in\mathbb{Z}}\).
		\end{proof}
	\end{proposition}
	\begin{definition}[Ordre d'un grup]
		\labelname{ordre d'un grup}
		\label{def:ordre d'un grup}
		Sigui \(G\) un grup amb l'operació \(\ast\). Direm que \(\abs{G}\) és l'ordre del grup.
		Si \(\abs{G}\) és finit direm que \(G\) és un grup d'ordre finit, i si \(\abs{G}\) no és finit direm que \(G\) és un grup d'ordre infinit.
	\end{definition}
	\begin{proposition}
		\label{prop:potencia element neutre en un grup}
		Siguin \(G\) un grup amb l'operació \(\ast\) i element neutre \(e\) i \(g\) un element de \(G\). Aleshores
		\[\abs{\langle\{g\}\rangle}=n\sii n=\min\{k\in\mathbb{N}\mid g^{k}=e\}.\]
		\begin{proof}
			Comencem amb la implicació cap a l'esquerra (\(\implicatper\)). Suposem doncs que \(n=\min\{k\in\mathbb{N}\mid g^{k}=e\}\). Pel \myref{thm:divisió euclidiana} tenim que per a tot \(t\in\mathbb{Z}\) existeixen uns únics \(Q\in\mathbb{Z}\), \(r\in\mathbb{N}\), amb \(r<n\) tals que \(t=Qn+r\). Per tant
			\begin{align*}
			g^{t}&=g^{Qm+r}\\
			&=g^{Qm}\ast g^{r}\\
			&=(g^{m})^{Q}\ast g^{r}\\
			&=e^{Q}\ast g^{r}=g^{r}.
			\end{align*}
			Per tant, com que \(0\leq r<n\), \(\abs{\langle\{g\}\rangle}=n\).
			
			Fem ara la implicació cap a la dreta (\(\implica\)). Suposem doncs que \(\abs{\langle\{g\}\rangle}=n\). Com que el grup és finit per a cada \(i\in\mathbb{Z}\) existeix \(j\in\mathbb{Z}\) tal que \(g^{i}=g^{j}\) i, com que \(\langle\{g\}\rangle\) és un grup, per la definició de \myref{def:grup} existeix \(g^{-j}\in\langle\{g\}\rangle\) tal que \(g^{i-j}=e\).
			
			Sigui doncs \(t\in\mathbb{N}\) tal que \(g^{t}=e\). Aleshores, pel \myref{thm:divisió euclidiana} existeixen uns únics \(Q,r\in\mathbb{N}\), amb \(r<n\) tals que \(t=Qn+r\). Per tant
			\begin{align*}
			g^{t}&=g^{Qn+r}\\
			&=g^{Qn}\ast g^{r}\\
			&=(g^{n})^{Q}\ast g^{r}\\
			&=e^{Q}\ast g^{r}\\
			&=g^{r}=e.
			\end{align*}
			i per tant \(r=0\), i tenim \(t=Qn\), i per tant \(n=\min\{k\in\mathbb{N}\mid g^{k}=e\}\).
		\end{proof}
	\end{proposition}
	\begin{definition}[Conjugació entre conjunts sobre grups]
		\labelname{conjugació entre conjunts sobre grups}
		\label{def:conjugació entre conjunts sobre grups}
		Siguin \(G\) un grup amb l'operació \(\ast\) i \(H\) un subconjunt de \(G\). Aleshores definim
		\[GH=\{g\ast h\mid g\in G,h\in H\}\quad\text{i}\quad HG=\{h\ast g\mid g\in G,h\in H\}.\]
	\end{definition}
	\begin{definition}[Subgrup normal]
		\labelname{subgrup normal}
		\label{def:subgrup normal}
		Siguin \(G\) un grup amb l'operació \(\ast\) i \(H\) un subgrup de \(G\). Aleshores direm que \(H\) és un subgrup normal de \(G\) si per a tot \(x\in G\) tenim
		\[\{x\}H=H\{x\}.\]
		Ho denotarem com \(H\trianglelefteq G\).
	\end{definition}
	\begin{proposition}
		\label{prop:condicions equivalents a subgrup normal}
		Siguin \(G\) un grup amb l'operació \(\ast\) i \(H\) un subgrup de \(G\). Aleshores són equivalents
		\begin{enumerate}
			\item\label{enum:condicions equivalents a subgrup normal 1} \(\{x\}H=H\{x\}\) per a tot \(x\in G\).
			\item\label{enum:condicions equivalents a subgrup normal 2} \(\{x^{-1}\}H\{x\}=H\) per a tot \(x\in G\).
			\item\label{enum:condicions equivalents a subgrup normal 3} \(\{x^{-1}\}H\{x\}\subseteq H\) per a tot \(x\in G\).
		\end{enumerate}
		\begin{proof}
			Comencem demostrant \eqref{enum:condicions equivalents a subgrup normal 1}\(\implica\)\eqref{enum:condicions equivalents a subgrup normal 2}. Suposem que \(H\) és un subgrup normal de \(G\), per la definició de \myref{def:subgrup normal} tenim \(\{x\}H=H\{x\}\) per a tot \(x\in G\). Aleshores tenim
			\begin{align*}
			\{x\}H\{x^{-1}\}&=H\{x\}\{x^{-1}\}\\
			&=\{h\ast x\ast x^{-1}\mid h\in H\}\\
			&=\{h\mid h\in H\}=H.
			\end{align*}
			
			Continuem demostrant \eqref{enum:condicions equivalents a subgrup normal 2}\(\implica\)\eqref{enum:condicions equivalents a subgrup normal 3}. Suposem que \(\{x\}H\{x^{-1}\}=H\). Tenim que \(\{x\}H\{x^{-1}\}=H\subseteq H\).
			
			Demostrem ara \eqref{enum:condicions equivalents a subgrup normal 3}\(\implica\)\eqref{enum:condicions equivalents a subgrup normal 1}. Suposem doncs que \(\{x^{-1}\}H\{x\}\subseteq H\) per a tot \(x\in G\). Això significa que per a tot \(h\in H\) existeix un \(h'\in H\) tal que \(x\ast h\ast x^{-1}=h'\), i aleshores, per la definició de \myref{def:grup}, \(x\ast h=h'\ast x\in H\), i per tant \(x\ast h\in H\{x\}\) per a tot \(x\in G\). Així hem vist que \(\{x\}H\subseteq H\{x\}\). Per veure l'altre inclusió es pot donar un argument anàleg, i per tant \(\{x\}H= H\{x\}\). %REFERENCIA I POTSER MENYS MANDRA
			
			I així hem vist que \eqref{enum:condicions equivalents a subgrup normal 1}\(\implica\)\eqref{enum:condicions equivalents a subgrup normal 2}\(\implica\)\eqref{enum:condicions equivalents a subgrup normal 3}\(\implica\)\eqref{enum:condicions equivalents a subgrup normal 1} i hem acabat.
		\end{proof}
	\end{proposition}
	\subsection{Grups cíclics i grups abelians}
	\begin{definition}[Grup abelià]
		\labelname{grup abelià}
		\label{def:grup abelià}
		Sigui \(G\) un grup amb l'operació \(+\) tal que per a tot \(x,y\in G\) satisfà
		\[x+y=y+x.\]
		Aleshores direm que \(G\) és un grup abelià.
	\end{definition}
	\begin{proposition}
		\label{prop:condició equivalent a grup abelià}
		Sigui \(G\) un grup amb l'operació \(+\). Aleshores \(G\) és un grup abelià si i només si per a tot \(a,b\in G\) es compleix
		\[-(a+b)=-a-b.\]
		\begin{proof}
			Que la condició és necessària (\(\implica\)) ho podem veure amb la definició de \myref{def:grup abelià} i la proposició \myref{prop:invers de a b = b invers a invers}.
			
			Demostrem ara que la condició és suficient (\(\implicatper\)). Diem que l'element neutre de \(G\) és \(e\). Per la definició de \myref{def:grup} tenim que
			\begin{equation}
			\label{hipotesi:inverses grups}
			(a+b)-(a+b)=e,
			\end{equation}
			que és equivalent a
			\[(a+b)-(a+b)-\left(-(a+b)\right)=-\left(-(a+b)\right),\]
			i aleshores
			\begin{align*}
			a+b&=-\left(-(a+b)\right)\\
			&=-\left(-b-a\right)\tag{Proposició \myref{prop:invers de a b = b invers a invers}}\\
			&=-\left(-(b+a)\right)\tag{Hipótesi \eqref{hipotesi:inverses grups}}\\
			&=b+a,\tag{Proposició \myref{prop:grups:l'invers de l'invers d'un element es l'element}}
			\end{align*}
			i per la definició de \myref{def:grup abelià}, \(G\) és un grup abelià.
		\end{proof}
	\end{proposition}
	\begin{definition}[Grup cíclic]
		\labelname{grup cíclic}
		\label{def:grup cíclic}
		Siguin \(G\) un grup amb l'operació \(\ast\) i \(g\) un element de \(G\). Aleshores diem que el grup \(\langle\{g\}\rangle\) amb l'operació \(\ast\) és un grup cíclic i que \(g\) és un generador del grup.
	\end{definition}
	\begin{proposition}
		Sigui \(G\) un grup cíclic amb l'operació \(\ast\). Aleshores \(G\) és un grup abelià.
		\begin{proof}
			Com que \(G\) és un grup cíclic, per la definició de \myref{def:grup cíclic} tenim que existeix un \(g\) tal que \(\langle\{g\}\rangle=G\). Siguin \(a,b\) dos  elements de \(G\), i com que \(G\) és un grup cíclic, ha de ser \(a=g^{m}\) i \(b=g^{n}\) per a certs \(m,n\in\mathbb{N}\). Ara bé, tenim que \(g^{m}\ast g^{n}=g^{m}\ast g^{n}\), ja que
			\begin{align*}
			g^{n}\ast g^{m}&=g^{n+m}\\
			&=g^{m+n}=g^{m}\ast g^{n}
			\end{align*}
			i aleshores \(a\ast b=b\ast a\), i per la definició de \myref{def:grup abelià} hem acabat.
		\end{proof}
	\end{proposition}
	\begin{proposition}
		Sigui \(G\) un grup cíclic amb l'operació \(\ast\) i \(H\) un subgrup de \(G\). Aleshores \(H\) és un grup cíclic.
		\begin{proof}
			Sigui \(e\) l'element neutre de \(G\). Per l'observació \myref{obs:l'element neutre d'un grup pertany als subgrups} tenim que \(e\in H\). Si \(H=\{e\}\) no hi ha res a demostrar. Suposem que \(H\neq\{e\}\), aleshores existeix un \(g\in G\) tal que \(g^{n}\in H\) per a cert \(n\in\mathbb{N}\). Sigui doncs \(m\) l'enter més petit tal que \(g^{m}\in H\); volem demostrar que \(H=\langle\{g^{m}\}\rangle\).
			
			Sigui \(a\) un element de \(H\). Aleshores com que \(a\in H\subseteq G\), \(a=g^{t}\) per a cert \(t\in\mathbb{N}\), i pel \myref{thm:divisió euclidiana} existeixen \(Q,r\in\mathbb{N}\), amb \(r<m\) tals que \(t=Qm+r\), i per tant \(g^{t}=g^{Qm+r}\). Aleshores tenim
			\[g^{r}=\left(g^{m}\right)^{-Q}\ast g^{t},\]
			i ha de ser \(g^{r}\in H\), ja que \(g^{m}\in H\), i per la definició de \myref{def:grup} tenim que \(\left(g^{m}\right)^{-1}\in H\). Per tant \(g^{r}\in H\). Però ara bé, \(m\) era el mínim enter tal que \(g^{m}\in H\), i \(r<m\), per tant ha de ser \(g^{r}=e\), és a dir, \(r=0\) i per tant \(t=Qm\); el que significa que \(H=\{\left(g^{m}\right)^{Q}\mid Q\in\mathbb{N}\}\) i per la proposició \myref{prop:forma grups cíclics} \(H=\langle\{g^{m}\}\rangle\) i hem acabat.
		\end{proof}
	\end{proposition}
	\begin{proposition}
		\label{prop:existència de subgrups cíclics d'ordres divisors}
		Siguin \(G\) un grup cíclic amb l'operació \(\ast\) d'ordre \(n\) finit i \(d\in\mathbb{N}\) un divisor de \(n\). Aleshores existeix un únic subgrup de \(G\) d'ordre \(d\).
		\begin{proof}
			%TODO
		\end{proof}
	\end{proposition}
	\subsection{Grup quocient}
	\begin{proposition}
		\label{prop:relacio entre grups és d'equivalència}
		\label{TODO:grup quocient}
		Siguin \(G\) un grup amb l'operació \(\ast\) i \(H\) un subgrup de \(G\). Aleshores la relació
		\[x\sim y\sii x\ast y^{-1}\in H\text{ per a tot }x,y\in G\]
		és una relació d'equivalència.
		\begin{proof}
			Sigui \(e\) l'element neutre del grup \(G\). Comprovem les propietats de la definició de relació d'equivalència:
			\begin{enumerate}
				\item Reflexiva: Sigui \(x\in G\). Aleshores \(x\ast x^{-1}=e\), i tenim l'observació \myref{obs:l'element neutre d'un grup pertany als subgrups}.
				\item Simètrica: Siguin \(x,y\in G\) i suposem que \(x\sim y\), això significa que \(x\ast y^{-1}\in H\), i per la definició de \myref{def:grup} tenim que \((x\ast y^{-1})^{-1}\in H\), ja que per hipòtesi \(H\) és un grup, i per les proposicions \myref{prop:invers de a b = b invers a invers} i \myref{prop:grups:l'invers de l'invers d'un element es l'element} tenim que
				\[(x\ast y^{-1})^{-1}=y\ast x^{-1},\]
				i això és \(y\sim x\).
				\item Transitiva: Siguin \(x,y,z\in G\) i suposem que \(x\sim y\) i \(y\sim z\). Per tant \(x\ast y^{-1}\in H\), i \(y\ast z^{-1}\in H\). Com que per hipòtesi \(H\) és un grup, tenim que
				\[(x\ast y^{-1})\ast(y\ast z^{-1})\in H,\]
				que és equivalent a \(x\ast z^{-1}\in H\), i per tant \(x\sim z\).
			\end{enumerate}
			I per la definició de \myref{def:relació d'equivalència} hem acabat.
		\end{proof}
	\end{proposition}
	\begin{proposition}\label{prop:grup quocient}
		Siguin \(G\) un grup amb l'operació \(\ast\) i \(H\) un subgrup de \(G\) i \(\sim\) una relació d'equivalència tal que
		\[x\sim y\sii x\ast y^{-1}\in H\text{ per a tot }x,y\in G.\]
		Aleshores el conjunt quocient \(G/H\) amb l'operació
		\begin{align*}
		\ast\colon G/H\times G/H&\longrightarrow G/H\\
		[x]\ast[y]&\longmapsto[x\ast y]
		\end{align*}
		és un grup si i només si \(H\) és un subgrup normal de \(G\).
		\begin{proof}
			%TODO
		\end{proof}
	\end{proposition}
	\begin{definition}[Grup quocient]
		\labelname{grup quocient}\label{def:grup quocient}\label{def:relació d'equivalència entre grups}\label{def:producte entre classes versió grups}
		Siguin \(G\) un grup amb l'operació \(\ast\) i \(N\) un subgrup normal de \(G\). Aleshores direm que el grup \(G/N\) amb l'operació
		\begin{align*}
		\ast\colon G/N\times G/N&\longrightarrow G/N\\
		[x]\ast[y]&\longmapsto[x\ast y]
		\end{align*}
		és el grup quocient \(G\) mòdul \(N\).
		
		Aquesta definició té sentit per la proposició \myref{prop:grup quocient}.
	\end{definition}
	\begin{lemma}
		\label{lema:operar en grups és bijectiu}
		Siguin \(G\) un grup amb l'operació \(\ast\), \(H\) un subgrup de \(G\), \(x\) un element de \(G\) i
		\begin{align*}
		f_{x}\colon H&\longrightarrow\{x\}H\\
		h&\longmapsto x\ast h
		\end{align*}
		una aplicació. Aleshores \(f_{x}\) és bijectiva.
		\begin{proof}
			Veiem que aquesta funció és bijectiva trobant la seva inversa:
			\begin{align*}
			f_{x}^{-1}\colon\{x\}H&\longrightarrow H\\
			y&\longmapsto x^{-1}\ast y
			\end{align*}
			i comprovant \(f_{x}(f_{x}^{-1}(h))=h\) i \(f_{x}^{-1}(f_{x}(h))=h\). Per tant \(f\) és bijectiva\footnote{de fet, \(f_{x}^{-1}=f_{x^{-1}}\)}. %REFERENCIES fonaments
		\end{proof}
	\end{lemma}
	\begin{observation}
		\(\{x\}G=G\).
	\end{observation}
	\begin{theorem}[Teorema de Lagrange]
		\labelname{Teorema de Lagrange}
		\label{thm:Teorema de Lagrange}
		Siguin \(G\) un grup d'ordre finit amb l'operació \(\ast\) i \(H\) un subgrup de \(G\). Aleshores \(\abs{H}\) divideix \(\abs{G}\).
		\begin{proof}
			Fixem \(x\in G\) i considerem la funció
			\begin{align*}
			f_{x}\colon H&\longrightarrow\{x\}H\\
			h&\longmapsto x\ast h
			\end{align*}
			
			Pel lema \myref{lema:operar en grups és bijectiu} trobem \(\abs{H}=\abs{\{x\}H}\). Tenim que \(\abs{G}\) és el resultat de multiplicar el número de classes d'equivalència pel nombre d'elements d'una de les classes, és a dir %REFERENCIES fonaments
			\[\abs{G}=\abs{G/H}\abs{H}\]
			i per tant \(\abs{H}\) divideix \(\abs{G}\).
		\end{proof}
	\end{theorem}
	\begin{corollary}
		Sigui \(G\) un grup d'ordre \(p\) primer amb l'operació \(\ast\). Aleshores \(G\) és un grup cíclic.
		\begin{proof}
			Sigui \(e\) l'element neutre de \(G\).
			
			Prenem un element \(g\in G\) diferent de \(e\) i considerem el subgrup de \(G\) generat per \(g\). Pel \myref{thm:Teorema de Lagrange} tenim que l'ordre de \(\langle\{g\}\rangle\) divideix l'ordre de \(G\) i com que per hipòtesi l'orde de \(G\) és primer i \(g^{1}\neq e\), ja que \(g\) és per hipòtesi diferent de l'element neutre, tenim que \(\abs{\langle\{g\}\rangle}=p\), i per tant \(\langle\{g\}\rangle=G\) i per la definició de \myref{def:grup cíclic} tenim que \(G\) és un grup cíclic.
		\end{proof}
	\end{corollary}
	\begin{definition}[l'índex d'un subgrup en un grup]\label{def:l'índex d'un subgrup en un grup}
		Siguin \(G\) un grup amb l'operació \(\ast\) i \(H\) un subgrup de \(G\). Aleshores definim
		\[[G:H]=\frac{\abs{G}}{\abs{H}}\]
		com l'índex de \(H\) a \(G\).
	\end{definition}
	\section{Tres Teoremes d'isomorfisme entre grups}
	\subsection{Morfismes entre grups}
	\begin{definition}[Morfisme entre grups]
		\labelname{morfisme entre grups}
		\label{def:morfisme entre grups}
		\labelname{monomorfisme entre grups}
		\label{def:monomorfisme entre grups}
		\labelname{epimorfisme entre grups}
		\label{def:epimorfisme entre grups}
		\labelname{isomorfisme entre grups}
		\label{def:isomorfisme entre grups}
		\labelname{endomorfisme entre grups}
		\label{def:endomorfisme entre grups}
		\labelname{automorfisme entre grups}
		\label{def:automorfisme entre grups}
		Siguin \(G_{1}\) un grup amb l'operació \(\ast\), \(G_{2}\) un grup amb l'operació \(\circ\) i \(f\colon G_{1}\to G_{2}\) una aplicació que, per a tot \(x,y\in G_{1}\) satisfà
		\[f(x\ast y)=f(x)\circ f(y).\]
		Aleshores diem que \(f\) és un morfisme entre grups. Definim també
		\begin{enumerate}
			\item Si \(f\) és injectiva direm que \(f\) és un monomorfisme entre grups.
			\item Si \(f\) és exhaustiva direm que \(f\) és un epimorfisme entre grups.
			\item Si \(f\) és bijectiva direm que \(f\) és un isomorfisme entre grups. També escriurem \(G_{1}\cong G_{2}\) i direm qe \(G_{1}\) i \(G_{2}\) són grups isomorfs.
			\item Si \(G_{1}=G_{2}\) direm que \(f\) és un endomorfisme entre grups.
			\item Si \(G_{1}=G_{2}\) i \(f\) és bijectiva direm que \(f\) és un automorfisme entre grups.
		\end{enumerate}
	\end{definition}
	\begin{proposition}
		\label{prop:morfismes conserven neutre i l'invers commuta amb el morfisme}
			Siguin \(G_{1}\) un grup amb l'operació \(\ast\) amb element neutre \(e\), \(G_{2}\) un grup amb l'operació \(\circ\) amb element neutre \(e'\) i \(f\colon G_{1}\to G_{2}\) un morfisme entre grups. Aleshores
		\begin{enumerate}
			\item\label{enum:morfismes conserven neutre i l'invers commuta amb el morfisme 1} \(f(e)=e'\).
			\item\label{enum:morfismes conserven neutre i l'invers commuta amb el morfisme 2} \(f\left(x^{-1}\right)=f(x)^{-1}\) per a tot \(x\in G_{1}\).
		\end{enumerate}
		\begin{proof}
			Demostrem primer el punt \eqref{enum:morfismes conserven neutre i l'invers commuta amb el morfisme 1}. Per la definició de morfisme tenim que per a tot \(x\in G_{1}\)
			\begin{align*}
			f(x)\circ f(e)&=f(x\ast e)\tag{\myref{def:morfisme entre grups}}\\
			&=f(x)\tag{\myref{def:l'element neutre del grup}}\\
			&=f(x)\circ e'
			\end{align*}
			i per la proposició \myref{prop:podem tatxar pels costats en grups} tenim \(f(e)=e'\).
			
			Per demostrar el punt \eqref{enum:morfismes conserven neutre i l'invers commuta amb el morfisme 2} en tenim prou en veure que per a tot \(x\in G\)
			\begin{align*}
			f(x)\circ f(x^{-1})&=f(x\ast x^{-1})\tag{\myref{def:morfisme entre grups}}\\
			&=f(e)\tag{\myref{def:l'invers d'un element d'un grup}}\\
			&=f(x^{-1}\ast x)\tag{\myref{def:morfisme entre grups}}\\
			&=f(x^{-1})\circ f(x)
			\end{align*}
			i pel punt \eqref{enum:morfismes conserven neutre i l'invers commuta amb el morfisme 1} d'aquesta proposició \(f(x)\circ f(x^{-1})=f(x^{-1})\circ f(x)=e'\), i per la proposició \myref{prop:unicitat inversa en grups} tenim que \(f(x^{-1})=f(x)^{-1}\), com volíem.
		\end{proof}
	\end{proposition}
	\begin{proposition}\label{prop:conjugació de morfismes entre grups és morfisme entre grups}
		Siguin \(G\), \(H\) i \(K\) tres grups amb les operacions \(\ast\), \(\circ\) i \(+\), respectivament, i \(f\colon G\longrightarrow H\) i \(g\colon H\longrightarrow K\) dos morfismes entre grups. Aleshores \(g(f)\colon G\longrightarrow K\) és un morfisme entre grups.
		\begin{proof}
			Per la definició de \myref{def:morfisme entre grups} tenim que per a tot \(g_{1},g_{2}\in H\) i \(h_{1},h_{2}\in H\) tenim \(f(g_{1}\ast g_{2})=f(g_{1})\circ f(g_{2})\) i \(g(h_{1}\circ h_{2})=g(h_{1})+g(h_{2})\). Per tant
			\[g(f(g_{1}\ast g_{2}))=g(f(g_{1})\circ f(g_{2}))=g(f(g_{1}))+g(f(g_{2})),\]
			i per la definició de \myref{def:morfisme entre grups} hem acabat.
		\end{proof}
	\end{proposition}
	\begin{proposition}
		\label{prop:condicions equivalents abelià i cíclic per isomorfismes}
		Siguin \(G_{1}\) un grup amb l'operació \(\ast\) i \(G_{2}\) un grup amb l'operació \(\circ\) tal que
		\[G_{1}\cong G_{2}.\]
		Aleshores
		\begin{enumerate}
			\item\label{enum:condicions equivalents abelià i cíclic per isomorfismes 1} \(G_{1}\) és un grup abelià si i només si \(G_{2}\) és un grup abelià.
			\item\label{enum:condicions equivalents abelià i cíclic per isomorfismes 2} \(G_{1}\) és un grup cíclic si i només si \(G_{2}\) és un grup cíclic.
		\end{enumerate}
		\begin{proof}
			Sigui \(f\colon G_{1}\to G_{2}\) un isomorfisme entre grups.
			
			Comencem demostrant el punt \eqref{enum:condicions equivalents abelià i cíclic per isomorfismes 1}. Suposem doncs que \(G_{1}\) és un grup abelià. Per la definició de \myref{def:grup abelià} tenim que per a tot \(a,b\in G_{1}\) es compleix \(a\ast b=b\ast a\). Aleshores tenim
			\[f(a\ast b)=f(b\ast a)\]
			i per la definició de \myref{def:morfisme entre grups} tenim que
			\[f(a)\circ f(b)=f(b)\circ f(a),\]
			i per tant, com que per la definició de \myref{def:isomorfisme entre grups} \(f\) és un bijectiu, \(G_{2}\) satisfà la definició de \myref{def:grup abelià}.
			
			Demostrem ara el punt \eqref{enum:condicions equivalents abelià i cíclic per isomorfismes 2}. Suposem doncs que \(G_{1}\) és un grup cíclic. Per la definició de \myref{def:grup cíclic} tenim que \(G_{1}=\{g^{i}\}_{i\in\mathbb{Z}}\) per a un cert \(g\in G_{1}\). Per tant, com que \(f\) és bijectiva per la definició de \myref{def:isomorfisme entre grups} tenim que per a tot \(x\in G_{2}\) es compleix \(x=f(g^{i})\) per a un cert \(i\in\mathbb{Z}\), i per la definició de \myref{def:morfisme entre grups} tenim que\footnote{el primer és amb l'operació \(\ast\) i el segon amb l'operació \(\circ\).} \(f(g^{i})=f(g)^{i}\), i per la definició de \myref{def:grup cíclic} \(G_{2}\) és un grup cíclic.
		\end{proof}
	\end{proposition}
	\begin{definition}[Nucli i imatge d'un morfisme entre grups]
		\labelname{nucli d'un morfisme entre grups}
		\label{def:nucli d'un morfisme entre grups}
		\labelname{imatge d'un morfisme entre grups}
		\label{def:imatge d'un morfisme entre grups}
			Siguin \(G_{1}\) un grup amb l'operació \(\ast\) amb element neutre \(e\), \(G_{2}\) un grup amb l'operació \(\circ\) amb  element neutre \(e'\) i \(f\colon G_{1}\to G_{2}\) un morfisme entre grups. Aleshores definim el nucli de \(f\) com
		\[\ker(f)=\{x\in G_{1}\mid f(x)=e'\},\]
		i la imatge de \(f\) com
		\[\Ima(f)=\{f(x)\in G_{2}\mid x\in G_{1}\}.\]
	\end{definition}
	\begin{observation}
		\label{obs:nucli d'un morfisme entre grups es subconjunt del grup d'entrada, imatge n'és del de sortida}
		\(\ker(f)\subseteq G_{1}\), \(\Ima(f)\subseteq G_{2}\).
	\end{observation}
	\begin{proposition}
		\label{prop:el nucli d'un morfisme és un subgrup normal del grup de sortda}
		\label{prop:la imatge d'un morfisme és un subgrup del grup d'arribada}
			Siguin \(G_{1}\) un grup amb l'operació \(\ast\) amb element neutre \(e\), \(G_{2}\) un grup amb l'operació \(\circ\) amb element neutre \(e'\), i \(f\colon G_{1}\to G_{2}\) un morfisme entre grups. Aleshores
		\begin{enumerate}
			\item\label{enum:el nucli d'un morfisme és un subgrup normal del grup de sortda 1} \(\ker(f)\) és un subgrup normal de \(G_{1}\).
			\item\label{enum:la imatge d'un morfisme és un subgrup del grup d'arribada 2} \(\Ima(f)\) és un subgrup de \(G_{2}\).
		\end{enumerate}
		\begin{proof}
			Aquest enunciat té sentit per l'observació \myref{obs:nucli d'un morfisme entre grups es subconjunt del grup d'entrada, imatge n'és del de sortida}.
			
			Primer comprovem el punt \eqref{enum:el nucli d'un morfisme és un subgrup normal del grup de sortda 1}. Comencem veient que \(\ker(f)\) és un subgrup de \(G_{1}\). Per la proposició \myref{prop:condició equivalent a subgrup} tenim que ens cal amb veure que si \(a,b\in\ker(f)\), aleshores \(a\ast b^{-1}\in\ker(f)\). Això és cert ja que si \(a,b\in\ker(f)\) aleshores \(f(a)=e'\) i \(f(b^{-1})=e'\), i per tant \(a\ast b^{-1}=e\ast e^{-1}=e\), el que significa que \(f(a\ast b^{-1})=e'\), i tenim \(a\ast b^{-1}\in\ker(f)\).
			
			Comprovem ara que el subgrup és normal. Per la proposició \myref{prop:condicions equivalents a subgrup normal} en tenim prou en veure que per a tot \(x\in\ker(f)\) i \(g\in G\), \(x\ast g\ast x^{-1}\in\ker(f)\). Això ho veiem notant que si \(g\in\ker(f)\), \(f(g)=e'\), i per tant \(f(x\ast g\ast x^{-1})=f(x)\circ e'\circ f(x^{-1})\)i això és \(f(x\ast x^{-1})=e'\), i per tant \(x\ast g\ast x^{-1}\in\ker(f)\).
			
			Acabem veient el punt \eqref{enum:la imatge d'un morfisme és un subgrup del grup d'arribada 2}. De nou per la proposició \myref{prop:condició equivalent a subgrup} tenim que si per a tot \(f(a),f(b)\in\Ima(f)\) tenim \(f(a)\circ f(b)^{-1}\in\Ima(f)\) aleshores \(\Ima(f),\) és un subgrup de \(G_{2}\). Això és cert, ja que per la definició de \myref{def:morfisme entre grups} i la proposició \myref{prop:morfismes conserven neutre i l'invers commuta amb el morfisme} tenim \(f(a)\circ f(b)^{-1}=f(a\ast b^{-1})\); i per la definició de grup \(a\ast b^{-1}\in G_{1}\), i per la definició de \myref{def:morfisme entre grups} tenim que \(f(a)\circ f(b)^{-1}\in\Ima(f)\), i per tant \(\Ima(f)\) és un subgrup de \(G_{2}\), com volíem veure.
		\end{proof}
	\end{proposition}
	\begin{proposition}
		\label{prop:condicions equivalents a monomorfisme i epimorfisme per nucli i imatge}
		Siguin \(G_{1}\) un grup amb l'operació \(\ast\) amb element neutre \(e\), \(G_{2}\) un grup amb l'operació \(\circ\) amb element neutre \(e'\), i \(f\colon G_{1}\to G_{2}\) un morfisme entre grups. Aleshores
		\begin{enumerate}
			\item\label{enum:condicions equivalents a monomorfisme i epimorfisme per nucli i imatge 1} \(f\) és un monomorfisme si i només si \(\ker(f)=\{e\}\).
			\item\label{enum:condicions equivalents a monomorfisme i epimorfisme per nucli i imatge 2} \(f\) és un epimorfisme si i només si \(\Ima(f)=G_{2}\).
		\end{enumerate}
		\begin{proof}
			Comencem fent la demostració del punt \eqref{enum:condicions equivalents a monomorfisme i epimorfisme per nucli i imatge 1} per la implicació cap a la dreta (\(\implica\)). Suposem doncs que \(f\) és un monomorfisme, i per tant injectiva. Per la definició de \myref{def:nucli d'un morfisme entre grups} tenim que \(\ker(f)=\{x\in G_{1}\mid f(x)=e'\}\). Suposem \(x\in G_{1}\), és a dir, \(f(x)=e'\). Ara bé, com que \(f\) és injectiva per la proposició \myref{prop:morfismes conserven neutre i l'invers commuta amb el morfisme} ha de ser \(\ker(f)=\{e\}\).%ref injectiva
			
			Demostrem ara la implicació cal a l'esquerra (\(\implicatper\)). Suposem doncs que \(\ker(f)=\{e\}\). Siguin \(x,y\in G_{1}\) dos elements que satisfacin \(f(x)=f(y)\). Com que, per la proposició \myref{prop:el nucli d'un morfisme és un subgrup normal del grup de sortda} \(\ker(f)\) és un subgrup de \(G_{1}\), tenim que \(x\ast y^{-1}\in G_{1}\), i per tant
			\begin{align*}
			f(x\ast y^{-1})&=f(x)\circ f(y^{-1})\tag{\myref{def:morfisme entre grups}}\\
			&=f(x)\circ f(y)^{-1}\tag{Proposició \myref{prop:morfismes conserven neutre i l'invers commuta amb el morfisme}}\\
			&=f(y)\circ f(y)^{-1}=e',
			\end{align*}
			i per tant \(x\ast y^{-1}\in\ker(f)\), però per hipòtesi teníem \(\ker(f)=\{e\}\), i per tant ha de ser \(x\ast y^{-1}=e\), el que és equivalent a \(x=y\), i per tant \(f\) és injectiva. %ref definició
			
			Demostrem ara el punt \eqref{enum:condicions equivalents a monomorfisme i epimorfisme per nucli i imatge 2} començant per la implicació cap a la dreta (\(\implica\)). Suposem doncs que \(f\) és un epimorfisme, i per tant exhaustiva, i per tant per a cada \(y\in G_{2}\) existeix un \(x\in G_{1}\) tal que \(f(x)=y\), i per la definició d'\myref{def:imatge d'un morfisme entre grups} tenim que \(\Ima(f)=G_{2}\).
			
			Acabem demostrant la implicació cap a l'esquerra (\(\implicatper\)). Suposem doncs que \(\Ima(f)=G_{2}\) i prenem \(y\in G_{2}\). Aleshores per la definició d'\myref{def:imatge d'un morfisme entre grups} tenim que existeix un \(x\in G_{1}\) tal que \(f(x)=y\), i per tant \(f\) és exhaustiva. %ref definició
		\end{proof}
	\end{proposition}
	\begin{proposition}
		\label{prop:les imatges dels subgrups son subgrups + el mateix amb subgrups normals}
		Siguin \(G_{1}\) un grup amb l'operació \(\ast\), \(G_{2}\) un grup amb l'operació \(\circ\), i \(f\colon G_{1}\to G_{2}\) un morfisme entre grups. Aleshores
		\begin{enumerate}
			\item\label{enum:les imatges dels subgrups son subgrups + el mateix amb subgrups normals 1} Si \(H_{1}\leq G_{1}\implica\{f(h)\in G_{2}\mid h\in H_{1}\}\leq G_{2}\).
			\item\label{enum:les imatges dels subgrups son subgrups + el mateix amb subgrups normals 2} Si \(H_{2}\leq G_{2}\implica\{h\in G_{1}\mid f(h)\in H_{2}\}\leq G_{1}\).
			\item\label{enum:les imatges dels subgrups son subgrups + el mateix amb subgrups normals 3} Si \(H_{2}\trianglelefteq G_{2}\implica\{h\in G_{1}\mid f(h)\in H_{2}\}\trianglelefteq G_{1}\).
		\end{enumerate}
		\begin{proof}
			Comprovem primer el punt \eqref{enum:les imatges dels subgrups son subgrups + el mateix amb subgrups normals 1}. Suposem doncs que \(H_{1}\) és un subgrup de \(G_{1}\). Denotarem \(H=\{f(h)\in G_{2}\mid h\in H_{1}\}\). Siguin \(x,y\in H_{1}\); per la proposició \myref{prop:condició equivalent a subgrup} només ens cal veure que \(f(x)\circ f(y)^{-1}\in H\). Això és
			\begin{align*}
			f(x)\circ f(y)^{-1}&=f(x)\circ f(y^{-1})\tag{Proposició \myref{prop:morfismes conserven neutre i l'invers commuta amb el morfisme}}\\
			&=f(x\ast y^{-1}).\tag{\myref{def:morfisme entre grups}}
			\end{align*}
			Ara bé, com que \(x,y\in H_{1}\) i \(H_{1}\) és un subgrup de \(G_{1}\), per la proposició \myref{prop:condició equivalent a subgrup} tenim que \(x\ast y^{-1}\in H_{1}\), i per tant \(f(x\ast y^{-1})\in H\), i per la definició de \myref{def:morfisme entre grups} i la proposició \myref{prop:morfismes conserven neutre i l'invers commuta amb el morfisme} tenim que \(f(x)\circ f(y)^{-1}\in H\), i per tant \(H\) és un subgrup de \(G_{2}\), com volíem veure.
			
			Comprovem ara el punt \eqref{enum:les imatges dels subgrups son subgrups + el mateix amb subgrups normals 2}. Suposem doncs que \(H_{2}\) és un subgrup de \(G_{2}\) i denotem \(H=\{h\in G_{1}\mid f(h)\in H_{2}\}\). Per la proposició \myref{prop:condició equivalent a subgrup} només ens cal veure que per a tot \(x,y\in H\) es satisfà \(x\ast y^{-1}\in H\). Si \(x,y\in H\) aleshores tenim que \(f(x),f(y)\in H_{2}\), i com que \(H_{2}\) és un grup, aleshores per la definició de \myref{def:grup} ha de ser \(f(x)\circ f(y^{-1})\in H_{2}\) Aleshores, per la definició de \myref{def:morfisme entre grups} tenim \(f(x)\circ f(y^{-1})=f(x\ast y^{-1})\), i per tant \(x\ast y^{-1}\in H\) i així tenim que \(H\) és un subgrup de \(G_{1}\).
			
			Veiem el punt \eqref{enum:les imatges dels subgrups son subgrups + el mateix amb subgrups normals 3} per acabar. Suposem doncs que \(H_{2}\) és un subgrup normal de \(G_{2}\) i definim \(H=\{h\in G_{1}\mid f(h)\in H_{2}\}\). Per demostrar-ho prenem \(g\in G_{1}\), \(h\in H_{1}\) tal que \(f(h)\in H\) i fem
			\begin{align*}
			f(g)\circ f(h)\circ f(g)^{-1}&=f(g)\circ f(h)\circ f(g^{-1})\tag{Proposició \myref{prop:morfismes conserven neutre i l'invers commuta amb el morfisme}}\\
			&=f(g\ast h\ast g^{-1})\tag{\myref{def:morfisme entre grups}}
			\end{align*}
			Ara bé, com que \(H_{2}\) és un subgrup normal de \(G_{2}\), tenim que, per a tot \(g\in G_{1}\), \(f(g\ast h\ast g^{-1})\in H_{2}\), i per tant \(g\ast h\ast g^{-1}\in H\), que satisfà la definició de \myref{def:subgrup normal} per la proposició \myref{prop:condicions equivalents a subgrup normal}.
		\end{proof}
	\end{proposition}
	\begin{theorem}[Teorema de representació de Cayley]
		\labelname{Teorema de representació de Cayley}
		\label{thm:Cayley Teorema de representació}
		Sigui \(G\) un grup amb l'operació \(\ast\). Aleshores \(G\) és isomorf a un subgrup de \(\GrupSimetric_{G}\) amb l'operació \(\circ\), on \(\GrupSimetric_{G}\) és el grup simètric dels elements de \(G\). %REPASSAR
		\begin{proof}
			Definim
			\begin{alignat}{2}
			\varphi\colon G&\longrightarrow\GrupSimetric_{G}&&\label{eq:thm:Cayley Teorema de representació 1}\\
			g&\longmapsto\sigma_{g}&\colon G&\longrightarrow G\label{eq:thm:Cayley Teorema de representació 2}\\
			&&x&\longmapsto g\ast x\nonumber
			\end{alignat}
			Tenim que \(\sigma_{g}\) és bijectiva ja que és una permutació. %REFERENCIES fonaments + demostrar donant inversa
			Comprovarem que \(\varphi\) és un monomorfisme entre grups. Veiem primer que és un morfisme entre grups. Prenem \(g,g'\in G\). Per la definició \eqref{eq:thm:Cayley Teorema de representació 1} tenim que \(\varphi(g\ast g')=\sigma_{g\ast g'}\). Per veure que \(\sigma_{g\ast g'}=\sigma_{g}\circ\sigma_{g'}\) observem que per a tot \(x\in G\)
			\begin{align*}
			\sigma_{g\ast g'}(x)&=g\ast g'\ast x\\
			&=g\ast\sigma_{g'}(x)\\
			&=\sigma_{g}\circ\sigma_{g'}(x),
			\end{align*}
			i per la definició de \myref{def:morfisme entre grups} tenim que \(\varphi\) és un morfisme entre grups. Veiem ara que \(\varphi\) és un monomorfisme. Per la definició de \myref{def:nucli d'un morfisme entre grups} tenim que
			\[\ker(\varphi)=\{x\in G\mid f(x)=\text{Id}_{G}\}.\]
			Ara bé, \(\sigma_{g}=\text{Id}\) és, per la definició \eqref{eq:thm:Cayley Teorema de representació 2}, equivalent a dir que \(g\ast x=x\) per a tota \(x\in G\), i per la definició de \myref{def:l'element neutre del grup} això és si i només si \(g=e\), i per tant
			\[\ker(\varphi)=\{e\},\]
			i per la proposició \myref{prop:condicions equivalents a monomorfisme i epimorfisme per nucli i imatge} tenim que \(\varphi\) és un monomorfisme, com volíem veure.
			
			Per tant, per la proposició \myref{prop:les imatges dels subgrups son subgrups + el mateix amb subgrups normals} tenim que
			\[G\cong \Ima(\varphi)\leq\GrupSimetric_{G}.\qedhere\]
		\end{proof}
	\end{theorem}
	\begin{corollary}
		Si \(G\) té ordre \(n!\) aleshores \(G\cong\GrupSimetric_{n}\).
	\end{corollary}
	\subsection{Teoremes d'isomorfisme entre grups}
	\begin{theorem}
		\label{thm:Teorema fonamental dels isomorfismes}%[Teorema Fonamental dels Isomorfismes]
			Siguin \(G_{1}\) un grup amb l'operació \(\ast\), \(G_{2}\) un grup amb l'operació \(\circ\) i \(f\colon G_{1}\to G_{2}\) un morfisme entre grups. Aleshores \(G_{1}/\ker(f)\cong\Ima(f)\).
		\begin{proof}
			Siguin \(e\) l'element neutre de \(G_{1}\) i \(e'\) l'element neutre de \(G_{2}\).
			Definim l'aplicació
			\begin{align}
			\label{eq:8}
			\varphi\colon G_{1}/\ker(f)&\longleftrightarrow\Ima(f)\\
			[x]&\longmapsto f(x)\nonumber
			\end{align}
			Comprovem primer que aquesta aplicació està ben definida:
			
			Suposem que \([x]=[x']\). Això és que \(x'\in\{x\}\ker(f)\), i equivalentment \(x'=x\ast h\) per a cert \(h\in\ker(f)\). Per tant
			\begin{align*}
			\varphi([x'])&=\varphi([x\ast h])\tag{Definició \eqref{def:producte entre classes versió grups}}\\
			&=f(x\ast h)\tag{Definició \eqref{eq:8}}\\
			&=f(x)\circ f(h)\tag{\myref{def:morfisme entre grups}}\\
			&=f(x)\circ e'\tag{\myref{def:nucli d'un morfisme entre grups}}\\
			&=f(x)=\varphi([x])\tag{Definició \eqref{eq:8}}
			\end{align*}
			i per tant \(\varphi\) està ben definida.
			Veiem ara que \(\varphi\) és un morfisme entre grups. Tenim que
			\begin{align*}
			\varphi([x]\ast[y])&=\varphi([x\ast y])\tag{Definició \eqref{def:producte entre classes versió grups}}\\
			&=f(x\ast y)\tag{Definició \eqref{eq:8}}\\
			&=f(x)\circ f(y)\tag{\myref{def:morfisme entre grups}}\\
			&=\varphi([x])\circ \varphi([y]),\tag{Definició \eqref{eq:8}}
			\end{align*}
			i per la definició de \myref{def:morfisme entre grups} \(\varphi\) és un morfisme entre grups.
			Continuem demostrant que \(\varphi\) és injectiva. Per la definició de \myref{def:nucli d'un morfisme entre grups} tenim que \(\ker(\varphi)=\{[x]\in G/\ker(f)\mid\varphi([x])=e\}\), i per tant \(\ker(\varphi)=\ker(f)\), ja que \(f(x)=e\) si i només si \(x\in\ker(f)\), i per tant \(\ker(\varphi)=[e]\) i per la proposició \myref{prop:condicions equivalents a monomorfisme i epimorfisme per nucli i imatge} \(\varphi\) és injectiva.
			
			Per veure que \(\varphi\) és exhaustiva veiem que si \([x]\in G_{1}/\ker(f)\), per la definició de \myref{def:relació d'equivalència entre grups} tenim que \(x=x'\ast y\) per a uns certs \(x'\in G_{2}\), \(h\in\ker(f)\), i per tant
			\begin{align*}
			\varphi([x])&=\varphi([x'\ast h])\\
			&=\varphi([x']\ast[h])\tag{\myref{def:producte entre classes versió grups}}\\
			&=f(x')\circ f(e)\tag{\myref{def:morfisme entre grups}}\\
			&=f(x')\tag{\myref{def:grup}}\\
			&=f(x\ast h^{-1})\\
			&=f(x)\circ f(h^{-1})\tag{\myref{def:morfisme entre grups}}\\
			&=f(x)\circ f(h)^{-1}\tag{Proposició \myref{prop:morfismes conserven neutre i l'invers commuta amb el morfisme}}\\
			&=f(x)\circ e^{-1}=f(x).\tag{Proposició \myref{prop:l'invers de l'element neutre d'un grup és ell mateix}}
			\end{align*}
			
			Així veiem que \(\Ima(\varphi)=\Ima(f)\). Per tant \(\varphi\) és un isomorfisme, i per la definició d'\myref{def:isomorfisme entre grups} tenim \(G_{1}/\ker(f)\cong\Ima(f)\), com volíem veure. %REFERENCIES BIECTIVITAT ALS ISO-
		\end{proof}
	\end{theorem}
	\begin{theorem}[Primer Teorema de l'isomorfisme]
		\labelname{Primer Teorema de l'isomorfisme entre grups}
		\label{thm:Primer Teorema de l'isomorfisme entre grups}
			Siguin \(G_{1}\) un grup amb l'operació \(\ast\), \(G_{2}\) un grup amb l'operació \(\circ\) i \(f\colon G_{1}\to G_{2}\) un epimorfisme entre grups. Aleshores
		\begin{enumerate}
			\item\label{enum:Primer Teorema de l'isomorfisme entre grups 1} \(G_{1}/\ker(f)\cong G_{2}\).
			\item\label{enum:Primer Teorema de l'isomorfisme entre grups 2} L'aplicació
			\begin{align}
			\label{eq:primer teorema de l'isomorfisme eq2}
			\varphi_{1}\colon\{H\mid\ker(f)\leq H\leq G\}&\longleftrightarrow\{K\mid K\leq G_{2}\}\\
			H&\longmapsto\{f(h)\in G_{2}\mid h\in H\}\nonumber
			\end{align}
			és bijectiva.
			\marginpar{Si ningú ve del futur per aturar-te, com de dolenta pot ser la decisió que estàs prenent?}
			\item\label{enum:Primer Teorema de l'isomorfisme entre grups 3} L'aplicació
			\begin{align}
			\label{eq:primer teorema de l'isomorfisme eq3}
			\varphi_{2}\colon\{H\mid\ker(f)\leq H\trianglelefteq G\}&\longleftrightarrow\{K\mid K\trianglelefteq G_{2}\}\\
			H&\longmapsto\{f(h)\in G_{2}\mid h\in H\}\nonumber
			\end{align}
			és bijectiva.
		\end{enumerate}
		\begin{proof}
			Siguin \(e\) l'element neutre de \(G_{1}\) i \(e'\) l'element neutre de \(G_{2}\).
			
			El punt \eqref{enum:Primer Teorema de l'isomorfisme entre grups 1} és conseqüència del Teorema \myref{thm:Teorema fonamental dels isomorfismes}, ja que si \(f\) és exhaustiva, \(\Ima(f)=G_{2}\), i per tant \(G_{1}/\ker(f)\cong G_{2}\).
			
			Per veure el punt \eqref{enum:Primer Teorema de l'isomorfisme entre grups 2} comencem demostrant que \(\varphi_{1}\) està ben definida. Siguin \(H_{1}=H_{2}\in\{H\mid\ker(f)\leq H\leq G\}\). Aleshores, per la hipòtesi \eqref{eq:primer teorema de l'isomorfisme eq2} tenim \(\varphi_{1}(H_{1})=\{f(h)\in G_{2}\mid h\in H_{1}\}\) i \(\varphi_{1}(H_{2})=\{f(h)\in G_{2}\mid h\in H_{2}\}\), i com que \(f\) és una aplicació, i per tant ben definida, \(\varphi_{1}(H_{1})=\varphi_{1}(H_{2})\).
			
			Continuem comprovant que \(\varphi_{1}\) és bijectiva. Per veure que és injectiva prenem \(H_{1},H_{2}\in\{H\mid\ker(f)\leq H\leq G\}\) tals que \(\varphi_{1}(H_{1})=\varphi_{1}(H_{2})\). Això, per la hipòtesi \eqref{eq:primer teorema de l'isomorfisme eq2} és
			\[\{f(h)\in G_{2}\mid h\in H_{1}\}=\{f(h)\in G_{2}\mid h\in H_{2}\}.\]
			Per tant siguin \(h_{1}\in H_{1}\), \(h_{2}\in H_{2}\) tals que \(f(h_{1})=f(h_{2})\). Equivalentment, per la proposició \myref{prop:unicitat inversa en grups} i la definició de \myref{def:l'invers d'un element d'un grup} i la proposició \myref{prop:l'invers de l'element neutre d'un grup és ell mateix} tenim les igualtats \(f(h_{2}^{-1}\ast h_{1})=f(h_{1}^{-1}\ast h_{2})=e'\), i per la definició de \myref{def:nucli d'un morfisme entre grups} tenim \(h_{2}^{-1}\ast h_{1},h_{1}^{-1}\ast h_{2}\in\ker(f)\), i per la hipòtesi \eqref{eq:primer teorema de l'isomorfisme eq2} això és \(h_{2}^{-1}\ast h_{1}\in\ker(f)\subseteq H_{2}\) i \(h_{1}^{-1}\ast h_{2}\in\ker(f)\subseteq H_{1}\). Observem que això és que \(h_{1}\in\{h_{2}\}\ker(f))\subseteq H_{2}\) i \(h_{2}\in\{h_{1}\}\ker(f))\subseteq H_{1}\). Això vol dir que \(H_{1}\subseteq H_{2}\) i \(H_{2}\subseteq H_{1}\), i per doble inclusió això és \(H_{1}=H_{2}\), com volíem veure. %FER Referència doble inclusió
			
			Per veure que \(\varphi_{1}\) és exhaustiva tenim que per la proposició \myref{prop:les imatges dels subgrups son subgrups + el mateix amb subgrups normals} i per la hipòtesi \eqref{eq:primer teorema de l'isomorfisme eq2} tenim que donat un conjunt \(K\) tal que \(K\leq G_{2}\) aleshores el conjunt \(H=\{k\in G_{1}\mid f(h)\in K\}\) satisfà \(H\leq G_{1}\), i per la definició de \myref{def:nucli d'un morfisme entre grups} tenim que es compleix \(\ker(f)\leq H\leq G_{1}\), i per tant \(\varphi_{1}(H)=K\), i per tant \(\varphi\) és exhaustiva i per tant bijectiva. %FER referències fonaments.
			
			Es pot demostrar el punt \eqref{enum:Primer Teorema de l'isomorfisme entre grups 3} amb el mateix argument que hem donat per demostrar el punt \eqref{enum:Primer Teorema de l'isomorfisme entre grups 2}.
		\end{proof}
	\end{theorem}
	\begin{proposition}
		\label{prop:producte de subgrups amb un de normal es subgrup}
		Siguin \(G\) un grup amb l'operació \(\ast\) i \(H\), \(K\) subgrups de \(G\). Aleshores
		\begin{enumerate}
			\item\label{enum:producte de subgrups amb un de normal es subgrup 1} Si \(K\trianglelefteq G\), aleshores \(HK\leq G\).
			\item\label{enum:producte de subgrups amb un de normal es subgrup 2} Si \(H,K\trianglelefteq G\), aleshores \(HK\trianglelefteq G\).
		\end{enumerate}
		\begin{proof}
			Comencem veient el punt \eqref{enum:producte de subgrups amb un de normal es subgrup 1}. Per la proposició \myref{prop:condició equivalent a subgrup} només ens cal comprovar que per a tot \(x,y\in HK\) es satisfà \(x\ast y^{-1}\in HK\). Siguin doncs \(x,y\in HK\), que podem reescriure com \(x=h_{1}\ast k_{1}\) i \(y=h_{2}\ast k_{2}\). Calculem \(x\ast y^{-1}\):
			\begin{align*}
			x\ast y^{-1}&=h_{1}\ast k_{1}\ast (h_{2}\ast k_{2})^{-1}\\
			&=h_{1}\ast k_{1}\ast k_{2}^{-1}\ast h_{2}^{-1}\tag{Proposició \myref{prop:invers de a b = b invers a invers}}\\
			&=h_{1}\ast k_{1}\ast h_{2}^{-1}\ast k_{2}^{-1},\tag{\myref{def:subgrup normal}}\\
			&=h_{1}\ast h_{2}^{-1}\ast k_{1}\ast k_{2}^{-1},\tag{\myref{def:subgrup normal}}
			\end{align*}
			i com que, per la definició de \myref{def:grup} tenim \(h_{1}\ast h_{2}^{-1}\in H\) i \(k_{1}\ast k_{2}^{-1}\in K\), veiem que \(x\ast y^{-1}\in HK\), com volíem demostrar.
			
			La demostració del punt \eqref{enum:producte de subgrups amb un de normal es subgrup 2} és anàloga a la del punt \eqref{enum:producte de subgrups amb un de normal es subgrup 1}.
		\end{proof}
	\end{proposition}
	\begin{lemma}
		\label{lema:Segon Teorema de l'isomorfisme entre grups}
		Siguin \(G\) un grup amb l'operació \(\ast\), \(H\) un subgrup de \(G\) i \(K\) un subgrup normal de \(G\). Aleshores \(H\cap K\trianglelefteq H\).
		\begin{proof}
			Prenem \(x\in H\) i \(y\in H\cap K\). Per la proposició \myref{prop:condicions equivalents a subgrup normal} només hem de veure que per a tot \(x\in H\) i \(y\in H\cap K\), es satisfà \(x^{-1}\ast y\ast x\in H\). Ara bé, com que \(K\) és un grup normal, per la mateixa proposició \myref{prop:condicions equivalents a subgrup normal}, com que per hipòtesi \(y\in H\cap K\), i en particular \(y\in K\), tenim que \(x^{-1}\ast y\ast x\in K\). Per veure que \(x^{-1}\ast y\ast x\in H\) tenim prou amb veure que \(x,y\in H\), i com que \(H\) és un subgrup de \(G\), i per tant un grup, per la definició de \myref{def:grup} tenim que \(x^{-1}\ast y\ast x\in H\), i per tant \(x^{-1}\ast y\ast x\in H\cap K\), com volíem veure.
		\end{proof}
	\end{lemma}
	\begin{theorem}[Segon Teorema de l'isomorfisme]
		\labelname{Segon Teorema de l'isomorfisme entre grups}
		\label{thm:Segon Teorema de l'isomorfisme entre grups}
		Siguin \(G\) un grup amb l'operació \(\ast\), \(H\) un subgrup de \(G\) i \(K\) un subgrup normal de \(G\). Aleshores
		\[(HK)/K\cong H/(H\cap K).\]
		\begin{proof}
			Aquest enunciat té sentit pel lema \myref{lema:Segon Teorema de l'isomorfisme entre grups}.
			
			Definim
			\begin{align}\label{hipotesi:definició f per segon teorea de l'isomorfisme}
			f\colon HK&\longrightarrow H/(H\cap K)\\
			h\ast k&\longmapsto [h].\nonumber
			\end{align}
			Demostrarem que \(f\) és un epimorfisme; però primer cal veure que \(f\) està ben definida. Prenem doncs \(h_{1}\ast k_{1},h_{2}\ast k_{2}\in HK\) amb \(h_{1},h_{2}\in H\) i \(k_{1},k_{2}\in K\) tals que \(h_{1}\ast k_{1}=h_{2}\ast k_{2}\), i per tant \(h_{2}^{-1}\ast h_{1}=k_{2}\ast k_{1}^{-1}\). Ara bé, com que per hipòtesi i per la definició de \myref{def:l'invers d'un element d'un grup} tenim que \(h_{1},h_{2}^{-1}\in H\) i a la vegada \(k_{2},k_{1}^{-1}\in K\), per la definició de \myref{def:grup} tenim \(h_{2}^{-1}\ast h_{1}\in H\) i \(k_{2}\ast k_{1}^{-1}\in K\) i com que \(h_{2}^{-1}\ast h_{1}=k_{2}\ast k_{1}^{-1}\) tenim que \([h_{2}^{-1}\ast h_{1}]=[k_{2}^{-1}\ast k_{1}]=[e]\), i per la definició de \myref{def:relació d'equivalència entre grups} %areglar referències o definir producte individualment
			tenim que \([h_{1}]=[h_{2}]\) i per tant \(f\) està ben definida.
			
			Veiem ara que \(f\) és un morfisme entre grups. Prenem \(h_{1},h_{2}\in H\) i \(k_{1},k_{2}\in K\), i per tant \(h_{1}\ast k_{1},h_{2}\ast k_{2}\in HK\), i fem
			\begin{align*}
			f(h_{1}\ast k_{1}\ast h_{2}\ast k_{2})&=[h_{1}\ast h_{2}]\tag{Definició \eqref{hipotesi:definició f per segon teorea de l'isomorfisme}}\\
			&=[h_{1}]\ast[h_{2}]\tag{Definició \eqref{def:producte entre classes versió grups}}\\
			&=f(h_{1}\ast k_{1})\ast f(h_{2}\ast k_{2})\tag{Definició \eqref{hipotesi:definició f per segon teorea de l'isomorfisme}}
			\end{align*}
			i per tant \(f\) satisfà la definició de \myref{def:morfisme entre grups}.
			
			Continuem veient que \(f\) és exhaustiva. Prenem \([h]\in H/(H\cap K)\). Per la definició \eqref{hipotesi:definició f per segon teorea de l'isomorfisme} tenim que \(f(h\ast k)=[h]\) per a qualsevol \(k\in K\), i per tant \(f\) és exhaustiva.
			
			Per tant \(f\) és un epimorfisme, i per tant, pel \myref{thm:Primer Teorema de l'isomorfisme entre grups} tenim
			\[HK/\ker(f)\cong H/(H\cap K).\]
			Ara bé, per la definició de \myref{def:nucli d'un morfisme entre grups} tenim que \(\ker(f)=\{h_{1}\ast k_{2}\in HK\mid f(h_{1}\ast k_{1})=[e]\}\), i per tant \(\ker(f)=K\) i trobem
			\[HK/K\cong H/(H\cap K).\qedhere\]
		\end{proof}
	\end{theorem}
	\begin{theorem}[Tercer Teorema de l'isomorfisme]
		\labelname{Tercer Teorema de l'isomorfisme entre grups}
		\label{thm:Tercer Teorema de l'isomorfisme entre grups}
		Siguin \(G\) un grup amb l'operació \(\ast\) i \(H\), \(K\) dos subgrups normals de \(G\) amb \(K\subseteq H\). Aleshores
		\[G/H\cong (G/K)/(H/K).\]
		\begin{proof}
			Definim les aplicacions
			\begin{align*}
			\varphi_{1}\colon G&\longrightarrow G/K&\text{i}&&\varphi_{2}\colon G/K&\longrightarrow(G/K)/(H/K)\\
			g&\longmapsto[g]&&&[g]&\longmapsto\overline{[g]}.
			\end{align*}
			
			Veiem que \(\varphi_{1}\) i \(\varphi_{2}\) són morfismes.
			
			Per la proposició \myref{prop:conjugació de morfismes entre grups és morfisme entre grups} tenim que \(\varphi_{2}(\varphi_{1})\colon G\longrightarrow(G/K)/(H/K)\) és un epimorfisme entre grups, %fer referencies fonaments.
			i pel \myref{thm:Primer Teorema de l'isomorfisme entre grups} trobem
			\[G/\ker(\varphi_{2}(\varphi_{1}))\cong(G/K)/(H/K).\]
			
			Veiem ara que \(\ker(\varphi_{2}(\varphi_{1}))=H\). Per la definició de \myref{def:relació d'equivalència entre grups} tenim que \(G/K=\{gK\mid g\in G\}\) i \(H/K=\{hK\mid h\in H\}\), i per tant
			\begin{equation}
			\label{eq:Tercer Teorema de l'isomorfisme eq1}
			(G/K)/(H/K)=\{gKhK\mid g\in G, h\in H\},
			\end{equation}
			però com que, per hipòtesi, \(H\) i \(K\) són subgrups normals de \(G\), per la definició de \myref{def:subgrup normal} podem reescriure \eqref{eq:Tercer Teorema de l'isomorfisme eq1} com
			
			\begin{equation}\label{eq:Tercer Teorema de l'isomorfisme eq2}
			(G/K)/(H/K)=\{ghK\mid g\in G,h\in H\}.
			\end{equation}
			Ara bé, com que per hipòtesi \(K\subseteq H\) podem reescriure \eqref{eq:Tercer Teorema de l'isomorfisme eq2} com
			\[(G/K)/(H/K)=\{gH\mid g\in G\},\]
			i trobem, per la definició de \myref{def:nucli d'un morfisme entre grups}, que \(\ker(\varphi_{2}(\varphi_{1}))=H\), i per tant
			\[G/H\cong (G/K)/(H/K).\qedhere\]
		\end{proof}
	\end{theorem}
	\section{Tres Teoremes de Sylow}
	\subsection{Accions sobre grups}
	\begin{definition}[Acció d'un grup sobre un conjunt]
		\labelname{acció d'un grup sobre un conjunt}
		\label{def:acció d'un grup sobre un conjunt}
		Siguin \(G\) un grup amb l'operació \(\ast\) i element neutre \(e\), \(X\) un conjunt no buit i
		\begin{align*}
		\cdot\colon G\times X&\longrightarrow X\\
		(g,x)&\longmapsto g\cdot x
		\end{align*}
		una operació que satisfaci
		\begin{enumerate}
			\item \(e\cdot x=x\) per a tot \(x\in X\).
			\item \((g_{1}\ast g_{2})\cdot x=g_{1}\cdot(g_{2}\cdot x)\) per a tot \(x\in X\), \(g_{1},g_{2}\in G\).
		\end{enumerate}
		Aleshores direm que \(\cdot\) és una acció de \(G\) sobre \(X\). També direm que \(X\) és un \(G\)-conjunt amb l'acció \(\cdot\).
	\end{definition}
	\begin{proposition}
		\label{prop:relacio d'òrbites és d'equivalència}
		Siguin \(G\) un grup amb l'operació \(\ast\), \(X\) un conjunt, \(\cdot\) una acció de \(G\) sobre \(X\) i \(\sim\) una relació sobre \(X\) tal que per a tot \(x_{1},x_{2}\in X\) diem que \(x_{1}\sim x_{2}\) si i només si existeix un \(g\in G\) tal que \(x_{1}=g\cdot x_{2}\). Aleshores la relació \(\sim\) és una relació d'equivalència.
		\begin{proof}
			Sigui \(e\) l'element neutre del grup \(G\). Comprovem que \(\sim\) satisfà la definició de \myref{def:relació d'equivalència}:
			\begin{enumerate}
				\item Reflexiva: Sigui \(x\in X\). Per la definició de \myref{def:acció d'un grup sobre un conjunt} tenim que \(x=x\cdot e\), i per tant \(x\sim x\).
				\item Simètrica: Siguin \(x_{1},x_{2}\in X\) tals que \(x_{1}\sim x_{2}\). Per tant existeix \(g\in G\) tal que \(x_{1}=g\cdot x_{2}\). Per la definició de \myref{def:acció d'un grup sobre un conjunt} tenim que \(g\cdot x_{2}\in X\), i per tant podem prendre \(g^{-2}\cdot(g\ast x_{2})\), que és equivalent a \(g^{-2}\cdot(g\cdot x_{2})=g^{-2}\cdot x_{1}\), i així \(x_{2}=g^{-1}\cdot x_{1}\), i per tant \(x_{2}\sim x_{1}\).
				\item Transitiva: Siguin \(x_{1},x_{2},x_{3}\in X\) tals que \(x_{1}\sim x_{2}\) i \(x_{2}\sim x_{3}\). Per tant existeixen \(g_{1},g_{2}\in G\) tals que \(x_{1}=g_{1}\cdot x_{2}\) i \(x_{2}=g_{2}\cdot x_{3}\), i per tant \(x_{1}=g_{1}\cdot(g_{2}\cdot x_{3})\), i per la definició de \myref{def:acció d'un grup sobre un conjunt} això és \(x_{1}=(g_{1}\ast g_{2})\cdot x_{3}\), i com que \(G\) és un grup, \(g_{1}\ast g_{2}\in G\), i tenim que \(x_{1}\sim x_{3}\).
			\end{enumerate}
			per tant \(\sim\) és una relació d'equivalència.
		\end{proof}
	\end{proposition}
	\begin{definition}[Òrbita d'un element d'un \(G\)-conjunt]
		\labelname{l'òrbita d'un element d'un \ensuremath{G}-conjunt}\label{def:òrbita d'un element d'un G-conjunt}
		Siguin \(G\) un grup amb l'operació \(\ast\), \(\cdot\) una acció de \(G\) sobre un conjunt \(X\) i \(\sim\) una relació d'equivalència sobre \(X\) tal que per a tot \(x_{1},x_{2}\in X\) diem que \(x_{1}\sim x_{2}\) si i només si existeix un \(g\in G\) tal que \(x_{1}=g\cdot x_{2}\). Aleshores direm que \(\mathcal{O}(x)=[x]\) és l'òrbita de \(x\).
		
		Aquesta definició té sentit per la proposició \myref{prop:relacio d'òrbites és d'equivalència}.
	\end{definition}
	\begin{definition}[Estabilitzador d'un element per una acció]
		\labelname{l'estabilitzador d'un element per una acció}
		\label{def:l'estabilitzador d'un element per una acció}
		Siguin \(G\) un grup amb l'operació \(\ast\), \(X\) un conjunt i \(\cdot\) una acció de \(G\) sobre \(X\). Aleshores direm que el conjunt
		\[\St(x)=\{g\in G\mid g\cdot x=x\}\]
		és l'estabilitzador de \(x\) per l'acció \(\cdot\).
	\end{definition}
	\begin{proposition}
		\label{prop:l'invers d'un element pertany a l'estabilitzador per una acció}
		Siguin \(G\) un grup amb l'operació \(\ast\), \(\cdot\) una acció de \(G\) sobre un conjunt \(X\) i \(\St(x)\) l'estabilitzador d'un element \(x\) de \(X\) per l'acció \(\cdot\). Aleshores \(g\in\St(x)\) si i només si \(g^{-1}\in\St(x)\).
		\begin{proof}
			Per la definició de \myref{def:l'estabilitzador d'un element per una acció} tenim que \(g\in\St(x)\) si i només si \(g\cdot x=x\). Ara bé, si prenem \(g^{-1}\cdot (g\cdot x)=g^{-1}\cdot x\), i per la definició de \myref{def:acció d'un grup sobre un conjunt} tenim que \(g^{-1}\cdot(g\cdot x)=(g^{-1}\ast g)\cdot x\), i per la definició de \myref{def:l'invers d'un element d'un grup} tenim que \(g^{-1}\cdot x\) i per tant \(g^{-1}\in\St(x)\).
		\end{proof}
	\end{proposition}
	\begin{proposition}
		\label{prop:l'estabilitzador és un subgrup}
		Siguin \(G\) un grup amb l'operació \(\ast\), \(\cdot\) una acció de \(G\) sobre un conjunt \(X\) i \(\St(x)\) l'estabilitzador de \(x\) per l'acció \(\cdot\). Aleshores \(\St(x)\) és un subgrup de \(G\).
		\begin{proof}
			Per la proposició \myref{prop:condició equivalent a subgrup} només ens cal veure que per a tot \(g_{1},g_{2}\in\St(x)\) es compleix \(g_{1}\ast g_{2}^{-1}\in\St(x)\).
			
			Prenem doncs \(g_{1},g_{2}\in\St(x)\). Per la proposició \myref{prop:l'invers d'un element pertany a l'estabilitzador per una acció} tenim que \(g_{2}^{-1}\in\St(x)\), i per tant, per la definició de \myref{def:l'estabilitzador d'un element per una acció} tenim que \((g_{1}\ast g_{2}^{-1})\cdot x=x\), i per tant \(g_{1}\ast g_{2}^{-1}\in\St(x)\).
		\end{proof}
	\end{proposition}
	\begin{proposition}
		\label{prop:cardinal del grup dividit per cardinal de l'estabilitzador és el cardinal de l'òrbita}
		Siguin \(G\) un grup d'ordre finit amb l'operació \(\ast\), \(X\) un \(G\)-conjunt finit amb una acció \(\cdot\) i \(\St(x)\) l'estabilitzador d'un element \(x\) de \(X\) per l'acció \(\cdot\). Aleshores
		\[\abs{G/\St(x)}=\abs{\mathcal{O}(x)}=[G:\St(x)].\]
		\begin{proof}
			Considerem
			\begin{align*}
			f\colon G/\St(x)&\longrightarrow\mathcal{O}(x)\\
			[g]&\longmapsto g\cdot x
			\end{align*}
			Volem veure que \(f\) és una aplicació bijectiva, per tant mirem si està ben definida:
			
			Siguin \([g_{1}],[g_{2}]\in G/\St(x)\) tals que \([g_{1}]=[g_{2}]\). Per tant \(g_{1}=g_{2}\ast g'\) per a cert \(g'\in\St(x)\), i per la definició de \myref{def:l'estabilitzador d'un element per una acció} tenim \(g_{1}\cdot x=x\), \((g_{2}\ast g')\cdot x=x\), i per la definició de \myref{def:acció d'un grup sobre un conjunt} això és \(g_{2}\cdot(g'\cdot x)=x\), i per la proposició \myref{prop:l'invers d'un element pertany a l'estabilitzador per una acció} \(g_{2}\cdot x=x\), i per tant \(f\) està ben definida.
			
			Veiem ara que \(f\) és injectiva. Prenem \(g\cdot x,g'\cdot x\in\mathcal{O}(x)\) tals que \(g\cdot x=g'\cdot x\). Això, per la definició de \myref{def:acció d'un grup sobre un conjunt}, és equivalent a dir \(x=(g^{-1}\ast g')\cdot x\), i per la definició de \myref{def:l'estabilitzador d'un element per una acció} tenim que \(g^{-1}\ast g\in\St(x)\). Per tant, per la definició de \myref{def:relació d'equivalència entre grups} tenim que \([g]=[g']\), i per tant \(f\) és injectiva.
			
			Per veure que \(f\) és exhaustiva veiem que per a qualsevol \(g\cdot x\in\mathcal{O}(x)\), \(f([g])=g\cdot x\), i per tant \(f\) és exhaustiva i tenim que \(\abs{\St(x)}=\abs{\mathcal{O}(x)}\), ja que \(f\) és una bijecció. %referencies fonaments!
		\end{proof}
	\end{proposition}
	\subsection{Teoremes de Sylow}
	\begin{definition}[\(p\)-subgrup de Sylow]
		\labelname{\ensuremath{p}-subgrup de Sylow}
		\label{def:p-subgrup de Sylow}
		Siguin \(G\) un grup amb l'operació \(\ast\) tal que \(\abs{G}=p^{n}m\) amb \(p\) primer que no divideix \(m\) i \(P\) un subgrup de \(G\) amb \(\abs{P}=p^{n}\). Aleshores direm que \(P\) és un \(p\)-subgrup de Sylow de \(G\).
	\end{definition}
	\begin{lemma}
		\label{lema:Primer Teorema de Sylow}
		Siguin \(p\) un primer i \(m\) un enter positiu tal que \(p\) no divideixi \(m\). Aleshores per a tot \(n\) natural tenim que
		\begin{equation}
		\left(\begin{matrix}
		\label{eq:lema:Primer Teorema de Sylow 0}
		p^{n}m\\p^{n}
		\end{matrix}\right)
		\end{equation}
		no és divisible per \(p\).
		\begin{proof}
			Tenim que
			\begin{equation}
		\label{eq:lema:Primer Teorema de Sylow 1}
			\left(\begin{matrix}
			p^{n}m\\p^{n}
			\end{matrix}\right)
			=\frac{p^{n}m(p^{n}m-1)\cdots(p^{n}m-p^{n}+1)}{p^{n}(p^{n}-1)\cdots(p^{n}-p^{n}+1)}=\prod_{i=0}^{p^{n}-1}\frac{p^{n}m-i}{p^{n}-i}.
			\end{equation}
			Com que, per hipòtesi, \(p\) és primer aquesta expressió només serà divisible per \(p\) si ho són els elements del numerador. Fixem doncs \(i\in\{0,\dots,p^{n}+1\}\) i estudiem l'\(i\)-èsim terme del producte de \eqref{eq:lema:Primer Teorema de Sylow 1}. Notem primer que si \(i=0\) aquest terme no és divisible per \(p\). Imposem ara també que \(i\neq0\). Si el denominador, \(p^{n}m-i\), és divisible per \(p\) tindrem que \(p^{n}m-i=p^{k}m'\), on \(m'\) no és divisible per \(p\), i per tant \(k\) serà l'exponent més gran que satisfaci la igualtat. Si aïllem \(i\) obtindrem \(i=p^{k}(p^{n-k}m-m')\). Ara bé, tenim doncs que \(p^{n}-i=p^{n}-p^{k}(p^{n-k}m-m')\), i això és \(p^{n}-p^{k}(p^{n-k}m-m')=p^{k}(p^{n-k}(1-m)-m')\), i per tant tindrem que l'\(i\)-èsim terme del producte de \eqref{eq:lema:Primer Teorema de Sylow 1} serà de la forma
			\[\frac{p^{n}m-i}{p^{n}-i}=\frac{p^{k}m'}{p^{k}(p^{n-k}(1-m)-m')}=\frac{m'}{p^{n-k}(1-m)-m'},\]
			i així veiem que aquest \(i\)-èsim terme del producte no serà divisible per \(p\); i com que això és cert per a qualsevol \(i\in\{0,\dots,p^{n}+1\}\) tenim que \eqref{eq:lema:Primer Teorema de Sylow 0} tampoc ho serà, com volíem veure.
		\end{proof}
	\end{lemma}
	\begin{theorem}[Primer Teorema de Sylow]
		\labelname{Primer Teorema de Sylow}
		\label{thm:Primer Teorema de Sylow}
		Siguin \(G\) un grup amb l'operació \(\ast\) tal que \(\abs{G}=p^{n}m\) amb \(p\) primer que no divideix \(m\). Aleshores existeix un subconjunt \(P\subseteq G\) tal que \(P\) sigui un \(p\)-subgrup de Sylow de \(G\).
		\begin{proof}%Demostració de Wielandt
			Sigui \(\mathcal{P}_{p^{n}}(G)=\{H\subseteq G\mid\lvert H\rvert=p^{n}\}=\{H_{1},\dots,H_{k}\}\) el conjunt de subconjunts d'ordre \(p^{n}\) de \(G\). Aleshores tenim que
			\begin{displaymath}
			k=\abs{\mathcal{P}_{p^{n}}(G)}=\left(\begin{matrix}p^{n}m\\p^{n}\end{matrix}\right),
			\end{displaymath}
			i pel lema \myref{lema:Primer Teorema de Sylow} tenim que \(p\) no divideix \(k\). %Fer això a fonaments, definint \mathcal{P}_{n}(G)
			
			Sigui \(e\) l'element neutre de \(G\). Definim
			\begin{align}\label{eq:thm:Primer Teorema de Sylow 1}
			\cdot\colon G\times\mathcal{P}_{p^{n}}(G)&\longrightarrow \mathcal{P}_{p^{n}}(G)\\
			(g,X)&\longmapsto\{g\}X.\nonumber
			\end{align}
			Veurem que \(\cdot\) és una acció de \(G\) sobre \(\mathcal{P}_{p^{n}}(G)\). Primer hem de veure que, efectivament, \(\cdot\) està ben definida. Prenem \(g_{1},g_{2}\in G\) i \(X_{1},X_{2}\in \mathcal{P}_{p^{n}}(G)\) tals que \(g_{1}=g_{2}\) i \(X_{1}=X_{2}\). Per tant tenim \(g_{1}\cdot X_{1}=g_{2}\cdot X_{2}\) ja que \(\{g_{1}\}X_{1}=\{g_{2}\}X_{2}\). Per veure que \(g_{1}\cdot X_{1}\in\mathcal{P}_{p^{n}}(G)\) en tenim prou amb veure que, per a tot \(X\in\mathcal{P}_{p^{n}}(G)\), si fixem \(g\in G\) l'aplicació \(g\cdot X\) té inversa, que per la definició de \myref{def:l'invers d'un element d'un grup} és \(x^{-1}\cdot X\), i per tant \(X\in\mathcal{P}_{p^{n}}(G)\).
			
			Comprovem que \(\cdot\) satisfà la definició de \myref{def:acció d'un grup sobre un conjunt}. Tenim que \(e\cdot X=X\) per a tot \(X\in\mathcal{P}_{p^{n}}(G)\) ja que, per la definició de \myref{def:conjugació entre conjunts sobre grups}, \(eX=X\).
			
			De nou per la definició de \myref{def:conjugació entre conjunts sobre grups} veiem que per a tot \(g_{1},g_{2}\in G\) i \(X\in\mathcal{P}_{p^{n}}(G)\) tenim que \begin{align*}
			(g_{1}\ast g_{2})\cdot X&=\{g_{1}\ast g_{2}\}X\\
			&=\{g_{1}\}\{g_{2}\}X\\
			&=\{g_{1}\}(\{g_{2}\}X)=g_{1}\cdot(g_{2}\cdot X).
			\end{align*}
			i per tant \(\cdot\) satisfà la definició de \myref{def:acció d'un grup sobre un conjunt}.
			
			Veiem ara que existeix almenys un element \(X\in\mathcal{P}_{p^{n}}(G)\) tal que la seva òrbita, \(\mathcal{O}(X)\), tingui ordre no divisible per \(p\). Per veure això observem que per la definició de \myref{def:òrbita d'un element d'un G-conjunt} veiem que \(\mathcal{O}(X)\) és un classe d'equivalència, i per tant l'ordre del conjunt \(\mathcal{P}_{p^{n}}\) és la suma dels ordres de les òrbites dels seus elements, \(\mathcal{O}(X)\), i si \(p\) dividís l'ordre de \(\mathcal{O}(X)\) per a tot \(X\in\mathcal{P}_{p^{n}}\) tindríem que \(p\) també divideix \(k\), però ja hem vist que això no pot ser. Per tant existeix almenys un element \(X\in\mathcal{P}_{p^{n}}\) tal que \(\abs{\mathcal{O}(X)}\) no és divisible per \(p\). Fixem aquest conjunt \(X\).
			
			Prenem l'estabilitzador de \(X\), \(\St(X)\). Per la proposició \myref{prop:cardinal del grup dividit per cardinal de l'estabilitzador és el cardinal de l'òrbita} tenim que \(\abs{\St(X)}\) divideix \(p^{n}\). Prenem també \(x_{o}\in X\) i \(g\in\St(X)\). Per la definició de \myref{def:l'estabilitzador d'un element per una acció} tenim que \(\{g\}X=X\), i per tant \(g\ast x_{0}\in X\), i equivalentment \(g\in X\{x_{0}^{-1}\}\). Així veiem que \(\St(X)\subseteq X\{x_{0}^{-1}\}\), i per tant tenim que \(\abs{\St(X)}\leq\abs{X\{x_{0}\}}\). Observem que \(X\{x_{0}\}\in\mathcal{P}_{p^{n}}(G)\) i per tant \(\abs{X\{x_{0}\}}=p^{n}\). Ara bé, l'ordre de \(\St(X)\) divideix \(p^{n}\), però acabem de veure que \(\abs{\St(X)}\leq p^{n}\). Per tant ha de ser \(\abs{\St(X)}=p^{n}\), i per tant, per la proposició \myref{prop:l'estabilitzador és un subgrup} tenim que \(\St(X)\leq G\), i per tant, per la definició de \myref{def:p-subgrup de Sylow}, \(\St(X)\) és un \(p\)-subgrup de Sylow.
		\end{proof}
	\end{theorem}
	\begin{corollary}[Teorema de Cauchy per grups]
		\labelname{Teorema de Cauchy per grups}
		\label{thm:Teorema de Cauchy per grups}
		Siguin \(G\) un grup d'ordre finit amb l'operació \(\ast\) i \(p\) un primer que divideix l'ordre de \(G\). Aleshores existeix un element \(g\in G\) tal que l'ordre de \(g\) sigui \(p\).
		\begin{proof}
			Direm que \(e\) és l'element neutre de \(G\). Pel \myref{thm:Primer Teorema de Sylow} tenim que existeix un \(p\)-subgrup de Sylow \(P\) de \(G\), que per la definició de \myref{def:p-subgrup de Sylow} té ordre \(p^{n}\) per a cert \(n\in\mathbb{N}\). Ara bé, pel \myref{thm:Teorema de Lagrange} tenim que per a tot \(x\in P\) diferent del neutre el grup cíclic generar per \(x\) ha de tenir ordre \(p^{t}\) amb \(t\in\{2,\dots,n\}\), i per tant l'element \(x^{p^{t-1}}\) té ordre \(p\), ja que 
			\[\left(x^{p^{t-1}}\right)^{p}=x^{p^{t}}=e.\qedhere\]
		\end{proof}
	\end{corollary}
	\begin{lemma}
		\label{lema:Segon Teorema de Sylow}
		Siguin \(G\) un grup d'ordre \(p^{n}\) on \(p\) és un primer amb l'operació \(\ast\), \(X\) un \(G\)-conjunt d'ordre finit amb una acció \(\cdot\) i \(X_{G}=\{x\in X\mid g\cdot x=x\text{ per a tot }g\in G\}\) un conjunt. Aleshores
		\[\abs{X_{G}}\equiv\abs{X}\pmod{p}.\]
		\begin{proof}
			Siguin \(\mathcal{O}(x_{1}),\dots,\mathcal{O}(x_{r})\) les òrbites dels elements de \(X\). Aleshores, com que per la definició de \myref{def:òrbita d'un element d'un G-conjunt} aquestes són classes d'equivalència, tenim que %REFERENCIES fonaments
			\[X=\bigcup_{i=1}^{r}\mathcal{O}(x_{i}),\]
			i com que aquestes òrbites són disjuntes per ser classes d'equivalència
			\begin{equation}
		\label{eq:lema:Segon Teorema de Sylow 1}
			\abs{X}=\sum_{i=0}^{r}\abs{\mathcal{O}(x_{i})}.
			\end{equation}
			Ara bé, per les proposicions \myref{prop:cardinal del grup dividit per cardinal de l'estabilitzador és el cardinal de l'òrbita} i \myref{prop:l'estabilitzador és un subgrup} i el \myref{thm:Teorema de Lagrange} tenim que l'ordre \(\mathcal{O}(x_{i})\) divideix l'ordre de \(X\), i per tant els únics elements amb òrbites que tinguin un ordre que no sigui divisible per \(p\) són els elements del conjunt \(X_{G}\); i com que les òrbites d'aquests elements tenen un únic element tenim que
			\[\abs{X_{G}}=\sum_{x\in X_{G}}\abs{\mathcal{O}(x)},\]
			i per tant, recordant que totes les altres òrbites tenen ordre divisible per \(p\), trobem, amb \eqref{eq:lema:Segon Teorema de Sylow 1}, que
			\[\abs{X_{G}}\equiv\abs{X}\pmod{p}.\qedhere\]
		\end{proof}
	\end{lemma}
	\begin{theorem}[Segon Teorema de Sylow]
		\labelname{Segon Teorema de Sylow}
		\label{thm:Segon Teorema de Sylow}
		Siguin \(G\) un grup d'ordre finit amb l'operació \(\ast\), \(p\) un primer que divideixi l'ordre de \(G\) i \(P_{1}\), \(P_{2}\) dos \(p\)-subgrups de Sylow de \(G\). Aleshores existeix \(g\in G\) tal que
		\[\{g\}P_{1}\{g^{-1}\}=P_{2}.\]
		\begin{proof}
			Primer observem que aquest enunciat té sentit pel \myref{thm:Primer Teorema de Sylow}.
			
			Definim el conjunt \(X=\{\{x\}P_{1}\mid x\in G\}\) i 
			\begin{align}\label{eq:thm:Segon Teorema de Sylow 1}
			\cdot\colon P_{2}\times X&\longrightarrow X\\
			(y,\{x\}P_{1})&\longmapsto \{y\}\{x\}P_{1}.\nonumber
			\end{align}
			
			Primer veurem que \(\cdot\) és una acció. Per veure que \(\cdot\) està ben definida prenem \(\{x\}P_{1},\{x'\}P_{1}\in X\) tals que \(\{x\}P_{1}=\{x'\}P_{1}\).  Aleshores per a tot \(y\in P_{2}\) tindrem \(y\cdot\{x\}P_{1}=\{y\}\{x\}P_{1}\) i \(y\cdot\{x'\}P_{1}=\{y\}\{x'\}P_{1}\), i com que per hipòtesi \(\{x\}P_{1}=\{x'\}P_{1}\), ha de ser \(\{y\}\{x\}P_{1}=\{y\}\{x'\}P_{1}\).
			
			Comprovem que \(\cdot\) satisfà la definició de \myref{def:acció d'un grup sobre un conjunt}. Veiem que per a tot \(y\in P_{2}\), \(\{x\}P_{1}\in X\) es compleix que \(y\cdot\{x\}P_{1}\in X\). Per la definició \eqref{eq:thm:Segon Teorema de Sylow 1} tenim \(y\cdot\{x\}P_{1}\in X=\{y\}\{x\}P_{1}=\{y\ast x\}P_{1}\), i com que per hipòtesi \(G\) és un grup i \(x,y\in G\), per la definició de \myref{def:grup} tenim que \(y\ast x\in G\), i per tant \(y\cdot\{x\}P_{1}\in X\).
			
			Sigui \(e\) l'element neutre de \(G\). Tenim que
			\begin{align*}
			e\cdot \{x\}P_{1}&=\{e\}\{x\}P_{1}\\
			&=\{e\ast x\}P_{1}=\{x\}P_{1}.\tag{\myref{def:l'element neutre del grup}}
			\end{align*}
			i per últim veiem que per a tot \(y,y'\in G\) tenim \((y\ast y')\cdot P_{1}=y\cdot(y'\cdot P_{1})\). Això és
			\begin{align*}
			(y\ast y')\cdot P_{1}&=\{y\ast y'\}P_{1}\\
			&=\{y\}\{y'\}P_{1}\\
			&=\{y\}(\{y'\}P_{1})=y\cdot(y'\cdot P_{1}).\tag{Definició \eqref{eq:thm:Segon Teorema de Sylow 1}}
			\end{align*}
			i per la definició de \myref{def:acció d'un grup sobre un conjunt} tenim que \(X\) és un \(G\)-conjunt.
			
			Definim el conjunt
			\begin{equation}
		\label{eq:thm:Segon Teorema de Sylow 2}
			X_{P_{2}}=\{\{x\}P_{1}\in X\mid y\cdot\{x\}P_{1}=\{x\}P_{1}\text{ per a tot }y\in P_{2}\}.
			\end{equation}
			Aleshores pel lema \myref{lema:Segon Teorema de Sylow} tenim que
			\begin{equation*}
			\abs{X_{P_{2}}}\equiv\abs{X}\pmod{p}.
			\end{equation*}
			Ara bé, per la definició de \myref{def:l'índex d'un subgrup en un grup} i \eqref{eq:thm:Segon Teorema de Sylow 1} tenim que \(\abs{X}=[G:P_{1}]\), i per hipòtesi tenim que \(\abs{X}=\frac{p^{n}m}{p^{n}}=m\), en particular \(\abs{X_{P_{2}}}\neq0\). Així veiem que existeix almenys un element que satisfà la definició de \(X_{P_{2}}\), \eqref{eq:thm:Segon Teorema de Sylow 2}; és a dir, existeix almenys un \(\{x\}P_{1}\) tal que \(y\cdot\{x\}P_{1}=\{x\}P_{1}\) per a tot \(y\in P_{2}\), i per tant tenim que \(\{y\}\{x\}P_{1}=\{x\}P_{1}\), i per tant \(\{x^{-1}\}\{y\}\{x\}P_{1}\in \{x^{-1}\}\{x\}P_{1}\), i equivalentment \(x^{-1}\ast y\ast x\in P_{1}\) per a tot \(y\in P_{2}\), i per tant \(\{x^{-1}\}P_{2}\{x\}\subseteq P_{1}\), però, per hipòtesi, al ser els dos \(p\)-subgrups de Sylow, per la definició de \myref{def:p-subgrup de Sylow} trobem \(\abs{P_{1}}=\abs{P_{2}}=\abs{\{x^{-1}\}P_{2}\{x\}}\) i tenim que \(\{x\}P_{1}\{x^{-1}\}=P_{2}\).
		\end{proof}
	\end{theorem}
	\begin{corollary}
		\label{corollary:Segon Teorema de Sylow}
		\(G\) té un únic \(p\)-subgrup de Sylow si i només si aquest és un subgrup normal.
	\end{corollary}
	\begin{theorem}[Tercer Teorema de Sylow]
		\labelname{Tercer Teorema de Sylow}
		\label{thm:Tercer Teorema de Sylow}
		Siguin \(G\) un grup d'ordre \(p^{n}m\) on \(p\) és un primer que no divideix \(m\) amb l'operació \(\ast\) i \(n_{p}\) el número de \(p\)-subgrups de Sylow de \(G\). Aleshores
		\(n_{p}\equiv1\pmod{p}\) i \(n_{p}\) divideix l'ordre de \(G\).
		\begin{proof}
			Definim el conjunt
			\[X=\{T\subseteq G\mid T\text{ és un }p\text{-subgrup de Sylow de }G\}.\]
			Pel \myref{thm:Primer Teorema de Sylow} tenim que \(X\) és no buit\footnote{Tindrem que \(\abs{X}=n_{p}\).} i fixem \(P\in X\).
			
			Definim
			\begin{align}
		\label{eq:thm:Tercer Teorema de Sylow 1}
			\cdot\colon P\times X&\longrightarrow X\\
			(g,T)&\longmapsto\{g\}T\{g^{-1}\}.\nonumber
			\end{align}
			Anem a veure que \(\cdot\) és una acció. Veiem que \(\cdot\) està ben definida, ja que si \(T\in X\), aleshores per a tot \(x\in G\), i en particular per a tot \(x\in P\) ja que \(P\leq G\), tenim \(\abs{\{x\}T\{x^{-1}\}}=\abs{T}\), i per tant \(\abs{\{x\}T\{x^{-1}\}}\in X\) per la definició de \myref{def:p-subgrup de Sylow}. Veiem ara que \(\cdot\) satisfà les condicions de la definició de \myref{def:acció d'un grup sobre un conjunt}. Sigui \(e\) l'element neutre de \(G\). Tenim que per a tot \(T\in X\)
			\begin{align*}
			e\cdot T&=\{e\}T\{e^{-1}\}\tag{Definició \eqref{eq:thm:Tercer Teorema de Sylow 1}}\\
			&=T
			\end{align*}
			i per a tot \(g_{1},g_{2}\in P\) i \(T\in X\)
			\begin{align*}
			(g_{1}\ast g_{2})\cdot T&=\{g_{1}\ast g_{2}\}T\{{g_{1}\ast g_{2}}^{-1}\}			\tag{Definició \eqref{eq:thm:Tercer Teorema de Sylow 1}}\\
			&=\{g_{1}\ast g_{2}\}T\{g_{2}^{-1}\ast g_{1}^{-1}\}\tag{Proposició \myref{prop:invers de a b = b invers a invers}}\\
			&=\{g_{1}\}\{g_{2}\}T\{g_{2}^{-1}\}\{g_{1}^{-1}\}\\
			&=\{g_{1}\}(g_{2}\cdot T)\{g_{1}^{-1}\}\tag{Definició \eqref{eq:thm:Tercer Teorema de Sylow 1}}\\
			&=g_{1}\cdot(g_{2}\cdot T)\tag{Definició \eqref{eq:thm:Tercer Teorema de Sylow 1}}
			\end{align*}
			i per tant, per la definició de \myref{def:acció d'un grup sobre un conjunt} \(X\) és un \(P\)-conjunt amb l'acció \(\cdot\).
			
			Definim el conjunt
			\[X_{P}=\{T\in X\mid g\cdot T=T\text{ per a tot }g\in P\},\]
			i per la definició \eqref{eq:thm:Tercer Teorema de Sylow 1} tenim que si \(T\in X_{P}\) aleshores per a tot \(g\in G\) es compleix \(\{g\}T\{g^{-1}\}=T\). Ara bé, això és que \(T=P\) per a tot \(T\in X_{P}\), i per tant \(\abs{X_{P}}=1\). Aleshores pel lema \myref{lema:Segon Teorema de Sylow} tenim que
			\[\abs{X}\equiv\abs{X_{P}}\pmod{p},\]
			o equivalentment
			\[\abs{X}\equiv1\pmod{p}.\]
			
			Per veure que \(\abs{X}\) divideix l'ordre de \(G\) prenem \(P\in X\) i tenim, pel \myref{thm:Segon Teorema de Sylow} i la definició de \myref{def:òrbita d'un element d'un G-conjunt}, que
			\[\mathcal{O}(P)=X,\]
			on \(\mathcal{O}(P)\) és l'òrbita de \(P\), i per tant
			\[\abs{\mathcal{O}(P)}=\abs{X},\]
			i per les proposicions \myref{prop:l'estabilitzador és un subgrup} i \myref{prop:cardinal del grup dividit per cardinal de l'estabilitzador és el cardinal de l'òrbita} i el \myref{thm:Teorema de Lagrange} tenim que \(\abs{X}\) divideix l'ordre de \(G\).
		\end{proof}
	\end{theorem}
	\begin{corollary}
		Si \(G\) té ordre \(p^{n}q^{m}\) on \(p,q\) són primers amb \(p<q\) aleshores \(n_{q}=1\), i pel corol·lari \myref{corollary:Segon Teorema de Sylow}, aquest és normal en \(G\).
	\end{corollary}
	\chapter{Teoria d'anells}
	\section{Anells}
	\subsection{Propietats bàsiques dels anells i subanells}
	\begin{definition}[Anell]
		\labelname{anell}
		\label{def:anell}
		\labelname{anell commutatiu}
		\label{def:anell commutatiu}
		\labelname{anell i element neutre pel producte}
		\label{def:anell i element neutre pel producte}
		Sigui \(R\) un conjunt no buit i
		\[+\colon R\times R\longrightarrow R\qquad\qquad\quad\cdot\colon R\times R\longrightarrow R\]
		dues operacions que satisfan
		\begin{enumerate}
			\item \(R\) amb l'operació \(+\) és un grup abelià.
			\item Existeix un element \(e\) de \(R\) tal que \(x\cdot e=e\cdot x=x\) per a tot \(x\in R\).
			\item Per a tot \(x,y,z\in R\) tenim
			\[x\cdot(y\cdot z)=(x\cdot y)\cdot z.\]
			\item Per a tot \(x,y,z\in R\) tenim
			\[x\cdot(y+z)=x\cdot y+x\cdot z\quad\text{i}\quad(x+y)\cdot z=x\cdot z+y\cdot z.\]
		\end{enumerate}
		Aleshores direm que \(R\) és un anell amb la suma \(+\) i el producte \(\cdot\). També direm que \(R\) és un anell amb element neutre pel producte \(e\).
		
		Si per a tot \(x,y\in R\) tenim \(x\cdot y=y\cdot x\) direm que \(R\) és un anell commutatiu.
	\end{definition}
	\begin{proposition}
		\label{prop:unicitat neutre del producte anell}
		Sigui \(R\) un anell amb la suma \(+\) i el producte \(\cdot\) i element neutre pel producte \(e\). Aleshores l'element neutre pel producte de \(R\) és únic.
		\begin{proof}
			Suposem que existeix un altre \(e'\neq e\) tal que \(x\cdot e'=e'\cdot x=x\) per a tot \(x\in R\). Aleshores tindríem
			\[e\cdot e'=e'\cdot e=e\]
			a la vegada que
			\[e\cdot e'=e'\cdot e=e'\]
			i per tant ha de ser \(e=e'\), que contradiu la hipòtesi que existeix un altre element neutre pel producte a \(R\), i en conseqüència aquest és únic.
		\end{proof}
	\end{proposition}
	\begin{definition}[Elements neutres d'un anell]
		\labelname{l'element neutre d'un anell per la suma}
		\label{def:l'element neutre d'un anell per la suma}
		\labelname{l'element neutre d'un anell pel producte}
		\label{def:l'element neutre d'un anell pel producte}
		Sigui \(R\) un anell amb la suma \(+\) i el producte \(\cdot\). Aleshores direm que \(0_{R}\) és l'element neutre de \(R\) per la suma i \(1_{R}\) és l'element neutre de \(R\) pel producte i els denotarem per \(0_{R}\) i \(1_{R}\), respectivament.
		
		Aquesta definició té sentit per la proposició \myref{prop:unicitat neutre del grup} i la proposició \myref{prop:unicitat neutre del producte anell}.
	\end{definition}
	\begin{notation}
		Donat un anell \(R\) un anell amb la suma \(+\) i el producte \(\cdot\), aprofitant que el producte \(\cdot\) és associatiu escriurem
		\[(x_{1}\cdot x_{2})\cdot x_{3}=x_{1}\cdot x_{2}\cdot x_{3}.\]
		
		També, si el context ho permet (quan treballem amb un únic anell \(R\)), denotarem \(1_{R}=1\) i \(0_{R}=0\).
	\end{notation}
	%EXPLICAR notació aditiva pel -a inversa de a per la suma.
	\begin{proposition}
		\label{prop:propietats bàsiques anells}
		Sigui \(R\) un anell amb la suma \(+\) i el producte \(\cdot\). Aleshores per a tot \(a,b,c\in R\) tenim
		\begin{enumerate}
			\item\label{enum:prop:propietats bàsiques anells 1} \(0\cdot a=a\cdot0=0\).
			\item\label{enum:prop:propietats bàsiques anells 2} \((-1)\cdot a=a\cdot(-1)=-a\).
			\item\label{enum:prop:propietats bàsiques anells 3} \((-a)\cdot(-b)=a\cdot b\).
			\item\label{enum:prop:propietats bàsiques anells 4} \((-a)\cdot b=a\cdot(-b)=-(a\cdot b)\).
		\end{enumerate}
		\begin{proof}
			Comprovem el punt \eqref{enum:prop:propietats bàsiques anells 1}. Només veurem que \(0\cdot a=0\) ja que l'altre demostració és anàloga. Com que \(0=0+0\) per la definició de \myref{def:l'element neutre d'un anell per la suma}, per la definició d'\myref{def:anell} tenim que
			\begin{align*}
			0\cdot a&=(0+0)\cdot a\\
			&=0\cdot a+0\cdot a
			\end{align*}
			i per tant, per la definició de \myref{def:grup}
			\[0\cdot a-(0\cdot a)=0\cdot a+0\cdot a-(0\cdot a)\]
			d'on veiem
			\[0\cdot a=0,\]
			com volíem veure.
			
			Veiem ara el punt \eqref{enum:prop:propietats bàsiques anells 2}. Per la definició d'\myref{def:anell} tenim
			\[1\cdot a+(-1)\cdot a=(1-1)\cdot a\quad\text{i}\quad(-1)\cdot a+1\cdot a=(-1+1)\cdot a\]
			i per tant, com que \(1-1=-1+1=0\) per la definició de \myref{def:l'element neutre d'un anell per la suma}, tenim que \(a+(-1)\cdot a=(-1)\cdot a+a=0\), però per la proposició \myref{prop:unicitat inversa en grups} tenim que \((-1)\cdot a=-a\). L'altre igualtat és anàloga.
			
			Continuem veient el punt \eqref{enum:prop:propietats bàsiques anells 3}. Pel punt \eqref{enum:prop:propietats bàsiques anells 2} tenim que
			\[(-a)\cdot(-b)=a\cdot (-1)\cdot(-1)\cdot b.\]
			Ara bé, pel punt \eqref{enum:prop:propietats bàsiques anells 2} de nou tenim que \((-1)\cdot(-1)=-(-1)\), i per la proposició \myref{prop:grups:l'invers de l'invers d'un element es l'element} tenim que \(-(-1)=1\) i per tant
			\[(-a)\cdot(-b)=a\cdot b.\]
			
			Per veure el punt \eqref{enum:prop:propietats bàsiques anells 4} només veurem que \((-a)\cdot b=-(a\cdot b)\) ja que l'altre demostració és anàloga. Pel punt \eqref{enum:prop:propietats bàsiques anells 2} tenim que
			\[-(a\cdot b)=(-1)\cdot(a\cdot b)\quad{\text{i}}\quad(-a)\cdot b=(-1)\cdot a\cdot b.\]
			Ara bé, per la definició d'\myref{def:anell}
			\[(-1)\cdot a\cdot b-(-1)\cdot(a\cdot b)=(-1-(-1))\cdot(a\cdot b),\]
			i per la proposició \myref{prop:grups:l'invers de l'invers d'un element es l'element} tenim que \(-(-1)=1\) i per tant, per la definició de \myref{def:grup} tenim \(-1+1=0\) i trobem
			\[(-1)\cdot a\cdot b-(-1)\cdot(a\cdot b)=0\cdot(a\cdot b)=0,\]
			i podem veure també que
			\[(-1)\cdot(a\cdot b)-(-1)\cdot a\cdot b=0\]
			de manera anàloga. Aleshores, per la proposició \myref{prop:unicitat inversa en grups} tenim que
			\[(-a)\cdot b=-(a\cdot b),\]
			com volíem veure.
		\end{proof}
	\end{proposition}
	\begin{proposition}
		\label{prop:unicitat invers en anells}
		Siguin \(R\) un anell amb la suma \(+\) i el producte \(\cdot\) i \(a\) un element de \(R\) tal que existeixi \(b\in R\) que satisfaci
		\[a\cdot b=b\cdot a=1.\]
		Aleshores \(b\) és únic.
		\begin{proof}
			Suposem que existeix un altre element \(b'\in R\) tal que
			\[a\cdot b'=b'\cdot a=1.\]
			Aleshores tenim
			\[a\cdot b=a\cdot b'\]
			i per tant
			\[b\cdot a\cdot b=b\cdot a\cdot b'\]
			i com que per hipòtesi \(b\cdot a=1\) per la definició de \myref{def:l'element neutre d'un anell pel producte} trobem
			\[b=b'.\qedhere\]
		\end{proof}
	\end{proposition}
	\begin{definition}[Element invertible]
		\labelname{element invertible}
		\label{def:element invertible pel producte d'un anell}
		Siguin \(R\) un anell amb la suma \(+\) i el producte \(\cdot\) i \(x\) un element de \(R\) tal que existeixi \(x'\in R\) tals que
		\[x\cdot x'=x'\cdot x=1.\] 
		Aleshores direm que \(x\) és invertible o que \(x\) és un element invertible de \(R\).
	\end{definition}
	\begin{definition}[L'invers d'un element]
		\labelname{l'invers d'un element invertible}
		\label{def:l'invers d'un element d'un anell}
		Siguin \(R\) un anell amb la suma \(+\) i el producte \(\cdot\) i \(x\) un element invertible de \(R\). Aleshores denotarem per \(x^{-1}\) l'element de \(R\) tal que
		\[x\cdot x^{-1}=x^{-1}\cdot x=1.\]
		Direm que \(x^{-1}\) és l'invers de \(x\).
		
		Aquesta definició té sentit per la proposició \myref{prop:unicitat invers en anells}.
	\end{definition}
	\begin{definition}[Subanell]
		\labelname{subanell}
		\label{def:subanell}
		Siguin \(R\) un anell amb la suma \(+\) i el producte \(\cdot\) i \(S\subseteq R\) un subconjunt amb \(1\in S\) tal que per a tot \(a,b\in S\) tenim \(a\cdot b,a+b\in S\) i \(S\) un anell amb la suma \(+\) i el producte \(\cdot\). Aleshores direm que \(S\) és un subanell de \(R\)
		
		Ho denotarem amb \(S\leq R\).
	\end{definition}
	\subsection{Ideals i ideals principals}%completar dividir lol
	\begin{definition}[Ideal]
		\labelname{ideal d'un anell}
		\label{def:ideal d'un anell}
		Siguin \(R\) un anell commutatiu amb la suma \(+\) i el producte \(\cdot\) amb \(1\neq0\) i \(I\) un subconjunt no buit de \(R\) tal que \(I\) sigui un subgrup del grup \(R\) amb la suma \(+\) i tal que per a tot \(x\in I\), \(r\in R\) tenim \(r\cdot x\in I\). Aleshores direm que \(I\) és un ideal de \(R\).
	\end{definition}
	\begin{observation}
		\label{obs:l'element neutre per la suma pertany a l'ideal}
		\(0\in I\).
	\end{observation}
	\begin{notation}
		Si \(I\) és un ideal d'un anell \(R\) denotarem \(I\triangleleft R\).
	\end{notation}
	\begin{proposition}
		\label{prop:condició equivalent a ideal d'un anell}
		Siguin \(R\) un anell commutatiu amb la suma \(+\) i el producte \(\cdot\) amb \(1\neq0\) i \(I\) un subconjunt no buit de \(R\). Aleshores tenim que \(I\) és un ideal de \(R\) si i només si
		\begin{enumerate}
			\item Per a tot \(x,y\in I\) tenim \(x-y\in I\).
			\item Per a tot \(r\in R\), \(x\in I\) tenim \(r\cdot x\in I\).
		\end{enumerate}
		\begin{proof}
			Veiem que la condició és suficient (\(\implica\)). Suposem que \(I\) és un ideal de \(R\). Hem de veure que per a tot \(x,y\in I\), \(r\in R\) tenim \(x-y\in I\) i \(r\cdot x\in I\). Per la definició de \myref{def:grup} tenim que \(x-y\in I\), ja que, per la definició d'\myref{def:ideal d'un anell} tenim que \(I\) és un grup amb la suma \(+\). També trobem \(r\cdot x\in I\) per la definició d'\myref{def:ideal d'un anell}.
			
			Veiem ara que la condició és necessària (\(\implicatper\)). Suposem que per a tot \(x,y\in I\), \(r\in R\) tenim \(x-y\in I\) i \(r\cdot x\in I\). Per la proposició \myref{prop:condició equivalent a subgrup} tenim que \(I\) és un subgrup del grup \(R\) amb la suma \(+\), i per la definició d'\myref{def:ideal d'un anell} tenim que \(I\) és un ideal de \(R\).
		\end{proof}
	\end{proposition}
	\begin{notation}
		Si \((a)\) és un ideal principal d'un anell \(R\) denotarem \((a)\trianglelefteq R\).
	\end{notation}
	\begin{proposition}
		\label{prop:combinació d'ideals per obtenir ideals}
		Siguin \(I,J\) dos ideals d'un anell \(R\) amb la suma \(+\) i el producte \(\cdot\). Aleshores els conjunts
		\begin{enumerate}
			\item\label{enum:prop:combinació d'ideals per obtenir ideals 1} \(I+J=\{x+y\mid x\in I, y\in J\}\).
			\item\label{enum:prop:combinació d'ideals per obtenir ideals 2} \(I\cap J\).
			\item\label{enum:prop:combinació d'ideals per obtenir ideals 3} \(IJ=\{x_{1}\cdot y_{1}+\dots x_{n}\cdot y_{n}\mid x_{1},\dots,x_{n}\in I, y_{1},\dots,y_{n}\in J, n\in\mathbb{N}\}\).
		\end{enumerate}
		són ideals de \(R\).
		\begin{proof}
			Comencem demostrant el punt \eqref{enum:prop:combinació d'ideals per obtenir ideals 1}. Per la proposició \myref{prop:condició equivalent a ideal d'un anell} només ens cal veure que per a tot \(a,b\in I+J\), \(r\in R\) tenim \(a-b\in I+J\) i \(r\cdot a\in I+J\). Prenem doncs \(a,b\in I+J\). Aleshores tenim \(a=a_{1}+a_{2}\) i \(b=b_{1}+b_{2}\) per a certs \(a_{1},b_{1}\in I\) i \(a_{2}, b_{2}\in J\). Volem veure que \(a-b\in I+J\). Això és
			\begin{align*}
			a_{1}+a_{2}-(b_{1}+b_{2})&=a_{1}+a_{2}-b_{1}-b_{2}\tag{\myref{prop:propietats bàsiques anells}}\\
			&=(a_{1}-b_{1})+(a_{2}-b_{2})\tag{\myref{def:grup abelià}}
			\end{align*}
			i com que, per hipòtesi, \(I,J\) són ideals de \(R\) per la proposició \myref{prop:condició equivalent a ideal d'un anell} tenim que \(a_{1}-b_{1}\in I\) i \(a_{2}-b_{2}\in J\), i per tant \(a-b=(a_{1}-b_{1})+(a_{2}-b_{2})\in I+J\).
			
			Veiem ara que per a tot \(r\in R\) es satisfà \(r\cdot a\in I+J\). Tenim
			\begin{align*}
			r\cdot a&=r\cdot(a_{1}+a_{2})\tag{\myref{def:anell}}\\
			&=r\cdot a_{1}+r\cdot a_{2}\tag{\myref{def:anell}}
			\end{align*}
			i per la proposició \myref{prop:condició equivalent a ideal d'un anell} tenim que \(r\cdot a_{1}\in I\) i \(r\cdot a_{2}\in J\), i per tant \(r\cdot a=r\cdot a_{1}+r\cdot a_{2}\in I+J\). Aleshores per la proposició \myref{prop:condició equivalent a ideal d'un anell} tenim que \(I+J\) és un ideal de \((R,+,\cdot)\).
			
			Veiem ara el punt \eqref{enum:prop:combinació d'ideals per obtenir ideals 2}. Per la proposició \myref{prop:condició equivalent a ideal d'un anell} només ens cal veure que per a tot \(a,b\in I\cap J\), \(r\in R\)  tenim \(a-b\in I\cap J\) i \(r\cdot a\in I\cap J\). Prenem doncs \(a,b\in I\cap J\), i per tant \(a,b\in I\) i \(a,b\in J\), i per la proposició \myref{prop:condició equivalent a ideal d'un anell} tenim que \(a-b\in I\) i \(a-b\in J\), i tenim que \(a-b\in I\cap J\).
			
			Per veure que \(r\cdot a\in I\cap J\) per a tot \(r\in R\) tenim per la definició d'\myref{def:ideal d'un anell} que \(r\cdot a\in I\) i \(r\cdot a\in J\), i per tant \(r\cdot a\in I\cap J\) i per la proposició \myref{prop:condició equivalent a ideal d'un anell} \(I\cap J\) és un ideal de \(R\).
			
			Acabem veient el punt \eqref{enum:prop:combinació d'ideals per obtenir ideals 3}. Per la proposició \myref{prop:condició equivalent a ideal d'un anell} només ens cal veure que per a tot \(a,b\in IJ\), \(r\in R\)  tenim \(a-b\in IJ\) i \(r\cdot a\in IJ\). Prenem doncs \(a,b\in IJ\), i tenim que \(a=x_{1}\cdot y_{1}+\dots+x_{n}\cdot y_{n}\), \(b=x'_{1}\cdot y'_{1}+\dots+x'_{m}\cdot y'_{m}\) per a certs \(x_{1},\dots,x_{n},x'_{1},\dots,x'_{m}\in I\), \(y_{1},\dots,y_{n},y'_{1},\dots,y'_{m}\in J\). Per veure que \(a-b\in IJ\) fem, per la proposició \myref{prop:propietats bàsiques anells},
			\begin{multline*}
			a-b=x_{1}\cdot y_{1}+\dots+x_{n}\cdot y_{n}-(x'_{1}\cdot y'_{1}+\dots+x'_{m}\cdot y'_{m})=\\
			=x_{1}\cdot y_{1}+\dots+x_{n}\cdot y_{n}-x'_{1}\cdot y'_{1}-\dots-x'_{m}\cdot y'_{m}
			\end{multline*}
			i com que \(I\) és, per hipòtesi, un anell, per les proposicions \myref{prop:propietats bàsiques anells} i \myref{prop:condició equivalent a ideal d'un anell} tenim que \(-x'_{1},\dots,-x'_{m}\in I\) i \(a-b=x_{1}\cdot y_{1}+\dots+x_{n}\cdot y_{n}-x'_{1}\cdot y'_{1}-\dots-x'_{m}\cdot y'_{m}\in IJ\).
			
			Veiem ara que per a tot \(r\in R\) es satisfà \(r\cdot a\in I\). Això és
			\begin{align*}
			r\cdot a&=r\cdot(x_{1}\cdot y_{1}+\dots+x_{n}\cdot y_{n})\\
			&=r\cdot x_{1}\cdot y_{1}+\dots+r\cdot x_{n}\cdot y_{n}\tag{\myref{prop:propietats bàsiques anells}}
			\end{align*}
			i com que \(I\) és, per hipòtesi, un anell, per les proposicions \myref{prop:propietats bàsiques anells} i \myref{prop:condició equivalent a ideal d'un anell} tenim que \(r\cdot x_{1},\dots,r\cdot x_{n}\in I\), i per tant \(r\cdot a=r\cdot x_{1}\cdot y_{1}+\dots+r\cdot x_{n}\cdot y_{n}\in IJ\) i per la proposició \myref{prop:condició equivalent a ideal d'un anell} tenim que \(IJ\) és un ideal de \(R\).
		\end{proof}
	\end{proposition}
	\begin{definition}[Ideal principal]
		\labelname{ideal principal}
		\label{def:ideal principal}
		Sigui \(I\) un ideal d'un anell \(R\) amb la suma \(+\) i el producte \(\cdot\) tal que \(I=\{a\}R=R\{a\}=\{r\cdot a\mid r\in R\}\) per a cert \(a\in I\). Aleshores direm que \(I\) és un anell principal de \(R\). Ho denotarem amb \(I=(a)\).
	\end{definition}
	\subsection{Cossos i l'anell quocient}
	\begin{definition}[Cos]
		\labelname{cos}
		\label{def:cos per anells}
		Sigui \(\mathbb{K}\) un anell commutatiu amb la suma \(+\) i el producte \(\cdot\) amb \(1\neq0\) tal que \(\mathbb{K}\setminus\{0\}\) sigui un grup abelià amb el producte \(\cdot\). Aleshores direm que \(\mathbb{K}\) és un cos amb la suma \(+\) i el producte \(\cdot\).
	\end{definition}
	\begin{proposition}
		\label{prop:condició equivalent a cos per anells}
		Sigui \(\mathbb{K}\) un anell commutatiu amb la suma \(+\) i el producte \(\cdot\) amb \(1\neq0\). Aleshores tenim que \(\mathbb{K}\) és un cos si i només si els únics ideals de \(\mathbb{K}\) són \((0)\) i \(\mathbb{K}\).
		\begin{proof}
			Comencem comprovant que la condició és suficient (\(\implica\)). Suposem doncs que \(\mathbb{K}\) és un cos amb la suma \(+\) i el producte \(\cdot\) i que \(I\) és un ideal de \(\mathbb{K}\) amb \(I\neq(0)\), i per tant existeix \(a\in\mathbb{K}\), \(a\neq0\) tal que \(a\in I\). Com que, per hipòtesi, \((\mathbb{K}\setminus\{0\},\cdot)\) és un grup i \(a\ne0\) per la definició de \myref{def:l'invers d'un element d'un grup} existeix \(a^{-1}\in\mathbb{K}\) tal que \(a\cdot a^{-1}=1\), i per la definició d'\myref{def:ideal d'un anell} trobem que \(1\in I\), i per tant per la proposició \myref{prop:condició equivalent a ideal d'un anell} tenim que per a tot \(x\in\mathbb{K}\) tenim \(x\cdot1=x\in I\), i per tant \(I=\mathbb{K}\).
			
			Veiem ara que la condició és necessària (\(\implicatper\)). Suposem que els únics ideals de \(\mathbb{K}\) són \((0)\) i \(\mathbb{K}\). Prenem un element \(a\in\mathbb{K}\), \(a\neq0\) i considerem l'ideal principal \((a)\), que per hipòtesi ha de ser \((a)=(1)=\mathbb{K}\), i per la definició d'\myref{def:ideal principal} tenim que existeix \(a'\in(1)\) tal que \(a\cdot a'=1\), i per tant, per la definició de \myref{def:grup} tenim que \(\mathbb{K}\setminus\{0\}\) és un grup amb el producte \(\cdot\) i per la definició de \myref{def:cos per anells} tenim que \(\mathbb{K}\) és un cos amb la suma \(+\) i el producte \(\cdot\).
		\end{proof}
	\end{proposition}
	\begin{proposition}
		\label{prop:relació d'equivalència en anells per ideals}
		Sigui \(I\) un ideal d'un anell \(R\) amb la suma \(+\) i el producte \(\cdot\). Aleshores la relació
		\[x\sim y\sii x-y\in I\quad\text{per a tot }x,y\in R\]
		és una relació d'equivalència.
		\begin{proof}
			Comprovem que \(\sim\) satisfà la definició de \myref{def:relació d'equivalència}:
			\begin{enumerate}
				\item Reflexiva: Prenem \(x\in R\). Per l'observació \myref{obs:l'element neutre per la suma pertany a l'ideal} tenim que \(0\in I\), i per tant \(x-x=0\in I\) i veiem que \(x\sim x\).
				\item Simètrica: Siguin \(x_{1},x_{2}\in I\) tals que \(x_{1}\sim x_{2}\). Això és que \(x_{1}-x_{2}\in I\). Per la definició d'\myref{def:ideal d'un anell} tenim que \((-1)\cdot(x_{1}-x_{2})\in I\), i per la proposició \myref{prop:propietats bàsiques anells} tenim que \(x_{2}-x_{1}\in I\), és a dir, \(x_{2}\sim x_{1}\).
				\item Transitiva: Siguin \(x_{1},x_{2},x_{3}\in R\) tals que \(x_{1}\sim x_{2}\) i \(x_{2}\sim x_{3}\). Per la definició d'\myref{def:ideal d'un anell} tenim que \(x_{3}-x_{2}\in I\), i per la proposició \ref{prop:condició equivalent a ideal d'un anell} tenim que \(x_{1}-x_{2}-(x_{3}-x_{2})\in I\). Ara bé, per la proposició \myref{prop:propietats bàsiques anells} tenim que això és \(x_{1}-x_{3}\in I\), i per tant \(x_{1}\sim x_{3}\).
			\end{enumerate}
			Per tant \(\sim\) és una relació d'equivalència.
		\end{proof}
	\end{proposition}
	\begin{proposition}
		\label{prop:anell quocient}
		Sigui \(I\) un ideal d'un anell \(R\) amb la suma \(+\) i el producte \(\cdot\). Aleshores \(R/I\) amb la suma \([x]+[y]=[x+y]\) i el producte \([x]\cdot[y]=[x\cdot y]\) és un anell.
		\begin{proof}
			Aquest enunciat té sentit per la proposició \myref{prop:relació d'equivalència en anells per ideals}.
			
			Per la proposició \myref{TODO:grup quocient} tenim que \((R/I,+)\) és un grup, i com que
			\begin{align*}
			[x]+[y]&=[x+y]\\
			&=[y+x]\tag{\myref{def:grup abelià}}\\
			&=[y]+[x]
			\end{align*}
			tenim que \(R/I\) és un grup abelià amb la suma \(+\). Veiem ara que per a tot \(x,y,z\in R/I\) tenim \([x]\cdot([y]\cdot[z])=([x]\cdot[y])\cdot[z]\) i \([x]\cdot([y]+[z])=[x]\cdot[y]+[x]\cdot[z]\). Tenim
			\begin{align*}
			[x]\cdot([y]\cdot[z])&=[x]\cdot[y\cdot z]\\
			&=[x\cdot(y\cdot z)]\\
			&=[(x\cdot y)\cdot z]\\
			&=[x\cdot y]\cdot[z]=([x]\cdot[y])\cdot[z]
			\end{align*}
			i
			\begin{align*}
			[x]\cdot([y]+[z])&=[x]\cdot[y+z]\\
			&=[x\cdot(y+z)]\\
			&=[x\cdot y+x\cdot z]=[x]\cdot[y]+[x]\cdot[z]
			\end{align*}
			i per la definició d'\myref{def:anell} tenim que \(R/I\) és un anell amb la suma \(+\) i el producte \(\cdot\).
		\end{proof}
	\end{proposition}
	\begin{definition}[Anell quocient]
		\labelname{anell quocient}
		\label{def:anell quocient}
		Siguin \(R\) un anell commutatiu amb la suma \(+\) i el producte \(\cdot\) amb \(1\neq0\) i \(I\) un ideal de \(R\). Aleshores direm que \(R/I\) és un anell quocient.
		
		Aquesta definició té sentit per la proposició \myref{prop:anell quocient}.
	\end{definition}
	\section{Tres Teoremes d'isomorfisme entre anells}
	\subsection{Morfismes entre anells}
	\begin{definition}[Morfisme entre anells]
		\labelname{morfisme entre anells}
		\label{def:morfisme entre anells}
		\labelname{epimorfisme entre anells}
		\label{def:epimorfisme entre anells}
		\labelname{monomorfisme entre anells}
		\label{def:monomorfisme entre anells}
		\labelname{isomorfisme entre anells}
		\label{def:isomorfisme entre anells}
		\labelname{endomorfisme entre anells}
		\label{def:endomorfisme entre anells}
		Siguin \(R\) un anell commutatiu amb la suma \(+_{R}\) i el producte \(\ast_{R}\), \(S\) un anell commutatiu amb la suma \(+_{S}\) i el producte \(\ast_{S}\) amb \(1_{R}\neq0_{R}\) i \(1_{S}\neq0_{S}\) i \(f\colon R\longrightarrow S\) una aplicació tal que
		\begin{enumerate}
			\item \(f(x+_{R}y)=f(x)+_{S}f(y)\) per a tot \(x,y\in R\).
			\item \(f(x\ast_{R}y)=f(x)\ast_{S}f(y)\) per a tot \(x,y\in R\).
			\item \(f(1_{R})=1_{S}\).
		\end{enumerate}
		Aleshores diem que \(f\) és un morfisme entre anells. Definim també
		\begin{enumerate}
			\item Si \(f\) és injectiva direm que \(f\) és un monomorfisme entre anells.
			\item Si \(f\) és exhaustiva direm que \(f\) és un epimorfisme entre anells.
			\item Si \(f\) és bijectiva direm que \(f\) és un isomorfisme entre anells. També escriurem \(R\cong S\).
			\item Si \(R=S\) direm que \(f\) és un endomorfisme entre anells.
			\item Si \(R=S\) i \(f\) és bijectiva direm que \(f\) és un automorfisme entre anells.
		\end{enumerate}
	\end{definition}
	\begin{observation}
		\label{obs:morfisme entre anells és morfisme entre grups}
		Si \(f\) és un morfisme entre anells aleshores \(f\) és un morfisme entre grups.
	\end{observation}
	\begin{proposition}
		\label{prop:propietats morfismes entre anells}
			Siguin \(R\) un anell commutatiu amb la suma \(+_{R}\) i el producte \(\ast_{R}\), \(S\) un anell commutatiu amb la suma \(+_{S}\) i el producte \(\ast_{S}\) amb \(1_{R}\neq0_{R}\) i \(1_{S}\neq0_{S}\) i \(f\colon R\longrightarrow S\) un morfisme entre anells. Aleshores
		\begin{enumerate}
			\item \(f(0_{R})=0_{S}\).
			\item \(f(-x)=-f(x)\) per a tot \(x\in R\).
		\end{enumerate}
		\begin{proof}
			Per l'observació \myref{obs:morfisme entre anells és morfisme entre grups} tenim que \(f\) és un morfisme entre els grups \(R\) amb la suma \(+_{R}\) i \(S\) amb la suma \(+_{S}\), i per la proposició \myref{prop:morfismes conserven neutre i l'invers commuta amb el morfisme} tenim que \(f(0_{R})=0_{S}\) i \(f(-x)=-f(x)\) per a tot \(x\in R\).
		\end{proof}
	\end{proposition}
	\begin{definition}[Nucli i imatge]
		\labelname{nucli d'un morfisme entre anells}
		\label{def:nucli d'un morfisme entre anells}
		\labelname{imatge d'un morfisme entre anells}
		\label{def:imatge d'un morfisme entre anells}
			Siguin \(R\) un anell commutatiu amb la suma \(+_{R}\) i el producte \(\ast_{R}\), \(S\) un anell commutatiu amb la suma \(+_{S}\) i el producte \(\ast_{S}\) amb \(1_{R}\neq0_{R}\) i \(1_{S}\neq0_{S}\) i \(f\colon R\longrightarrow S\) un morfisme entre anells. Aleshores definim
		\[\ker(f)=\{x\in R\mid f(x)=0_{S}\}\]
		com el nucli de \(f\), i
		\[\Ima(f)=\{y\in S\mid f(x)=y\text{ per a cert }x\in R\}\]
		com la imatge de \(f\).
	\end{definition}
	\begin{observation}
		\label{obs:nucli d'un morfisme entre anells es subconjunt del grup d'entrada, imatge n'és del de sortida}
		\(\ker(f)\subseteq R\), \(\Ima(f)\subseteq S\).
	\end{observation}
	\begin{proposition}
		\label{prop:el nucli d'un morfisme entre anells és ideal, la imatge d'un morfisme entre anells és subanell}
			Siguin \(R\) un anell commutatiu amb la suma \(+_{R}\) i el producte \(\ast_{R}\), \(S\) un anell commutatiu amb la suma \(+_{S}\) i el producte \(\ast_{S}\) amb \(1_{R}\neq0_{R}\) i \(1_{S}\neq0_{S}\) i \(f\colon R\longrightarrow S\) un morfisme entre anells. Aleshores
		\begin{enumerate}
			\item\label{enum:prop:el nucli d'un morfisme entre anells és ideal, la imatge d'un morfisme entre anells és subanell 1} \(\ker(f)\triangleleft R\).
			\item\label{enum:prop:el nucli d'un morfisme entre anells és ideal, la imatge d'un morfisme entre anells és subanell 2} \(\Ima(f)\leq S\).
		\end{enumerate}
		\begin{proof}
			Aquest enunciat té sentit per l'observació \myref{prop:el nucli d'un morfisme entre anells és ideal, la imatge d'un morfisme entre anells és subanell}
			
			Comencem veient el punt \eqref{enum:prop:el nucli d'un morfisme entre anells és ideal, la imatge d'un morfisme entre anells és subanell 1}. Com que, per la proposició \myref{prop:propietats morfismes entre anells}, tenim que \(f(0_{R})=0_{S}\) veiem, per la definició de \myref{def:nucli d'un morfisme entre anells}, que \(\ker(f)\neq\emptyset\). Prenem doncs \(a\in\ker(f)\). Observem que, per la definició de \myref{def:morfisme entre anells}, tenim que \(f(r\ast_{R}a)=f(r)\ast_{S}f(a)\) per a tot \(r\in R\), i per tant, com que per la definició de \myref{def:nucli d'un morfisme entre anells} es compleix \(f(a)=0_{S}\) tenim que \[f(r\ast_{R}a)=f(r)\ast_{S}f(a)=f(r)\ast_{S}0_{S}=0_{S}\]
			i per tant \(r\ast_{R}a\in\ker(f)\) per a tot \(r\in R\), \(a\in\ker(f)\). Ara bé per la definició de \myref{def:ideal d'un anell} tenim que \(\ker(f)\) és un ideal de \(R\).
			
			Veiem ara el punt \eqref{enum:prop:el nucli d'un morfisme entre anells és ideal, la imatge d'un morfisme entre anells és subanell 2}. Veiem que per a tot \(x,y\in\Ima(f)\) tenim \(x\ast_{S}y\in\Ima(f)\). Per la definició d'\myref{def:imatge d'un morfisme entre anells} tenim que existeixen \(a,b\in R\) tals que \(f(a)=x\) i \(f(b)=y\). Ara bé, per la definició d'\myref{def:anell} tenim que \(a\ast_{R}b=c\in R\), i per tant per la definició d'\myref{def:imatge d'un morfisme entre anells} tenim que \(f(c)=x\ast_{S}y\in\Ima(f)\). Veiem ara que \(\Ima(f)\) és un anell amb la suma \(+_{S}\) i el producte \(\ast_{S}\). Com que, per l'observació \myref{obs:morfisme entre anells és morfisme entre grups} tenim que \(f\) és un morfisme entre grups per la proposició \myref{prop:la imatge d'un morfisme és un subgrup del grup d'arribada} tenim que \(\Ima(f)\) és un subgrup del grup \(S\) amb la suma \(+_{S}\); i per la definició d'\myref{def:anell} tenim que per a tot \(x,y,z\in R\) es satisfà
			\[x\ast_{R}(y\ast_{R}z)=(x\ast_{R}y)\ast_{R}z\quad\text{i}\quad x\ast_{R}(y+_{R}z)=x\ast_{R}y+_{R}x\ast_{R}z,\]
			i per la definició de \myref{def:subanell} tenim que \(\Ima(f)\) és un subanell de \(S\), com volíem veure.
		\end{proof}
	\end{proposition}
%	\begin{proposition}
%		\label{prop:ideals per morfismes d'anells}
%			Siguin \(R\) un anell commutatiu amb la suma \(+_{R}\) i el producte \(\ast_{R}\), \(S\) un anell commutatiu amb la suma \(+_{S}\) i el producte \(\ast_{S}\) satisfent \(1_{R}\neq0_{R}\) i \(1_{S}\neq0_{S}\), \(I\) un ideal de \((R,+_{R},\ast_{R})\), \(J\) un ideal de \((S,+_{S},\ast_{S})\) i \(f\colon R\longrightarrow S\) un morfisme entre anells. Aleshores
%		\begin{enumerate}
%			\item\label{eq:prop:ideals per morfismes d'anells 1} \(\{x\in R\mid f(x)\in J\}\) és un ideal de \((R,+_{R},\ast_{R})\).
%			\item\label{eq:prop:ideals per morfismes d'anells 2} Si \(f\) és exhaustiva, \(\{f(x)\in S\mid x\in J\}\) és un ideal de \((S,+_{S},\ast_{S})\).
%		\end{enumerate}
%		\begin{proof}
%			%TODO
%		\end{proof}
%	\end{proposition}
%%	\begin{proposition}
%			Siguin \(R\) un anell commutatiu amb la suma \(+_{R}\) i el producte \(\ast_{R}\), \(S\) un anell commutatiu amb la suma \(+_{S}\) i el producte \(\ast_{S}\) amb \(1_{R}\neq0_{R}\) i \(1_{S}\neq0_{S}\), \((R',+_{R},\ast_{R})\) un subanell de \((R,+_{R},\ast_{R})\), \((S',+_{S},\ast_{S})\) un subanell de \((S,+_{S},\ast_{S})\) i \(f\colon R\longrightarrow S\) un morfisme entre anells. Aleshores
%		\begin{enumerate}
%			\item \((\{f(x)\in S\mid x\in R\},+_{S},\ast_{S})\) és un subanell de \((S,+_{S},\ast_{S})\).
%			\item \((\{x\in R\mid f(x)\in S\},+_{R},\ast_{R})\) és un subanell de \((R,+_{R},\ast_{R})\).
%		\end{enumerate}
%		\begin{proof}
%			%TODO
%		\end{proof}
%	\end{proposition}
	\begin{proposition}
		\label{prop:operació morfismes entre anells és morfisme entre anells}
		Siguin \(R\) un anell commutatiu amb la suma \(+_{R}\) i el producte \(\ast_{R}\), \(S\) un anell commutatiu amb la suma \(+_{S}\) i el producte \(\ast_{S}\), \(D\) un anell commutatiu amb la suma \(+_{D}\) i el producte \(\ast_{D}\) tres anells commutatius amb \(1_{R}\neq0_{R}\), \(1_{S}\neq0_{S}\) i \(1_{D}\neq0_{D}\) i \(f\colon R\longrightarrow S\), \(g\colon S\longrightarrow D\) dos morfismes entre anells.
		Aleshores \(g\circ f\colon R\longrightarrow D\) és un morfisme entre anells.
		\begin{proof}
			Per la definició de \myref{def:morfisme entre anells} trobem que
			\begin{align*}
			g(f(x+_{R}y))&=g(f(x)+_{S}g(y))\\
			&=g(f(x))+_{D}g(f(x)),
			\end{align*}
			i
			\begin{align*}
			g(f(x\ast_{R}y))&=g(f(x)\ast_{S}g(y))\\
			&=g(f(x))\ast_{D}g(f(x)).
			\end{align*}
			També tenim
			\[g(f(1_{R}))=g(1_{S})=1_{D}\]
			i per la definició de \myref{def:morfisme entre anells} tenim que \(g\circ f\) és un morfisme entre anells.
		\end{proof}
	\end{proposition}
	\begin{corollary}
		\label{corollary:conjugació isomorfismes enre anells és isomorfisme entre anells}
		Si \(f,g\) són isomorfismes aleshores \(g\circ f\) és isomorfisme.
	\end{corollary}
	\begin{lemma}
		\label{lema:ker(f) és ideal de l'anell d'entrada}
		\label{lema:Ima(f) és subanell de l'anell de sortida}
			Siguin \(R\) un anell commutatiu amb la suma \(+_{R}\) i el producte \(\ast_{R}\), \(S\) un anell commutatiu amb la suma \(+_{S}\) i el producte \(\ast_{S}\) amb \(1_{R}\neq0_{R}\) i \(1_{S}\neq0_{S}\) i \(f\colon R\longrightarrow S\) un morfisme entre anells. Aleshores
		\[\ker(f)\text{ és un ideal de }R\quad\text{i}\quad\Ima(f)\text{ és un subanell de }S.\]
		\begin{proof}
			Comencem veient que \(\ker(f)\) és un ideal de \(R\). Per l'observació \myref{obs:morfisme entre anells és morfisme entre grups} tenim que \(\ker(f)\) és un morfisme entre grups, i per la proposició \ref{prop:el nucli d'un morfisme és un subgrup normal del grup de sortda} tenim que \(\ker(f)\) és un subgrup del grup \(R\) amb la suma \(+_{R}\).
			
			Prenem \(x\in\ker(f)\) i \(r\in R\). Volem veure que \(r\ast_{R}r\in\ker(f)\). Tenim
			\begin{align*}
				f(r\ast_{R}x)&=f(r)\ast_{S}f(x)\tag{\myref{def:morfisme entre anells}}\\
				&=f(r)\ast_{S}0_{S}\tag{\myref{def:nucli d'un morfisme entre anells}}\\
				&=0_{S}\tag{\myref{def:l'element neutre d'un anell pel producte}}
			\end{align*}
			i per la definició de \myref{def:nucli d'un morfisme entre anells} tenim que \(r\ast_{R}x\in\ker(f)\), i per la proposició \myref{prop:condició equivalent a ideal d'un anell} tenim que \(\ker(f)\) és un ideal de \(R\).
			
			Veiem ara que \(\Ima(f)\) és un subanell de \(S\). Per l'observació \myref{obs:morfisme entre anells és morfisme entre grups} tenim que \(\Ima(f)\) és un morfisme entre grups, i per la proposició \ref{prop:la imatge d'un morfisme és un subgrup del grup d'arribada} tenim que \(\Ima(f)\) és un subgrup del grup \(S\) amb la suma \(+_{S}\).
			
			Prenem \(x,y\in\Ima(f)\). Volem veure que \(x\ast_{S}y\in\Ima(f)\). Per la definició d'\myref{def:imatge d'un morfisme entre anells} tenim que existeixen \(x',y'\in R\) tals que \(f(x')=x\) i \(f(y')=y\). Ara bé, per la definició d'\myref{def:anell} tenim que \(x'\ast_{R}y'\in R\), i per tant
			\begin{align*}
			f(x'\ast_{R}y')&+f(x')\ast_{S}f(y')\tag{\myref{def:morfisme entre anells}}\\
			&=x\ast_{S}y
			\end{align*}
			i per la definició de \myref{def:imatge d'un morfisme entre grups} trobem que \(x\ast_{S}y\in\Ima(f)\).
			
			També tenim, per la definició de \myref{def:morfisme entre anells}, que \(1_{S}\in\Ima(f)\), ja que \(f(1_{R})=1_{S}\), i per tant, per la definició de \myref{def:subanell}, tenim que \(\Ima(f)\) és un subanell de \(S\).
		\end{proof}
	\end{lemma}
	\subsection{Teoremes d'isomorfisme entre anells} %TODO
	\begin{theorem}[Primer Teorema de l'isomorfisme]
		\labelname{Primer Teorema de l'isomorfisme entre anells}\label{thm:Primer Teorema de l'isomorfisme entre anells}
		Siguin \(R\) un anell amb la suma \(+_{R}\) i el producte \(\ast_{R}\), \(S\) un anell amb la suma \(+_{S}\) i el producte \(\ast_{S}\) i \(\varphi\colon R\longrightarrow S\) un morfisme entre anells. Aleshores
		\[R/\ker(\varphi)\cong\Ima(\varphi).\]
		\begin{proof}
			Aquest enunciat té sentit per la proposició \myref{prop:el nucli d'un morfisme entre anells és ideal, la imatge d'un morfisme entre anells és subanell}. %TODO
		\end{proof}
	\end{theorem}
	\begin{theorem}[Segon Teorema de l'isomorfisme]
		\labelname{Segon Teorema de l'isomorfisme entre anells}\label{thm:Segon Teorema de l'isomorfisme entre anells}
		Siguin \(R\) un anell commutatiu amb la suma \(+\) i el producte \(\cdot\) amb \(1\neq0\) i \(I,J\) dos ideals de \(R\). Aleshores
		\[(I+J)/I\cong J/(I\cap J).\]
		\begin{proof}
			%TODO 
		\end{proof}
	\end{theorem}
	\begin{lemma}
		\label{lema:Tercer Teorema de l'isomorfisme entre anells}
		Siguin \(R\) un anell commutatiu amb la suma \(+\) i el producte \(\cdot\) amb \(1\neq0\) i \(I\), \(J\) dos ideals de \(R\) tals que \(I\subseteq J\). Aleshores \(J/I\) és un ideal de \(R/I\). %Definr \(I\subseteq J\), (projecció)
		\begin{proof}
			%TODO % J/I hauria de ser \bar{J}, \bar{J}^{I} o algo aixi.
		\end{proof}
	\end{lemma}
	\begin{theorem}[Tercer Teorema de l'isomorfisme]
		\labelname{Tercer Teorema de l'isomorfisme entre anells}\label{thm:Tercer Teorema de l'isomorfisme entre anells}
		Siguin \(R\) un anell commutatiu amb la suma \(+\) i el producte \(\cdot\) amb \(1\neq0\) i \(I\), \(J\) dos ideals de \(R\) tals que \(I\subseteq J\). Aleshores
		\[(R/I)/(J/I)\cong R/J.\]
		\begin{proof}
			Aquest enunciat té sentit pel lema \myref{lema:Tercer Teorema de l'isomorfisme entre anells}. %TODO
		\end{proof}
	\end{theorem}
	\subsection{Característica d'un anell}
	\begin{definition}[Característica]
		\labelname{característica d'un anell}
		\label{def:característica d'un anell}
		Sigui \(R\) un anell amb la suma \(+\) i el producte \(\cdot\). Direm que \(R\) té característica \(n>0\) si \(n=\min_{n\in\mathbb{N}}\{n\cdot1=0\}\). Ho denotarem amb \(\ch(R)=n\). Si aquest \(n\) no existeix diem que \(R\) té característica \(0\) i \(\ch(R)=0\).
	\end{definition}
	\begin{proposition}
		\label{prop:morfisme entre anells per trobar característica}
		Siguin \(R\) un anell amb la suma \(+\) i el producte \(\cdot\) i
		\begin{align*}
		f\colon\mathbb{Z}&\longrightarrow R\\
		n&\longmapsto n\cdot1=1+\overset{n)}{\cdots}+1
		\end{align*}
		una aplicació. Aleshores \(f\) és un morfisme entre anells i \(\ker(f)=(\ch(R))\).
		\begin{proof}
			Aquest enunciat té sentit per la proposició \myref{prop:el nucli d'un morfisme entre anells és ideal, la imatge d'un morfisme entre anells és subanell}. %Té sentit ja que \(\mathbb{N}\) és un anell. %AFEGIR
			
			Comencem veient que \(f\) és un morfisme entre anells. Veiem que \(f\) és un morfisme entre els grup. Tenim que per a tot \(n,m\in\mathbb{Z}\)
			\begin{align*}
			f(n+m)&=1+\overset{n+m)}{\cdots}+1\\
			&=(1+\overset{n)}{\cdots}+1)+(1+\overset{m)}{\cdots}+1)\\
			&=f(n)+f(m)
			\end{align*}
			i per la definició de \myref{def:morfisme entre grups} tenim que \(f\) és un morfisme entre grups. Veiem ara que \(f(1)=1\). Tenim que \(f(1)=1\cdot1=1\) i per tant per la definició de \myref{def:morfisme entre anells} tenim que \(f\) és un morfisme entre anells.
			
			Veiem ara que \(\ker(f)=(\ch(n))\). Per la definició de \myref{def:nucli d'un morfisme entre anells} tenim que \(\ker(f)=\{x\in\mathbb{N}\mid f(n)=0\}\).	Per tant
			\[\ker(f)=\{n\in\mathbb{N}\mid n\cdot1=0\}\]
			i per la definició de \myref{def:característica d'un anell} tenim que \(n=\ch(R)\), i per tant \(\ker(f)=\{k\cdot n\mid k\in\mathbb{N}\}\). Ara bé, per la definició de \myref{def:ideal d'un anell} tenim que \(\ker(f)=(n)\), com volíem veure.
		\end{proof}
	\end{proposition}
	\begin{corollary}
		\label{corollary:subcos isomorf respecte caracteristica}
		Si \(\ch(R)=0\) aleshores existeix un subanell \(S\) de \(R\) tal que \(S\cong\mathbb{Z}\).
		
		Si \(\ch(R)=n\) aleshores existeix un subanell \(S\) de \(R\) tal que \(S\cong\mathbb{Z}/(n)\).
	\end{corollary}
	%\begin{proposition}\(\ch(K)=0\) ó \(\ch(K)=p\) primer.\end{proposition}
	%	\begin{corollary}
	%\(x\in R\) invertible, \(S\) subanell de \(R\) amb \(x\in S\) no implica \(x^{-1}\in S\).
	%		Sigui \(\mathbb{K}\) un cos amb la suma \(+\) i el producte \(\cdot\) i \(p\) un primer. Aleshores
	%		\begin{enumerate}
	%			\item Si \(\ch(\mathbb{K})=0\), el cos \((\mathbb{K},+,\cdot)\) conté un subcòs isomorf a \(\mathbb{Q}\).
	%			\item Si \(\ch(\mathbb{K})=p\), el cos \((\mathbb{K},+,\cdot)\) conté un subcòs isomorf a \(\mathbb{Z}/(p)\).
	%		\end{enumerate}
	%		\begin{proof}
	%			Comencem veient el punt 
	%		\end{proof}
	%	\end{corollary}	
	\section{Dominis}
	\subsection{Dominis d'integritat, ideals primers i maximals}
	\begin{definition}[Divisor de 0]
		\labelname{divisor de 0 en un anell}
		\label{def:divisor de 0 en un anell}
		Siguin \(R\) un anell commutatiu amb la suma \(+\) i el producte \(\cdot\) amb \(1\neq0\) i \(a,b\neq0\) dos elements de \(R\) tals que \(a\cdot b=0\). Aleshores diem que \(a\) és un divisor de 0 en \(R\).
	\end{definition}
	\begin{definition}[Dominis d'integritat]
		\labelname{domini d'integritat}
		\label{def:domini d'integritat}
		\label{def:DI}
		Sigui \(D\) un anell commutatiu amb la suma \(+\) i el producte \(\cdot\) amb \(1\neq0\) tal que no existeix cap \(a\in D\) tal que \(a\) sigui un divisor de \(0\) en \(D\). Aleshores direm que \(D\) és un domini d'integritat.
	\end{definition}
	\begin{proposition}
		\label{prop:podem tatxar pels costats en DI}
		Siguin \(D\) un domini d'integritat amb la suma \(+\) i el producte \(\cdot\) i \(a\neq0\) un element de \(D\). Aleshores
		\[a\cdot x=a\cdot y\Longrightarrow x=y.\]
		\begin{proof}
			Tenim
			\[a\cdot x-a\cdot y=0\]
			i per la proposició \myref{prop:propietats bàsiques anells} tenim que
			\[(x-y)\cdot a=0.\]
			
			Ara bé, com que, per hipòtesi, \(D\) és un domini d'integritat i \(a\neq0\) tenim que ha de ser \(x-y=0\), i per tant trobem \(x=y\).
		\end{proof}
	\end{proposition}
	\begin{definition}[Ideal primer]
		\labelname{ideal primer}
		\label{def:ideal primer}
		Sigui \(I\) un ideal d'un anell \(R\) amb la suma \(+\) i el producte \(\cdot\) amb \(I\neq R\) tal que si \(a\cdot b\in I\) tenim \(a\in I\) o \(b\in I\). Aleshores direm que \(I\) és un ideal primer de \(R\).
	\end{definition}
	\begin{proposition}
		\label{prop:R/I domini d'integritat sii I ideal primer}
		Sigui \(I\) un ideal d'un anell \(R\) amb la suma \(+\) i el producte \(\cdot\) amb \(I\neq R\). Aleshores
		\[R/I\text{ és un domini d'integritat}\sii I\text{ és un ideal primer de }R.\]
		\begin{proof}
			Comencem demostrant que la condició és suficient (\(\implica\)). Suposem doncs que \(R/I\) és un domini d'integritat i prenem \([a],[b]\in R/I\) tals que \([a]\cdot[b]=[0]\). Aleshores per la definició de \myref{def:anell quocient} tenim que \(a\cdot b\in I\). Ara bé, com que per hipòtesi \(R/I\) és un domini d'integritat tenim que ha de ser \([a]=[0]\) ó \([b]=[0]\), i per tant trobem que ha de ser \(a\in I\) o \(b\in I\), i per la definició d'\myref{def:ideal primer} trobem que \(I\) és un ideal primer de \(R\).
			
			Veiem ara que la condició és necessària (\(\implicatper\)). Suposem doncs que \(I\) és un ideal primer de \(R\). Per la proposició \ref{prop:anell quocient} tenim que \(R/I\) és un anell commutatiu amb \(1\neq0\). Prenem doncs \(a\in R\), \(a\notin I\) i suposem que existeix \(b\in R\) tal que \([a]\cdot[b]=[0]\). Això és que \(a\cdot b\in I\), i com que per hipòtesi \(I\) és un ideal primer, per la definició d'\myref{def:ideal primer} trobem que ha de ser \(b\in I\), i per tant \([b]=[0]\) i per la definició de \myref{def:domini d'integritat} tenim que \(R/I\) és un domini d'integritat.
		\end{proof}
	\end{proposition}
	\begin{corollary}
		\label{corollary:domini d'integritat sii (0) ideal primer}
		\(R\) és un domini d'integritat si i només si \((0)\) és un ideal primer.
	\end{corollary}
	\begin{definition}[Ideal maximal]
		\labelname{ideal maximal}
		\label{def:ideal maximal}
		Sigui \(M\) un ideal d'un anell \(R\) amb la suma \(+\) i el producte \(\cdot\) amb \(M\neq R\) tal que  per a tot ideal \(I\) de \(R\) amb \(M\subseteq I\subseteq R\) ha de ser \(I=M\) o \(I=R\). Aleshores direm que \(M\) és un ideal maximal de \(R\).
	\end{definition}
	\begin{proposition}
		\label{prop:condició equivalent a ideal maximal per R/M cos}
		Sigui \(M\) un ideal d'un anell \(R\) amb la suma \(+\) i el producte \(\cdot\) amb \(I\neq R\). Aleshores
		\[R/M\text{ és un cos}\sii M\text{ és un ideal maximal de }R.\]
		\begin{proof}
			Aquest enunciat té sentit per la proposició \myref{prop:anell quocient}.
			
			Comencem veient la implicació cap a la dreta (\(\implica\)). Suposem doncs que \(R/M\) és un cos i prenem un ideal \(I/M\) %REF lema teorema d'isomorfia anells per donar-li sentit
			 de \(R/M\). Aleshores ha de ser \(M\subseteq I\subseteq R\). Per la proposició \myref{prop:condició equivalent a cos per anells} tenim que els únics ideals de \(R/M\) són \(([0])\) i \(R/M\), i per tant ha de ser \(I=M\) o \(I=R\), i per la definició d'\myref{def:ideal maximal} tenim que \(M\) és un ideal maximal de \(R\).
			
			Veiem ara la implicació cap a l'esquerra (\(\implicatper\)). Suposem doncs que \(M\) és un ideal maximal de l'anell \(R\) i considerem, per la proposició \myref{prop:anell quocient}, l'anell \(R/M\). Per la proposició \myref{prop:condició equivalent a cos per anells} tenim que només hem de veure que els únics ideals de \(R/M\) són \((0)\) i \(R\). Prenem un ideal \(I/M\) de \(R/M\). Aquest ha de ser tal que \(M\subseteq I\subseteq R\), i per la definició d'\myref{def:ideal maximal} tenim que ha de ser \(I=M\) o \(I=R\), i per tant \(I/M\) ha de ser \(([0])\) o \(R/M\) i per la proposició \myref{prop:anell quocient} tenim que \(R/M\) és un cos.
		\end{proof}
	\end{proposition}
	\begin{corollary}
		\label{corollary:M maximal implica M primer}
		\(M\) és maximal \(\Longrightarrow\) \(M\) és primer.
	\end{corollary}
	\begin{theorem}
		Sigui \(R\) un anell commutatiu amb la suma \(+\) i el producte \(\cdot\) amb \(1\neq0\). Aleshores \(R\) és un domini d'integritat si i només si \(R\setminus\{0\}\) és un cos amb la suma \(+\) i el producte \(\cdot\). %Ha de ser \abs{R} finit?
		\begin{proof}
			%TODO
		\end{proof}
	\end{theorem}
	\begin{proposition}
		\label{prop:ideal primer en DI es maximal}
		Sigui \(I\neq(0)\) un ideal primer d'un domini d'integritat \(D\) amb la suma \(+\) i el producte \(\cdot\). Aleshores \(I\) és maximal.
		\begin{proof}
			Posem \(I=(a)\). Per hipòtesi tenim que \(a\neq0\) i que \(I\) és un ideal primer. Sigui \(b\in D\) tal que \((a)\subseteq(b)\). Aleshores tenim que \(a\in(b)\), i per la definició d'\myref{def:ideal principal} tenim que \(a=a'\cdot b\) per a cert \(a'\in D\). Aleshores, per la definició d'\myref{def:ideal primer}, tenim que \(a'\in(a)\) o \(b\in(a)\).
			
			Suposem que \(a'\in(a)\). Aleshores tenim que \(a'=a\cdot\beta\) per a cert \(\beta\in D\), i per tant \(a=a\cdot\beta\cdot b\), i per la proposició \myref{prop:podem tatxar pels costats en DI} tenim que \(1=\beta\cdot b\), i per tant \(1\in(b)\), d'on trobem \((b)=R\). Suposem ara que \(b\in I\). Aleshores \((a)=(b)\), i per la definició de \myref{def:ideal maximal} tenim que \(I=(a)\) és un ideal maximal de \(D\).
		\end{proof}
	\end{proposition}
	\subsection{Lema de Zorn}
	\begin{definition}[Relació d'ordre]
		\labelname{relació d'ordre}
		\label{def:relació d'ordre}
		Sigui \(A\) un conjunt no buit i \(\leq\) una relació binària en \(A\) que satisfaci
		\begin{enumerate}
			\item Reflexiva: \(a\leq a\) per a tot \(a\in A\).
			\item Antisimètrica: \(a\leq b\) i \(b\leq a\) impliquen \(a=b\) per a tot \(a,b\in A\).
			\item Transitiva: Si \(a\leq b\) i \(b\leq c\), aleshores \(a\leq c\) per a tot \(a,b,c\in A\).
		\end{enumerate}
		Aleshores direm que \(\leq\) és una relació d'ordre.
	\end{definition}
	\begin{definition}[Cadena]
		\labelname{cadena}
		\label{def:cadena}
		Siguin \(\mathcal{C}\) un conjunt i \(\leq\) una relació d'ordre en \(A\) tal que per a tot \(a,b\in A\) es satisfà \(a\leq b\) ó \(b\leq a\). Aleshores direm que \(\mathcal{C}\) amb \(\leq\) és una cadena.
	\end{definition}
	\begin{proposition}
		\label{prop:subconjunts d'un conjunt amb inclusió són una cadena}
		Siguin \(Y\) i \(\mathcal{X}\subseteq\mathcal{P}(Y)\) dos conjunts tals que per a tot \(A,B\in X\) tenim \(A\subseteq B\) o \(B\subseteq A\). Aleshores \(\mathcal{X}\) amb \(\subseteq\) és una cadena.
		\begin{proof}
			Comprovem  que \(\subseteq\) satisfà les condicions de la definició de \myref{def:relació d'ordre}:
			\begin{enumerate}
				\item Reflexiva: Si \(A\in\mathcal{X}\) tenim \(A=A\), i en particular \(A\subseteq A\). %REFERENCIES
				\item Antisimètrica: Si \(A,B\in\mathcal{X}\) tals que \(A\subseteq B\) i \(B\subseteq A\) tenim, per doble inclusió, que \(A=B\). %REFERENCIA doble inclusió
				\item Transitiva: Si \(A,B,C\in\mathcal{X}\) tals que \(A\subseteq B\) i \(B\subseteq C\) aleshores \(A\subseteq C\).
			\end{enumerate}
			per tant, per les definicions de \myref{def:relació d'ordre} i \myref{def:cadena} tenim que \(\mathcal{X}\) amb \(\subseteq\) és una cadena.
		\end{proof}
	\end{proposition}
	\begin{definition}[Cota superior d'una cadena]
		\labelname{cota superior d'una cadena}
		\label{def:cota superior d'una cadena}
		\labelname{element maximal d'una cadena}
		\label{def:element maximal d'una cadena}
		Siguin \(\mathcal{C}\) amb \(\leq\) una cadena, \(a\) un element de \(\mathcal{C}\) i \({B}\) un subconjunt de \(\mathcal{C}\) tal que per a tot \(b\in{B}\) es compleix \(b\leq A\). Aleshores direm que \(a\) és una cota superior de \({B}\).
		
		Si \(a\leq b\) implica \(b=a\) per a tot \(b\in{B}\) direm que \(a\) és maximal per \({B}\).
	\end{definition}
	\begin{axiom}[Lema de Zorn]
		\label{lema de Zorn}
		Sigui \(\mathcal{A}\) amb \(\leq\) una cadena tal que per a tot subconjunt \(\mathcal{C}\subseteq\mathcal{A}\) la cadena \(\mathcal{C}\) té alguna cota superior. Aleshores \(\mathcal{A}\) té algun element maximal.
	\end{axiom}
	\begin{theorem}
		\label{thm:ideal maximal exsiteix}
		Sigui \(R\) un anell amb la suma \(+\) i el producte \(\cdot\) commutatiu amb \(1\neq0\). Aleshores existeix \(M\subseteq R\) tal que \(M\) sigui un ideal maximal de \(R\).
		\begin{proof}
			Definim el conjunt
			\[A=\{I\triangleleft R\mid I\neq R\}.\]
			i amb un subconjunt \(\mathcal{C}\subseteq A\) considerem, per la proposició \myref{prop:subconjunts d'un conjunt amb inclusió són una cadena}, la cadena \((\mathcal{C},\subseteq)\). Considerem ara el conjunt
			\[J=\bigcup_{I\in\mathcal{C}}I\]
			i veiem que \(J\) és un ideal de \(R\), ja que si \(x,y\in J\) tenim \(x\in J_{1}\) i \(y\in J_{2}\) per a certs \(J_{1},J_{2}\in\mathcal{C}\). Ara bé, si \(J_{2}\subseteq J_{1}\) tenim que \(x-y\in J_{1}\), i per tant \(x-y\in J\), i si \(J_{1}\subseteq J_{2}\) tenim que \(x-y\in J_{2}\), i per tant \(x-y\in J\). Si prenem \(x\in J\) i \(r\in R\) aleshores \(r\cdot x\in J\), ja que tenim \(x\in J_{1}\) per a cert \(J_{1}\in\mathcal{C}\), i per tant \(r\cdot x\in J_{1}\), i en particular \(r\cdot x\in J\). Per tant per la definició d'\myref{def:ideal d'un anell} tenim que \(J\) és un ideal de \(R\). Per veure que \(J\in A\) hem de comprovar que \(J\neq R\). Ho fem per contradicció. Suposem que \(J=R\). Aleshores \(1\in J\), i per tant \(1\in I\) per a cert \(I\in A\), però això no pot ser ja que si \(I\in A\) s'ha de complir \(I\neq A\), i per tant \(1\notin I\). Per tant \(J\neq R\) i tenim que \(J\in A\).
			
			Ara bé, pel \myref{lema de Zorn} tenim que existeix \(M\in\mathcal{C}\) tal que per a tot \(I\in\mathcal{C}\) tenim \(I\subseteq M\) i per la definició d'\myref{def:ideal maximal} tenim que \(M\) és un ideal maximal de \((R,+,\cdot)\).
		\end{proof}
	\end{theorem}
	\subsection{Divisibilitat}
	\begin{definition}[Divisors i múltiples]
		\labelname{divisor}\label{def:divisor per anells}
		\labelname{múltiple}\label{def:múltiple per anells}
		Siguin \(D\) un domini d'integritat amb la suma \(+\) i el producte \(\cdot\) i \(a,b\in D\) tals que existeix \(c\in D\) tal que \(b=a\cdot c\). Aleshores direm que \(a\) divideix \(b\) o que \(b\) és múltiple de \(a\). Ho denotarem amb \(a\divides b\).
	\end{definition}
	\begin{observation}
		\label{obs:divisors són ideals continguts}
		\(b\divides a\sii(a)\subseteq(b)\).
	\end{observation}
	\begin{proposition}
		\label{prop:podem passar els múltiples de costat a costat}
		Siguin \(D\) un domini d'integritat amb la suma \(+\) i el producte \(\cdot\) i \(a,b,c,c'\) quatre elements, amb \(a\neq0\), \(b\neq0\), tals que \(a\divides b\) i \(b\divides a\), i \(a=c\cdot b\) i \(b=c'\cdot a\). Aleshores \(c'=c^{-1}\).
		\begin{proof}
			Tenim que \(b=c'\cdot c\cdot b\), i per la proposició \myref{prop:podem tatxar pels costats en DI} tenim que \(1=c'\cdot c\), i per la definició d'\myref{def:element invertible pel producte d'un anell} tenim que \(c'=c^{-1}\).
		\end{proof}
	\end{proposition}
	\begin{proposition}
		\label{prop:associats és relació d'equivalència}
		Sigui \(R\) un anell amb la suma \(+\) i el producte \(\cdot\) commutatiu amb \(1\neq0\) i \(\sim\) una relació binària tal que per a tot \(x,y\in R\) tenim
		\[x\sim y\Longrightarrow x=u\cdot y\text{ per a algun }u\in R\text{ invertible}.\]
		Aleshores \(\sim\) és una relació d'equivalència.
		\begin{proof}
			Comprovem les condicions de la definició de \myref{def:relació d'equivalència}:
			\begin{enumerate}
				\item Simètrica: Per a tot \(x\in R\) tenim \(x=1\cdot x\).
				\item Reflexiva: Siguin \(x,y\in R\) tals que \(x\sim y\). Aleshores tenim que existeix \(u\in R\) invertible tal que \(x=u\cdot y\). Ara bé, com que \(u\) és invertible tenim per la definició de \myref{def:element invertible pel producte d'un anell} que \(y=u^{-1}\cdot x\), i per tant \(y\sim x\).
				\item Transitiva: Siguin \(x,y,z\in R\) tals que \(x\sim y\) i \(y\sim z\). Aleshores tenim que \(x=u\cdot y\) i \(y=u'\cdot z\) per a certs \(u,u'\in R\) invertibles, i per tant \(x=u\cdot u'\cdot z\), i com que \(1=u\cdot u'\cdot {u'}^{-1}\cdot{u}^{-1}\) per la definició de \myref{def:element invertible pel producte d'un anell} tenim que \(x\sim z\).
			\end{enumerate}
			i per la definició de \myref{def:relació d'equivalència} tenim que \(\sim\) és una relació d'equivalència.
		\end{proof}
	\end{proposition}
	\begin{definition}[Elements associats]
		\labelname{elements associats}
		\label{def:elements associats}
		Siguin \(R\) un anell commutatiu amb la suma \(+\) i el producte \(\cdot\) amb \(1\neq0\) i \(a,b\in R\) dos elements tals que existeix un element invertible \(u\in R\) tal que \(a=u\cdot b\). Aleshores direm que \(a\) i \(b\) són associats i escriurem \(a\sim b\).
		
		Aquesta definició té sentit per la proposició \myref{prop:associats és relació d'equivalència}.
	\end{definition}
	\begin{proposition}
		\label{prop:màxim comú divisor anells}
		Siguin \(D\) un domini d'integritat amb la suma \(+\) i el producte \(\cdot\), \(a,b\) dos elements de \(D\) i \(X\subseteq D\) un conjunt tal que per a tot \(d\in X\) tenim \(d\divides a\), \(d\divides b\) i per a tot \(c\in D\) tal que \(c\divides a\), \(c\divides b\) es compleix \(c\divides d\). Aleshores tenim que per a tot \(d'\in X\) si i només si \(d\sim d'\).
		\begin{proof}
			Comencem amb la implicació cap a la dreta (\(\implica\)). Suposem que \(d'\in X\). Hem de veure que \(d\sim d'\). Tenim \(d\divides d'\) i \(d'\divides d\) i per la definició de \myref{def:divisor per anells} trobem que \(d\sim d'\).
			
			Fem ara la implicació cap a l'esquerra (\(\implicatper\)). Suposem que \(d'\sim d\). Hem de veure que \(d'\in X\). Per hipòtesi tenim que \(d\divides a\) i \(d\divides b\). Per tant existeixen \(\alpha,\beta\in D\) tals que \(a=\alpha d\) i \(b=\beta d\), i per la proposició \myref{prop:podem passar els múltiples de costat a costat} tenim que si \(d'=d\cdot u\) amb \(u\in D\) invertible aleshores \(d=d'\cdot u^{-1}\). Per tant
			\[a=\alpha\cdot d'\cdot u^{-1}\quad\text{i}\quad b=\beta\cdot d'\cdot u^{-1}\]
			i per tant \(d'\divides a\) i \(d'\divides b\). Ara bé, com que per hipòtesi \(d\sim d'\), per la definició d'\myref{def:elements associats} tenim que \(d'\in X\).
		\end{proof}
	\end{proposition}
	\begin{definition}[Màxim comú divisor]
		\labelname{màxim comú divisor}
		\label{def:màxim comú divisor anells}
		\label{def:mcd anells}
		Siguin \(D\) un domini d'integritat amb la suma \(+\) i el producte \(\cdot\) i \(a,b,d\in D\) tres elements tals que \(d\divides a\) i \(d\divides b\) i tals que per a tot \(c\in D\) que satisfaci \(c\divides a\) i \(c\divides b\) tenim \(c\divides d\). Aleshores direm que \(d\) és el màxim comú divisor de \(a\) i \(b\). Direm que \(d\) és un màxim comú divisor de \(a\) i \(b\) o que \(d\sim\mcd(a,b)\).
		Entendrem que \(\mcd(a,b)\) és un element de \(D\).
		
		Aquesta definició té sentit per la proposició \myref{prop:màxim comú divisor anells}.
	\end{definition}
	\begin{proposition}
		\label{prop:mínim comú múltiple anells}
		Siguin \(D\) un domini d'integritat amb la suma \(+\) i el producte \(\cdot\), \(a,b\) dos elements de \(D\) i \(X\subseteq D\) un conjunt tal que per a tot \(m\in X\) tenim \(a\divides m\), \(b\divides m\) i per a tot \(c\in D\) tal que \(a\divides c\), \(b\divides c\) es compleix \(m\divides c\). Aleshores tenim que per a tot \(m'\in X\) si i només si \(m\sim m'\).
		\begin{proof}
			Comencem amb la implicació cap a la dreta (\(\implica\)). Suposem que \(m'\in X\). Hem de veure que \(m\sim m'\). Tenim \(m'\divides m\) i \(m\divides m'\) i per la definició de \myref{def:múltiple per anells} trobem que \(m\sim m'\). %REVISAR. Em feia mandra pensar i he copiat la de dalt :S
			
			Fem ara la implicació cap a l'esquerra (\(\implicatper\)). Suposem que \(m'\sim m\). Hem de veure que \(m'\in X\). Per hipòtesi tenim que \(a\divides m\) i \(b\divides m\). Per tant existeixen \(\alpha,\beta\in D\) tals que \(m=\alpha\cdot a\) i \(m=\beta\cdot b\), i per la proposició \myref{prop:podem passar els múltiples de costat a costat} tenim que si \(m'=u\cdot m\) amb \(u\in D\) invertible aleshores \(m=m'\cdot u^{-1}\). Per tant
			\[m'=\alpha\cdot a\cdot u^{-1}\quad\text{i}\quad m'=\beta\cdot b\cdot{u'}^{-1}\]
			i per tant \(m'\divides a\) i \(m'\divides b\). Ara bé, com que per hipòtesi \(m\sim m'\), per la definició d'\myref{def:elements associats} tenim que \(m'\in X\). %REVISAR. Em feia mandra pensar i he copiat la de dalt :S
		\end{proof}
	\end{proposition}
	\begin{definition}[Mínim comú múltiple]
		\labelname{mínim comú múltiple}
		\label{def:mínim comú múltiple anells}
		\label{def:mcm anells}
		Siguin \(D\) un domini d'integritat amb la suma \(+\) i el producte \(\cdot\) i \(a,b,m\in D\) tres elements tals que \(a\divides m\) i \(b\divides m\) i tals que per a tot \(c\in D\) que satisfaci \(a\divides c\) i \(b\divides c\) tenim \(m\divides c\). Aleshores direm que \(m\) és el mínim comú múltiple de \(a\) i \(b\). Direm que \(m\) és un mínim comú múltiple de \(a\) i \(b\) o que \(m\sim\mcm(a,b)\).
		Entendrem que \(\mcm(a,b)\) és un element de \(D\).
		
		Aquesta definició té sentit per la proposició \myref{prop:mínim comú múltiple anells}.
	\end{definition}
	\begin{proposition}
		\label{prop:combinació d'ideals principals per obtenir ideals principals}
		Siguin \((a),(b)\) dos ideals principals d'un domini d'integritat \((D,+,\cdot)\). Aleshores tenim les igualtats
		\begin{enumerate}
			\item\label{enum:prop:combinació d'ideals principals per obtenir ideals principals 1} \((a)+(b)=(\mcd(a,b))\).
			\item\label{enum:prop:combinació d'ideals principals per obtenir ideals principals 2} \((a)\cap(b)=(\mcm(a,b))\).
			\item\label{enum:prop:combinació d'ideals principals per obtenir ideals principals 3} \((a)(b)=(a\cdot b)\).
		\end{enumerate}
		\begin{proof}%pulir
			Comencem veient el punt \eqref{enum:prop:combinació d'ideals principals per obtenir ideals principals 1}. Per la proposició \myref{prop:combinació d'ideals per obtenir ideals} tenim que \((a)+(b)=\{x+y\mid x\in(a),y\in(b)\}\), i per la definició d'\myref{def:ideal principal} això és
			\[(a)+(b)=\{r_{1}\cdot a+r_{2}\cdot b\mid r_{1},r_{2}\in R\},\]
			que podem reescriure com
			\[(a)+(b)=\{x\mid\text{existeixen }m,n\in R\text{ tals que }x=n\cdot m+b\cdot n\}\]
			i per tant \((a)+(b)=(\mcd(a,b))\) és un ideal principal de \(R\). %REFERENCIES
			
			Continuem veient el punt \eqref{enum:prop:combinació d'ideals principals per obtenir ideals principals 2}. Per la proposició \myref{prop:combinació d'ideals per obtenir ideals} tenim que
			\[(a)\cap(b)=\{x\mid x\in(a),x\in(b)\},\]
			que, per la definició d'\myref{def:ideal principal}, podem reescriure com
			\[(a)\cap(b)=\{x\mid x\text{ divideix }a\text{ i }b\}\]
			i per tant \((a)\cap(b)=(\mcm(a,b))\) és un ideal principal de \(R\). %REFERENCIES
			
			Acabem veient el punt \eqref{enum:prop:combinació d'ideals principals per obtenir ideals principals 3}. Per la proposició \myref{prop:combinació d'ideals per obtenir ideals} tenim que
			\[(a)(b)=\{x_{1}\cdot y_{1}+\dots+x_{n}\cdot y_{n}\mid x_{1},\dots,x_{n}\in(a),y_{1},\dots,y_{n}\in(b)\},\]
			que, per la definició d'\myref{def:ideal principal} i la proposició \myref{prop:propietats bàsiques anells}, podem reescriure com
			\begin{align*}
			(a)(b)&=\{(r_{1}\cdot a)(r'_{1}\cdot b)+\dots+(r_{n}\cdot a)(r'_{n}\cdot b)\mid r_{1},\dots,r_{n},r'_{1},\dots,r'_{n}\in R\}\\
			&=\{(r_{1}\cdot r'_{1}+\dots+r_{n}\cdot r'_{n})\cdot(a\cdot b)\mid r_{1},\dots,r_{n},r'_{1},\dots,r'_{n}\in R\},
			\end{align*}
			i si fixem \(r_{2}=\dots r_{n}=0\) i \(r'_{1}=1\) tenim, amb \(r_{1}=r\) que
			\[(a)(b)=\{r\cdot(a\cdot b)\mid r\in R\},\]
			i per la definició d'\myref{def:ideal principal} tenim que \((a)(b)\) és un ideal principal de \(R\) amb
			\[(a)(b)=(a\cdot b).\qedhere\]
		\end{proof}
	\end{proposition}
	%	\begin{proposition}
	%		\label{prop:propietats mcm mcd}
	%		Siguin \(D\) un domini d'integritat amb la suma \(+\) i el producte \(\cdot\) i \(a,b\) dos elements de \(D\). Aleshores
	%		\begin{enumerate}
	%			\item\label{enum:prop:propietats mcm mcd 1} \(\mcd(a,0)\sim0\) i \(\mcm(a,0)\sim0\).
	%			\item\label{enum:prop:propietats mcm mcd 2} Si \(d\neq0\) és un element de \(D\) tal que \(d\sim\mcd(a,b)\) i \(a',b'\in D\) són tals que \(a=a'\cdot d\) i \(b=b'\cdot d\)  aleshores \(\mcd(a,b)\sim1\).
	%			\item\label{enum:prop:propietats mcm mcd 3} Si \(c\) és un element de \(D\) aleshores \(\mcd(c\cdot a,c\cdot b)\sim c\cdot\mcd(a,b)\).
	%			\item Si \(a\neq0\) i \(b\neq0\), i existeix \(c\in D\) tal que, si \(m\sim\mcm(a,b)\), tenim \(a\cdot b=m\cdot c\), aleshores \(c\sim\mcd(a,b)\).
	%		\end{enumerate}
	%		\begin{proof}
	%			Comencem veient el punt \eqref{enum:prop:propietats mcm mcd 1}. 
	%		\end{proof}
	%	\end{proposition}
	\begin{definition}[Primer]
		\labelname{primer}
		\label{def:primer en un anell}
		Siguin \(D\) un domini d'integritat amb la suma \(+\) i el producte \(\cdot\), \(p\neq0\) un element de \(D\) tal que per a tot \(a,b\) dos elements de \(D\) que satisfacin \(p\divides a\cdot b\) tenim \(p\divides a\) ó \(p\divides b\). Aleshores direm que \(p\) és primer.
	\end{definition}
	\begin{observation}
		\label{obs:ideals primer iff primer}
		\(a\neq0\), \((a)\) és un ideal primer si i només si \(a\) és primer.
	\end{observation}
	\begin{definition}[Element irreductible]
		\labelname{irreductible}
		\label{def:irreductible en un anell}
		Siguin \(D\) un domini d'integritat amb la suma \(+\) i el producte \(\cdot\), \(a\neq0\) un element no invertible de \(D\) i \(b,c\) dos elements de \(D\) tals que \(a=b\cdot c\). Aleshores direm que \(a\) és irreductible si \(b\) ó \(c\) són invertibles.
	\end{definition}
	\begin{proposition}
		Siguin \(D\) un domini d'integritat amb la suma \(+\) i el producte \(\cdot\) i \(p\) un element primer de \(D\). Aleshores \(p\) és irreductible.
		\begin{proof}
			Suposem que \(a,b\) són dos elements de \(D\) tals que \(p=a\cdot b\). Per la definició de \myref{def:primer en un anell} tenim que ha de ser \(p\divides a\) ó \(p\divides b\). Si \(p\divides a\) tenim que \(a=\alpha\cdot p\) per a cert \(\alpha\in D\). Ara bé, per hipòtesi, tenim que \(p=a\cdot b\). Per tant \(a=\alpha\cdot a\cdot b\), i per la proposició \myref{prop:podem tatxar pels costats en DI} tenim que \(1=\alpha\cdot b\), i per la definició d'\myref{def:element invertible pel producte d'un anell} tenim que \(b\) és invertible i per la definició d'\myref{def:irreductible en un anell} tenim que \(p\) és irreductible.
			
			El cas \(p\divides b\) és anàleg.
		\end{proof}
	\end{proposition}
	\subsection{Dominis de factorització única}
	\begin{definition}[Domini de factorització única]
		\labelname{domini de factorització única}
		\label{def:domini de factorització única}
		\label{def:DFU}
		Sigui \(D\) un domini d'integritat amb la suma \(+\) i el producte \(\cdot\) tal que per a tot element no invertible \(a\neq0\) de \(D\)
		\begin{enumerate}
			\item Existeixen \(p_{1},\dots,p_{n}\) elements irreductibles de \(D\) tals que
			\[a=p_{1}\cdot\ldots\cdot p_{n}.\]
			\item Si existeixen \(p_{1},\dots,p_{r}\) i \(q_{1},\dots,q_{s}\) elements irreductibles de \(D\) tals que
			\[a=p_{1}\cdot\ldots\cdot p_{r}=q_{1}\cdot\ldots\cdot q_{s}\]
			aleshores \(r=s\) i existeix \(\sigma\in S_{r}\) tal que
			\[p_{1}\cdot\ldots\cdot p_{r}=q_{\sigma(1)}\cdot\ldots\cdot q_{\sigma(r)},\]
			amb \(p_{i}\sim q_{\sigma(i)}\) per a tot \(i\in\{1,\dots,r\}\).
		\end{enumerate}
		Aleshores direm que \(D\) és un domini de factorització única.
	\end{definition}
	\begin{theorem}
		\label{thm:condició equivalent per DI sii DFU}
		Sigui \(D\) un domini d'integritat amb la suma \(+\) i el producte \(\cdot\). Aleshores \(D\) és un domini de factorització única si i només si tenim
		\begin{enumerate}
			\item\label{enum:thm:condició equivalent per DI sii DFU 1} Per a tot \(a\neq0\) element no invertible de \(D\) existeixen \(p_{1},\dots,p_{r}\) elements irreductibles de \(D\) tals que
			\[a=p_{1}\cdot\ldots\cdot p_{r}\]
			\item\label{enum:thm:condició equivalent per DI sii DFU 2} Si \(a\) és in element irreductible de \(D\) aleshores \(a\) és primer.
		\end{enumerate}
		\begin{proof}
			Comencem demostrant que la condició és suficient (\(\implica\)). Suposem doncs que \(D\) és un domini de factorització única. El punt \eqref{enum:thm:condició equivalent per DI sii DFU 1} és conseqüència de la definició de \myref{def:domini de factorització única}. Per tant només ens queda veure que tot element irreductible és primer. Siguin \(p\) un element irreductible de \(D\) i \(a,b\) dos elements no invertibles no nuls de \(D\) tals que \(p\divides a\cdot b\). Per la definició de \myref{def:DFU} tenim que existeixen \(p_{1},\dots,p_{r},q_{1},\dots,q_{s}\) elements irreductibles de \(D\) tals que
			\[a=p_{1}\cdot\ldots\cdot p_{r}\quad\text{i}\quad b=q_{1}\cdot\ldots\cdot q_{s}\]
			i per tant
			\[a\cdot b=p_{1}\cdot\ldots\cdot p_{r}\cdot q_{1}\cdot\ldots\cdot q_{s}\]
			i com que, per hipòtesi, \(p\divides a\cdot b\) i la definició de \myref{def:DFU} tenim que
			\[a\cdot b=p\cdot\alpha_{1}\cdot\ldots\cdot\alpha_{t}\]
			per a certs \(\alpha_{1},\cdots,\alpha_{t}\) elements irreductibles de \(D\). Per tant tenim
			\[a\cdot b=p\cdot\alpha_{1}\cdot\ldots\cdot\alpha_{t}=p_{1}\cdot\ldots\cdot p_{r}\cdot q_{1}\cdot\ldots\cdot q_{s}\]
			i, de nou, per la definició de \myref{def:DFU} tenim que \(p\sim p_{i}\) ó \(p\sim q_{j}\) per a certs \(i\in\{1,\dots,r\}\), \(j\in\{1,\dots,s\}\), i per tant \(p\divides a\) ó \(p\divides b\), i per la definició de \myref{def:primer en un anell} tenim que \(p\) és primer.
			
			Veiem ara que la condició és necessària (\(\implicatper\)). Suposem doncs que
			\begin{enumerate}
				\item Per a tot \(a\) element no invertible de \(D\) existeixen \(p_{1},\dots,p_{r}\) elements irreductibles de \(D\) tals que
				\[a=p_{1}\cdot\ldots\cdot p_{r}\]
				\item Si \(a\) és in element irreductible de \(D\) aleshores \(a\) és primer.
			\end{enumerate}
			Sigui \(a\) un element no invertible de \(D\). Pel punt \eqref{enum:thm:condició equivalent per DI sii DFU 1} tenim que existeixen \(p_{1},\dots,p_{r}\) elements irreductibles de \(D\) tals que
			\[a=p_{1}\cdot\ldots\cdot p_{r}.\]
			
			Suposem que existeixen també \(q_{1},\dots,q_{s}\) elements irreductibles de \(D\) tals que
			\[a=q_{1},\dots,q_{s}.\]
			Aleshores volem veure que \(r=s\) i que existeix \(\sigma\in S_{r}\) tal que
			\[p_{1}\cdot\ldots\cdot p_{r}=q_{\sigma(1)}\cdot\ldots\cdot q_{\sigma(r)},\]
			amb \(p_{i}\sim q_{\sigma(i)}\) per a tot \(i\in\{1,\dots,r\}\).
			
			Tenim que \(p_{1}\divides a\), i com que pel punt \eqref{enum:thm:condició equivalent per DI sii DFU 2} tenim que \(p_{1}\) és primer, per la definició de \myref{def:primer en un anell} tenim que \(p_{1}\divides q_{j}\) per a cert \(j\in\{1,\dots,s\}\), i per la definició de \myref{def:irreductible en un anell} i la definició d'\myref{def:elements associats} tenim que \(p_{1}\sim q_{j}\). Sigui doncs \(\sigma\in S_{s}\) tal que \(p_{1}\divides q_{\sigma(1)}\). Aleshores tenim
			\[p_{1}\cdot p_{2}\cdot\ldots\cdot p_{r}=u_{1}\cdot q_{\sigma(1)}\cdot\ldots\cdot q_{s}\]
			per a cert \(u_{1}\) element invertible de \(D\). Podem iterar aquest argument per a \(p_{2},\dots,p_{t}\), on \(t=\min(r,s)\) per obtenir
			\[p_{1}\cdot\ldots\cdot p_{t}\cdot p_{t+1}\cdot\ldots\cdot p_{r}=(u_{1}\cdot q_{1})\cdot\ldots\cdot(u_{t}\cdot q_{t})\cdot p_{t+1}\cdot\ldots \cdot p_{s}\]
			per a certs \(u_{1},\dots, u_{t}\) elements invertibles de \(D\). Ara bé, tenim que \(r=s\), ja que si \(r>s\) tindríem que \(p_{s+1},\dots,p_{r}\) són invertibles, i si \(s>r\) tindríem que \(q_{r+1},\dots,q_{s}\) són invertibles, però per la definició d'\myref{def:irreductible en un anell} i la definició de \myref{def:element invertible pel producte d'un anell} tenim que això no pot ser, i per tant \(r=s\) i per la definició de \myref{def:DFU} tenim que \(D\) és un domini de factorització única, com volíem veure.
		\end{proof}
	\end{theorem}
	\begin{proposition}
		Siguin \(D\) un domini de factorització única amb la suma \(+\) i el producte \(\cdot\) i \(a,b\) dos elements no invertibles i no nuls de \(D\) tals que existeixen \(p_{1},\dots,p_{r}\) tals que
		\[a=p_{1}^{\alpha_{1}}\cdot\ldots\cdot p_{r}^{\alpha_{r}}\quad\text{i}\quad b=p_{1}^{\beta_{1}}\cdot\ldots\cdot p_{r}^{\beta_{r}}\]
		per a certs \(\alpha_{1},\dots,\alpha_{r},\beta_{1},\dots,\beta_{r}\in\mathbb{N}\). Aleshores
		\[\prod_{i=1}^{r}p_{i}^{\min(\alpha_{i},\beta_{i})}\sim\mcd(a,b)\quad\text{i}\quad\prod_{i=1}^{r}p_{i}^{\max(\alpha_{i},\beta_{i})}\sim\mcm(a,b).\]
		\begin{proof}
			Denotem \(d=\prod_{i=1}^{r}p_{i}^{\min(\alpha_{i},\beta_{i})}\) i \(m=\prod_{i=1}^{r}p_{i}^{\max(\alpha_{i},\beta_{i})}\).
			
			Prenem \(c\) un element de \(D\) tal que \(c\divides a\) i \(c\divides b\). Aleshores tenim que
			\[c=p_{1}^{\gamma_{1}}\cdot\ldots\cdot p_{r}^{\gamma_{r}}\]
			per a certs \(\gamma_{i}\leq\min(\alpha_{i},\beta_{i})\) per a tot \(i\in\{1,\dots,r\}\). Ara bé, com que \(\gamma_{i}\leq\min(\alpha_{i},\beta_{i})\) per a tot \(i\in\{1,\dots,r\}\), i per tant \(d\divides c\), i per la definició de \myref{def:màxim comú divisor anells} tenim que \(d\sim\mcd(a,b)\).
			
			Prenem ara \(c\) un element de \(D\) tal que \(a\divides c\) i \(b\divides c\). Aleshores tenim que
			\[c=q\cdot p_{1}^{\gamma_{1}}\cdot\ldots\cdot p_{r}^{\gamma_{r}}\]
			per a cert \(q\) element de \(D\) i certs \(\gamma_{i}\geq\max(\alpha_{i},\beta_{i})\) per a tot \(i\in\{1,\dots,r\}\). Ara bé, com que \(\gamma_{i}\geq\max(\alpha_{i},\beta_{i})\) per a tot \(i\in\{1,\dots,r\}\), i per tant \(m\divides c\), i per la definició de \myref{def:mínim comú múltiple anells} tenim que \(m\sim\mcm(a,b)\).
		\end{proof}
	\end{proposition}
	\subsection{Anells Noetherians}
	\begin{definition}[Anell Noetherià]
		\labelname{anell Noetherià}
		\label{def:anell Noetherià}
		Sigui \(N\) un anell commutatiu amb la suma \(+\) i el producte \(\cdot\) amb \(1\neq0\) tal que si
		\[I_{1}\subseteq I_{2}\subseteq I_{3}\subseteq\dots\]
		són ideals de \(N\) existeix \(n_{0}\) tal que per a tot \(i\geq n_{0}\) tenim \(I_{i}=I_{i+1}\). Aleshores diem que \(N\) és Noetherià.
	\end{definition}
	\begin{observation}
		\((\{I_{1},I_{2},I_{3}\dots\},\subseteq)\) és una cadena.
	\end{observation}
	\begin{lemma}
		\label{lema:DIP és DFU}
		Siguin \(N\) un domini d'integritat Noetherià amb la suma \(+\) i el producte \(\cdot\) i \(a\neq0\) un element no invertible de \(N\). Aleshores existeixen \(p_{1},\dots,p_{n}\) elements irreductibles de \(N\) tals que
		\[a=p_{1}\cdot\ldots\cdot p_{n}.\]
		\begin{proof}
			Ho farem per reducció a l'absurd. Definim el conjunt
			\[X=\{a\in N\text{ invertible}\mid a\neq p_{1}\cdot\ldots\cdot p_{n}\text{ per a }p_{1},\dots,p_{n}\in N\text{ irreductibles}\}.\]
			Volem veure que \(X=\emptyset\). Suposem doncs que \(X\neq\emptyset\) i prenem \(a_{1}\in X\). Per la definició d'\myref{def:irreductible en un anell} tenim que \(a_{1}\) no és irreductible, i per tant existeixen \(b_{1},c_{1}\in N\) no invertibles tals que
			\[a_{1}=b_{1}\cdot c_{1}\]
			i ha de ser \(b_{1}\in X\) o \(c_{1}\in X\).
			
			Suposem que \(b_{1}\in X\), la demostració de l'altre opció és anàloga. Aleshores tenim, per l'observació \myref{obs:divisors són ideals continguts}, que \((a)\subset(b)\). Ara bé, també tindríem que \(b_{1}=b_{2}\cdot c_{2}\) per a certs \(b_{2},c_{2}\) elements no invertibles de \(N\) amb \(b_{2}\in X\) o \(c_{2}\in X\), i podem iterar aquest argument per construir
			\[(a_{1})\subset(b_{1})\subset(b_{2})\subset(b_{3})\subset\dots\]
			però això entra en contradicció amb la definició de \myref{def:anell Noetherià}, i per tant \(X=\emptyset\), com volíem veure.
		\end{proof}
	\end{lemma}
	\subsection{Dominis d'ideals principals}
	\begin{definition}[Domini d'ideals principals]
		\labelname{domini d'ideals principals}
		\label{def:domini d'ideals principals}
		\label{def:DIP}
		Sigui \(D\) un domini d'integritat amb la suma \(+\) i el producte \(\cdot\) tal que tot ideal de \(D\) és un ideal principal. Aleshores direm que \(D\) és un domini d'ideals principals.
	\end{definition}
	\begin{proposition}
		\label{prop:irreductible sii ideal maximal}
		Sigui \(D\) un domini d'ideals principals amb la suma \(+\) i el producte \(\cdot\). Aleshores un element \(a\in D\) és irreductible si i només si \((a)\) és un ideal maximal.
		\begin{proof}
			Comencem veient que la condició és suficient (\(\implicatper\)). Suposem doncs que \(a\) és un element irreductible de \(D\) i prenem \(b\in D\) tal que \((a)\subseteq(b)\neq D\). Aleshores, per l'observació \myref{obs:divisors són ideals continguts} tenim que \(b\divides a\), és a dir, existeix \(r\in D\) tal que \(a=b\cdot r\), i per la definició d'\myref{def:irreductible en un anell} tenim que \(r\) ó \(b\) són invertibles. Ara bé, com que, per hipòtesi, \((b)\neq D\) tenim que \(b\) no és invertible, %REF
			per tant ha de ser \(r\) invertible per la definició d'\myref{def:element invertible pel producte d'un anell} tenim que \(a\cdot r^{-1}=b\), per l'observació \myref{obs:divisors són ideals continguts} tenim que \((a)=(b)\), i per la definició d'\myref{def:ideal maximal} tenim que \((a)\) és un ideal maximal.
			
			Tenim que la condició és necessària (\(\implica\)) per la proposició \myref{prop:ideal primer en DI es maximal}. %REVISAR
		\end{proof}
	\end{proposition}
	\begin{proposition}
		\label{prop:en DIP irreductible implica primer}
		Siguin \(D\) un domini d'ideals principals amb la suma \(+\) i el producte \(\cdot\) i \(a\) un element irreductible de \(D\). Aleshores \(a\) és primer.
		\begin{proof}
			Per la proposició \myref{prop:irreductible sii ideal maximal} tenim que \((a)\) és maximal, pel corol·lari \myref{corollary:M maximal implica M primer} veiem que \((a)\) és primer, i per l'observació \myref{obs:ideals primer iff primer} trobem que \(a\) és primer, com volíem veure.
		\end{proof}
	\end{proposition}
	\begin{theorem}
		\label{thm:DIP es Noetherià}
		Sigui \(D\) un domini d'ideals principals amb la suma \(+\) i el producte \(\cdot\). Aleshores \(D\) és Noetherià.
		\begin{proof}
			Siguin \(I_{1},\dots,I_{i},\dots\) ideals de \(D\) tals que
			\[I_{1}\subseteq I_{2}\subseteq I_{3}\subseteq\dots\]
			i
			\[\mathcal{I}=\bigcup_{i=1}^{\infty}I_{i}.\]
			Aleshores tenim que \(\mathcal{I}\) és un ideal. També veiem que si \(x\in\mathcal{I}\) existeix \(n\) tal que \(x\in I_{n}\), i per la definició d'\myref{def:ideal d'un anell} tenim que si \(y\in D\) aleshores \(x\cdot y\in I_{n}\).
			
			Ara bé, com que per hipòtesi \(D\) és un domini d'ideals principals tenim, per la definició de \myref{def:DIP} que existeix \(a\in D\) tal que \(\mathcal{I}=(a)\), i per tant existeix \(n\) tal que \(a\in I_{n}\), i trobem que
			\[\mathcal{I}=(a)\subseteq I_{n}\subseteq I_{n+k}\subseteq\mathcal{I}\]
			per a tot \(k\in\mathbb{N}\), i per tant \(I_{n}=I_{n+k}\) per a tot \(k\in\mathbb{N}\), i per la definició de \myref{def:anell Noetherià} trobem que \(D\) és un anell Noetherià.
		\end{proof}
	\end{theorem}
	\begin{theorem}
		\label{thm:DIP és DFU}
		Sigui \(D\) un domini d'ideals principals amb la suma \(+\) i el producte \(\cdot\). Aleshores \(D\) és un domini de factorització única.
		\begin{proof}
			Pel Teorema \myref{thm:DIP es Noetherià} tenim que \(D\) és un anell Noetherià, i pel lema \myref{lema:DIP és DFU} tenim que per a tot element no irreductible \(a\) de \(D\) existeixen \(p_{1},\dots,p_{n}\) elements irreductibles de \(N\) tals que
			\[a=p_{1}\cdot\ldots\cdot p_{n}.\]
			
			També tenim, per la proposició \myref{prop:en DIP irreductible implica primer} que si \(a\) és un element irreductible de \(D\) aleshores \(a\) és primer.
			
			Per acabar, pel Teorema \myref{thm:condició equivalent per DI sii DFU} tenim que \(D\) és un domini de factorització única.
		\end{proof}
	\end{theorem}
	\subsection{Dominis Euclidians}
	\begin{definition}[Domini Euclidià]
		\labelname{domini Euclidià}
		\label{def:domini Euclidià}
		\label{def:DE}
		Siguin \(D\) un domini d'integritat amb la suma \(+\) i el producte \(\cdot\) i \(U\mid D\setminus\{0\}\longrightarrow\mathbb{N}\) una aplicació tal que
		\begin{enumerate}
			\item \(U(x)\leq U(x\cdot y)\) per a tot \(x,y\in D\setminus\{0\}\).
			\item Per a tot \(x,y\in D\), \(y\neq0\) existeixen \(Q,r\in D\) tals que \(x=Q\cdot y+r\), amb \(r=0\) ó \(U(r)<U(y)\).
		\end{enumerate}
		Aleshores direm que \(D\) és un domini Euclidià amb la norma \(U\).
	\end{definition}
	\begin{proposition}
		\label{prop:norma de 1 és la més petita en DE}
		Sigui \(D\) un domini Euclidià amb la suma \(+\) i el producte \(\cdot\) amb la norma \(U\). Aleshores
		\[U(1)\leq U(x)\quad\text{per a tot }x\in D\setminus\{0\}.\]
		\begin{proof}
			Per la definició de \myref{def:DE} tenim que \(U(x)\leq U(x\cdot y)\) per a tot \(x,y\in D\setminus\{0\}\). Per tant
			\[U(1)\leq U(1\cdot x)=U(x)\quad\text{per a tot }x\in D\setminus\{0\}.\qedhere\]
		\end{proof}
	\end{proposition}
	\begin{proposition}
		Sigui \(D\) un domini Euclidià amb la suma \(+\) i el producte \(\cdot\) amb la norma \(U\). Aleshores
		\[U(u)=U(1)\sii u\text{ és un element invertible de }D.\]
		\begin{proof}			
			Comencem veient l'implicació cap a la dreta (\(\implica\)). Suposem doncs que \(u\) és un element invertible de \(D\).
			
			Per la proposició \myref{prop:norma de 1 és la més petita en DE} tenim que \(U(1)\leq U(u)\) i que \(U(u)\leq U(u\cdot u^{-1})\). Ara bé, per la definició de \myref{def:l'invers d'un element d'un anell} tenim que \(u\cdot u^{-1}=1\), i per tant
			\[U(1)\leq U(u)\leq U(u\cdot u^{-1})=U(1),\]
			i trobem \(U(u)=U(1)\).
			
			Veiem ara l'implicació cap a l'esquerra (\(\implicatper\)). Suposem que \(U(u)=U(1)\).
			
			Per la definició de \myref{def:DE} tenim que existeixen \(Q,r\) elements de \(D\) tals que
			\[1=Q\cdot u+r\]
			amb \(r=0\) ó \(U(r)<U(u)\). Ara bé, per hipòtesi \(U(u)=U(1)\), i per la proposició \myref{prop:norma de 1 és la més petita en DE} trobem que ha de ser \(r=0\). Per tant tenim
			\[1=Q\cdot u\]
			i per la definició d'\myref{def:element invertible pel producte d'un anell} trobem que \(u\) és invertible.
		\end{proof}
	\end{proposition}
	\begin{theorem}
		Sigui \(D\) un domini Euclidià amb la suma \(+\) i el producte \(\cdot\). Aleshores \((D,+,\cdot)\) és un domini d'ideals principals.
		\begin{proof}
			Siguin \(U\) una norma de \(D\) i \(I\) un ideal de \(D\). Si \(I=\{0\}\) aleshores \(I=(0)\). Suposem doncs que \(I\neq(0)\) i prenem \(b\in I\) tal que \(U(b)\leq U(x)\) per a tot \(x\in I\), \(x\neq0\).
			
			Prenem ara \(a\in I\). Per la definició de \myref{def:DE} tenim que existeixen \(Q,r\in D\) tals que
			\[a=Q\cdot b+r\]
			amb \(r=0\) ó \(U(r)<U(b)\). I com que, per la definició de \myref{def:anell} tenim que \((D,+)\) és un grup tenim
			\[r=a-Q\cdot b,\]
			i per la proposició \myref{prop:condició equivalent a ideal d'un anell} tenim que
			\[r=a-Q\cdot b\in I\]
			i per tant, \(r\in I\) amb \(r=0\) ó \(U(r)<U(b)\). Ara bé, per hipòtesi tenim que \(U(r)\geq U(b)\), per tant ha de ser \(r=0\) i tenim que
			\[a=Q\cdot b,\]
			d'on trobem que \(I\) és un ideal principal, i per la definició de \myref{def:DIP} tenim que \(D\) és un domini d'ideals principals.
		\end{proof}
	\end{theorem}
	\begin{theorem}
		Sigui \(\mathbb{K}\) un cos amb la suma \(+\) i el producte \(\cdot\). Aleshores \(\mathbb{K}\) és un domini Euclidià.
		\begin{proof}
			%TODO
		\end{proof}
	\end{theorem}
	\section{Anells de polinomis}
	\subsection{Cos de fraccions d'un domini d'integritat}
	\begin{proposition}
		\label{prop:relació d'equivalència cos de fraccions}
		Siguin \(D\) un domini d'integritat amb la suma \(+\) i el producte \(\cdot\) i \(\sim\) definida a \(D\times D\setminus\{0\}\) una relació binària tal que per a tot \(a,c\in D\), \(b,d\in D\setminus\{0\}\) 
		\[(a,b)\sim(c,d)\sii a\cdot d=b\cdot c.\]
		Aleshores \(\sim\) és una relació d'equivalència.
		\begin{proof}
			%TODO
		\end{proof}
	\end{proposition}
	\begin{notation}
		Denotarem el conjunt quocient \(D\times D\setminus\{0\}/\sim\) com \(\mathbb{Q}(D)\) i la classe d'equivalència \(\overline{(a,b)}\in\mathbb{Q}(D)\) com \(\frac{a}{b}\).
	\end{notation}
	\begin{lemma}
		\label{lema:cos de fraccions}
		Siguin \(D\) un domini d'integritat amb la suma \(+\) i el producte \(\cdot\). Aleshores \(\mathbb{Q}(D)\) és un anell commutatiu amb \(1\neq0\) amb les operacions
		\[\frac{a}{b}+\frac{c}{d}=\frac{a\cdot d+c\cdot b}{b\cdot d}\quad\text{i}\quad\frac{a}{b}\cdot\frac{c}{d}=\frac{a\cdot c}{b\cdot d}.\quad\text{per a tot }a,c\in D\text{, }b,d\in D\setminus\{0\}.\]
		\begin{proof}
			%TODO
		\end{proof}
	\end{lemma}
	\begin{theorem}
		\label{thm:cos de fraccions}
		Sigui \(D\) un domini d'integritat amb la suma \(+\) i el producte \(\cdot\). Aleshores \(\mathbb{Q}(D)\) és un cos amb la suma \(+\) i el producte \(\cdot\).
		\begin{proof}
			%TODO
		\end{proof}
	\end{theorem}
	\begin{theorem}[Unicitat de \(\mathbb{Q}(D)\)]
		\labelname{Teorema d'unicitat del cos de fraccions d'un domini}\label{tmh:unicitat del cos de fraccions d'un domini}
		Siguin \(D\) un domini d'integritat amb la suma \(+\) i el producte \(\cdot\) i \(\mathbb{Q}_{1}(D)\) i \(\mathbb{Q}_{2}(D)\) dos cossos  amb la suma \(+\) i el producte \(\cdot\). Aleshores
		\[\mathbb{Q}_{1}(D)\cong\mathbb{Q}_{2}(D)\]
		\begin{proof}
			%TODO%Té sentit per l'anterior?
		\end{proof}
	\end{theorem}
	\begin{definition}[Cos de fraccions]
		\labelname{cos de fraccions d'un domini}\label{def:cos de fraccions}
		Siguin \(D\) un domini d'integritat amb la suma \(+\) i el producte \(\cdot\). Aleshores direm que \(\mathbb{Q}(D)\) és el cos de fraccions de \(D\).
		%Té sentit per l'anterior
	\end{definition}
	\subsection{El Teorema de Gauss}
	\begin{proposition}
		\label{prop:l'anell de polinomis és un anell}
		Siguin \(R\) un anell amb la suma \(+\) i el producte \(\cdot\) i
		\[R[x]=\{a_{0}+a_{1}x+a_{2}x^{2}+\dots+a_{n}x^{n}\mid n\in\mathbb{N}, a_{0},\dots,a_{n}\in R\}\]
		un conjunt. Aleshores \(R[x]\) és un anell amb la suma \(+\) i el producte \(\cdot\).
		\begin{proof}
			%TODO
		\end{proof}
	\end{proposition}
	\begin{observation}
		\label{obs:els anells de polinomis conserven neutre i unitat}
		\(1_{R}=1_{R[x]}\), \(0_{R}=0_{R[x]}\).
	\end{observation}
	\begin{observation}
		\label{obs:els anells de polinomis conserven commutativitat}
		Si \(R\) és un anell commutatiu aleshores \(R[x]\) també és un anell commutatiu.
	\end{observation}
	\begin{definition}[Anell de polinomis]
		\labelname{anell de polinomis}
		\label{def:anell de polinomis}
		Siguin \(R\) un anell amb la suma \(+\) i el producte \(\cdot\) i
		\[R[x]=\{a_{0}+a_{1}x+a_{2}x^{2}+\dots+a_{n}x^{n}\mid n\in\mathbb{N},a_{0},\dots,a_{n}\in R\}.\]
		Aleshores direm que l'anell \(R[x]\) és l'anell de polinomis de \(R\).
		
		Aquesta definició té sentit per la proposició \myref{prop:l'anell de polinomis és un anell}.
	\end{definition}
	\begin{observation}
		\(D\subseteq D[x]\).
	\end{observation}
	\begin{theorem}[Teorema de la base de Hilbert]
		Sigui \(R\) un anell amb la suma \(+\) i el producte \(\cdot\) Noetherià. Aleshores l'anell de fraccions de \(R\), \(R[x]\) és un anell Noetherià.
		\begin{proof}
			%TODO
		\end{proof}
	\end{theorem}
	\begin{definition}[Contingut d'un polinomi]
		\labelname{contingut d'un polinomi}
		\label{def:contingut d'un polinomi}
		Siguin \(D\) un domini de factorització única amb la suma \(+\) i el producte \(\cdot\) i \(f(x)=\sum_{i=0}^{n}a_{i}x^{i}\) un element de \(D[x]\). Aleshores definim
		\[\cont(f)\sim\mcd(a_{0},\dots,a_{n})\]
		com el contingut de \(f\). %Veure que en DFU el mcd existeix (última pagina lila) i associativitat mcd (crec que ja ho tinc fet).
		
		Interpretarem \(\cont(f(x))\) com un element de \(D\).
	\end{definition}
	\begin{definition}[Polinomi primitiu]
		Siguin \(D\) un domini de factorització única amb la suma \(+\) i el producte \(\cdot\) i \(f(x)\) un element de \(D[x]\) tal que
		\[\cont(f(x))\sim1.\]
		Aleshores direm que \(f(x)\) és un polinomi primitiu.
	\end{definition}
	\begin{lemma}[Lema de Gauss]
		\labelname{lema de Gauss}
		\label{lema:lema de Gauss}
		Siguin \(D\) un domini de factorització única amb la suma \(+\) i el producte \(\cdot\) i \(f(x),g(x)\) dos polinomis primitius de \(D[x]\). Aleshores \(f(x)\cdot g(x)\) és un polinomi primitiu.
		\begin{proof}
			%TODO
		\end{proof}
	\end{lemma}
	\begin{corollary}
		Siguin \(D\) un domini de factorització única amb la suma \(+\) i el producte \(\cdot\) i \(f(x)\), \(g(x)\) dos elements de \(D[x]\). Aleshores
		\[\cont(f(x)\cdot g(x))\sim\cont(f(x))\cdot\cont(g(x)).\]
		\begin{proof}
			%TODO
		\end{proof}
	\end{corollary}
	\begin{lemma}
		Siguin \(D\) un domini d'integritat amb la suma \(+\) i el producte \(\cdot\) i \(p\) un element irreductible de \(D\). Aleshores tenim que \(p\) és un element irreductible de \(D[x]\).
		\begin{proof}
			%TODO
		\end{proof}
	\end{lemma}
	\begin{theorem}
		Sigui \(D\) un domini de factorització única amb la suma \(+\) i el producte \(\cdot\) i \(f(x)\) un polinomi de \(D[x]\). Aleshores \(\grau(f(x))\geq1\) i \(f(x)\) és un polinomi irreductible de \(D[x]\) si i només si \(\cont(f(x))\sim1\) i \(f(x)\) és irreductible en \(\mathbb{Q}(D)[x]\).
		\begin{proof}
			%TODO
		\end{proof}
	\end{theorem}
	\begin{theorem}[Teorema de Gauss]
		Sigui \(D\) un domini de factorització única amb la suma \(+\) i el producte \(\cdot\). Aleshores \(D[x]\) és un domini de factorització única.
		\begin{proof}
			%TODO
		\end{proof}
	\end{theorem}
	\begin{theorem}
		Sigui \(D\) un domini d'integritat amb la suma \(+\) i el producte \(\cdot\). Aleshores són equivalents
		\begin{enumerate}
			\item \(D\) és un domini de factorització única.
			\item \(D[x]\) és un domini de factorització única.
			\item \(D[x_{1},\dots,x_{n}]\) és un domini de factorització única.
		\end{enumerate}
		\begin{proof}
			%TODO
		\end{proof}
	\end{theorem}
	\subsection{Criteris d'irreductibilitat}
	\begin{definition}[Arrel]
		\labelname{arrel d'un polinomi}\label{def:arrel d'un polinomi}
		Siguin \(R\) un anell commutatiu amb la suma \(+\) i el producte \(\cdot\) amb \(1\neq0\), \(f(x)\) un element de \(R[x]\) i \(\alpha\) un element de \(R\) tal que \(f(\alpha)=0\). Aleshores direm que \(\alpha\) és una arrel de \(f(x)\).
	\end{definition}
	\begin{proposition}
		Siguin \(\mathbb{K}\) un cos amb la suma \(+\) i el producte \(\cdot\) i \(f(x)\) un element de \(\mathbb{K}[x]\). Aleshores
		\begin{enumerate}
			\item Si \(\grau(f(x))=1\) aleshores \(f(x)\) és irreductible.
			\item Si \(\grau(f(x))=2\text{ ó }3\) aleshores \(f(x)\) és irreductible si i nomé si \(f(x)\) no té cap arrel.
		\end{enumerate}
		\begin{proof}
			%TODO
		\end{proof}
	\end{proposition}
	\begin{proposition}
		Siguin \(D\) un domini de factorització única amb la suma \(+\) i el producte \(\cdot\), \(f(x)=a_{0}+a_{1}\cdot x+\dots+a_{n}\cdot x^{n}\) un polinomi de \(D[x]\) i \(\frac{a}{b}\in\mathbb{Q}(D)\) una arrel de \(f(x)\) amb \(\mcd(a,b)\sim1\). Aleshores tenim que \(a\divides a_{0}\) ó \(b\divides a_{n}\).
		\begin{proof}
			%TODO
		\end{proof}
	\end{proposition}
	\begin{theorem}[Criteri modular]
		\labelname{Teorema del criteri modular}\label{thm:Criteri modular}
		Siguin \(D\) un domini de factorització única amb la suma \(+\) i el producte \(\cdot\), \(f(x)=a_{0}+a_{1}\cdot x+\dots+a_{n}\cdot x^{n}\) un polinomi primitiu de \(D[x]\) i \(p\) un element irreductible de \(D\) amb \(p\ndivides a_{n}\) tals que
		\[\overline{f(x)}=\overline{a_{0}}+\overline{a_{1}}\cdot\overline{x}+\dots+\overline{a_{n}}\cdot\overline{x}^{n}\]
		sigui un polinomi irreductible de \(D/(p)[x]\). Aleshores \(f(x)\) és un polinomi irreductible de \(D[x]\).
		\begin{proof}
			%TODO
		\end{proof}
	\end{theorem}
	\begin{theorem}[Criteri d'Eisenstein]
		\labelname{Teorema del Criteri d'Eisenstein}\label{thm:Criteri d'Eisenstein}
		Siguin \(D\) un domini de factorització única amb la suma \(+\) i el producte \(\cdot\), \(f(x)=a_{0}+a_{1}\cdot x+\dots+a_{n}\cdot x^{n}\) un polinomi primitiu de \(D[x]\) amb \(n\geq1\) i \(p\) un element irreductible de \(D\) tal que \(p\divides a_{0},\dots,p\divides a_{n-1}\) i \(p\ndivides a_{n},p^{2}\ndivides a_{0}\). Aleshores \(f(x)\) és un polinomi irreductible de \(D[x]\).
		\begin{proof}
			%TODO
		\end{proof}
	\end{theorem}
	\begin{corollary}
		Siguin \(D\) un domini de factorització única amb la suma \(+\) i el producte \(\cdot\), \(f(x)=a_{0}+a_{1}\cdot x+\dots+a_{n}\cdot x^{n}\) un polinomi de \(D[x]\) amb \(n\geq1\) i \(p\) un element irreductible de \(D\) tal que \(p\divides a_{0},\dots,p\divides a_{n-1}\) i \(p\ndivides a_{n},p^{2}\ndivides a_{0}\). Aleshores \(f(x)\) és un polinomi irreductible de \(\mathbb{Q}(D)[x]\).
		\begin{proof}
			%TODO
		\end{proof}
	\end{corollary}
	\chapter{Teoria de cossos finits}
	\section{Cossos finits}
	\subsection{Propietats bàsiques dels cossos finits}
	\begin{proposition}
		Siguin \(\mathbb{K}\) i \(\mathbb{E}\) dos cossos amb la suma \(+\) i el producte \(\cdot\) tals que \(\mathbb{K}\subseteq\mathbb{E}\). Aleshores \(\mathbb{E}\) és un \(\mathbb{K}\)-espai vectorial.
		\begin{proof}
			%TODO - Fer àlgebra lineal primer.
		\end{proof}
	\end{proposition}
	\begin{definition}[Cos finit]
		\labelname{cos finit}\label{def:cos finit}
		Sigui \(\mathbb{K}\) un cos amb la suma \(+\) i el producte \(\cdot\) tal que \(\abs{\mathbb{K}}\in\mathbb{N}\). Aleshores direm que \(\mathbb{K}\) és un cos finit.
	\end{definition}
	\begin{observation}
		Sigui \(\mathbb{K}\) un cos finit amb la suma \(+\) i el producte \(\cdot\). Aleshores \(\ch(\mathbb{K})\) és primer.
	\end{observation}
	\begin{theorem}
		Sigui \(\mathbb{K}\) un cos finit amb la suma \(+\) i el producte \(\cdot\). Aleshores
		\[\ch(\mathbb{K})=p\sii\abs{\mathbb{K}}=p^{n}\text{ per a cert }n\in\mathbb{N}.\]
		\begin{proof}
			%TODO
		\end{proof}
	\end{theorem}
	\begin{corollary}
		Siguin \(\mathbb{K}\) un cos finit amb la suma \(+\) i el producte \(\cdot\) i \(\mathbb{F}\) un subcòs de \(\mathbb{K}\) amb \(\abs{\mathbb{K}}=p^{n}\). Aleshores \(\abs{\mathbb{F}}=p^{d}\) amb \(d\divides n\).
		\begin{proof}
			%TODO
		\end{proof}
	\end{corollary}
	\begin{theorem}[Teorema de l'element primitiu]
		\labelname{Teorema de l'element primitiu}\label{thm:teorema de l'element primitiu}
		Sigui \(\mathbb{K}\) un cos finit amb la suma \(+\) i el producte \(\cdot\). Aleshores \(\mathbb{K}\setminus\{0\}\) és un grup cíclic amb el producte \(\cdot\).
		\begin{proof}
			%TODO
		\end{proof}
	\end{theorem}
	\begin{definition}[Element primitiu]
		\labelname{element primitiu d'un cos finit}\label{def:element primitiu d'un cos finit}
		Sigui \(\mathbb{K}\) un cos finit amb la suma \(+\) i el producte \(\cdot\) i \(\beta\) un element de \(\mathbb{K}\) tal que \(\langle\{\beta\}\rangle=\mathbb{K}\setminus\{0\}\). Aleshores direm que \(\beta\) és un element primitiu de \(\mathbb{K}\).
		
		Aquesta definició té sentit pel \myref{thm:teorema de l'element primitiu}.
	\end{definition}
	\begin{theorem}
		Sigui \(\mathbb{K}\) un cos finit amb la suma \(+\) i el producte \(\cdot\) amb \(\abs{\mathbb{K}}=p\). Aleshores existeix un polinomi irreductible \(f(x)\) en \(\mathbb{Z}/(p)[x]\) tal que
		\[\mathbb{K}\cong\mathbb{Z}/(p)[x]/(f(x)).\]
		\begin{proof}
			%TODO
		\end{proof}
	\end{theorem}
	\subsection{Arrels d'un polinomi}
	\begin{definition}[Descomposició d'un polinomi]
		\labelname{descomposició d'un polinomi}\label{def:descomposició d'un polinomi}
		Siguin \(\mathbb{K}\) un cos amb la suma \(+\) i el producte \(\cdot\) i \(f(x)\) un polinomi de \(\mathbb{K}\) tal que existeixen \(\alpha_{1},\dots,\alpha_{n}\in\mathbb{K}\) satisfent \(f(x)=(x-\alpha_{1})\cdot\ldots\cdot(x-\alpha_{n})\). Aleshores direm que \(f(x)\) descompon en \(\mathbb{K}\).
	\end{definition}
	\begin{theorem}[Teorema de Kronecker]
		\labelname{Teorema de Kronecker}\label{thm:Teorema de Kronecker}
		Siguin \(\mathbb{K}\) un cos amb la suma \(+\) i el producte \(\cdot\) i \(f(x)\) un polinomi de \(\mathbb{K}[x]\). Aleshores existeix un cos \(\mathbb{L}\), amb \(\mathbb{K}\subseteq\mathbb{L}\), tal que \(f(x)\) descompon en \(\mathbb{L}\).
		\begin{proof}
			%TODO
		\end{proof}
	\end{theorem}
	\begin{definition}[Cos de descomposició]
		\labelname{cos de descomposició d'un polinomi}\label{def:cos de descomposició d'un polinomi}
		Siguin \(\mathbb{K}\) un cos amb la suma \(+\) i el producte \(\cdot\), \(f(x)\) un polinomi de \(\mathbb{K}\) i \(\mathbb{L}\) el mínim cos on \(f(x)\) descompon amb \(f(x)=(x-\alpha_{1})\cdot\ldots\cdot(x-\alpha_{n})\), amb \(\alpha_{1},\dots,\alpha_{n}\in\mathbb{L}\). Aleshores direm que \(\mathbb{L}\) és el cos descomposició de \(f(x)\). Denotarem \(\mathbb{L}=\mathbb{K}(f(x))\).
		
		Aquesta definició té sentit pel \myref{thm:Teorema de Kronecker}.
	\end{definition}
	\begin{definition}[Derivada formal]
		\labelname{derivada formal}\label{def:derivada formal}
		Siguin \(\mathbb{K}\) un cos amb la suma \(+\) i el producte \(\cdot\) i \(f(x)=a_{0}+a_{1}\cdot x+\dots+a_{n}\cdot x^{n}\) un polinomi de \(\mathbb{K}\). Aleshores definim la derivada formal de \(f(x)\) com
		\[f'(x)=a_{1}+2\cdot a_{2}x+3\cdot a_{3}\cdot x^{2}+\dots+n\cdot a_{n}\cdot x^{n-1}.\]
	\end{definition}
	\begin{proposition}
		Siguin \(\mathbb{K}\) un cos amb la suma \(+\) i el producte \(\cdot\) i \(f(x)\), \(g(x)\) dos polinomis de \(\mathbb{K}\). Aleshores es compleix
		\begin{enumerate}
			\item \((a\cdot f(x))'=a\cdot f'(x)\) per a tot \(a\in\mathbb{K}\).
			\item \((f(x)+g(x))'=f'(x)+g'(x)\).
			\item \((f(x)\cdot g(x))'=f'(x)\cdot g(x)+f(x)\cdot g'(x)\).
			\item \({(f(x)^{n})}'=n\cdot f(x)^{n-1}\).
		\end{enumerate}
		\begin{proof}
			%TODO
		\end{proof}
	\end{proposition}
	\begin{proposition}
		Siguin \(\mathbb{K}\) un cos amb la suma \(+\) i el producte \(\cdot\) amb \(\ch(\mathbb{K})=0\) i \(f(x)=a_{0}+a_{1}\cdot x+\cdots+a_{n}\cdot x^{n}\) un polinomi de \(\mathbb{K}\) amb \(n\geq1\). Aleshores \(n\cdot a_{n}\neq0\) i \(f'(x)\neq0\).
		\begin{proof}
			%TODO
		\end{proof}
	\end{proposition}
	\begin{proposition}
		Siguin \(\mathbb{K}\) un cos amb la suma \(+\) i el producte \(\cdot\) amb \(\ch(\mathbb{K})=p\) no nul i \(f(x)=a_{0}+a_{1}\cdot x+\cdots+a_{n}\cdot x^{n}\) un polinomi de \(\mathbb{K}\) amb \(n\geq1\). Aleshores
		\[f'(x)\neq0\sii p\divides i\text{ per a tot }i\geq1\text{ tal que }a_{i}\neq0.\]
		\begin{proof}
			%TODO
		\end{proof}
	\end{proposition}
	\begin{definition}[Arrels múltiples]
		\labelname{arrel múltiple}\label{def:arrel múltiple}
		Siguin \(\mathbb{K}\) un cos amb la suma \(+\) i el producte \(\cdot\) i \(f(x)\) un polinomi de \(\mathbb{K}[x]\), \(\alpha\) una arrel de \(f(x)\) i \(g(x)\) un polinomi de \(\mathbb{K}(f(x))\) tal que 
		\[f(x)=(x-\alpha)^{m}\cdot g(x)\]
		amb \(m\geq2\). Aleshores direm que \(\alpha\) és una arrel múltiple de \(f(x)\).
	\end{definition}
	\begin{proposition}
		Siguin \(\mathbb{K}\) un cos amb la suma \(+\) i el producte \(\cdot\) i \(f(x)\) un polinomi de \(\mathbb{K}[x]\). Aleshores \(\alpha\) és una arrel múltiple de \(f(x)\) si i només si \(\alpha\) és una arrel de \(f'(x)\).
		\begin{proof}
			%TODO
		\end{proof}
	\end{proposition}
	\begin{corollary}
		Siguin \(\mathbb{K}\) un cos amb la suma \(+\) i el producte \(\cdot\) i \(f(x)\) un polinomi de \(\mathbb{K}[x]\) amb \(\grau(f(x))\geq1\). Aleshores \(\mcd(f(x),f'(x))=1\) si i només si \(f(x)\) no té arrels múltiples.
		\begin{proof}
			%TODO
		\end{proof}
	\end{corollary}
	\section{Caracterització dels cossos finits i els seus subcossos}
	\subsection{Teoremes d'existència i unicitat dels cossos finits}
	\begin{theorem}[Teorema d'existència dels cossos finits]
		Siguin \(p\) un primer i \(n\) un natural. Aleshores existeix un cos \(\mathbb{K}\) amb la suma \(+\) i el producte \(\cdot\) tal que \(\abs{\mathbb{K}}=p^{n}\).
		\begin{proof}
			%TODO
		\end{proof}
	\end{theorem}
	\begin{corollary}
		Siguin \(p\) un primer i \(n\) un natural. Aleshores existeix un polinomi \(f(x)\) de \(\mathbb{Z}/(p)[x]\) amb \(\grau(f(x))=n\).
		\begin{proof}
			%TODO
		\end{proof}
	\end{corollary}
	\begin{lemma}
		Siguin \(n\), \(d\) dos naturals tal que \(d\divides n\), \(p\) un primer i \(f(x)\) un polinomi irreductible de \(\mathbb{Z}/(p)[x]\). Aleshores \(f(x)\divides(x^{p^{n}}-x)\).
		\begin{proof}
			%TODO
		\end{proof}
	\end{lemma}
	\begin{theorem}[Teorema d'unicitat dels cossos finits]
		Siguin \(\mathbb{K}\) un cos finit amb la suma \(+_{\mathbb{K}}\) i el producte \(\ast_{\mathbb{K}}\) i \(\mathbb{F}\) un cos finit amb la suma \(+_{\mathbb{F}}\) i el producte \(\ast_{\mathbb{F}}\) amb \(\abs{\mathbb{K}}=\abs{\mathbb{F}}=p^{n}\). Aleshores
		\[(\mathbb{K},+_{\mathbb{K}},\ast_{\mathbb{K}})\cong(\mathbb{F},+_{\mathbb{F}},\ast_{\mathbb{F}}).\]
		\begin{proof}
			%TODO
		\end{proof}
	\end{theorem}
	\begin{theorem}[Teorema d'existència dels subcossos finits]
		Siguin \(\mathbb{K}\) un cos amb la suma \(+\) i el producte \(\cdot\) amb \(\abs{\mathbb{K}}=p^{n}\) i \(d\) un natural tal que \(d\divides n\). Aleshores existeix un \(\mathbb{L}\subseteq\mathbb{K}\) amb \(\abs{\mathbb{L}}=p^{d}\) tal que \(\mathbb{L}\) és un subcòs de \(\mathbb{K}\).
		\begin{proof}
			%TODO
		\end{proof}
	\end{theorem}
	\begin{theorem}[Teorema d'unicitat dels subcossos finits]
		Siguin \(\mathbb{K}\) un cos amb la suma \(+\) i el producte \(\cdot\) amb \(\abs{\mathbb{K}}=p^{n}\), \(d\) un natural tal que \(d\divides n\) i \(\mathbb{L}_{1}\), \(\mathbb{L}_{2}\) dos subcossos de \(\mathbb{K}\) amb \(\abs{\mathbb{L}_{1}}=\abs{\mathbb{L}_{2}}=p^{d}\). Aleshores \(\mathbb{L}_{1}=\mathbb{L}_{2}\).
		\begin{proof}
			%TODO
		\end{proof}
	\end{theorem}
	\begin{notation}
		Denotarem el cos de \(p^{n}\) elements com \(\mathbb{F}_{p^{n}}\).
	\end{notation}
	\subsection{El morfisme de Frobenius}	%Juntar arrels amb això?
	\begin{theorem}
		Siguin \(n\) un natural, \(p\) un primer i
		\[\mathcal{F}=\{f(x)\in\mathbb{Z}/(p)[x]\mid f(x)\text{ és un polinomi mónic irreductible de grau }d\divides n\}.\]
		Aleshores
		\[x^{p^{n}}-x=\prod_{f(x)\in\mathcal{F}}f(x).\]
		\begin{proof}
			%TODO
		\end{proof}
	\end{theorem}
	\begin{proposition}[Morfisme de Frobenius]
		Sigui \(\mathbb{K}\) un cos finit amb la suma \(+\) i el producte \(\cdot\) amb \(\ch(\mathbb{K})=p\). Aleshores l'aplicació
		\begin{align*}
		\Phi\colon\mathbb{K}&\longrightarrow\mathbb{K}\\
		a&\longmapsto a^{p}
		\end{align*}
		és un automorfisme.
		\begin{proof}
			%TODO
		\end{proof}
	\end{proposition}
	\begin{theorem}
		Siguin \(p\) un primer, \(f(x)\) un polinomi irreductible de l'anell de polinomis \(\mathbb{Z}/(p)[x]\) amb \(\grau(f(x))=n\) i \(\alpha\) una arrel de \(f(x)\) en \(\mathbb{K}(p(x))\). Aleshores les arrels de \(f(x)\) són \(\alpha,\alpha^{p^{2}},\alpha^{p^{3}},\dots,\alpha^{p{n-1}}\) i \(\alpha^{p^{n}}=\alpha\).
		\begin{proof}
			%TODO %Aquest enunciat té sentit pel \myref{thm:Teorema de Kronecker}
		\end{proof}
	\end{theorem}
	\begin{theorem}
		Siguin \(p\) un primer, \(f(x)\) un polinomi irreductible de l'anell de polinomis \(\mathbb{Z}/(p)[x]\) amb \(\grau(f(x))=n\) i \(\alpha\) una arrel de \(f(x)\) en \(\mathbb{K}(p(x))\). Aleshores les arrels de \(f(x)\) són \(\alpha,\alpha^{p^{2}},\alpha^{p^{3}},\dots,\alpha^{p^{n-1}}\) i \(\alpha^{p^{i}}\neq\alpha^{p^{j}}\) per a tot \(i\neq j\), \(i,j\in\{0,\dots,n-1\}\).
		\begin{proof}
			%TODO. sentit pel teorema previ
		\end{proof}
	\end{theorem}
	\printbibliography
	El capítol de teoria de grups està molt ben explicat en \cite{NumerosGruposyAnillos}, i la teoria de cossos finits està complementada amb \cite{AntoineRosaCampsMoncasiIntroduccioAlgebraAbstracta} sobre la teoria de classe.
	
	La bibliografia del curs inclou els textos \cite{AntoineRosaCampsMoncasiIntroduccioAlgebraAbstracta,CedoAlgebraBasica,CohnBasicAlgebra,NumerosGruposyAnillos,FelixConcepcionSebastianIntroduccionAlAlgebra,FraleighFirstCourseAbstractAlgebra,HungerfordAlgebra}.
\end{document}

% Teorema d'òrbita-estabilitzador i lema de Burnside